\documentclass[letter-paper]{tufte-book}

%%
% Book metadata
\title{Statistical Mechanics}
\author[]{Inusuke Shibemoto}
%\publisher{Research Institute of Valinor}

%%
% If they're installed, use Bergamo and Chantilly from www.fontsite.com.
% They're clones of Bembo and Gill Sans, respectively.
\IfFileExists{bergamo.sty}{\usepackage[osf]{bergamo}}{}% Bembo
\IfFileExists{chantill.sty}{\usepackage{chantill}}{}% Gill Sans

%\usepackage{microtype}
\usepackage{amssymb}
\usepackage{amsmath}
%%
% For nicely typeset tabular material
\usepackage{booktabs}

%% overunder braces
\usepackage{oubraces}

%% 
\usepackage{xcolor}
\usepackage{tcolorbox}

\newtcolorbox[auto counter,number within=section]{derivbox}[2][]{colback=TealBlue!5!white,colframe=TealBlue,title=Box \thetcbcounter:\ #2,#1}                                                          

\makeatletter
\@openrightfalse
\makeatother

%%
% For graphics / images
\usepackage{graphicx}
\setkeys{Gin}{width=\linewidth,totalheight=\textheight,keepaspectratio}
\graphicspath{{figs/}}

% The fancyvrb package lets us customize the formatting of verbatim
% environments.  We use a slightly smaller font.
\usepackage{fancyvrb}
\fvset{fontsize=\normalsize}

\usepackage[plain]{fancyref}
\newcommand*{\fancyrefboxlabelprefix}{box}
\fancyrefaddcaptions{english}{%
  \providecommand*{\frefboxname}{Box}%
  \providecommand*{\Frefboxname}{Box}%
}
\frefformat{plain}{\fancyrefboxlabelprefix}{\frefboxname\fancyrefdefaultspacing#1}
\Frefformat{plain}{\fancyrefboxlabelprefix}{\Frefboxname\fancyrefdefaultspacing#1}

%%
% Prints argument within hanging parentheses (i.e., parentheses that take
% up no horizontal space).  Useful in tabular environments.
\newcommand{\hangp}[1]{\makebox[0pt][r]{(}#1\makebox[0pt][l]{)}}

%% 
% Prints an asterisk that takes up no horizontal space.
% Useful in tabular environments.
\newcommand{\hangstar}{\makebox[0pt][l]{*}}

%%
% Prints a trailing space in a smart way.
\usepackage{xspace}
\usepackage{xstring}

%%
% Some shortcuts for Tufte's book titles.  The lowercase commands will
% produce the initials of the book title in italics.  The all-caps commands
% will print out the full title of the book in italics.
\newcommand{\vdqi}{\textit{VDQI}\xspace}
\newcommand{\ei}{\textit{EI}\xspace}
\newcommand{\ve}{\textit{VE}\xspace}
\newcommand{\be}{\textit{BE}\xspace}
\newcommand{\VDQI}{\textit{The Visual Display of Quantitative Information}\xspace}
\newcommand{\EI}{\textit{Envisioning Information}\xspace}
\newcommand{\VE}{\textit{Visual Explanations}\xspace}
\newcommand{\BE}{\textit{Beautiful Evidence}\xspace}

\newcommand{\TL}{Tufte-\LaTeX\xspace}

% Prints the month name (e.g., January) and the year (e.g., 2008)
\newcommand{\monthyear}{%
  \ifcase\month\or January\or February\or March\or April\or May\or June\or
  July\or August\or September\or October\or November\or
  December\fi\space\number\year
}


\newcommand{\urlwhitespacereplace}[1]{\StrSubstitute{#1}{ }{_}[\wpLink]}

\newcommand{\wikipedialink}[1]{http://en.wikipedia.org/wiki/#1}% needs \wpLink now

\newcommand{\anonymouswikipedialink}[1]{\urlwhitespacereplace{#1}\href{\wikipedialink{\wpLink}}{Wikipedia}}

\newcommand{\Wikiref}[1]{\urlwhitespacereplace{#1}\href{\wikipedialink{\wpLink}}{#1}}

% Prints an epigraph and speaker in sans serif, all-caps type.
\newcommand{\openepigraph}[2]{%
  %\sffamily\fontsize{14}{16}\selectfont
  \begin{fullwidth}
  \sffamily\large
  \begin{doublespace}
  \noindent\allcaps{#1}\\% epigraph
  \noindent\allcaps{#2}% author
  \end{doublespace}
  \end{fullwidth}
}

% Inserts a blank page
\newcommand{\blankpage}{\newpage\hbox{}\thispagestyle{empty}\newpage}

\usepackage{units}

% Typesets the font size, leading, and measure in the form of 10/12x26 pc.
\newcommand{\measure}[3]{#1/#2$\times$\unit[#3]{pc}}

% Macros for typesetting the documentation
\newcommand{\hlred}[1]{\textcolor{Maroon}{#1}}% prints in red
\newcommand{\hangleft}[1]{\makebox[0pt][r]{#1}}
\newcommand{\hairsp}{\hspace{1pt}}% hair space
\newcommand{\hquad}{\hskip0.5em\relax}% half quad space
\newcommand{\TODO}{\textcolor{red}{\bf TODO!}\xspace}
\newcommand{\na}{\quad--}% used in tables for N/A cells
\providecommand{\XeLaTeX}{X\lower.5ex\hbox{\kern-0.15em\reflectbox{E}}\kern-0.1em\LaTeX}
\newcommand{\tXeLaTeX}{\XeLaTeX\index{XeLaTeX@\protect\XeLaTeX}}
% \index{\texttt{\textbackslash xyz}@\hangleft{\texttt{\textbackslash}}\texttt{xyz}}
\newcommand{\tuftebs}{\symbol{'134}}% a backslash in tt type in OT1/T1
\newcommand{\doccmdnoindex}[2][]{\texttt{\tuftebs#2}}% command name -- adds backslash automatically (and doesn't add cmd to the index)
\newcommand{\doccmddef}[2][]{%
  \hlred{\texttt{\tuftebs#2}}\label{cmd:#2}%
  \ifthenelse{\isempty{#1}}%
    {% add the command to the index
      \index{#2 command@\protect\hangleft{\texttt{\tuftebs}}\texttt{#2}}% command name
    }%
    {% add the command and package to the index
      \index{#2 command@\protect\hangleft{\texttt{\tuftebs}}\texttt{#2} (\texttt{#1} package)}% command name
      \index{#1 package@\texttt{#1} package}\index{packages!#1@\texttt{#1}}% package name
    }%
}% command name -- adds backslash automatically
\newcommand{\doccmd}[2][]{%
  \texttt{\tuftebs#2}%
  \ifthenelse{\isempty{#1}}%
    {% add the command to the index
      \index{#2 command@\protect\hangleft{\texttt{\tuftebs}}\texttt{#2}}% command name
    }%
    {% add the command and package to the index
      \index{#2 command@\protect\hangleft{\texttt{\tuftebs}}\texttt{#2} (\texttt{#1} package)}% command name
      \index{#1 package@\texttt{#1} package}\index{packages!#1@\texttt{#1}}% package name
    }%
}% command name -- adds backslash automatically
\newcommand{\docopt}[1]{\ensuremath{\langle}\textrm{\textit{#1}}\ensuremath{\rangle}}% optional command argument
\newcommand{\docarg}[1]{\textrm{\textit{#1}}}% (required) command argument
\newenvironment{docspec}{\begin{quotation}\ttfamily\parskip0pt\parindent0pt\ignorespaces}{\end{quotation}}% command specification environment
\newcommand{\docenv}[1]{\texttt{#1}\index{#1 environment@\texttt{#1} environment}\index{environments!#1@\texttt{#1}}}% environment name
\newcommand{\docenvdef}[1]{\hlred{\texttt{#1}}\label{env:#1}\index{#1 environment@\texttt{#1} environment}\index{environments!#1@\texttt{#1}}}% environment name
\newcommand{\docpkg}[1]{\texttt{#1}\index{#1 package@\texttt{#1} package}\index{packages!#1@\texttt{#1}}}% package name
\newcommand{\doccls}[1]{\texttt{#1}}% document class name
\newcommand{\docclsopt}[1]{\texttt{#1}\index{#1 class option@\texttt{#1} class option}\index{class options!#1@\texttt{#1}}}% document class option name
\newcommand{\docclsoptdef}[1]{\hlred{\texttt{#1}}\label{clsopt:#1}\index{#1 class option@\texttt{#1} class option}\index{class options!#1@\texttt{#1}}}% document class option name defined
\newcommand{\docmsg}[2]{\bigskip\begin{fullwidth}\noindent\ttfamily#1\end{fullwidth}\medskip\par\noindent#2}
\newcommand{\docfilehook}[2]{\texttt{#1}\index{file hooks!#2}\index{#1@\texttt{#1}}}
\newcommand{\doccounter}[1]{\texttt{#1}\index{#1 counter@\texttt{#1} counter}}

\newcommand{\studyq}[1]{\marginnote{Q: #1}}

\hypersetup{colorlinks}% uncomment this line if you prefer colored hyperlinks (e.g., for onscreen viewing)

% Generates the index
\usepackage{makeidx}
\makeindex

\setcounter{tocdepth}{3}
\setcounter{secnumdepth}{3}

%%%%%%%%%%%%%%%%%%%%%%%%%%%%%%%%%%%%%%%%%%%%%%%%%%%%%%%%%%%%%%
% custom commands

\newtheorem{theorem}{\color{pastel-blue}Theorem}[section]
\newtheorem{lemma}[theorem]{\color{pastel-blue}Lemma}
\newtheorem{proposition}[theorem]{\color{pastel-blue}Proposition}
\newtheorem{corollary}[theorem]{\color{pastel-blue}Corollary}

\newenvironment{proof}[1][Proof]{\begin{trivlist}
\item[\hskip \labelsep {\bfseries #1}]}{\end{trivlist}}
\newenvironment{definition}[1][Definition]{\begin{trivlist}
\item[\hskip \labelsep {\bfseries #1}]}{\end{trivlist}}
\newenvironment{example}[1][Example]{\begin{trivlist}
\item[\hskip \labelsep {\bfseries #1}]}{\end{trivlist}}
\newenvironment{remark}[1][Remark]{\begin{trivlist}
\item[\hskip \labelsep {\bfseries #1}]}{\end{trivlist}}

\hyphenpenalty=5000

% more pastel ones
\xdefinecolor{pastel-red}{rgb}{0.77,0.31,0.32}
\xdefinecolor{pastel-green}{rgb}{0.33,0.66,0.41}
\definecolor{pastel-blue}{rgb}{0.30,0.45,0.69} % crayola blue
\definecolor{gray}{rgb}{0.2,0.2,0.2} % dark gray

\xdefinecolor{orange}{rgb}{1,0.45,0}
\xdefinecolor{green}{rgb}{0,0.35,0}
\definecolor{blue}{rgb}{0.12,0.46,0.99} % crayola blue
\definecolor{gray}{rgb}{0.2,0.2,0.2} % dark gray

\xdefinecolor{cerulean}{rgb}{0.01,0.48,0.65}
\xdefinecolor{ust-blue}{rgb}{0,0.20,0.47}
\xdefinecolor{ust-mustard}{rgb}{0.67,0.52,0.13}

%\newcommand\comment[1]{{\color{red}#1}}

\newcommand{\dy}{\partial}
\newcommand{\ddy}[2]{\frac{\dy#1}{\dy#2}}

\newcommand{\ab}{\boldsymbol{a}}
\newcommand{\bb}{\boldsymbol{b}}
\newcommand{\cb}{\boldsymbol{c}}
\newcommand{\db}{\boldsymbol{d}}
\newcommand{\eb}{\boldsymbol{e}}
\newcommand{\lb}{\boldsymbol{l}}
\newcommand{\nb}{\boldsymbol{n}}
\newcommand{\tb}{\boldsymbol{t}}
\newcommand{\ub}{\boldsymbol{u}}
\newcommand{\vb}{\boldsymbol{v}}
\newcommand{\xb}{\boldsymbol{x}}
\newcommand{\wb}{\boldsymbol{w}}
\newcommand{\yb}{\boldsymbol{y}}

\newcommand{\Xb}{\boldsymbol{X}}

\newcommand{\ex}{\mathrm{e}}
\newcommand{\zi}{{\rm i}}

\newcommand\Real{\mbox{Re}} % cf plain TeX's \Re and Reynolds number
\newcommand\Imag{\mbox{Im}} % cf plain TeX's \Im

\newcommand{\zbar}{{\overline{z}}}

\newcommand\Def[1]{\textbf{#1}}

\newcommand{\qed}{\hfill$\blacksquare$}
\newcommand{\qedwhite}{\hfill \ensuremath{\Box}}

%%%%%%%%%%%%%%%%%%%%%%%%%%%%%%%%%%%%%%%%%%%%%%%%%%%%%%%%%%%%%%
% some extra formatting (hacked from Patrick Farrell's notes)
%  https://courses.maths.ox.ac.uk/node/view_material/4915
%

% chapter format
\titleformat{\chapter}%
  {\huge\rmfamily\itshape\color{pastel-red}}% format applied to label+text
  {\llap{\colorbox{pastel-red}{\parbox{1.5cm}{\hfill\itshape\huge\color{white}\thechapter}}}}% label
  {1em}% horizontal separation between label and title body
  {}% before the title body
  []% after the title body

% section format
\titleformat{\section}%
  {\normalfont\Large\itshape\color{pastel-green}}% format applied to label+text
  {\llap{\colorbox{pastel-green}{\parbox{1.5cm}{\hfill\color{white}\thesection}}}}% label
  {1em}% horizontal separation between label and title body
  {}% before the title body
  []% after the title body

% subsection format
\titleformat{\subsection}%
  {\normalfont\large\itshape\color{pastel-blue}}% format applied to label+text
  {\llap{\colorbox{pastel-blue}{\parbox{1.5cm}{\hfill\color{white}\thesubsection}}}}% label
  {1em}% horizontal separation between label and title body
  {}% before the title body
  []% after the title body

%%%%%%%%%%%%%%%%%%%%%%%%%%%%%%%%%%%%%%%%%%%%%%%%%%%%%%%%%%%%%%%%%%%%%%%%%%%%%%%%

\begin{document}

% Front matter
%\frontmatter

% r.3 full title page
%\maketitle

% v.4 copyright page

\chapter*{}

\begin{fullwidth}

\par \begin{center}{\Huge Statistical Mechanics 3/4H}\end{center}

\vspace*{5mm}

\par \begin{center}{\Large typed up by B. S. H. Mithrandir}\end{center}

\vspace*{5mm}

\begin{itemize}
  \item \textit{Last compiled: \monthyear}
  \item Adapted from notes of Veronika Hubeny and someone else (maybe Mukund?), Durham
  \item This was part of the Durham 3/4H elective. Includes introduction to
  classical thermodynamics and some quantum systems.
  \item It is sometimes said that there are probably two ways to approach
  thermodynamics: start from classical thermodynamics first, or start from a
  stat mech approach. These notes take the latter approach. From a physical
  understanding point of view (and my personal point of view), these particular
  notes might be more informative / useful if the intended user already has some
  exposure already to the traditional way of approaching thermodynamics.
  \item[]
  \item \TODO diagrams
  \item \TODO to possibly merge in some other notes (e.g. expand a bit more on
  classical thermodynamics; or maybe not and keep it separate to highlight
  differences in approach)
\end{itemize}

\par

\par Licensed under the Apache License, Version 2.0 (the ``License''); you may not
use this file except in compliance with the License. You may obtain a copy
of the License at \url{http://www.apache.org/licenses/LICENSE-2.0}. Unless
required by applicable law or agreed to in writing, software distributed
under the License is distributed on an \smallcaps{``AS IS'' BASIS, WITHOUT
WARRANTIES OR CONDITIONS OF ANY KIND}, either express or implied. See the
License for the specific language governing permissions and limitations
under the License.
\end{fullwidth}

%===============================================================================

\chapter{Classical thermodynamics}

The aim is to describe thermodynamics in \Def{equilibrium}.

\Def{Extensive variables} $X_i$ of a thermodynamic system are those that depend
and scales proportionally to the system size. The set of all extensive variables
are sufficient to describe the system in equilibrium. Examples include
\begin{itemize}
  \item energy
  \item volume
  \item charge
  \item number of particles
\end{itemize}

Thermodynamic system can either be \Def{isolated} or \Def{interacting} with an
external environment. Isolated systems to not exchange any of the extensive
parameters the environment.

\vspace*{3mm}

\Def{Thermodynamic postulate 1}:

\begin{center}\textit{There exists and equilibrium state.}\end{center}

\begin{proposition}[Generalised first law of thermodynamics]
At equilibrium,
\begin{equation}
  \frac{\mathrm{d}X_i}{\mathrm{d}t} = 0
\end{equation}
for all extensive variables $X_i$ in a system.
\end{proposition}

The above is usually phrased as\sidenote{Sign of $W$ depends on convention.
Positive sign here as energy transferring into system.}
\begin{equation}
  \delta U = \delta Q + \delta W,
\end{equation}
where $U$ is the \Def{internal energy}, $Q$ is the \Def{heat}, and $W$ is
\Def{work done}. Ther conservation of internal energy gives
\begin{equation}
  \mathrm{d}U = T\ \mathrm{d}S - P\ \mathrm{d}V + \phi_i q_i,
\end{equation}
where $S$ is the \Def{entropy} (see later), $T$ is temperature, $P$ is
\Def{pressure}, $V$ is \Def{volume}, and $\phi_i$ and $q_i$ denote some other
form of change. The exact definitions of some of these variables with respect to
the more familiar thermodynamic quantities will be provided later.

\vspace*{3mm}

\Def{Thermodynamic postulate 2}:

\begin{center}\textit{There exists a function $S$ called the \Def{entropy} for all thermodynamic systems such that}\end{center}
\textit{
\begin{enumerate}
  \item $S = S(X)$, $X = \{X_i\}$, and $S$ is itself extensive;
  \item $S \in C^{\infty}$ (usually)
  \item $S$ is maximised as a function of $X$ when the system is in thermal
  equilibrium.
\end{enumerate}
}

The two postulates constrain the allowed physical processes that go on within a
thermodynamic system.

\begin{example}
\marginnote{Superscripts denote system, subscripts denote variable within a system.}
Given two systems $\mathcal{S}^{1,2}$ with extensive parameters $X^{1,2}$
respectively, for $\mathcal{S} = \mathcal{S}^1 \cup \mathcal{S}^2$, the total
set of extensive parameters is $X = X^1 + X^2$. Since entropy is maximised,
$S(X) = S(X^1 + X^2) \geq S_1(X^1) + S_2(X^2)$.

Suppose $S(X) = S_1(X^1 + \delta X^1) + S_2(X^2 + \delta X^2)$, where $\delta
X^2$ is the required change to bring the system total system to equilibrium.
Then we want $\delta X^! + \delta X^2 = 0$. Considering the variation, we have
\begin{equation*}
  \delta S = \frac{\partial S_1}{\partial X^1} \delta X^1 + \frac{\partial S_2}{\partial X^2} \delta X^2 = \delta X^1 \left(\frac{\partial S_1}{\partial X^1} - \frac{\partial S_2}{\partial X^2}\right).
\end{equation*}
Since $\delta S \geq 0$, the sign of $\delta X^1$ is the same as $(\partial S_1
/ \partial X^1 - \partial S_2 / \partial X^2)$, thus determining the direction
of flow for $X$.
\end{example}

Some properties of entropy:
\begin{itemize}
  \item $S(X)$ is a homogeneous function of degree 1, i.e. $S(\alpha X) =
  \alpha S(X)$ for some scalar $\alpha$;
  \item the defining relation of entropy in terms of other extensive parameters
  is called the \Def{entropy fundamental relation}, which provides a complete
  specific formulation of the thermodynamic system;
  \item physically, the entropy measures the amount of disorder in the system.
\end{itemize}

\Def{Intensive parameters} $Z_i$ are those that are \Def{conjugate} to the
extensive parameters $X_i$, i.e.
\begin{equation}
  Z_i = \frac{\partial S}{\partial X_i}.
\end{equation}
Clearly $Z_i$ has to be homogeneous functions of degree 0, and $Z_i(X)$ is
called the \Def{equation of state}, but requires additional information in order
to have a complete specification of the thermodynamic system.

\begin{lemma}
If two systems $\mathcal{S}^{1,2}$ interact and exchange extensive variables
$X^{1,2}$, at equilibrium, we have $Z^1 = Z^2$.
\end{lemma}
\begin{proof}
Since we are at equilibrium, we aim to maximise $S = S^1 + S^2$ keeping $X = X^1
+ X^2$ fixed. This can be done either by variation, with
\begin{equation*}
  0 = \delta S = \delta X^1 \left(\frac{\partial S_1}{\partial X^1} - \frac{\partial S_2}{\partial X^2}\right) = (Z^1 - Z^2) \delta X^1 \qquad \Rightarrow \qquad Z^1 = Z^2,
\end{equation*}
or by the method of Lagrange multiplies, we aim to maximise $S - \lambda(X^1 +
X^2)$ to get
\begin{equation*}
  \frac{\partial S}{\partial X^1} - \lambda = 0, \qquad \frac{\partial S}{\partial X^2} - \lambda = 0
\end{equation*}
which implies $\lambda = Z^1 = Z^2$.
\qed
\end{proof}

\vspace*{3mm}

\Def{Zeroth law of thermodynamics}\sidenote{More commonly, if two systems are in
thermal equilibrium with a third system, then they are in equlibrium with each
other. This a statement about \emph{transitivity} and establishes thermal
equilibrium as an equivalence relation from a mathematical point of view.}:

\begin{center}\textit{For two systems in thermal equilibrium with each other,
$Z^1_i = Z^2_i$ for all $i$.}\end{center}

When two systems $\mathcal{S}^{1,2}$ are not in equilibrium and $Z^1 > Z^2$,
$\mathrm{d}S > 0$ implies $X$ flows from $\mathcal{S}^1$ to $\mathcal{S}^2$, and
we have a \Def{directed flow}.

Instead of entropy $S$, we can take the internal energy $U$ as the fundamental
variable\sidenote{And really is what would be the more traditional way of
introducing classical thermodynamics.}. We have the \Def{energetic fundamental
relations} with intensive parameters
\begin{equation}
  Y_i = \frac{\partial U}{\partial X_i}.
\end{equation}
From this, we note that
\begin{itemize}
  \item \Def{temperature} $T$ is conjugate to entropy $S$, i.e.
  \begin{equation}
    T = \frac{\partial U}{\partial S};
  \end{equation}
  \item \Def{pressure} $P$ is conjugate to volume $V$, i.e.
  \begin{equation}
    P = -\frac{\partial U}{\partial V};
  \end{equation}
  \item \Def{chemical potential} $\mu$ is conjugate to particle number $N$, i.e.
  \begin{equation}
    \mu = \frac{\partial U}{\partial N}.
  \end{equation}
\end{itemize}
All signs are by convention. Occasionally, we employ the \Def{inverse
temperature} variable $\beta$ given by
\begin{equation}
  \beta = \frac{1}{T} = \frac{\partial S}{\partial U}.
\end{equation}

\begin{example}
An alternative phrasing of the first law of thermodynamics\sidenote{Actually the
more traditional way of stating the first law.} is given as energy conservation
(Einstein summation convention implied)
\begin{equation}
  \mathrm{d}U = T\ \mathrm{d}S + \frac{\partial U}{\partial X_i}\ \mathrm{d}X_i = T\ \mathrm{d}S + Y_i\ \mathrm{d}X_i.
\end{equation}
Then, by rephrasing in the conjugate variables, we have
\begin{align*}
  \mathrm{d}S &= \beta\ \mathrm{d}U + Z_i\ \mathrm{d}X_i \\
    &= \beta(T\ \mathrm{d}S + Y_i\ \mathrm{d}X_i) + Z_i\ \mathrm{d}X_i \\
    &= \mathrm{d}S + (\beta Y_i + Z_i)\ \mathrm{d}X_i,
\end{align*}
implying
\begin{equation}
  Y_i = -T Z_i,
\end{equation}
i.e. all individual intensive parameters associated with $T$ and $S$ are related
by a factor of temperature.
\end{example}

A system is said to undergo a \Def{quasi-static} process if, at every instant of
time along the path in the phase space, one can approximate the system as being
in equilibrium. A quasi-static process is \Def{adiabatic} if the system does not
exchange heat with its surroundings, i.e. $\mathrm{d}S = 0$ along the
appropriate curve in phase space.

\begin{example}[Ideal gas]
Consider an \Def{ideal gas}\sidenote{Randomly moving point particles that are
not subject to inter-particle interactions.} in equilibrium, which is made to
undergo a quasi-static expansion from $V_i$ to $V_f$ adiabatically. During this
process, $PV^\gamma = \textnormal{constant}$, where $\gamma$ is some number (the
\Def{adaibatic index} or the \Def{heat capacity ratio}). We aim to calculated
the change in internal energy of the gas, and the work done and heat absorbed by
the gas when it goes from $(P_i, V_i)$ to $(P_f, V_f)$.

% expand at constant P, then decreasing P at constant V
\end{example}
%===============================================================================

%%%%%%%%%%%%%%%%%%%%%%%%%%%%%%%%%%%%%%%%%

% r.5 contents
%\tableofcontents

%\listoffigures

%\listoftables

% r.7 dedication
%\cleardoublepage
%~\vfill
%\begin{doublespace}
%\noindent\fontsize{18}{22}\selectfont\itshape
%\nohyphenation
%Dedicated to those who appreciate \LaTeX{} 
%and the work of \mbox{Edward R.~Tufte} 
%and \mbox{Donald E.~Knuth}.
%\end{doublespace}
%\vfill

% r.9 introduction
% \cleardoublepage

%%%%%%%%%%%%%%%%%%%%%%%%%%%%%%%%%%%%%%%%%
% actual useful crap (normal chapters)
\mainmatter

%\part{Basics (?)}


%\backmatter

%\bibliography{refs}
\bibliographystyle{plainnat}

%\printindex

\end{document}

