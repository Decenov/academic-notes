%% ****** Start of file aiptemplate.tex ****** %
%%
%%   This file is part of the files in the distribution of AIP substyles for REVTeX4.
%%   Version 4.1 of 9 October 2009.
%%

%
% This is a template for producing documents for use with 
% % the REVTEX 4.1 document class and the AIP substyles.
% 
% Copy this file to another name and then work on that file.
% That way, you always have this original template file to use.

%\documentclass[aip,pof,graphicx,11pt]{revtex4-1}
\documentclass[10pt,notitlepage]{revtex4-1}
\usepackage{amssymb,amsbsy,times,fancyhdr,color}
\usepackage{amsmath}
\usepackage{mathrsfs,amsfonts}
%\usepackage[dvips]{epsfig,graphics,lscape}
%\usepackage{natbib}
\usepackage{latexsym,afterpage}
\usepackage{graphicx,subfigure,multirow}
\usepackage{color}
\newcommand{\dfd}{\mathrm{d}}
\newcommand{\imi}{\mathrm{i}}

%\draft % marks overfull lines with a black rule on the right

% extra macros; disable accordingly
\linespread{1}
%\setlength{\parindent}{0pt}

% ===========================================================================

\usepackage{psfrag}
%\usepackage{showlabels}
\usepackage{color}

% ===========================================================================

\newtheorem{theorem}{Theorem}[section]
\newtheorem{lemma}[theorem]{Lemma}
\newtheorem{proposition}[theorem]{Proposition}
\newtheorem{corollary}[theorem]{Corollary}

\newenvironment{proof}[1][Proof]{\begin{trivlist}
\item[\hskip \labelsep {\bfseries #1}]}{\end{trivlist}}
\newenvironment{definition}[1][Definition]{\begin{trivlist}
\item[\hskip \labelsep {\bfseries #1}]}{\end{trivlist}}
\newenvironment{example}[1][Example]{\begin{trivlist}
\item[\hskip \labelsep {\bfseries #1}]}{\end{trivlist}}
\newenvironment{remark}[1][Remark]{\begin{trivlist}
\item[\hskip \labelsep {\bfseries #1}]}{\end{trivlist}}

\newcommand{\qed}{\nobreak \ifvmode \relax \else
      \ifdim\lastskip<1.5em \hskip-\lastskip
      \hskip1.5em plus0em minus0.5em \fi \nobreak
      \vrule height0.75em width0.5em depth0.25em\fi}

% ===========================================================================

% For multiletter symbols
\newcommand\Real{\mbox{Re}} % cf plain TeX's \Re and Reynolds number
\newcommand\Imag{\mbox{Im}} % cf plain TeX's \Im
\newcommand\Rey{\mbox{\textit{Re}}}  % Reynolds number
\newcommand\Pran{\mbox{\textit{Pr}}} % Prandtl number, cf TeX's \Pr product
\newcommand\Pen{\mbox{\textit{Pe}}}  % Peclet number
\newcommand\Ai{\mbox{Ai}}            % Airy function
\newcommand\Bi{\mbox{Bi}}            % Airy function

% Our general macros:

\newcommand{\dy}{\partial}
\newcommand{\ddy}[2]{\frac{\dy#1}{\dy#2}}

\newcommand{\ex}{\mathrm{e}}

\newcommand{\zi}{{\rm i}}

\newcommand{\grad}{\nabla}

\newcommand{\sech}{\mathrm{sech}}

% Notational macros for this ms only: ---------------------------------------

\newcommand{\eb}{\boldsymbol{e}}
\newcommand{\xb}{\boldsymbol{x}}
\newcommand{\ub}{\boldsymbol{u}}
\newcommand{\vb}{\boldsymbol{v}}
\newcommand{\wb}{\boldsymbol{w}}

\newcommand{\Ab}{{\boldsymbol{A}}}
\newcommand{\ab}{{\boldsymbol{a}}}
\newcommand{\Bb}{{\boldsymbol{B}}}
\newcommand{\bb}{{\boldsymbol{b}}}
\newcommand{\cb}{{\boldsymbol{c}}}
\newcommand{\Db}{{\boldsymbol{D}}}
\newcommand{\db}{{\boldsymbol{d}}}
\newcommand{\Fb}{{\boldsymbol{F}}}
\newcommand{\Ib}{{\boldsymbol{I}}}
\newcommand{\jb}{{\boldsymbol{j}}}
\newcommand{\Jb}{{\boldsymbol{J}}}
\newcommand{\kb}{{\boldsymbol{k}}}
\newcommand{\Lb}{{\boldsymbol{L}}}
\newcommand{\lb}{{\boldsymbol{l}}}
\newcommand{\mb}{{\boldsymbol{m}}}
\newcommand{\nb}{{\boldsymbol{n}}}
\newcommand{\pb}{{\boldsymbol{p}}}
\newcommand{\Sb}{{\boldsymbol{S}}}
\newcommand{\Vb}{{\boldsymbol{V}}}
\newcommand{\Ub}{{\boldsymbol{U}}}
\newcommand{\tb}{{\boldsymbol{t}}}
\newcommand{\Xb}{{\boldsymbol{X}}}
\newcommand{\yb}{\boldsymbol{y}}

\newcommand{\Ubar}{{\overline{U}}}
\newcommand{\ubar}{{\overline{u}}}
\newcommand{\Bbar}{{\overline{B}}}
\newcommand{\psibar}{{\overline{\psi}}}
\newcommand{\qbar}{{\overline{q}}}
\newcommand{\zbar}{{\overline{z}}}

\newcommand{\Ri}{R}
\newcommand{\Fr}{F}

\newcommand{\qhat}{{\hat{q}}}
\newcommand{\jhat}{{\hat{j}}}

\newcommand\comment[1]{{\color{red}#1}}

% ===========================================================================

\begin{document}

% Use the \preprint command to place your local institutional report number 
% on the title page in preprint mode.
% Multiple \preprint commands are allowed.
%\preprint{}

\title{Academic notes: 1A Geometry} %Title of paper

% repeat the \author .. \affiliation  etc. as needed
% \email, \thanks, \homepage, \altaffiliation all apply to the current author.
% Explanatory text should go in the []'s, 
% actual e-mail address or url should go in the {}'s for \email and \homepage.
% Please use the appropriate macro for the type of information
% \affiliation command applies to all authors since the last \affiliation command. 
% The \affiliation command should follow the other information.

\author{J. Mak (\today) [From notes of R. Gregory, Durham]}
\email[]{julian.c.l.mak@googlemail.com}

%\author{D. W. Hughes}

%\author{S. D. Griffiths}

%\email[]{Your e-mail address}
%\homepage[]{Your web page}
%\thanks{}
%\altaffiliation{}

%\affiliation{}

% Collaboration name, if desired (requires use of superscriptaddress option in \documentclass). 
% \noaffiliation is required (may also be used with the \author command).
%\collaboration{}
%\noaffiliation
%\date{\today}

%\begin{abstract}
%Notes on the vorticity wave mechanism in 2d non-Boussinesq setting.
%\end{abstract}

%\pacs{}% insert suggested PACS numbers in braces on next line

\maketitle %\maketitle must follow title, authors, abstract and \pacs

% Body of paper goes here. Use proper sectioning commands. 
% References should be done using the \cite, \ref, and \label commands

%===============================================================================

\section*{Misc. notes}

\begin{itemize}
  \item This was part of the Durham Core A module given in the first year. This
  is standard Euclidean geometry involving some manipulations with matrices, as
  a precursor of linear algebra.
  \item The original course has geometry of complex numbers here, but for
  consistency reasons this has been moved to Complex Analaysis 2H in this
  organisation of notes.
  \item[]
  \item \comment{(to be fixed?)} I seem to have managed to do the notes with
  absolutely no diagrams, probably should be fixed
\end{itemize}

%===============================================================================

\section{Geometry on the plane}

Vectors in 2-dimensional space $\mathbb{R}^{2}$ may be written as $(x,y)$, $x$
and $y$ being co-ordinates for vector $\vb$, encoding how far along the axis
they are. Elementary vector operations are \underline{addition} and
\underline{multiplication by a scalar}, given respectively by
\begin{equation}
	\ub+\vb=(u_1 + v_1, u_2 + v_2),\qquad 
	\lambda\vb=(\lambda v_1, \lambda v_2),
\end{equation}
with $\lambda\in\mathbb{R}$. These operations satisfy the axioms of vector
space. For $\ub,\vb,\wb\in\mathbb{R}^2$ and $\lambda,\mu\in\mathbb{R}$
\begin{itemize}
	\item \underline{closure}
	\begin{enumerate}
		\item $(\ub+\vb)\in\mathbb{R}^2$.
		\item $\lambda\vb\in\mathbb{R}^2$.
	\end{enumerate}
	\item \underline{Addition}
	\begin{enumerate}
		\item \underline{Associativity}: $(\ub+\vb)+\wb=\ub+(\vb+\wb)$.
		\item \underline{Identity}: There exists a $\mathbf{0}\in\mathbb{R}^2$
		where $\vb+\mathbf{0}=\mathbf{0}+\vb=\vb$.
		\item \underline{Inverse}: There exists a $-\vb\in\mathbb{R}^2$ where
		$\vb+(-\vb)=\mathbf{0}$.
		\item \underline{Commutativity}: $\vb+\wb=\wb+\vb$.
	\end{enumerate}
	\item \underline{Multiplication},
	\begin{enumerate}
		\item \underline{Associativity}: $(\lambda\mu)\vb=\lambda(\mu\vb)$.
		\item \underline{Zero}: $0\vb=\mathbf{0}$.
		\item \underline{One}: $1\vb=\vb$.
		\item \underline{Distributive}: $(\lambda+\mu)\vb=\lambda\vb+\mu\vb$.
	\end{enumerate}
\end{itemize}

It is convenient to write $(x,y)=x(1,0)+y(0,1)=x\eb_1 + y\eb_2$. Then the set
$\{\eb_1,\eb_2\}$ form a \underline{basis} of $\mathbb{R}^2$, known as the
\underline{Cartesian basis}.

%-------------------------------------------------------------------------------

\subsection{Scalar product, lengths and angles}

Given $\vb_1,\vb_2\in\mathbb{R}^2$, the \underline{scalar product} is defines as
\begin{equation}
	\ub\dot\vb=(u_1 v_1 + u_2 v_2)\in\mathbb{R}.
\end{equation}
We note that $\vb\cdot\vb=v_1^2 + v_2^2\geq0$. The \underline{modulus} of the
vector $\vb$ is defined as
\begin{equation}
	|\vb|=\sqrt{v_1^2 + v_2^2}.
\end{equation}
The modulus coincides with the \underline{length} of the vector in this case,
under the Euclidean norm (defined by the dot product here).

%-------------------------------------------------------------------------------

\subsection{Polar co-ordinates}

If we let $r=|\vb|$, then by considering the angle $\vb$ makes with respect to
the base line (e.g., $x$-axis), we can write a vector $\vb$ in polar
co-ordinates $(r,\theta$), where $r\in\mathbb{R}^+$ and $\theta\in[0,2\pi)$, so
that
\begin{equation}
	v_1 = r\cos\theta,\qquad v_2 = r\sin\theta.
\end{equation}
(The zero vector is not well-defined since any choice of $\theta$ will do for
$r=0$).

\begin{lemma}
	Suppose $\ub,\vb\in\mathbb{R}^2$ and $\theta\in[0.\pi]$ is the angle between
	the two vectors. Then $\ub\dot\vb=|\ub||\vb|\cos\theta$.
\end{lemma}
\begin{proof}
	For $\ub=(|\ub|,\phi)$, $\vb=(|\vb|,\psi)$, $\theta=|\phi-\psi|$. So
	\begin{equation}
		\ub\dot\vb=|\ub||\vb|(\cos\phi\cos\psi+\sin\phi\sin\psi)
		=|\ub||\vb|\cos(\phi-\psi)=|\ub||\vb|\cos\theta.
	\end{equation}
\end{proof}

$\ub$ and $\vb$ are \underline{orthogonal} iff $\ub\cdot\vb=0$. (This is a more
general way for saying two vectors are perpendicular, as we do not necessarily
have to restrict ourselves to the Euclidean inner product).

%-------------------------------------------------------------------------------

\subsection{Simultaneous equations}

Suppose we have the system of equations
\begin{equation}
	ax+by=e,\qquad cx+dy=f,\qquad a\cdots f\in\mathbb{R}.
\end{equation}
Then multiplying and eliminating accordingly gives
\begin{equation}
	x=\frac{de-bf}{ad-bc},\qquad y=\frac{ce-af}{ad-bc},
\end{equation}
so unique solution exists if $(ad-bc)\neq0$. If $(ad-bc)=0$, then there can be
two scenarios:
\begin{enumerate}
	\item Solution is under-determined. e.g., $2x+6y=4$ and $3x+9y=6$, and the
	two equations differ by a constant factor, so all solutions lie on the same
	line.
	\item There is no solution (lines do not intersect in the plane). e.g.,
	$2x+6y=6$ and $3x+9y=4$.
\end{enumerate}

If we instead write the system of equations in terms of a matrix, for example,
\begin{equation}
	\begin{cases}2x+6y=6\\ 3x+9y=4\end{cases}\qquad\equiv\qquad
	\begin{pmatrix}2 & 6\\ 3 & 9\end{pmatrix}\begin{pmatrix}x\\ y\end{pmatrix}
	=\begin{pmatrix}6\\ 4\end{pmatrix},
\end{equation}
by using matrix multiplication rules and inverses of matrices, an alternative
method can be used to solve simultaneous equations, not necessarily of two
variables. In the general case in $\mathbb{R}^2$,
\begin{equation}
	\mathsf{A}\xb=\bb\rightarrow \xb=\mathsf{A}^{-1}\bb,
\end{equation}
where
\begin{equation}
	\mathsf{A}^{-1}=\frac{1}{ad-bc}\begin{pmatrix}d & -b\\ -c & a\end{pmatrix},
	\qquad \mathsf{A}=\begin{pmatrix}a & b\\ c & d\end{pmatrix}.
\end{equation}
$\mathsf{A}^{-1}$ is the \underline{inverse} of $\mathsf{A}$, and
$|\mathsf{A}|=(ad-bc)$ is known as the \underline{determinant} of a $2\times2$
matrix. The \underline{adjoint} $\mbox{adj}\mathsf{A}$ is the inverse without
division by the determinant. The geometric interpretation is then, if
$|\mathsf{A}|\neq0$, two lines intersect uniquely, otherwise there is no
intersection, or the lines lie on top of each other.

%-------------------------------------------------------------------------------

\subsection{Lines on a plane}

We know that $ax+by=c$ is the equation of a straight line. There are several
ways to describe lines:
\begin{itemize}
	\item \underline{Parametric form}. If $\vb$ passes through the origin, we
	can write the line as a collection of all scalar multiples of a direction
	vector along the line, i.e., $\xb=\lambda\vb$. More generally, if $\ab$ is
	any point on the line, then $\xb=\ab+\lambda\vb$.
	\item \underline{Normal vector}. Let $\nb$ be a vector orthogonal to $\vb$,
	then $\nb\cdot\vb=0$. Thus
	\begin{equation}
		\nb\cdot\xb=\nb\cdot\ab+\lambda\nb\cdot\vb = \nb\cdot\ab.
	\end{equation}
	A line may therefore be written in terms of $\ab$ and $\nb$ as
	$\nb\cdot\xb=n_1 x + n_2 y = \nb\cdot\ab$. It is often convenient to have
	$|\nb|=1$, and denote it $\hat{\nb}$.
\end{itemize}

%-------------------------------------------------------------------------------

\subsection{Determinants and area}

\begin{lemma}
	Let $\ub=(a,c)^{T}$ and $\vb=(b,d)^{T}$. Then $\ub$ and $\vb$ are parallel
	iff $|\mathsf{A}|=0$.
\end{lemma}
\begin{proof}
	If $\ub$ is parallel to $\vb$ then $\ub=\lambda\vb$. Then $a=\lambda b$ and
	$c=\lambda=d$. Then $ad-bc=\lambda(bd-bd)=0$. If $|\mathsf{A}|=0$, then
	$ad=bc$, and so $a/b=c/d=\lambda$, and so we arrive at $\ub=\lambda\vb$.
\end{proof}

Now consider the parallelogram formed from $\ub$ and $\vb$, with one vertex on
the origin wlog. The claim is that the area of this parallelogram is equal to
$|\mathsf{A}|$. The argument goes that the area is base multiplied by the
height. Taking $|\ub|$ to be the case, the height is given by $|\vb|\sin\theta$,
and so the area is $|\ub||\vb|\sin\theta$. On the other hand,
$\ub\cdot\vb=ab+cd=|\ub||\vb|\cos\theta$, so squaring both sides gives
\begin{equation}
	|\ub|^2 |\vb|^2 \cos^2\theta = (ab+cd)^2,
\end{equation}
and
\begin{equation}
	\mbox{Area}=|\ub|^2 |\vb|^2 \sin^2\theta = -(ab+cd)^2 + |\ub|^2 |\vb|^2
	=\cdots=(ad-bc)^2=|\mathsf{A}|^2,
\end{equation}
from which result follows.

\begin{example}
	Take a parallelogram with vertices
	\begin{equation}
		\ab=(1,3),\qquad\bb=(4,4),\qquad,\cb=(5,6),\qquad\db=(2,5).
	\end{equation}
	Then, noticing that $\bb-\ab$ is parallel to $\cb-\bb$, we take the spanning
	vectors as $\bb-\ab=(3,1)$ and $\cb-\db=(1,2)$, so
	\begin{equation}
		|\mathsf{A}|=\left|\begin{matrix}3 & 1\\ 1 & 2\end{matrix}\right|=6-1=5.
	\end{equation}
\end{example}

From the above, we deduce that the area of a triangle is half of the
determinant of the matrix with the spanning vectors of the parallelogram as the
entries.

\begin{example}
	For a triangle with vertices
	\begin{equation}
		\ab=(-1,2),\qquad\bb=(1,1),\qquad\cb=(3,4),
	\end{equation}
	we have $\bb-\ab=(2,-1)$ and $\cb-\db=(4,2)$, so
	\begin{equation}
		\mbox{Area}=\frac{1}{2}|\mathsf{A}|
		=\frac{1}{2}\left|\begin{matrix}2 & 4\\ -1 & 2\end{matrix}\right|
		=(4+4)/2=4.
	\end{equation}
\end{example}

%-------------------------------------------------------------------------------

\subsection{Curves in the plane}

The most familiar curves are graphs of functions. Let $f(x)$ be a real valued
function of $x\in\mathbb{R}$, $x\in(a,b)$. The graph $y=f(x)$ is a curve in
$\mathbb{R}^2$. The curve is differentiable if we can find a tangent to the
graph of $f$ for each $x\in(a,b)$. This is done by taking the limit of chords
with
\begin{equation}
	y=f(x+\delta x)-f(x),\qquad x=x+\delta x - x,
\end{equation}
so that the line has gradient $[f(x+\delta x)-f(x)]/\delta x$. By re-scaling and
taking the limit as $\delta x\rightarrow0$, this gives the direction vector of
the tangent line, or, as a vector,
\begin{equation}
	\tb=\lim_{\delta x\rightarrow0}\left(\begin{matrix}
	1\\ [f(x+\delta x)-f(x)]/\delta x\end{matrix}\right)=\left(\begin{matrix}
	1\\ f'(x)\end{matrix}\right).
\end{equation}
For fixed $x_0$, the tangent line is
\begin{equation}
	\begin{pmatrix}x\\ y\end{pmatrix}=\begin{pmatrix}x_0 \\ f(x_0)\end{pmatrix}
	+\tb\begin{pmatrix}1\\ f'(x_0)\end{pmatrix},
\end{equation}
with $\tb=x-x_0$, $y=f(x_0)+\tb f'(x_0)=f(x_0)+(x-x_0)f'(x_0)$. This is of
course the first Taylor expansion approximation of $y=f(x)$ at $x=x_0$.

\begin{example}
	Let $f(x)=x^3-6x^2 + 9x$.
	\begin{enumerate}
		\item Find tangent of $y=f(x)$ at $x=0,1,2,3,4$.
		\item Use this to sketch $y=f(x)$ for $x\in[-1,5]$.
		\item Find the area below the curve and above the $x$-axis between their
		two pairs of intersection.
	\end{enumerate}
	
	We have $f'(x)=3x^2-12x+9$. So
	
	\begin{tabular}{|c|c|c|c|c|c|}
		\hline
		$x$ & $0$ & $1$ & $2$ & $3$ & $4$\\
		\hline
		$y=f(x)$ & $0$ & $4$ & $2$ & $0$ & $4$\\
		\hline
		$f'(x)$ & $9$ & $0$ & $-3$ & $0$ & $9$\\
		\hline
		tangent & $y=9x$ & $y=4$ & $y=3x+8$ & $y=0$ & $y=9x-32$\\
		\hline
	\end{tabular}
	
	A sketch of the graph shows a single crossing at $x=0$ and double crossing
	at $x=3$ (can show this by factorising the cubic). So the area under the
	curve is
	\begin{equation}
		\int_0^3 (x^3-6x^2+9x)\ \mathrm{d}x=27/4.
	\end{equation}
\end{example}

%-------------------------------------------------------------------------------

\subsection{Parametric curve}

These can deal with places where the gradient becomes infinity, or where there
is more than one value of $y$ for a particular $x$. For example,
$(x,y)=(r\cos\theta,r\sin\theta)$ describes a circle centred at the origin,
since $x^2 + y^2 = r^2$. A \underline{parametric curve} in $\mathbb{R}^2$ is a
differentiable map
\begin{equation}
	\alpha(a,b)\rightarrow\mathbb{R}^2,\qquad \alpha(t)=(x(t),y(t)),\qquad
	t\in(a,b).
\end{equation}

The curve is differentiable if, for all $t\in(a,b)$, there exists
$\alpha'(t)=(x'(t),y'(t))$. Geometrically, $\alpha'(t)$ is the vector tangent to
the curve at $t$. This definition captures the essence of a vector and
generalises even to curved spaces. This may be seen via the definition of
chords, taking $\delta t\rightarrow0$ for each component.

\begin{example}
	For a circle, $\alpha(t)=(r\cos t,r\sin t)$, and $\alpha'(t)=(-r\sin t,r\cos
	t)$. More generally, the circle of radius $r$ centred at $\vb$ is
	parametrised as $\alpha(t)=(r\cos t + v_1, r\sin t + v_2)$.
	
	For an ellipse centred at the origin, the parametrisation is
	$\alpha(t)=(a\cos t, b\cos t)$, and $\alpha'(t)=(-a\sin t, b\cos t)$. In
	Cartesian co-ordinates, this yields
	\begin{equation}
		\frac{x^2}{a^2} + \frac{y^2}{b^2} = c^2,\qquad a,b>0.
	\end{equation}
	The general parametrise form is similar to the circle case.
\end{example}

A parametrised curve is \underline{regular} is $\alpha'(t)$ exists for
$t\in(a,b)$. This means that the parameter is constantly moving along the curve,
which allows us to measure distances along the curve. The \underline{arc length}
of a regular smooth parametrised curve, measured from a reference point $t_0$,
is given by
\begin{equation}
	s(t)=\int_{t_0}^t |\alpha'(t)|\ \mathrm{d}t = \int_{t_0}^{t}
	\sqrt{[x'(t)]^2+[y'(t)]^2}\ \mathrm{d}t.
\end{equation}
This is because, for an arc length element,
\begin{equation}
	\delta s=\sqrt{\left(\frac{\delta x}{\delta t}\right)^2 +
	\left(\frac{\delta y}{\delta t}\right)^2},\qquad\Rightarrow\qquad
	\sum\delta s=\int_a^b\sqrt{\left(\frac{\mathrm{d} x}{\mathrm{d} t}\right)^2 
	+\left(\frac{\mathrm{d} y}{\mathrm{d} t}\right)^2}\ \mathrm{d}t.
\end{equation}
The curve is said to be \underline{parametrised by arc length} if
$|\alpha'(t)|=1$. The is equivalent to saying $\delta s\approx\delta t$.

\begin{lemma}
	Let $\alpha(t)$ be a regular curve parametrised by arc length. Then
	$\alpha''(t)=(x''(t), y''(t))$ is normal to $\alpha(t)$.
\end{lemma}
\begin{proof}
	We need to show that $\alpha'(t)\cdot\alpha''(t)=0$. For $t$ the arc length,
	$|\alpha'(t)|^2=\alpha'(t)\cdot\alpha'(t)=1$. So
	$(\alpha'\cdot\alpha')'=2\alpha'\cdot\alpha''=0$, as required.
\end{proof}

\begin{example}
	For a circle with $\alpha(t)=(r\cos t, r\sin t)$, $|\alpha'(t)|=r$, so
	\begin{equation}
		s(t)=\int_{t_0}^t r\ \mathrm{d}t = r(t-t_0),\qquad
		\sum s = 2\pi r
	\end{equation}
	since $t\in[0,2\pi)$. To parametrise the curve by arc length, we take
	$t=s/r$, so $\alpha(s)=(r\cos (s/r), r\sin(s/r))$, $\alpha'(s)=(-\sin (s/r),
	\cos(s/r))$, and $\alpha''(s)=(-1/r)(\cos(s/r),
	\sin(s/r))=-(1/r^2)\alpha(s)$. It is easy to see that
	$\alpha'(s)\cdot\alpha''(s)=0$. Geometrically, $\alpha$ points away from the
	circle centre, $\alpha'(s)$ is tangent to the circle, and $\alpha''(s)$
	points towards the circle centre. From circular motion, $\alpha=\xb$, the
	position vector, $\alpha'=\vb$ the velocity vector, and $\alpha''=\ab$ is
	the acceleration vector.
\end{example}

\begin{example}
	The \underline{cardinod} is given by the parametrisation
	\begin{equation}
		\alpha(t)=2a(1-\cos t)(\cos t, \sin t),
	\end{equation}
	and looks like a heart shape lying on its side. At $t=0$, there is a cusp,
	which will need separate consideration since it is an end point for the
	cardinod, and the curve is not necessarily smooth there. We have
	\begin{equation}
		\alpha'(t)=2a(-\sin t+2\cos t\sin t, \cos t-\cos^2 t + \sin^2),
	\end{equation}
	and, after some algebra (do it yourself), $|\alpha'(t)|=4a\sin (t/2)$ after
	using some double angle formulas. The arc length is then
	\begin{equation}
		s(t)=\int_0^t 4a\sin(t/2)\ \mathrm{d}t=8a(1-\cos(t/2)).
	\end{equation}
	The total length of a cardinoid is then	$s(2\pi)=16a$.
\end{example}

%-------------------------------------------------------------------------------

\subsection{Central conics}

Consider a curve $C\in\mathbb{R}^2$, given by the constraint
\begin{equation}
	ax^2=2bxy+cy^2 = \begin{pmatrix}x & y\end{pmatrix}
	\begin{pmatrix}a & b\\ b & c\end{pmatrix}
	\begin{pmatrix}x\\ y\end{pmatrix}=1.
\end{equation}
Writing this in polar co-ordinates gives
\begin{equation}
	r^2 f(\theta)=1,\qquad f(\theta)=a\cos^2\theta + 2b\cos\theta\sin\theta
	+ c\sin^2\theta.
\end{equation}
Since the origin is not within the curve, $r>0$, so $r=1/\sqrt{f(\theta)}$.
Curve then only exists if $f(\theta)>0$. Now,
\begin{equation}
	\frac{f(\theta)}{\sin^2 \theta}=a\cot^2 \theta + 2b\cot\theta + c.
\end{equation}
If $f(\theta)=0$, $\cot\theta=(-b/a)\pm(\sqrt{b^2-ac}/a)$, and $\theta$ is real
only if $b^2-ac\geq0$. Now, $b^2-ac=-|\mathsf{A}|$, so if $|\mathsf{A}|>0$,
$f(\theta)$ does not exist. If $ac-b^2>0$, then if:
\begin{itemize}
	\item $a,c>0\qquad\Rightarrow\qquad f(\theta)>0$.
	\item $a,c<0\qquad\Rightarrow\qquad f(\theta)<0$.
\end{itemize}
If instead $|\mathsf{A}|=0$, $ac=b^2$, so multiplying both sides by $c$, it
maybe shown that $c=(bx+cy)^2$, thus $bx+cy=\pm c$, i.e., two parallel straight
lines with gradient $-b/c=-a/b$.

In summary:
\begin{enumerate}
	\item If $|\mathsf{A}|>0$ and $a,c>0$, we have a circle, and there are
	solutions for all $\theta$.
	\item If $|\mathsf{A}|>0$ and $a,c<0$, there are no solutions since
	$f(\theta)<0$ and $r=1/\sqrt{f(\theta)}$.
	\item If $|\mathsf{A}|=0$, the curve are two parallel straight lines.
	\item If $|\mathsf{A}|<0$, we have hyperbola. There is a solution of $F$ for
	$\theta$ in two open intervals, each of length less than $\pi$.
\end{enumerate}

\begin{example}
	For $a=c=1$, $b=0$, $|\mathsf{A}|=1$, and this describes the unit circle
	centred at the origin. There is a solution for all $\theta$.
	
	for $a=c=1$, $b=-1/2$, $|\mathsf{A}|=3/4$, we have the ellipse $x^2 + y^2
	-xy = 1$, and may be factorised into
	\begin{equation}
		\frac{3}{4}(x-y)^2 + \frac{1}{4}(x+y)^2 = 1.
	\end{equation}
	There is a solution for all $\theta$.
	
	For $a=b=c=1$, $|\mathsf{A}|=0$, and $x+y=\pm1$, two parallel straight
	lines.
	
	For $a=1$, $c=-1$, $b=0$, $|\mathsf{A}|=-1$, we have $x^2 - y^2 = 1$, and
	describes hyperbola for $\theta\in[0,\pi/4)\cup(7\pi/4,2\pi)$ and
	$\theta\in(3\pi/4,5\pi/4)$.
\end{example}

It is often useful to find the points of $C$ closest/furthest to/from the
origin. These are given by the turning points of $C$, i.e.,
$f'(\theta)=(c-a)\sin2\theta+2b\cos2\theta=0$. An observation we first make is
that, for $f'(\theta_0)=0$, then $f'(\theta_0+\pi/2)=0$ also (via
straightforward substitution). Then $f'(\theta_0)$ and $f'(\theta_0 + \pi/2)$
are the extremum values of $f(\theta)$. Thus turning points occur at orthogonal
lines. The lines $\theta=\theta_0$ and $\theta_0+\pi/2$ are the
\underline{principal axes} of the conic $C$.
\begin{example}
	For which values of $\theta$ do there exist
	\begin{equation}
		1=\begin{pmatrix}x & y\end{pmatrix}
		\begin{pmatrix}-1 & 2 \\ 2 & -1\end{pmatrix}
		\begin{pmatrix}x\\y\end{pmatrix},
	\end{equation}
	and what is the smallest value of $r=\sqrt{x^2+y^2}(=1/\sqrt{f(\theta)})$
	for which solutions exist?
	
	$|\mathsf{A}|=-3$, so solution forms a hyperbola.
	$f(\theta)=-\cos^2\theta+4\cos\theta\sin\theta-\sin^2\theta
	=4\cos\theta\sin\theta-1$, and solution exists for $f(\theta)>0$. Thus we
	require $\sin\theta>1/2$, and so $\theta\in(\pi/12,5\pi/12)$ and
	$\theta\in(13\pi/12,17\pi/12)$.
	
	Now, $f'(\theta)=4(\cos^2\theta-\sin^2\theta)$, so $f'=0$ if
	$\cos\theta=|\sin\theta|$. This occurs at $\cos\theta=1/\sqrt{2}$, and so
	$\theta=\pi/4,3\pi/4,5\pi/4,17\pi/4$. Taking into account the domain of
	relevance, $\theta=\pi/4$ and $\theta=5\pi/4$ are the relevant turning
	points. It is easy to see the extremum in this case is at $\pi/4$, with
	$r_{\min}=1$; since this is a hyperbola, there is no maximum.
\end{example}

Sometimes $\theta_0$ is an extremum point for $f$, then going back to the
relation for $f(\theta_0)$, we notice that
\begin{equation}\begin{aligned}
	f(\theta_0)\cos\theta_0 &= a\cos^3\theta_0 + 2b\cos^2\theta_0 \sin\theta_0
	+c\sin^2\theta_0 \cos\theta_0\\
	&= a\cos^3\theta_0 + 2b\cos^2\theta_0
	+\sin\theta_0[a\cos\theta_0 \sin\theta_0 +b(\sin^2\theta_0 -\cos^2\theta_0)]
	\\ &=a\cos\theta_0 + b\sin\theta_0
\end{aligned}\end{equation}
up on substituting the value of $c\sin2\theta$ when $f'(\theta_0)=0$. Similarly,
we have
\begin{equation}
	f(\theta_0)\sin\theta_0 = b\cos\theta_0 + c\sin\theta_0,
\end{equation}
so
\begin{equation}
	f(\theta_0)\begin{pmatrix}\cos\theta_0 \\ \sin\theta_0\end{pmatrix}
	=\begin{pmatrix}a & b\\ b & c\end{pmatrix}
	\begin{pmatrix}\cos\theta_0 \\ \sin\theta_0\end{pmatrix}.
\end{equation}
For $\hat{\mathbf{r}}=(\cos\theta_0, \sin\theta_0)^T$ (this is a unit vector),
we have $\mathsf{A}\hat{\mathbf{r}}=f(\theta_0)\hat{\mathbf{r}}$, and so
$\hat{\mathbf{r}}$ is the \underline{eigenvector} with associated
\underline{eigenvalue} $f(\theta_0)$.

As we will see, matrices are associated with linear mappings, and the
eigen-equation says the action of a map on a position vector is to leave it
unchanged up to a scale factor $\lambda$.
\begin{example}
	The matrix
	\begin{equation}
		\mathsf{A}=\begin{pmatrix}-1 & 0\\ 0 & 1\end{pmatrix}
	\end{equation}
	represents a reflection in the $y$-axis. In this case, the eigenvalues and
	eigenvectors are
	\begin{equation}
		\eb_1=\begin{pmatrix}1\\0\end{pmatrix},\qquad \lambda_1=-1,\qquad
		\eb_2=\begin{pmatrix}0\\1\end{pmatrix},\qquad \lambda_1=1.
	\end{equation}
\end{example}

Going back to conics, $\theta_0$ and $\theta_0+\pi/2$ gives the extremum values
of $r$, which also corresponds to the direction of the eigenvectors. The
eigenvalues gives the distance to the origin via $r=1/\sqrt{f(\theta)}$, with
\begin{equation}
	f(\theta_0)\begin{pmatrix}\cos\theta_0 \\ \sin\theta_0\end{pmatrix}
	=\mathsf{A}\begin{pmatrix}\cos\theta_0 \\ \sin\theta_0\end{pmatrix},\quad
	f(\theta_0+\pi/2)
	\begin{pmatrix}\cos\theta_0+\pi/2 \\ \sin\theta_0+\pi/2\end{pmatrix}
	=\mathsf{A}
	\begin{pmatrix}\cos\theta_0+\pi/2 \\ \sin\theta_0+\pi/2\end{pmatrix}.
\end{equation}
For general $\theta$, we observe that
\begin{equation}
	\begin{pmatrix}\cos\theta \\ \sin\theta\end{pmatrix}=
	\begin{pmatrix}\cos(\theta-\theta_0+\theta_0)
	 \\ \sin(\theta-\theta_0+\theta_0)\end{pmatrix}=
	 \begin{pmatrix}\cos(\theta-\theta_0)\cos\theta_0
	 +\sin\theta_0 \sin(\theta-\theta_0)
	 \\ \sin(\theta-\theta_0)\cos\theta_0 + \sin\theta_0 \cos(\theta-\theta_0)
	 \end{pmatrix},
\end{equation}
so then
\begin{equation}\begin{aligned}
	\mathsf{A}\begin{pmatrix}\cos\theta \\ \sin\theta\end{pmatrix}
	&=\cos(\theta-\theta_0)\mathsf{A}
	\begin{pmatrix}\cos\theta_0 \\ \sin\theta_0\end{pmatrix}
	+\sin(\theta-\theta_0)\mathsf{A}
	\begin{pmatrix}-\sin\theta_0 \\ \cos\theta\end{pmatrix}\\
	&=\cos(\theta-\theta_0)f(\theta_0)
	\begin{pmatrix}\cos\theta_0 \\ \sin\theta_0\end{pmatrix}
	+\sin(\theta-\theta_0)f(\theta_0+\pi/2)
	\begin{pmatrix}-\sin\theta_0 \\ \cos\theta\end{pmatrix}.
\end{aligned}\end{equation}
Now, we recall that we have
\begin{equation}
	1=r^2(\cos\theta,\sin\theta)\mathsf{A}
	\begin{pmatrix}\cos\theta \\ \sin\theta\end{pmatrix}.
\end{equation}
Using the above identity, the RHS is
\begin{equation}\begin{aligned}
	\mbox{RHS}=&r^2 f(\theta_0)\cos(\theta-\theta_0)
	[\cos\theta\cos\theta_0 + \sin\theta\sin\theta_0]\\
	&\qquad+ r^2 f(\theta_0+\pi/2)\sin(\theta-\theta_0)
	[-\cos\theta\sin\theta_0 + \sin\theta\cos\theta_0].
\end{aligned}\end{equation}
Using double angle formulae, we arrive at
\begin{equation}
	1=r^2 f(\theta_0)\cos^2(\theta-\theta_0) 
	+r^2 f(\theta_0+\pi/2)\sin^2(\theta-\theta_0).
\end{equation}
So we see that if we have (i) an ellipse if both eigenvalues are positive, (ii)
a hyperbola if only one of the eigenvalues is positive, (iii) no solution if
both eigenvalues are negative.

\begin{example}
	Characterise the conic $1=5x^2 + 2\sqrt{3} xy + 3y^2$, an find its largest
	and smallest distance from the origin.
	
	Now,
	\begin{equation}
		\mathsf{A}=\begin{pmatrix}5 & \sqrt{3}\\ \sqrt{3} & 3\end{pmatrix},
	\end{equation}
	with $|\mathsf{A}|=12$, $a,c>0$, therefore solutions exists for all
	$\theta$, so we have an ellipse (alternative we could find the eigenvalues).
	We also have
	\begin{equation}
	 	f(\theta)=5\cos^2\theta+2\sqrt{3}\cos\theta\sin\theta+3\sin^2\theta,
	 	\qquad
	 	f'(\theta)=2\sqrt{3}\cos2\theta-2\sin2\theta,
	 \end{equation}
	 so extremum values occur where $\tan2\theta=\sqrt{3}$, and so the principal
	 axes occurs at $\theta=(\pi/6,2\pi/3,7\pi/6,5\pi/3)$. It is seen that
	 $r_{\min}=1/\sqrt{f(\pi/6)}=1/\sqrt{6}$ and
	 $r_{\max}=1/\sqrt{f(2\pi/3)}=\sqrt{2}$ (and similarly with the values
	 another $\pi$ radians along).
\end{example}

%===============================================================================

\section{Geometry of $\mathbb{R}^3$}

%-------------------------------------------------------------------------------

\subsection{Vectors and their products}

We assume vectors $\vb\in\mathbb{R}^3$ satisfy the same axioms of addition and
multiplication as for $\mathbb{R}^2$. Taking the usual basis, we take the dot
product to be
\begin{equation}
	\ub\cdot\vb=u_1 v_1 + u_2 v_2 + u_3 v_3,
\end{equation}
so that
\begin{equation}
	|\vb|=\sqrt{\vb\cdot\vb}=\sqrt{v_1^2 + v_2^2 + v_3^2},\qquad
	\ub\cdot\vb=|\ub||\vb|\cos\theta.
\end{equation}
Again, $\ub$ and $\vb$ are orthogonal if $\ub\cdot\vb=0$.

In $\mathbb{R}^3$, we define the \underline{vector (cross) product} as
\begin{equation}
	\ub\times\vb=\begin{pmatrix}u_2 v_3 -u_3 v_2\\ u_3 v_1 - u_1 v_3\\
	u_1 v_2 - u_2 v_1\end{pmatrix}.
\end{equation}
It may be seen that $\eb_1\times\eb_2=\eb_3$, and the sign is kept if the
indices are permuted in a cyclic fashion. We also notice that
$\vb\cdot\ub=-\ub\cdot\vb$. In general, $\ub\times\vb$ generates a vector
orthogonal to both of the vectors, such that $\ub\times\vb$ points according to
the clockwise/right-hand screw convention.

\begin{lemma}
	For non-trivial $\ub,\vb\in\mathbb{R}^3$ and $\theta\in[0,\pi]$ be the angle
	between them, then $|\ub\times\vb|=|\ub||\vb|\sin\theta$.
\end{lemma}
\begin{proof}
	It may be shown that $|\ub\times\vb|^2=
	\cdots=|\ub|^2|\vb|^2-(\ub\cdot\vb)^2$. Then, using the identity for
	$\ub\cdot\vb$, we obtain the identity.
\end{proof}

%-------------------------------------------------------------------------------

\subsection{Simultaneous equations in $\mathbb{R}^3$}

Suppose we want to solve the simultaneous system of equations
\begin{equation}
	\begin{cases}ax+by+cz&=l,\\ dx+ey+fz&=m,\\ gx+hy+jz&=n.\end{cases}
\end{equation}
A similar approach using matrices results in
\begin{equation}
	\mathsf{A}=\begin{pmatrix}a & b & c\\ d & e & f &\\ g & h & j \end{pmatrix},
	\qquad \mathsf{A}\xb=\bb.
\end{equation}
Supposing the inverse $\mathsf{A}^{-1}$ exists, then we may solve the system
uniquely; the existence of an unique solution again depends on the determinant
of the matrix. By brute force or otherwise,
\begin{equation}
	|\mathsf{A}|=\left|\begin{matrix}a & b & c\\ d & e & f \\ g & h & j
	\end{matrix}\right|=
	a\left|\begin{matrix}e & f \\ h & j\end{matrix}\right|
	-b\left|\begin{matrix} d & f \\ g & j \end{matrix}\right|
	+c\left|\begin{matrix} d & e \\ g & h \end{matrix}\right|.
\end{equation}
(This is the expression of the determinant by expanding the first row; it may be
done by expanding any column or row, although one needs to take into account of
extra minus signs in entries where $i+j=2n$, with $\mathsf{A}=(A_{ij})$,
$A_{ij}$ the entry at the $i$-th row and $j$-th column.) With the determinant,
$\mathsf{A}^{-1}$ may be found by the following steps:
\begin{enumerate}
	\item \underline{Matrix of minors}. We find the determinant of each
	$2\times2$ matrix, where the element corresponding to that row and column is
	covered, i.e.,
	\begin{equation}
		\begin{pmatrix}ej-fh & dj-fg & dh-eg\\ bj-ch & aj-cg & ah-bg\\
		bf-ce & af-cd & ae-bd
		\end{pmatrix}.
	\end{equation}
	\item \underline{Matrix of co-factors}. Change the sign of the places where
	$i+j=2n$, where $i$ and $j$ are the row and column number (starting the
	count from one), i.e.,
	\begin{equation}
		\begin{pmatrix}ej-fh & -(dj-fg) & dh-eg\\ -(bj-ch) & aj-cg & -(ah-bg)\\
		bf-ce & -(af-cd) & ae-bd
		\end{pmatrix}=
		\begin{pmatrix}ej-fh & fg-dj & dh-eg\\ ch-bj & aj-cg & bg-ah\\
		bf-ce & cd-af & ae-bd
		\end{pmatrix}.
	\end{equation}
	\item \underline{Adjoint}. Take a transpose of the matrix of co-factors,
	i.e.,
	\begin{equation}
		\mbox{adj}(\mathsf{A})=\begin{pmatrix}ej-fh & ch-bj & bf-ce \\ 
		fg-dj & aj-cg & cd-af\\ dh-eg & bg-ah & ae-bd
		\end{pmatrix}.
	\end{equation}
	(Think of the transpose as reflecting everything about the main diagonal.)
	\item \underline{Inverse}. Divide the adjoint by the determinant, i.e.,
	$\mathsf{A}^{-1}=\mbox{adj}(\mathsf{A})/|\mathsf{A}|$.
\end{enumerate}

%-------------------------------------------------------------------------------

\subsection{Planes in $\mathbb{R}^3$}

Consider the constraint $ax+by+cz=l$, which describes a plane in $\mathbb{R}^3$.
This is the case because, assuming $c\neq0$ wlog, we have
\begin{equation}
	z=\frac{l-ax-by}{c},\qquad\Rightarrow\qquad
	\frac{\dfd z}{\dfd x}=-\frac{a}{c},\qquad\frac{\dfd z}{\dfd y}=-\frac{b}{c},
\end{equation}
and there exists a unique $z$ for every $x$ and $y$. The derivatives are
constant which implies that $z(x,y)$ traces out a two-dimensional plane in
$\mathbb{R}^3$.

An alternative description of the plane can be given in terms of vectors, again
as $\nb\cdot\boldsymbol{r}=\ab\cdot\nb$, where $\nb$ is a normal vector to the
plane, $\boldsymbol{r}=(x,y,z)$, and $\ab$ is a position vector in the plane. As
a trivial example, the $(x,y)$ plane is described by the equation with $a,b,l=0$
and $c=1$. We have $\nb=\eb_z$, $\ab=\boldsymbol{0}$.
\begin{example}
	For $3x+4y-2z=12$, we have $\nb=(3,4,-2)$, and one possibility for $\ab$ is
	$(4,0,0)$.
\end{example}

The description is as in $\mathbb{R}^2$, except now we require two direction
vectors. The equation of the plane is given as $\boldsymbol{r}=\ab+\lambda\db_1
+ \mu\db_2$. Eliminating $\lambda$ and $\mu$, it may be shown that
$\boldsymbol{r}\cdot(\db_1\times\db_2) =\ab\cdot(\db_1\times\db_2)$, i.e.,
$\db_1\times\db_2$ serves as a normal vector to describe the plane.
\begin{example}
	Find the equation of the plane that goes includes the points
	\begin{equation}
		\ab=\begin{pmatrix}1\\1\\1\end{pmatrix},\qquad
		\bb=\begin{pmatrix}1\\-1\\2\end{pmatrix},\qquad
		\cb=\begin{pmatrix}3\\1\\-1\end{pmatrix}.
	\end{equation}
	
	We have
	\begin{equation}
		\db_1=\bb-\ab=\begin{pmatrix}0\\-2\\1\end{pmatrix},\qquad
		\db_2=\cb-\ab=\begin{pmatrix}1\\0\\-2\end{pmatrix},\qquad\Rightarrow
		\qquad \nb=\db_1\times\db_2=\begin{pmatrix}4\\1\\2\end{pmatrix},
	\end{equation}
	with $\nb\cdot\ab=7$. So one possibility is $4x+y+2z=7$.
\end{example}

%-------------------------------------------------------------------------------

\subsection{Lines in $\mathbb{R}^3$}

A line is given by a point and a direction vector as
$\boldsymbol{r}=\ab+\lambda\db$. It is then given by two constraints:
\begin{equation}
	\frac{x-a_1}{d_1}=\frac{y-a_2}{d_2}=\frac{z-a_3}{d_3}=\lambda.
\end{equation}
This line also has two normal vectors. Since only one of these can ever be
eliminated, there are two independent solutions. Both
\begin{equation}
	\nb=(d_2, -d_1, 0),\qquad\textnormal{and}\qquad
	\nb\times\db=(-d_1 d_3, -d_2 d_3, d_1^2 + d_2^2)
\end{equation}
may serve as normal vectors.

Two planes in $\mathbb{R}^3$ are either parallel or intersect in a line. Suppose
the two normal vectors are $\nb_1$ and $\nb_2$, then if they are parallel,
$\nb_1=\lambda \nb_2$, otherwise they intersect in a line
$\boldsymbol{r}=\ab+\lambda\db$, with $\db=\nb_1\times\nb_2$. To find a point on
the line, we may take one of the variables to be zero, and solve for the
remaining linear system.
\begin{example}
	Find the intersection of the planes $x-2y+z=3$ and $y+2z=1$.
	
	For $\nb_1=(1,-2,1)$ and $\nb_2=(0,1,2)$, $\db=(-5,-2,1)$. Taking $z=0$,
	$y=1$ from the second equation, so $x=5$, thus
	$\boldsymbol{r}=(5,1,0)=\lambda(-5,-2,1)$, or, in equation form,
	\begin{equation}
		\frac{x-2}{-5}=\frac{y-1}{-2}=z.
	\end{equation}
\end{example}
\begin{example}
	Find the parametric form of the line $(x-2)/3=(y-4)/-1=(z+5)/-2$.
	
	$\boldsymbol{r}=(2,4,-5)+\lambda(3,-1,-2)$.
\end{example}

If $\boldsymbol{r}=(a,b,c)+\lambda(d,e,0)$, then the constraints are
\begin{equation}
	\frac{x-a}{d}=\frac{y-b}{e}\qquad\textnormal{and}\qquad z=c.
\end{equation}

Two lines in $\mathbb{R}^3$ do not necessarily intersect even if they are
non-parallel. We now try to find the shortest distance between two lines that do
not intersect (if they do, this is of course zero). Suppose
$\boldsymbol{r}_1=\ab_1+\lambda\db_1$ and $\boldsymbol{r}_2=\ab_2+\lambda\db_2$.
Then we may have
\begin{itemize}
	\item Two lines are \underline{parallel}, with $\db_1=k\db_2$. Then the
	distance is $D=|\ab_1-\ab_2|\sin\theta$, since $\ab_1$ and $\ab_2$ are just
	position vectors on the line. Using the cross product corollary, we have
	\begin{equation}
		D=\frac{|\ab_1-\ab_2|\times\db_1|}{|\db_1|}.
	\end{equation}
	\item If the two lines are \underline{skewed}, then
	$\nb=\db_1\times\db_2\neq\boldsymbol{0}$ is a normal to the lines. The
	minimum distance between two points on the line corresponds to the
	difference vector between the pairs parallel to $\nb$, i.e.,
	\begin{equation}
		D\frac{\nb}{|\nb|}=\pm(\ab_1+\lambda\db_1-\ab_2-\mu\db_2).
	\end{equation}
	Taking a dot product of the above with $\nb/|\nb|$ and noting that
	$\nb\cdot\db_1=\nb\cdot\db_2=0$, we have
	\begin{equation}
		D=|\ab_1-\ab_2|\cdot\frac{\nb}{|\nb|}=
		\frac{|(\ab_1-\ab_2)\cdot(\db_1\times\db_2)|}{|\db_1 \times \db_2|}.
	\end{equation}
\end{itemize}

\begin{example}
	Find the distance between the lines described by
	\begin{equation}
		\frac{x-2}{-3}=\frac{y-2}{6}=\frac{z+1}{9}\qquad\textnormal{and}\qquad
		\frac{x+1}{2}=\frac{y}{-4}=\frac{z-2}{-6}.
	\end{equation}
	
	$\nb=(-3,6,9)\times(2,-4,-6)=\boldsymbol{0}$ so two lines are parallel, and
	using the appropriate formula, we have $D=\sqrt{122/7}$.
\end{example}
\begin{example}
	Find the distance between the two lines described by
	\begin{equation}
		\boldsymbol{r}_1=\begin{pmatrix}2\\2\\-1\end{pmatrix}
		+\lambda\begin{pmatrix}-3\\6\\9\end{pmatrix}\qquad\textnormal{and}\qquad
		\boldsymbol{r}_2=\begin{pmatrix}-1\\0\\2\end{pmatrix}
		+\mu\begin{pmatrix}2\\4\\6\end{pmatrix}.
	\end{equation}
	
	$\db_1\times\db_2=(-72,0,-24)$, and $\ab_1-\ab_2=(3,2,-3)$, so
	$D=144/(24\sqrt{10})=3\sqrt{10}/5$.
\end{example}

%-------------------------------------------------------------------------------

\subsection{Vector products and volumes of polyhedra}

Given three vectors $\ab,\bb,\cb\in\mathbb{R}^3$, we define the
\underline{scalar triple product} as $[\ab,\bb,\cb]=\ab\cdot(\bb\times\cb)$.
\begin{lemma}
	The scalar triple product is invariant under cyclic permutations, i.e.,
	$[\ab,\bb,\cb]=[\cb,\ab,\cb]=[\bb,\cb,\ab]$. (May beshown via brute force
	calculation.)
\end{lemma}
\begin{lemma}
	The scalar triple product is the determinant of the matrix with
	$\ab,\bb,\cb$ as its columns. (Thus $|\mathsf{A}|$ is unchanged if we
	cyclically permute its columns.)
\end{lemma}

\begin{example}
	The three vectors $(\ab,\bb,\cb)=(\eb_1,\eb_2,\eb_3)$ describes three edges
	of the unit cube. The scalar triple product of these three vectors is of
	course $1$, since $(\ab|\bb|\cb)=\mathsf{I}$, which coincidentally is the
	volume of the unit cube.
	
	For a sheared cubed described by
	$(\ab,\bb,\cb)=(\eb_1,\eb_2,\eb_1+\eb_2+\eb3)$, which has the same volume as
	the cube, the scalar triple product is also $1$.
\end{example}

The latter object is a \underline{parallelpiped}, which is a three-dimensional
polyhedron with six quadrilateral faces where opposite faces are parallel. Its
relation to the cube is like that of a parallelogram to a square. The volume of
the parallelpiped is give by the base area multiplied by the height; these are
respectively given by $|\ab||\bb|\sin\theta=|\ab\times\bb|$ and
$|\cb|\cos\phi=\cb\cdot\hat{\nb}=|\cb\cdot(\ab\times\bb)/|\ab\times\bb||$. Thus
\begin{equation}
	V=(|\ab||\bb|\sin\theta)(|\cb|\cos\phi)=|\cb\cdot(\ab\times\bb)|,
\end{equation}
and the volume of a parallelpiped spanned by three vectors is given by the
determinant of a matrix formed by the three vectors.

\begin{example}
	Find the volume of the parallelpiped with vertices
	\begin{equation}
		\begin{pmatrix}-1\\-1\\-1\end{pmatrix},\
		\begin{pmatrix}0\\1\\-1\end{pmatrix},\
		\begin{pmatrix}2\\1\\0\end{pmatrix},\
		\begin{pmatrix}1\\-1\\0\end{pmatrix},\
		\begin{pmatrix}-1\\0\\1\end{pmatrix},\
		\begin{pmatrix}0\\2\\1\end{pmatrix},\
		\begin{pmatrix}2\\2\\2\end{pmatrix},\
		\begin{pmatrix}1\\0\\2\end{pmatrix}.
	\end{equation}
	
	The first vector has the lowest value of $x,y,z$ so we take it as the origin
	of the spanning vectors. We see that three of the vectors are invariant in
	$x,y,z$ in respect to the first vector, given by the fifth, fourth and
	second vector respectively. Then the spanning vectors are $(0,1,2)$,
	$(2.0.1)$ and $(1,2,0)$ which gives a volume of $9$.
\end{example}

For a \underline{tetrahedron} (triangular bottom pyramid), the volume is a third
of the base area multiplied by height; for triangles, this is
\begin{equation}
	V=\frac{1}{3}\frac{1}{2}|\ab||\bb|\sin\theta|\cb\cos\phi|=
	\frac{1}{6}|\cb\cdot(\ab\times\bb)|.
\end{equation}
\begin{example}
	Find the volume of a tetrahedron with vertex
	\begin{equation}
		\ab=\begin{pmatrix}1\\0\\-1\end{pmatrix},\
		\bb=\begin{pmatrix}2\\0\\1\end{pmatrix},\
		\cb=\begin{pmatrix}3\\1\\2\end{pmatrix},\
		\db=\begin{pmatrix}1\\-1\\1\end{pmatrix}.
	\end{equation}
	
	Taking $\ab$ to be the origin, the vectors are $\bb-\ab=(1,0,2)$,
	$\cb-\ab=(2,1,3)$, $\db-\ab=(0,1,2)$, and the resulting determinant of the
	matrix is $1$, so the volume is $1/6$.
\end{example}

As a final topic, the \underline{triple vector product} is
$\ab\times(\bb\times\cb) = (\ab\cdot\cb)\bb-(\ab\cdot\bb)\cb$. This may be
checked by brute force, and analogous results exists for any number of vectors.

%-------------------------------------------------------------------------------

\subsection{Intersection of planes and simultaneous equations}

Two non-parallel planes intersect in a line. Suppose two planes $\pi_1$ and
$\pi_2$ are described by $ax+by+cz=l$ and $dx+ey+fz=m$, then the intersection
$\mathcal{C}=\ab+\lambda(\nb_1\times\nb_2)$, where $\nb_1$ and $\nb_2$ are the
respective normal vectors to the planes. Suppose we now have a third plane
$\pi_3$ described by $gx+hy+jz=n$, then this plane $\pi_3$ will intersect
$\mathcal{C}$ at a point if it is no parallel to $\pi_1$ or $\pi_2$. In equation
form, there is intersection if
\begin{equation}
	\boldsymbol{r}\cdot\nb_3=\ab\cdot\nb_3+\lambda[\nb_1,\nb_2,\nb_3]=n,
\end{equation}
so that
\begin{equation}
	\lambda=\frac{n-\ab\cdot\nb_3}{[\nb_1,\nb_2,\nb_3]},
\end{equation}
and a solution is well-defined if $[\nb_1,\nb_2,\nb_3]\neq0$, i.e., the
determinant of the matrix formed by the normal vectors is non-zero.

\begin{lemma}[Cramer's rule]
	It may be shown that, from algebraic manipulation, if
	$\mathsf{A}\boldsymbol{r}=\bb$, then
	\begin{equation}
		x=|\mathsf{A}_{1}|/|\mathsf{A}|,\qquad
		y=|\mathsf{A}_{2}|/|\mathsf{A}|,\qquad
		z=|\mathsf{A}_{3}|/|\mathsf{A}|,
	\end{equation}
	where $\mathsf{A}_{j}$ has the $j^{\textnormal{th}}$ column replaced by
	$\bb$.
\end{lemma}

Further rules for evaluating the determinant of matrices:
\begin{itemize}
	\item $|\mathsf{A}|=|\mathsf{A}^{T}|$.
	\item If $\mathsf{B}$ is formed by multiply a row or column of $\mathsf{A}$
	by $\lambda$, then $|\mathsf{B}|=\lambda|\mathsf{A}|$.
	\item If $\mathsf{B}$ is formed by interchanging two rows/columns of
	$\mathsf{A}$, then $|\mathsf{B}|=-|\mathsf{A}|$. A cyclic permutation
	involves two such operation, so the determinant is unchanged in this case.
	\item If $\mathsf{B}$ is formed by adding an arbitrary multiple of a single
	row/column of $\mathsf{A}$ to another row/column, then
	$\mathsf{B}=\mathsf{A}$. e.g.,
	\begin{equation}
		\left|\begin{matrix}a & b & c\\ d & e & f\\ g & h & j\end{matrix}\right|
		=\left|\begin{matrix}a-\lambda d & b-\lambda e & c-\lambda f
		\\ d & e & f\\ g & h & j\end{matrix}\right|.
	\end{equation}
\end{itemize}

%-------------------------------------------------------------------------------

\subsection{Method of row reduction}

Using the last rule above, a matrix may be manipulated into an
\underline{upper-triangular/echelon} form, i.e.,
\begin{equation}
	\mathsf{A}'=\begin{pmatrix}a' & b' & c'\\ 0 & e' & f'\\
	0 & 0 & j'\end{pmatrix},\qquad |\mathsf{A}'|=a'e'j'.
\end{equation}

\begin{example}
	Solve the simultaneous equations
	\begin{equation}\begin{aligned}
		2x + 3y + z &= 23,\\
		x + 7y + z &= 36,\\
		5x + 4y - 3z &= 16.
	\end{aligned}\end{equation}
	
	\begin{equation}\begin{aligned}
		\left(\begin{matrix}2&3&1\\ 1&7&1\\ 5&4&-3\end{matrix}\right|
		\left.\begin{matrix}23\\ 36\\ 10\end{matrix}\right)=
		\left(\begin{matrix}2&3&1\\ 1&7&1\\ 0&-7/2&-11/2\end{matrix}\right|
		\left.\begin{matrix}23\\ 36\\ -83/2\end{matrix}\right)&=
		\left(\begin{matrix}2&3&1\\ 0&11/2&1/2\\ 0&-7/2&-11/\end{matrix}\right|
		\left.\begin{matrix}23\\ 49/2\\ -83/2\end{matrix}\right)\\
		&=\left(\begin{matrix}2&3&1\\ 0&11/2&1/2\\ 0&0&-57/2/\end{matrix}\right|
		\left.\begin{matrix}23\\ 49/2\\ -285/11\end{matrix}\right).
	\end{aligned}\end{equation}
	From this, we can back substitute and see that $z=5$, $y=4$ and $x=3$.
\end{example}

%===============================================================================

\section{Curves and surfaces in $\mathbb{R}^3$}
%-------------------------------------------------------------------------------

\subsection{Parametric curves and surfaces}

We are used to functions that go from $\mathbb{R}$ to $\mathbb{R}$. A parametric
curve is simple the generalisation of this to the case where a function takes
$\mathbb{R}$ to $\mathbb{R}^3$,
\begin{equation}
	\gamma:\ \mathbb{R}\rightarrow\mathbb{R}^3,\quad
	t\mapsto(x(t),y(t),z(t)).
\end{equation}
\begin{example}
	$\lambda\mapsto\ab+\lambda\db$ is the parameteric mapping for a line.
\end{example}
\begin{example}
	$t\mapsto(\cos t,\sin t, t)$ represents a helix gyrating in the $xy$-plane.
\end{example}

Just as we can differentiate real functions, we can differentiate vector
functions as
\begin{equation}
	\frac{\mathrm{d}\gamma}{\mathrm{d}t}=\left(
	\frac{\mathrm{d}x}{\mathrm{d}t}, \frac{\mathrm{d}y}{\mathrm{d}t},
	\frac{\mathrm{d}z}{\mathrm{d}t}\right)=\boldsymbol{\tau}.
\end{equation}
\begin{example}
	In the above two examples, the tangent vectors are respectively $\db$ and
	$(-\sin t, \cos t,1)$.
\end{example}

The same set of points may be parametrised in different ways. For example, the
parabola on the $x=1$ plane may be parametrised by $\gamma=(1,t,t^2)$ or
$\beta=(1,\ex^\lambda,\ex^{2\lambda})$. However, $\gamma$ and $\beta$ are
strictly different because they are parametrised using a different variable.

As in $\mathbb{R}^2$, for $\boldsymbol{r}(t)=(x(t),y(t),z(t))$, the arc length
$s$ is given by
\begin{equation}
	s(t)=\int_{t_0}^{t}\sqrt{\dot{x}^2+\dot{y}^2+\dot{z}^2}\ \mathrm{d}t.
\end{equation}
We have $\mathrm{d}s/\mathrm{d}t=|\mathrm{d}\boldsymbol{r}/\mathrm{d}t|$, and,
at infinitesimal ranges, this is often written as
\begin{equation}
	\mathrm{d}s^2=\left|\frac{\mathrm{d}r}{\mathrm{d}t}\right|^2 \mathrm{d}t^2
	=(\dot{x}^2+\dot{y}^2+\dot{z}^2)\mathrm{d}t^2
	=(\mathrm{d}x)^2+(\mathrm{d}y)^2+(\mathrm{d}z)^2.
\end{equation}
(cf. 3D version of Pythagoras' theorem.) A curve is said to be parametrised by
arc length if $|\mathrm{d}\boldsymbol{r}/\mathrm{d}t|=1$, and in this case we
use $s$ instead of $t$ are the parameter. Note that a curve is a collection of
points, whilst the tangent is a vector, with a definite direction and length.

\begin{lemma}
	If $\gamma(s)$ is parametrised by arc length with tangent vector
	$\boldsymbol{\tau}(s)$, then $\mathrm{d}/\mathrm{d}t(\boldsymbol{\tau}(s))$
	is normal to the curve $\gamma$.
\end{lemma}
\begin{proof}
	Since $\gamma$ is parametrised by arc length,
	$\boldsymbol{\tau}\cdot\boldsymbol{\tau}=1$. So then
	\begin{equation}
		\frac{\mathrm{d}}{\mathrm{d}s}(\boldsymbol{\tau}\cdot\boldsymbol{\tau})
		=2\boldsymbol{\tau}\cdot\frac{\mathrm{d}\boldsymbol{\tau}}{\mathrm{d}s}
		=0.
	\end{equation}
\end{proof}

We can define a surface parametrically, as
\begin{equation}
	S:\mathbb{R}^2\rightarrow \mathbb{R}^3,\qquad
	(\mu,\lambda)\mapsto(x(\mu,\lambda),y(\mu,\lambda),z(\mu,\lambda))
	=\Xb(\mu,\lambda).
\end{equation}
$S$ then is a function of two variables.
\begin{example}
	$S^2=(\sin\theta\cos\phi,\sin\theta\sin\phi,\cos\theta)$ with
	$\theta\in[0,\pi]$ and $\phi\in[0,2\pi]$ is the unit 2-sphere.
\end{example}
\begin{example}
	$S=(\sinh\chi,\cos\phi,\sinh\chi\sin\phi,\cosh\chi)$ with
	$\chi\in\mathbb{R}$ and $\phi\in[0,2\pi]$ gives a hyperboloid.
\end{example}

If $\Xb(\mu,\lambda)$ traces a smooth surface $S$ in $\mathbb{R}^3$, then its
tangent vectors at some $(\mu_0,\lambda_0)$ are
\begin{equation}
	\boldsymbol{\tau}_1 (\mu_0,\lambda_0)
	=\left.\ddy{\Xb}{\mu}\right|_{(\mu_0,\lambda_0)},\qquad
	\boldsymbol{\tau}_2 (\mu_0,\lambda_0)
	=\left.\ddy{\Xb}{\lambda}\right|_{(\mu_0,\lambda_0)}.
\end{equation}
The \underline{tangent plane} of $S$ at $(\mu_0,\lambda_0)$ is then given by
\begin{equation}
	\Pi=\{\boldsymbol{r}\in\mathbb{R}^3\ |\ \boldsymbol{r}=\Xb_0 
	+ \alpha \boldsymbol{\tau}_1 + \beta \boldsymbol{\tau}_2,\
	\alpha,\beta\in\mathbb{R}\}
\end{equation}
\begin{example}
	Find the tangent plane of the unit 2-sphere at
	$(\theta_0,\phi_0)=(\pi/2,0)$. [at the equator]
	
	We have
	\begin{equation}
		\ddy{\Xb}{\theta}=(\cos\theta\cos\phi,\cos\theta\sin\phi,-\sin\theta),
		\qquad
		\ddy{\Xb}{\phi}=(-\sin\theta\sin\phi,\sin\theta\cos\phi,0),
	\end{equation}
	and, at the location of interest, $\Xb_0=\eb_1$,
	$\boldsymbol{\tau}_1=-\eb_3$, and $\boldsymbol{\tau}_2=\eb_2$, so
	$\pi=\eb_1-\alpha\eb_3+\beta\eb_2$, i.e., the $yz$-plane translated to
	$x=1$.
\end{example}

%-------------------------------------------------------------------------------

\subsection{Functions, surfaces and gradients}

Like planes, surfaces and tangent planes may be written in Cartesian form or
using a function $f$, i.e., as
\begin{equation}
	ax+by+cz=d\qquad\textnormal{or}\qquad f(x,y,z)=d.
\end{equation}
A \underline{scalar function} (or \underline{scalar field}) on $\mathbb{R}^3$ is
a map
\begin{equation}
	f:\mathbb{R}^3\rightarrow\mathbb{R},\ (x,y,z)\mapsto f(x,y,z).
\end{equation}
$f$ is \underline{continuous} at $\ab$ if
$|f(\boldsymbol{r})-f(\ab)|\rightarrow0$ as $|\boldsymbol{r}-\ab|\rightarrow0$
along any path, and is differentiable is the partial derivatives exists.
\begin{example}
	For $f=x^4+x^2y^2+xyz$,
	\begin{equation}
		\ddy{f}{x}=4x^3+2xy^2+yz,\qquad \ddy{f}{y}=2x^2y+xz,\qquad
		\ddy{f}{z}=xy,
	\end{equation}
	and
	\begin{equation}
		\grad f=\left(\ddy{f}{x},\ddy{f}{y},\ddy{f}{z}\right)
	\end{equation}
	is the gradient of the function $f$ at $\xb_0$.
\end{example}
$\grad f$ gives us the steepest rate of change of $f$, with the length of the
vector representing the rate of change.
\begin{example}
	$f(x,y)=1-(x^2+y^2)$ is a cone of height 1 with apex at the origin. $\grad
	f=-2(x,y)$, and points inwards and uphill.
\end{example}
The intuition is that $\grad f$ points orthogonally from contours of constraints
in $f$.
\begin{example}
	$f=\sqrt{x^2+y^2+z^2}=r$ is the distance from origin. $\grad
	f=(x,y,z)/r=\hat{\boldsymbol{r}}$, and $\grad f$ describes a sphere.
\end{example}
\begin{example}
	The hyperboloid $f=x^2+y^2-z^2$ has $\grad f=2(x,y,-z)$, and $|\grad
	f|=2\sqrt{x^2+y^2+z^2}$. This points radially outwards on the $xy$-plane but
	down/up in the upper/lower half of $\mathbb{R}^3$.
\end{example}
$\grad f$ defines a normal $\nb$ to the surface $f=\textnormal{const}$, and
hence defines the tangent plane via $\boldsymbol{r}\cdot\nb=\ab\cdot\nb$.
\begin{example}
	Find the tangent plane to $x^2-y^2-z^2=1$ at $(-1,0,0)$. [This is a rotated
	paraboloid with the bottom at $(-1,0,0)$.]
	
	$\grad f=2(x,-y,-z)$, and so $\nb\cdot\boldsymbol{r}=-2x$ whilst
	$\ab\cdot\nb=-2$. Thus $x=-1$ is the tangent plane.
\end{example}
\begin{example}
	Find the tangent plane to $x^2+yz=1$ at $(0,1,1)$. [An inclined
	hyperboloid.]
	
	$\nb=\grad f=(2x,z,y)$, $\boldsymbol{r}=(0,1,1)$, so we have
	$\nb\cdot\boldsymbol{r}=z+y$, $\ab\cdot\nb=2$, so the tangent at that
	location satisfies the relation $z+y=2$.
\end{example}
If $\dy f/\dy x=0$ at a location, then $f$ has a turning point in the
$x$-direction. In the second example, the turning point in the $x$-direction is
a minimum. Looking at $f$ for fixed $y$ and $z$ shows this also.

A point $P=\xb_0$ is a \underline{critical point} of $f$ if $\grad
f=\boldsymbol{0}$ there.
\begin{example}
	$f(x,y,z)=x^2+yz$, $\grad f=\boldsymbol{0}$ at $\xb_0=\boldsymbol{0}$. If
	$y$ and $z$ are of the same sign, it is a minimum, whilst if they are of
	different sign, it is a maximum. This an example of a \underline{saddle
	point}; it is a minimum when it is coming down the spine of the hyperboloid,
	but a maximum when coming up perpendicular to the spine.
\end{example}

We classify extrema by using the second order derivative, which is a matrix
called the \underline{Hessian}, with
\begin{equation}
	\mathsf{H}_{ij}=\ddy{^2 f}{x^i x^j}.
\end{equation}
\begin{example}
	Again, with $f=x^2+yz$, we have
	\begin{equation}
		\mathsf{H}=\begin{pmatrix}2&0&0\\0&0&1\\0&1&0\end{pmatrix}.
	\end{equation}
\end{example}
Note that $\mathsf{H}$ is symmetric assuming $f$ is sufficiently smooth so that
derivative order may be interchanged. The eigenvalue of $\mathsf{H}$ then give
us information about the nature of the extrema.
\begin{example}
	$\mathsf{H}\boldsymbol{r}=\lambda\boldsymbol{r}$ gives $\lambda=\pm1,2$.
	Positive eigenvalues means a local minimum along the direction of the
	eigenvector, and the opposite is true for negative eigenvalues. If
	\begin{itemize}
		\item All eigenvalues positive, we have an absolute minimum of $f$.
		\item All eigenvalues negative, we have an absolute maximum of $f$.
		\item A saddle point if two eigenvalues are positive and one is
		negative.
		\item Hyperboloid of two sheets if two eigenvalues are negative and one
		is positive.
	\end{itemize}
	So again we confirm that we have a saddle point at the relevant location.
\end{example}
%-------------------------------------------------------------------------------

\subsection{Eigenvalue and eigenvectors of $3\times3$ matrix}

Recall that $\vb$ is an eigenvector of $\mathsf{A}$ if
$\mathsf{A}\vb=\lambda=\vb$, and $\lambda$ is the eigenvalue associated with the
eigenvector.
\begin{lemma}
	If $\lambda$ is an eigenvalue of $\mathsf{A}$, then
	$|\mathsf{A}-\lambda\mathsf{I}|=0$.
\end{lemma}
\begin{proof}
	Assuming that we have an eigenvector $\vb$ associated with $\lambda$,
	suppose that $|\mathsf{A}-\lambda\mathsf{I}|\neq0$. Then there exists
	$\mathsf{B}$ that is the inverse of $(\mathsf{A}-\lambda\mathsf{I})$.
	However,
	\begin{equation}
		\vb=\mathsf{I}\vb=\mathsf{B}(\mathsf{A}-\lambda\mathsf{I})\vb=
		\mathsf{B}(\mathsf{A}\vb-\lambda\mathsf{I}),
	\end{equation}
	and since $\mathsf{A}\vb=\lambda\mathsf{I}$, for $\vb\neq\boldsymbol{0}$, we
	have a contradiction.
\end{proof}
$|\mathsf{A}-\lambda\mathsf{I}|$ gives a polynomial in $\lambda$, which is a
cubic in when we are concerned with vectors in $\mathbb{R}^3$. This polynomial
is called the \underline{characteristic polynomial}, and
$|\mathsf{A}-\lambda\mathsf{I}|=0$ is the \underline{characteristic equation} of
$\mathsf{A}$.
\begin{example}
	For $\mathsf{H}$ above,
	\begin{equation}
		|\mathsf{H}-\lambda\mathsf{I}|=\left|\begin{matrix}
		2-\lambda & 0 & 0\\ 0 & -\lambda & 1 \\ 0 & 1 & -\lambda\end{matrix}
		\right|= (2-\lambda)(\lambda^2-1)=0,
	\end{equation}
	so $\lambda=2,\pm1$. It may be shown that the associated eigenvectors are
	\begin{equation}
		\vb_2 = \eb_1,\qquad \vb_1 = \eb_2 + \eb_3,\qquad
		\vb_{-1} = \eb_2-\eb_3.
	\end{equation}
	We notice that the eigenvectors are mutually orthogonal.
\end{example}

\begin{theorem}
	Let $\mathsf{A}$ be a real symmetric $3\times3$ matrix with eigenvalues
	$\lambda_i\in\mathbb{R}$ and associated eigenvectors $\vb_i
	\in\mathbb{R}^3$. Then:
	\begin{enumerate}
		\item if $\lambda_i \neq \lambda_j$, then $\vb_i \cdot \vb_j = 0$;
		\item if $\lambda_i = \lambda_j$, we may choose $\vb_i \cdot \vb_j = 0$.
	\end{enumerate}
\end{theorem}
\begin{proof}
	By writing $\vb_i \cdot \vb_j$ as $\vb_i^T \vb_j = \vb_i \vb_j^T$, then we
	have $\lambda_i \vb_i\cdot\vb_j = \lambda_i \vb_j^T \vb_i = \vb_j^T
	\lambda_i \vb_i = \vb_j^T \mathsf{A}\vb_i$. Taking the transpose does
	nothing to this because it is a scalar, so $\lambda_i \vb_i \cdot \vb_j =
	(\vb_j^T \mathsf{A} \vb_i)^T = \vb_i^T \mathsf{A}^T \vb_j$. Now,
	$\mathsf{A}$ is symmetric, so $\lambda_i \vb_i \cdot \vb_j = \vb_i^T
	\mathsf{A} \vb_j = \lambda_j \vb_i^T \vb_j = \lambda_j \vb_i\cdot \vb_j$.
	Thus we have $(\lambda_j-\lambda_i)\vb_i \cdot\vb_j =0$, and the first
	result follows.
	
	Suppose now that $\lambda_i = \lambda_j$. Then $\vb_i$ and $\vb_j$ span a 2D
	subspace/plane in which all vectors of $\mathsf{A}$ have the same
	eigenvalue. If $\vb_i\cdot\vb_j \neq=0$, then we may redefine $\vb_i$ as
	\begin{equation}
		\vb_i' = \vb_j-\frac{(\vb_i \cdot \vb_j)\vb_j}{|\vb_j|^2},
	\end{equation}
	which is the projection of $\vb_i$ onto the space/plane that is orthogonal
	to $\vb_j$.
\end{proof}
(The latter process of using projections to generate an orthogonal basis is know
as \underline{Gram--Schmidt orthogonalisation}.)

%-------------------------------------------------------------------------------

\subsection{Quadric surfaces}

A quadric is the generalisation of the conic to surfaces in $\mathbb{R}^3$. This
has the form $\boldsymbol{r}^T\mathsf{A}\boldsymbol{r}+\bb\cdot\boldsymbol{r} +
c=0$. $\mathsf{A}$ here is symmetric and of the form
\begin{equation}
	\mathsf{A}=\begin{pmatrix}a&b&c\\b&d&e\\c&e&f\end{pmatrix}.
\end{equation}
\begin{example}
	Suppose all cross terms are zero except for the leading diagonal in
	$\mathsf{A}$, i.e., $ax^2+dy^2+fz^2=1$. Then we have the following cases:
	\begin{enumerate}
		\item An ellipsoid when $a,d,f>0$. This may be parametrised as
		\begin{equation}
			x=\frac{1}{\sqrt{a}}\sin\theta\cos\phi,\qquad
			y=\frac{1}{\sqrt{d}}\sin\theta\sin\phi,\qquad
			z=\frac{1}{\sqrt{f}}\cos\phi.
		\end{equation}
		\item A hyperboloid of one sheet when $a,d>0$, $f<0$. For fixed $z$, $x$
		and $y$ are constrained to be on an ellipse (a circle if $a=d$). This
		may be parametrised as
		\begin{equation}
			x=\frac{1}{\sqrt{a}}\cosh\chi\cos\phi,\qquad
			y=\frac{1}{\sqrt{d}}\sinh\chi\sin\phi,\qquad
			z=\frac{1}{\sqrt{f}}\sinh\chi.
		\end{equation}
		\item A hyperboloid of two sheets when $a,d<0$, $f>0$. There is no
		solution for a limited range of $z$. This may be parametrised as
		\begin{equation}
			x=\frac{1}{\sqrt{a}}\sinh\chi\cos\phi,\qquad
			y=\frac{1}{\sqrt{d}}\sinh\chi\sin\phi,\qquad
			z=\frac{1}{\sqrt{f}}\cosh\chi.
		\end{equation}
		\item No solutions when $a,d,f<0$.
	\end{enumerate}
\end{example}

Note that, generally, we have
\begin{equation}
	f(x,y,z)=ax^2+2bxy+2cxz+dy^2+2eyz+fz^2 = \textnormal{constant}.
\end{equation}
As with conics, it is useful to identify the nearest/furthest points from the
origin. This will occur when the normal is pointing directly from the origin,
i.e., $\nb=\grad f\propto\boldsymbol{r}$, or $\grad f=2\lambda\boldsymbol{r}$.
Since $\grad f=2\mathsf{A}\boldsymbol{r}$, we have $\mathsf{A}\boldsymbol{r} =
\lambda\boldsymbol{r}$. Thus eigenvectors of $\mathsf{A}$ gives the direction of
extrema of the quadric surface from the origin, and these principal axes are
orthogonal by the previous theorem (since $\mathsf{A}$ is symmetric).

Since any vector in $\mathbb{R}^3$ may be written as a sum of three orthogonal
eigenvectors $\vb_1$, $\vb_2$, $\vb_3$, we have $\boldsymbol{r}=r_1 \vb_1 + r_2
\vb_2 + r_3 \vb_3$, so
\begin{equation}
	\boldsymbol{r}^T \mathsf{A}\boldsymbol{r}=
	(r_1 \vb_1^T + r_2\vb_2^T + r_3 \vb_3^T)\mathsf{A}(r_1 \vb_1 + r_2 \vb_2 +
	r_3 \vb_3)=r_1^2 \lambda_1|\vb_1|^2 + r_2^2 \lambda_2|\vb_2|^2 +
	r_3^2 \lambda_3|\vb_3|^2.
\end{equation}
Choosing $\vb_1$ to be unit vectors, we have $\boldsymbol{r}^T
\mathsf{A}\boldsymbol{r}=r_1^2 \lambda_1 + r_2^2 \lambda_2 + r_3^2 \lambda_3$.
Since $\vb_i$ are orthonormal, we can think of them as the new $xyz$-axis.
\begin{example}
	Classify the quadric $5x^2-6xy+5y^2+9z^2=1$.
	
	\begin{equation}
		\mathsf{A}=\begin{pmatrix}5&-3&0\\-3&5&0\\0&0&9\end{pmatrix},\qquad
		|\mathsf{A}-\lambda\mathsf{I}|=(9-\lambda)(8-\lambda)(2-\lambda)=0.
	\end{equation}
	All eigenvalues are positive, so we have an ellipsoid. It may be shown the
	principal axes are
	\begin{equation}
		\vb_9 = \eb_3,\qquad \vb_8=(\eb_1-\eb_2)/\sqrt{2},\qquad
		\vb_2=(\eb_1+\eb_2)/\sqrt{2}.
	\end{equation}
\end{example}

%===============================================================================

\section{Linear maps}

A set of vectors $\{\vb_1,\vb_2,\vb_3\}\in\mathbb{R}^3$ is a \underline{basis}
for $\mathbb{R}^3$ if any vector $\ub\in\mathbb{R}^3$ may be written uniquely as
$\ub=u_1 \vb_1 + u_2 \vb_2 + u_3 \vb_3$. $u_i$ here are the components of the
vector $\ub$ with respect to the basis $\{\vb_i\}$.

Suppose $\boldsymbol{r}$ has co-ordinates $x_1, x_2, x_3$ with respect to
$\{\vb_i\}$. Let $\mathsf{P}=(\vb_1|\vb_2|\vb_3)$, the matrix with $\vb_i$ as
columns. Then, by inspection, $\vb_i=\mathsf{P}\eb_i$, where $\eb_i$ is the
standard Cartesian basis; $\mathsf{P}$ is the linear map that takes us from the
standard basis to the basis $\{\vb_i\}$.

\begin{theorem}
	Let $S=\{\vb_1,\vb_2,\vb_3\}$ be a set of three vectors in $\mathbb{R}^3$,
	and $\mathsf{P}=(\vb_1|\vb_2|\vb_3)$. $S$ is a basis of $\mathbb{R}^3$ if
	$|\mathsf{P}|\neq0$.
\end{theorem}
\begin{proof}
	Suppose that $|\mathsf{P}|\neq0$. Then there exists $\mathsf{P}^{-1}$. For
	an arbitrary vector $\xb\in\mathbb{R}^3$,
	$\xb=\mathsf{P}\mathsf{P}^{-1}\xb=\yb$ is unique, with
	$y=\mathsf{P}^{-1}\xb$. Now,
	\begin{equation}
		\xb=\mathsf{P}\yb=y_1 \mathsf{P}\eb_1 + y_2 \mathsf{P}\eb_2 +
		y_3 \mathsf{P}\eb_3 = y_1 \vb_1 + y_2 \vb_2 + y_3 \vb_3,
	\end{equation}
	so $S$ is a basis.
	
	Suppose now that $S$ is a basis. Then, in particular, the standard basis has
	the representation
	\begin{equation}
		\eb_i = \sum_{j=1}^3 e_i^j \vb_j.
	\end{equation}
	The scalar triple product of the standard basis is 1, so
	\begin{equation}
		1=[\eb_1,\eb_2,\eb_3]=\sum_{i,j,k,=1}^3 e_1^i e_2^j e_3^k
		[\vb_i,\vb_j,\vb_k]\quad\Rightarrow\quad
		[\vb_i,\vb_j,\vb_k]=|\mathsf{P}|=1.
	\end{equation}
	Thus $|\mathsf{P}\neq0$ if $i\neq j\neq k$.
\end{proof}
\begin{example}
	Which of
	\begin{equation}
		B_1=\{(4,1,2),(2,5,1),(0,1,0)\},\qquad B_2=\{(4,1,2),(2,5,1),(0,0,1)\}
	\end{equation}
	forms a basis in $\mathbb{R}^3$?
	
	$\mathsf{P}_1=0$ whilst $\mathsf{P}_2=18$, so only $B_2$ is a basis.
\end{example}

%-------------------------------------------------------------------------------

\subsection{Axioms}

A linear map $L:\mathbb{R}^3\rightarrow\mathbb{R}^3$ is a function such that,
for all $\ub,\vb\in\mathbb{R}^3$ and $\lambda,\mu\in\mathbb{R}$,
$L(\lambda\ub+\mu\vb)=\lambda L(\ub)+\mu L(\vb)$. Note then
$L(\boldsymbol{0})=\boldsymbol{0}$.
\begin{example}
	$L(\ub)=(3u_1 -u_2 +u_3, u_3, 0)$ is linear.
	
	$L(\xb)=(xy,x,z)$ is not linear since the first component is not a linear
	function.
	
	$L(\xb)=(x+1,y,z)$ is not linear since it is not a homogeneous function of
	degree $1$.
	
	$\mathsf{P}=(\vb_1 |\vb_2 |\vb_3)$ induces a linear map by definition.
\end{example}

\begin{theorem}
	A map $L:\mathbb{R}^3\rightarrow\mathbb{R}^3$ is linear iff there exists a
	$3\times3$ matrix $\mathsf{A}$ where $L(\vb)=\mathsf{A}\vb$ for all
	$\vb\in\mathbb{R}^3$. Moreover, $\mathsf{A}=(L(\eb_1)|L(\eb_2)|L(\eb_3))$.
\end{theorem}
\begin{proof}
	Assuming $L$ is a linear map, then by the linearity property, $L(\vb)=v_1
	L(\eb_1)+ v_2 L(\eb_2) + v_3 L(\eb_3) = (L(\eb_1)|L(\eb_2)|L(\eb_3))\vb$ as
	required. Assuming we have $L(\vb)=\mathsf{A}\vb$, by properties of
	matrices, $\mathsf{A}$ automatically induces a linear map.
\end{proof}
\begin{example}
	\begin{equation}
		L\xb=\begin{pmatrix}2x-y+z\\ z\\ 0\end{pmatrix},\qquad
		\mathsf{A}=\begin{pmatrix}3&-1&1\\0&0&1\\0&0&0\end{pmatrix}
	\end{equation}
	\begin{equation}
		L\xb=\begin{pmatrix}6x-2y+5z\\2x+3y-z\\x+5y+2z\end{pmatrix},\qquad
		\mathsf{A}=\begin{pmatrix}6&-2&5\\2&3&-1\\1&5&2\end{pmatrix}
	\end{equation}
\end{example}

Suppose $L_1$ and $L_2$ are linear maps associated with $\mathsf{A}_1$ and
$\mathsf{A}_2$ respectively. Then a combined linear map $L_1\circ L_2(\vb)$ may
be represented by $\mathsf{B}=\mathsf{A}_1 \mathsf{A}_2$.
\begin{lemma}
	Suppose $L:\mathbb{R}^3\rightarrow\mathbb{R}^3$ is a linear map represented
	by $\mathsf{A}$. If $|\mathsf{A}|\neq0$, then there exists a map $L^{-1}$
	such that $L^{-1}\circ L(\vb)=L\circ L^{-1}(\vb)=\vb$.
\end{lemma}
\begin{proof}
	This follows immediately from the fact that $\mathsf{A}^{-1}$ exists when
	$|\mathsf{A}|\neq0$.
\end{proof}

%-------------------------------------------------------------------------------

\subsection{Geometry of linear maps}

There are various linear maps with geometric interpretations. For example, a
reflection in the $yz$-plane is represented by
\begin{equation}
	\begin{pmatrix}x\\y\\z\end{pmatrix}\rightarrow
	\begin{pmatrix}-x\\y\\z\end{pmatrix},\qquad
	\mathsf{A}=\begin{pmatrix}-1&0&0\\0&1&0\\0&0&1\end{pmatrix}.
\end{equation}
A projection onto the $xy$ plane is represented by
\begin{equation}
	\begin{pmatrix}x\\y\\z\end{pmatrix}\rightarrow
	\begin{pmatrix}x\\y\\0\end{pmatrix},\qquad
	\mathsf{A}=\begin{pmatrix}1&0&0\\0&1&0\\0&0&0\end{pmatrix}.
\end{equation}
A rotation of $\theta$ radians in the $xy$-plane is
\begin{equation}
	\mathsf{A}=\begin{pmatrix}\cos\theta&-\sin\theta&0\\
	\sin\theta&\cos\theta&0\\0&0&1\end{pmatrix}.
\end{equation}

\subsubsection{Projection}

Let $\Pi$ be a plane through $\boldsymbol{0}$ with normal vector $\nb$. For
$\xb\in\mathbb{R}^3$,
\begin{equation}
	\xb=\frac{\xb\cdot\nb}{|\nb|^2}\nb
	+\left(\xb-\frac{\xb\cdot\nb}{|\nb|^2}\nb\right).
\end{equation}
We note that the first portion is normal to $\Pi$, whilst the second part is
orthogonal to $\nb$. Since the equation of $\Pi$ is $\boldsymbol{r}\cdot\nb=0$,
the projection of $\xb$ onto $\Pi$ is given by
\begin{equation}
	P(\xb)=\xb-\frac{\xb\cdot\nb}{|\nb|^2}\nb=\xb-\xb\cdot\hat{\nb}\hat{\nb}.
\end{equation}
This is a linear map because
\begin{equation}
	P(\lambda\ub+\vb)=\lambda\ub+\vb-(\lambda\ub+\vb)\cdot\hat{\nb}\hat{\nb}
	=\lambda(\ub-\ub\cdot\hat{\nb}\hat{\nb})+\vb-\vb\cdot\hat{\nb}\hat{\nb}.
\end{equation}

\subsubsection{Reflection}

Similarly, if $\xb$ is the sum of $P(\xb)$ and normal to $\Pi$, then to reflect
$\xb$ in $\Pi$, we subtract twice, so
\begin{equation}
	R(\xb)=\xb-2\xb\cdot\hat{\nb}\hat{\nb}.
\end{equation}

\subsubsection{Rotation}

In $\mathbb{R}^3$, a rotation always leaves a line/axis invariant. Let $\lb$ be
a line through $\boldsymbol{0}$ with direction vector $\db$, then
$\lb=\lambda\db$, so
\begin{equation}
	\xb=(\xb-\xb\cdot\hat{\db}\hat{\db})+\xb\cdot\hat{\db}\hat{\db}.
\end{equation}
The first part is normal to $\lb$ by construction, so a vector orthogonal to
$\lb$ would be $(\xb\cdot\hat{\db}\hat{\db})\times\hat{\db}=\xb\times\hat{\db}$.
By analogy with rotation around a basis vector, we have
\begin{equation}
	R(\xb)=\xb\cdot\hat{\db}\hat{\db}+(\xb-\xb\cdot\hat{\db}\hat{\db})\cos\theta
	-(\xb\times\hat{\db})\sin\theta.
\end{equation}

\begin{example}
	Let $\Pi$ be a plane through $\boldsymbol{0}$ and $\nb=(1,1,2)$ Find the
	matrix of projection onto $\Pi$ with respect to the standard basis.
	
	$\nb\cdot\xb=x+y+2z$, $|\nb|^2=6$, so
	\begin{equation}
		P(\xb)=\begin{pmatrix}x-(x+y+2z)/6\\y-(x+y+2z)/6\\
		z-(x+y+2z)/6\end{pmatrix},\qquad \mathsf{A}=\frac{1}{6}\begin{pmatrix}
		5 & -1 & -2 \\ -1 & 5 & -2 \\ -2 & -2 & 2\end{pmatrix}.
	\end{equation}
\end{example}

\begin{example}
	Let $\Pi$ be a plane through $\boldsymbol{0}$ with $\nb=(1,-1,1)$. Find the
	matrix of reflection in $\Pi$ with respect to the standard basis.
	
	$\nb\cdot\xb=x-y+z$, $|\nb|^2=3$, so
	\begin{equation}
		R(\xb)=\begin{pmatrix}x-(x-y+z)(2/3)\\y+(x-y+z)(2/3)\\
		z-(x-y+z)(2/3)\end{pmatrix},\qquad \mathsf{A}=\frac{1}{3}\begin{pmatrix}
		1 & -2 & -2 \\ 2 & 1 & 2 \\ -2 & 2 & 1\end{pmatrix}.
	\end{equation}
\end{example}

\begin{example}
	Find the matrix of rotation representing a $-\pi/3$ rotation about
	$\lb=\lambda(1,1,1)$.
	
	$\xb\cdot\db=x+y+z$, $|\db|^2=3$, and
	\begin{equation}
		\xb-\xb\cdot\hat{\db}\hat{\db}
		=\frac{1}{3}\begin{pmatrix}2x-y-z\\-x+2y-z\\-x-y+2z\end{pmatrix},\qquad
		\xb\times\hat{\db}
		=\frac{1}{\sqrt{3}}\begin{pmatrix}y-z\\z-x\\x-y\end{pmatrix}.
	\end{equation}
	Collecting this,
	\begin{equation}
		R_{-\pi/3}(\xb)
		=\frac{1}{3}\begin{pmatrix}x+y+z\\x+y+z\\x+y+z\end{pmatrix}+
		\frac{1}{2}\frac{1}{3}
		\begin{pmatrix}2x-y-z\\-x+2y-z\\-x-y+2z\end{pmatrix}
		+\frac{\sqrt{3}}{2}\frac{1}{\sqrt{3}}
		\begin{pmatrix}y-z\\z-x\\x-y\end{pmatrix},
	\end{equation}
	and
	\begin{equation}
		\mathsf{A}=\frac{1}{6}\begin{pmatrix}1 & 2 & -1\\
		-1 & 1 & 2\\2 & -1 & 1\end{pmatrix}.
	\end{equation}
\end{example}

\begin{example}
	\begin{equation}
		R(R(\xb))=R(\xb-2\xb\cdot\hat{\nb}\hat{\nb})
		=R(\xb)-2R(\xb)\cdot\hat{\nb}\hat{\nb}
		=\xb-2\xb\cdot\hat{\nb}\hat{\nb}
		-2(\xb-2\xb\cdot\hat{\nb}\hat{\nb})\cdot\hat{\nb}\hat{\nb}
		=\xb.
	\end{equation}
\end{example}
%-------------------------------------------------------------------------------

\subsection{Change of basis}

\begin{theorem}
	Let $L$ be a linear map given by $\mathsf{A}$ with respect to $\{\eb_i\}$.
	Let $S=\{\vb_i\}$ be another basis of $\mathbb{R}^3$, and $\mathsf{P}=(\vb_1
	|\vb_2| \vb_3)$. Then, with respect to $S$,
	$L=\mathsf{P}^{-1}\mathsf{A}\mathsf{P}$.
\end{theorem}
\begin{proof}
	From a previous theorem, we have $\xb=\mathsf{P}\yb$, where $\yb$ is
	expanded in terms of $S$. Then $L\yb=L(\mathsf{P}^{-1}\xb)$ for
	$|\mathsf{P}|\neq0$. Since $L$ is linear,
	$L\yb=\mathsf{P}^{-1}L(\xb)=\mathsf{P}^{-1}\mathsf{A}\mathsf{P}
	\mathsf{P}^{-1}\xb=\mathsf{P}^{-1}\mathsf{A}\mathsf{P}(\yb)$,
	as required.
\end{proof}

\begin{example}
	Suppose $L$ is represented by
	\begin{equation}
		\mathsf{A}=\begin{pmatrix}1&3&3\\-2&3&-2\\3&2&1\end{pmatrix}
	\end{equation}
	with respect to the standard basis. Find $\mathsf{A}$ with respect to
	$S=\{(1,0,0),(-1,1,0),(0,-1,1)\}$.
	
	\begin{equation}
		\mathsf{P}=\begin{pmatrix}1&-1&0\\0&1&-1\\0&0&1\end{pmatrix},
		\mathsf{P}^{-1}=\begin{pmatrix}1&1&1\\0&1&1\\0&0&1\end{pmatrix},\qquad
		\mathsf{P}^{-1}\mathsf{A}\mathsf{P}=
		\begin{pmatrix}2&5&-5\\1&4&-6\\3&-1&-1\end{pmatrix}.
	\end{equation}
\end{example}

\begin{example}
	Find the matrix of linear map $L$ with respect to the basis $S$, where
	\begin{equation}
		L\xb=\begin{pmatrix}4x-z\\2x+3y-2z\\2x+2y-z\end{pmatrix},\qquad
		S=\left\{\begin{pmatrix}1\\1\\1\end{pmatrix},
		\begin{pmatrix}1\\2\\2\end{pmatrix},\begin{pmatrix}1\\2\\2\end{pmatrix},
		\begin{pmatrix}1\\2\\3\end{pmatrix}\right\}.
	\end{equation}	
	
	\begin{equation}
		\mathsf{A}=\begin{pmatrix}4&0&-1\\2&3&-2\\2&2&-1\end{pmatrix},\qquad
		\mathsf{P}=\begin{pmatrix}1&1&1\\1&2&2\\1&2&3\end{pmatrix},\qquad
		\mathsf{P}^{-1}\begin{pmatrix}2&-1&0\\-1&2&-1\\0&-1&1\end{pmatrix},
	\end{equation}
	so
	\begin{equation}
		\mathsf{P}^{-1}\mathsf{A}\mathsf{P}=
		\begin{pmatrix}3&0&0\\0&2&1\\0&0&1\end{pmatrix}.
	\end{equation}
\end{example}

Finally, we return to rotations. Since $(R_\theta)^{-1}=R_{-\theta}$, we have
\begin{equation}
	\mathsf{A}=\begin{pmatrix}\cos\theta&\sin\theta&0\\
	-\sin\theta&\cos\theta&0\\0&0&1\end{pmatrix},\qquad
	\mathsf{A}^{-1}=\begin{pmatrix}\cos\theta&-\sin\theta&0\\
	\sin\theta&\cos\theta&0\\0&0&1\end{pmatrix}=\mathsf{A}^{T}.
\end{equation}
Now, the scalar product is preserved under rotation (since we are rotating
something that only cares about the magnitude). Any transformation that has the
property that $\mathsf{A}^{T}=\mathsf{A}^{-1}$ is called \underline{orthogonal},
and the set of orthogonal transformations form the \underline{group} $O(3)$,
which includes rotations and reflections. It we restrict $O(3)$ to linear maps
where the associated matrix has $|\mathsf{A}|>0$, then we only have rotations,
and the subset is in fact a subgroup called the \underline{special orthogonal}
group, denoted $SO(3)$.

%===============================================================================

\bibliography{/home/julian/Documents/Tex_Files/master_ref/refs}

%\begin{thebibliography}{48}
%\providecommand{\natexlab}[1]{#1}
%\providecommand{\url}[1]{\texttt{#1}}
%\expandafter\ifx\csname urlstyle\endcsname\relax
%  \providecommand{\doi}[1]{doi: #1}\else
%  \providecommand{\doi}{doi: \begingroup \urlstyle{rm}\Url}\fi

%\bibitem[Limaye(1986)]{lim86}
%S.S. Limaye.
%\newblock {J}upiter: new estimates of the mean zonal flow at the cloud level.
%\newblock \emph{Icarus}, 65:\penalty0 335--352, 1986.

%\end{thebibliography}

\end{document}




%%%%%%%%%%%%%%%%%%%%%%%%%%%%
% Appendices:
% \newpage
% 
% \appendix
% \ifnum\hylinks=1 \phantomsection \fi
% \addcontentsline{toc}{section}{Appendices}
% 
% \section{Sample Appendix}

%%%%%%%%%%%%%%%%%%%%%%%%%%%%%
