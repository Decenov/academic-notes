\documentclass[letter-paper]{tufte-book}

%%
% Book metadata
\title{Calculus 1H}
\author[]{Inusuke Shibemoto}
%\publisher{Research Institute of Valinor}

%%
% If they're installed, use Bergamo and Chantilly from www.fontsite.com.
% They're clones of Bembo and Gill Sans, respectively.
\IfFileExists{bergamo.sty}{\usepackage[osf]{bergamo}}{}% Bembo
\IfFileExists{chantill.sty}{\usepackage{chantill}}{}% Gill Sans

%\usepackage{microtype}
\usepackage{amssymb}
\usepackage{amsmath}
%%
% For nicely typeset tabular material
\usepackage{booktabs}

%% overunder braces
\usepackage{oubraces}

%% 
\usepackage{xcolor}
\usepackage{tcolorbox}

\newtcolorbox[auto counter,number within=section]{derivbox}[2][]{colback=TealBlue!5!white,colframe=TealBlue,title=Box \thetcbcounter:\ #2,#1}                                                          

\makeatletter
\@openrightfalse
\makeatother

%%
% For graphics / images
\usepackage{graphicx}
\setkeys{Gin}{width=\linewidth,totalheight=\textheight,keepaspectratio}
\graphicspath{{figs/}}

% The fancyvrb package lets us customize the formatting of verbatim
% environments.  We use a slightly smaller font.
\usepackage{fancyvrb}
\fvset{fontsize=\normalsize}

\usepackage[plain]{fancyref}
\newcommand*{\fancyrefboxlabelprefix}{box}
\fancyrefaddcaptions{english}{%
  \providecommand*{\frefboxname}{Box}%
  \providecommand*{\Frefboxname}{Box}%
}
\frefformat{plain}{\fancyrefboxlabelprefix}{\frefboxname\fancyrefdefaultspacing#1}
\Frefformat{plain}{\fancyrefboxlabelprefix}{\Frefboxname\fancyrefdefaultspacing#1}

%%
% Prints argument within hanging parentheses (i.e., parentheses that take
% up no horizontal space).  Useful in tabular environments.
\newcommand{\hangp}[1]{\makebox[0pt][r]{(}#1\makebox[0pt][l]{)}}

%% 
% Prints an asterisk that takes up no horizontal space.
% Useful in tabular environments.
\newcommand{\hangstar}{\makebox[0pt][l]{*}}

%%
% Prints a trailing space in a smart way.
\usepackage{xspace}
\usepackage{xstring}

%%
% Some shortcuts for Tufte's book titles.  The lowercase commands will
% produce the initials of the book title in italics.  The all-caps commands
% will print out the full title of the book in italics.
\newcommand{\vdqi}{\textit{VDQI}\xspace}
\newcommand{\ei}{\textit{EI}\xspace}
\newcommand{\ve}{\textit{VE}\xspace}
\newcommand{\be}{\textit{BE}\xspace}
\newcommand{\VDQI}{\textit{The Visual Display of Quantitative Information}\xspace}
\newcommand{\EI}{\textit{Envisioning Information}\xspace}
\newcommand{\VE}{\textit{Visual Explanations}\xspace}
\newcommand{\BE}{\textit{Beautiful Evidence}\xspace}

\newcommand{\TL}{Tufte-\LaTeX\xspace}

% Prints the month name (e.g., January) and the year (e.g., 2008)
\newcommand{\monthyear}{%
  \ifcase\month\or January\or February\or March\or April\or May\or June\or
  July\or August\or September\or October\or November\or
  December\fi\space\number\year
}


\newcommand{\urlwhitespacereplace}[1]{\StrSubstitute{#1}{ }{_}[\wpLink]}

\newcommand{\wikipedialink}[1]{http://en.wikipedia.org/wiki/#1}% needs \wpLink now

\newcommand{\anonymouswikipedialink}[1]{\urlwhitespacereplace{#1}\href{\wikipedialink{\wpLink}}{Wikipedia}}

\newcommand{\Wikiref}[1]{\urlwhitespacereplace{#1}\href{\wikipedialink{\wpLink}}{#1}}

% Prints an epigraph and speaker in sans serif, all-caps type.
\newcommand{\openepigraph}[2]{%
  %\sffamily\fontsize{14}{16}\selectfont
  \begin{fullwidth}
  \sffamily\large
  \begin{doublespace}
  \noindent\allcaps{#1}\\% epigraph
  \noindent\allcaps{#2}% author
  \end{doublespace}
  \end{fullwidth}
}

% Inserts a blank page
\newcommand{\blankpage}{\newpage\hbox{}\thispagestyle{empty}\newpage}

\usepackage{units}

% Typesets the font size, leading, and measure in the form of 10/12x26 pc.
\newcommand{\measure}[3]{#1/#2$\times$\unit[#3]{pc}}

% Macros for typesetting the documentation
\newcommand{\hlred}[1]{\textcolor{Maroon}{#1}}% prints in red
\newcommand{\hangleft}[1]{\makebox[0pt][r]{#1}}
\newcommand{\hairsp}{\hspace{1pt}}% hair space
\newcommand{\hquad}{\hskip0.5em\relax}% half quad space
\newcommand{\TODO}{\textcolor{red}{\bf TODO!}\xspace}
\newcommand{\na}{\quad--}% used in tables for N/A cells
\providecommand{\XeLaTeX}{X\lower.5ex\hbox{\kern-0.15em\reflectbox{E}}\kern-0.1em\LaTeX}
\newcommand{\tXeLaTeX}{\XeLaTeX\index{XeLaTeX@\protect\XeLaTeX}}
% \index{\texttt{\textbackslash xyz}@\hangleft{\texttt{\textbackslash}}\texttt{xyz}}
\newcommand{\tuftebs}{\symbol{'134}}% a backslash in tt type in OT1/T1
\newcommand{\doccmdnoindex}[2][]{\texttt{\tuftebs#2}}% command name -- adds backslash automatically (and doesn't add cmd to the index)
\newcommand{\doccmddef}[2][]{%
  \hlred{\texttt{\tuftebs#2}}\label{cmd:#2}%
  \ifthenelse{\isempty{#1}}%
    {% add the command to the index
      \index{#2 command@\protect\hangleft{\texttt{\tuftebs}}\texttt{#2}}% command name
    }%
    {% add the command and package to the index
      \index{#2 command@\protect\hangleft{\texttt{\tuftebs}}\texttt{#2} (\texttt{#1} package)}% command name
      \index{#1 package@\texttt{#1} package}\index{packages!#1@\texttt{#1}}% package name
    }%
}% command name -- adds backslash automatically
\newcommand{\doccmd}[2][]{%
  \texttt{\tuftebs#2}%
  \ifthenelse{\isempty{#1}}%
    {% add the command to the index
      \index{#2 command@\protect\hangleft{\texttt{\tuftebs}}\texttt{#2}}% command name
    }%
    {% add the command and package to the index
      \index{#2 command@\protect\hangleft{\texttt{\tuftebs}}\texttt{#2} (\texttt{#1} package)}% command name
      \index{#1 package@\texttt{#1} package}\index{packages!#1@\texttt{#1}}% package name
    }%
}% command name -- adds backslash automatically
\newcommand{\docopt}[1]{\ensuremath{\langle}\textrm{\textit{#1}}\ensuremath{\rangle}}% optional command argument
\newcommand{\docarg}[1]{\textrm{\textit{#1}}}% (required) command argument
\newenvironment{docspec}{\begin{quotation}\ttfamily\parskip0pt\parindent0pt\ignorespaces}{\end{quotation}}% command specification environment
\newcommand{\docenv}[1]{\texttt{#1}\index{#1 environment@\texttt{#1} environment}\index{environments!#1@\texttt{#1}}}% environment name
\newcommand{\docenvdef}[1]{\hlred{\texttt{#1}}\label{env:#1}\index{#1 environment@\texttt{#1} environment}\index{environments!#1@\texttt{#1}}}% environment name
\newcommand{\docpkg}[1]{\texttt{#1}\index{#1 package@\texttt{#1} package}\index{packages!#1@\texttt{#1}}}% package name
\newcommand{\doccls}[1]{\texttt{#1}}% document class name
\newcommand{\docclsopt}[1]{\texttt{#1}\index{#1 class option@\texttt{#1} class option}\index{class options!#1@\texttt{#1}}}% document class option name
\newcommand{\docclsoptdef}[1]{\hlred{\texttt{#1}}\label{clsopt:#1}\index{#1 class option@\texttt{#1} class option}\index{class options!#1@\texttt{#1}}}% document class option name defined
\newcommand{\docmsg}[2]{\bigskip\begin{fullwidth}\noindent\ttfamily#1\end{fullwidth}\medskip\par\noindent#2}
\newcommand{\docfilehook}[2]{\texttt{#1}\index{file hooks!#2}\index{#1@\texttt{#1}}}
\newcommand{\doccounter}[1]{\texttt{#1}\index{#1 counter@\texttt{#1} counter}}

\newcommand{\studyq}[1]{\marginnote{Q: #1}}

\hypersetup{colorlinks}% uncomment this line if you prefer colored hyperlinks (e.g., for onscreen viewing)

% Generates the index
\usepackage{makeidx}
\makeindex

\setcounter{tocdepth}{3}
\setcounter{secnumdepth}{3}

%%%%%%%%%%%%%%%%%%%%%%%%%%%%%%%%%%%%%%%%%%%%%%%%%%%%%%%%%%%%%%
% custom commands

\newtheorem{theorem}{\color{pastel-blue}Theorem}[section]
\newtheorem{lemma}[theorem]{\color{pastel-blue}Lemma}
\newtheorem{proposition}[theorem]{\color{pastel-blue}Proposition}
\newtheorem{corollary}[theorem]{\color{pastel-blue}Corollary}

\newenvironment{proof}[1][Proof]{\begin{trivlist}
\item[\hskip \labelsep {\bfseries #1}]}{\end{trivlist}}
\newenvironment{definition}[1][Definition]{\begin{trivlist}
\item[\hskip \labelsep {\bfseries #1}]}{\end{trivlist}}
\newenvironment{example}[1][Example]{\begin{trivlist}
\item[\hskip \labelsep {\bfseries #1}]}{\end{trivlist}}
\newenvironment{remark}[1][Remark]{\begin{trivlist}
\item[\hskip \labelsep {\bfseries #1}]}{\end{trivlist}}

\hyphenpenalty=5000

% more pastel ones
\xdefinecolor{pastel-red}{rgb}{0.77,0.31,0.32}
\xdefinecolor{pastel-green}{rgb}{0.33,0.66,0.41}
\definecolor{pastel-blue}{rgb}{0.30,0.45,0.69} % crayola blue
\definecolor{gray}{rgb}{0.2,0.2,0.2} % dark gray

\xdefinecolor{orange}{rgb}{1,0.45,0}
\xdefinecolor{green}{rgb}{0,0.35,0}
\definecolor{blue}{rgb}{0.12,0.46,0.99} % crayola blue
\definecolor{gray}{rgb}{0.2,0.2,0.2} % dark gray

\xdefinecolor{cerulean}{rgb}{0.01,0.48,0.65}
\xdefinecolor{ust-blue}{rgb}{0,0.20,0.47}
\xdefinecolor{ust-mustard}{rgb}{0.67,0.52,0.13}

%\newcommand\comment[1]{{\color{red}#1}}

\newcommand{\dy}{\partial}
\newcommand{\ddy}[2]{\frac{\dy#1}{\dy#2}}

\newcommand{\ab}{\boldsymbol{a}}
\newcommand{\bb}{\boldsymbol{b}}
\newcommand{\cb}{\boldsymbol{c}}
\newcommand{\db}{\boldsymbol{d}}
\newcommand{\eb}{\boldsymbol{e}}
\newcommand{\lb}{\boldsymbol{l}}
\newcommand{\nb}{\boldsymbol{n}}
\newcommand{\tb}{\boldsymbol{t}}
\newcommand{\ub}{\boldsymbol{u}}
\newcommand{\vb}{\boldsymbol{v}}
\newcommand{\xb}{\boldsymbol{x}}
\newcommand{\wb}{\boldsymbol{w}}
\newcommand{\yb}{\boldsymbol{y}}

\newcommand{\Xb}{\boldsymbol{X}}

\newcommand{\ex}{\mathrm{e}}
\newcommand{\zi}{{\rm i}}

\newcommand\Real{\mbox{Re}} % cf plain TeX's \Re and Reynolds number
\newcommand\Imag{\mbox{Im}} % cf plain TeX's \Im

\newcommand{\zbar}{{\overline{z}}}

\newcommand\Def[1]{\textbf{#1}}

\newcommand{\qed}{\hfill$\blacksquare$}
\newcommand{\qedwhite}{\hfill \ensuremath{\Box}}

%%%%%%%%%%%%%%%%%%%%%%%%%%%%%%%%%%%%%%%%%%%%%%%%%%%%%%%%%%%%%%
% some extra formatting (hacked from Patrick Farrell's notes)
%  https://courses.maths.ox.ac.uk/node/view_material/4915
%

% chapter format
\titleformat{\chapter}%
  {\huge\rmfamily\itshape\color{pastel-red}}% format applied to label+text
  {\llap{\colorbox{pastel-red}{\parbox{1.5cm}{\hfill\itshape\huge\color{white}\thechapter}}}}% label
  {1em}% horizontal separation between label and title body
  {}% before the title body
  []% after the title body

% section format
\titleformat{\section}%
  {\normalfont\Large\itshape\color{pastel-green}}% format applied to label+text
  {\llap{\colorbox{pastel-green}{\parbox{1.5cm}{\hfill\color{white}\thesection}}}}% label
  {1em}% horizontal separation between label and title body
  {}% before the title body
  []% after the title body

% subsection format
\titleformat{\subsection}%
  {\normalfont\large\itshape\color{pastel-blue}}% format applied to label+text
  {\llap{\colorbox{pastel-blue}{\parbox{1.5cm}{\hfill\color{white}\thesubsection}}}}% label
  {1em}% horizontal separation between label and title body
  {}% before the title body
  []% after the title body

%%%%%%%%%%%%%%%%%%%%%%%%%%%%%%%%%%%%%%%%%%%%%%%%%%%%%%%%%%%%%%%%%%%%%%%%%%%%%%%%

\begin{document}

% Front matter
%\frontmatter

% r.3 full title page
%\maketitle

% v.4 copyright page

\chapter*{}

\begin{fullwidth}

\par \begin{center}{\Huge Calculus 1H}\end{center}

\vspace*{5mm}

\par \begin{center}{\Large typed up by B. S. H. Mithrandir}\end{center}

\vspace*{5mm}

\begin{itemize}
  \item \textit{Last compiled: \monthyear}
  \item Adapted from notes of A. Taormina and I. MacPhee, Durham
  \item This was part of the Durham Core A module given in the first year.
  Re-arranged here slightly as some of the triple integral stuff was actually
  given in the probability section (for time constraint reasons I guess). Basic
  concepts of calculus and analysis, going up to multi-dimensional integrals.
  \item[]
  \item \TODO diagrams
\end{itemize}

\par

\par Licensed under the Apache License, Version 2.0 (the ``License''); you may not
use this file except in compliance with the License. You may obtain a copy
of the License at \url{http://www.apache.org/licenses/LICENSE-2.0}. Unless
required by applicable law or agreed to in writing, software distributed
under the License is distributed on an \smallcaps{``AS IS'' BASIS, WITHOUT
WARRANTIES OR CONDITIONS OF ANY KIND}, either express or implied. See the
License for the specific language governing permissions and limitations
under the License.
\end{fullwidth}

%===============================================================================

\chapter{Functions}

A value which depends on another value implies something is a
\Def{function} or another. For example, we may have day $d$ and
temperature $T$, assuming that day affects the temperature. Then $t=f(d)$ and
has a unique value of temperature associated with it. Finding a function which
represents a given situation is called \Def{mathematical modelling}.

The \Def{domain} is the set of values the independent variable can
take. The \Def{range} or \Def{image} is the set of values the
dependent variable can take. More formally, assuming we are dealing with
functions of a real variable $x$,
\begin{align*}
  \mbox{dom}\ f &= \{x\in\mathbb{R}\ |\ f(x)\ \textnormal{exists}\},\\
  \mbox{im}\ f  &= \{y\in\mathbb{R}\ |\ y=f(x),\ x\in\mbox{dom}\ f\}.
\end{align*}
Therefore we have
\begin{equation*}
  f:\mbox{dom}\ f\rightarrow\mbox{im} f,\quad x\mapsto f(x)=y.
\end{equation*}
\begin{example}
  For $f(x)=|x|$, $\mbox{dom}\ f=\mathbb{R}=(-\infty,+\infty)$, whilst 
  $\mbox{im}\ f=[0,\infty)$, that is, including $0$. For $f(x)=\sqrt{1-x^2}$, 
  $\mbox{dom}\ f=[-1,1]$ and $\mbox{im}\ f=[0,1]$.
\end{example}

The \Def{graph} of $f$ is defined to be the set 
$\{(x,y)\ |\ x\in\mbox{dom}\ f,\ y=f(x)\}$. We say a function is not 
\Def{well-defined} if it is multi-valued for a given element in the 
domain; for example, if the branch is not defined, then the square root function 
is ill-defined.

If $f(x)=f(-x)$ for all $x\in\mbox{dom}\ f$, then we say the function is
\Def{even}. If $f(x)=-f(-x)$ for all $x\in\mbox{dom}\ f$, then we say
the function is \Def{odd}. For example, $x^2$ is even whilst $x^3$ is
odd. Even functions are symmetrical about the $y$-axis, whilst odd functions
are symmetrical about the origin.
\begin{example}
  It may be shown that
  \begin{equation*}
    f(x)=\begin{cases}x^3, & x>0,\\ 0, & x=0,\\ -x^3, & x<0\end{cases}
  \end{equation*}
  is an even function.
\end{example}
Note that some functions are neither even nor odd. Also, in some calculations,
it pays to spot the parity of the functions (e.g., when integrating over a
symmetric domain).

any function $f(x)$ may be written as the sum of an even and odd function, as
\begin{equation*}
  f(x)=\frac{f(x)+f(-x)}{2}+\frac{f(x)-f(-x)}{2}.
\end{equation*}
We notice that the product of two odd/even functions are even, whilst the
product of an odd and even function is odd.

For $f(x)$ a function with $a,b,\in\mathbb{R}$ and $\alpha\in\mathbb{R}_0^+$, 
some basic manipulations are as follows:
\begin{itemize}
  \item $f(x-a)$ shifts the graph by $(a,0)$, i.e., a translation in $x$;
  \item $f(x)+b$ shifts the graph by $(0,b)$, i.e., a translation in $y$;
  \item $f(-x)$ reflects the graph in the $y$-axis;
  \item $-f(-x)$ reflects the graph about the origin;
  \item $|f(x)|$ reflects any part of the graph with $f(x)<0$ about the 
  $x$-axis;
  \item $af(x)$ stretches the graph in the vertical by $\alpha$;
  \item $f(\alpha x)$ stretches the graph in the horizontal by $1/\alpha$.
\end{itemize}

%-------------------------------------------------------------------------------

\section{Combination of functions}

A function $f$ is given by (i) a set $\mbox{dom}\ f$, (ii) a set $\mbox{im}\
f$, (iii) by a \Def{rule} which assigns, to each element of $\mbox{dom}\
f$, at most one element $f(x)$ in $\mbox{im}\ f$. Then we have the following:
\begin{itemize}
  \item the \Def{sum} of $f$ and $g$ is defined to be
  \begin{equation*}
    (f+g)(x)=f(x)+g(x),\qquad \mbox{dom}(f+g)(x)=\mbox{dom}\ f\cap\mbox{dom}\ g;
  \end{equation*}
  \item the \Def{product} of $f$ and $g$ is
  \begin{equation*}
    (f\cdot g)(x)=f(x)\cdot g(x),\qquad
    \mbox{dom}(f\cdot g)(x)=\mbox{dom}\ f\cap\mbox{dom}\ g;
  \end{equation*}
  \item the \Def{quotient} is
  \begin{align*}
    \left(\frac{f}{g}\right)(x)=\frac{f(x)}{g(x)},\qquad g(x)\neq0,\\
    \mbox{dom}\left(\frac{f}{g}\right)(x)=\mbox{dom}\ f\cap\mbox{dom}\ g.
  \end{align*}
  \item \Def{scalar multiplication} by $\alpha\in\mathbb{R}-\{0\}$ gives
  \begin{equation*}
    (\alpha f)(x)=\alpha f(x),\qquad \mbox{dom}(\alpha f)=\mbox{dom}\ f;
  \end{equation*}
  \item \Def{linear combinations} then give
  \begin{align*}
    (\alpha f+\beta g)(x)=\alpha f(x) + \beta g(x),\\
    \mbox{dom}(\alpha f+\beta g) = \mbox{dom}\ f \cap \mbox{dom}\ g;
  \end{align*}
  \item $f$ \Def{composed} with $g$ is
  \begin{equation*}
    (g \circ f)(x)=g(f(x)),\qquad f(x)\in\mbox{dom}\ g
  \end{equation*}
  for the composition to exist. Note that $(g\circ f)\neq(f\circ g)$ 
  necessarily.
\end{itemize}
\begin{example}
  Take $f(x)=x^2-1$, $g(x)=\sqrt{3-x}$, we have
  \begin{align*}
    (f\circ g)(x) &= f(g(x)) \\
      &= f(\sqrt{3-x}) \\
      &= (\sqrt{3-x})^2 - 1 \\
      &= 2-x,\qquad \mbox{dom}(f\circ g)=(-\infty,3],
  \end{align*}
  while
  \begin{align*}
    (g\circ f)(x) &= g(x^2-1) \\
      &= \sqrt{3-(x^2-1)} \\
      &= \sqrt{4-x^2},\qquad \mbox{dom}(g\circ f)=[-2,2].
  \end{align*}
\end{example}

%-------------------------------------------------------------------------------

\section{Inverses}

A function $f:A\rightarrow B$ is \Def{injective} iff, for all $x_1,
x_2\in\mbox{dom}\ f$, $f(x_1)=f(x_2)$ implies $x_1 = x_2$. So if $f$ is
injective, then any element of $B$ is the image of at most one element in $A$.
\begin{example}
  Consider $f:\mathbb{R}\rightarrow\mathbb{R}$, $x\mapsto x^2+x-6$. By drawing 
  the graph of $f(x)$, we realise that $g$ is not injective because for $y=0$, 
  $x=2,-3$.
\end{example}
A function $f:A\rightarrow B$ is \Def{surjective} iff for $y\in B$,
there exists at least one element in $A$ with $y=f(x)$. A function is 
\Def{bijective} if $f$ is both injective and surjective.

\begin{theorem}
  If $f$ is an injective function, then there exists a function $f^{-1}$ such 
  that $\mbox{dom}\ f^{-1}=\mbox{im}\ f$, and $(f\circ f^{-1})(x)
  =(f^{-1}\circ f)(x)=x$, with $x\in\mbox{im}\ f$. \qedwhite
\end{theorem}
$f^{-1}$ is called the \Def{inverse} of $f$ in the sense of function 
composition.
\begin{example}
  Suppose $f(x)=\sin x$, then $f(x):[-\pi/2,\pi/2]\rightarrow[-1,1]$. Then 
  $f^{-1}(x)=\arcsin x$, with $f^{-1}(x):[-1,1]\rightarrow[-\pi/2,\pi/2]$.
\end{example}

%-------------------------------------------------------------------------------

\section{Periodic and hyperbolic functions}

A function is \Def{periodic} if it satisfies $f(x+p)=f(x)$ for
$p\in\mathbb{R}-\{0\}$. The smallest value of such a $p$ is known as the
\Def{period} of the function. For example, $\sin x$ and $\tan x$ are 
periodic with period $2\pi$ and $\pi$ respectively.

The \Def{hyperbolic sine} and \Def{hyperbolic cosine} function are
\begin{equation*}
  \sinh x=\frac{\ex^x - \ex^{-x}}{2},\qquad \cosh x=\frac{\ex^x + \ex^{-x}}{2},
\end{equation*}
and notice they are odd and even functions respectively. We also have 
\Def{hyperbolic tangent} function to be $\tanh x = \sinh x / \cosh x$. 
Several properties to note:
\begin{itemize}
  \item $\sinh 0 = 0$ and $\cosh 0 = 1$;
  \item $\cosh^2 x- \sinh^2 x = 1$;
  \item the two functions traces out the right-hand portion of a hyperbola;
  \item $0 < \sinh x < \cosh x$ for $x > 0$;
  \item $\ex^x = \cosh x + \sinh x$ while $\ex^{-x} = \cosh x - \sinh x$.
\end{itemize}

%===============================================================================

\chapter{Limits, continuity and differentiation}

%-------------------------------------------------------------------------------

\section{Limits}

A function may have a \Def{limit} even if the the limit is not in the
image of the function, or that the input value associated with the limit is
not in the domain of the function.
\begin{example}
  For $f(x)=(x^2-9)/(x-3)$, $\mbox{dom}\ f=\mathbb{R}-\{3\}$. However, we 
  observe that $f(x)=x+3$, so $\lim_{x\rightarrow3} f(x)=6$.
\end{example}

In $\mathbb{R}$, a limit may be approach from below or above, and we may have 
the case that
\begin{equation*}
  \lim_{x\nearrow c}f(x)=a,\qquad \lim_{x\searrow c}=b,
\end{equation*}
but $a$ does not have to be equal to $b$ necessarily (e.g., at a discontinuity). 
A well-defined limit only exists if $a=b$.
\begin{example}
  So the following do not have well-defined limits:
  \begin{enumerate}
    \item $f(x)=1/(x-2)$ for $x\rightarrow2$. We have that, for $x>2$, $f(x)>0$, 
    while for $x<2$, $f(x)<0$, so that
    \begin{equation*}
      \lim_{x\searrow2}=+\infty,\qquad \lim_{x\nearrow2}=-\infty,
    \end{equation*}
    and limit does not exist.
    
    \item $f(x)=\sin\pi/x$ for $x\searrow0$. We may approach $x$ as $\{1/n\}$, 
    in which case $f(x)\rightarrow 0$, whilst $\{1/(4n-1)\}$ gives $f(x)
    \rightarrow 1$, and $\{2/(4n-1)\}$ gives $f(x)\rightarrow -1$, so limit does
    not exist as $x\rightarrow 0$.
  \end{enumerate}
\end{example}

In summary:
\begin{itemize}
  \item $\lim_{x\nearrow c}\neq \lim_{x\searrow c}$;
  \item $\lim f(x)=\pm\infty$;
  \item $f(x)$ oscillates as $\lim_{x\nearrow c}$ or $\lim_{x\searrow c}$.
\end{itemize}

%-------------------------------------------------------------------------------

\section{Calculus of limits}

There are several ways to calculates $\lim_{x\rightarrow c} f(x)$, the easiest 
of which is just just evaluate $f(c)$ if it exists. For $l,m,k\in\mathbb{R}$ and 
$\lim_{x\rightarrow b}f(x)=l$, $\lim_{x\rightarrow b}=m$, we have
\begin{itemize}
  \item $\lim_{x\rightarrow b}f(x)\pm g(x)=l\pm m$ (\Def{sum/difference});
  \item $\lim_{x\rightarrow b}f(x)\cdot g(x)=lm$ (\Def{product});
  \item $\lim_{x\rightarrow b}f(x)/g(x)=l/m$ for $m\neq0$ 
  (\Def{quotient});
  \item $\lim_{x\rightarrow b}[f(x)]^{r/s}=l^{r/s}$ for $s\neq 0$ 
  (\Def{power}).
\end{itemize}

Otherwise, there are several things we could try.

\subsection{Eliminating zero from denominator}

Sometimes it is possible to remove the apparently singularity. Such as
\begin{equation*}
  \lim_{x\rightarrow 1}\frac{x^2+x-2}{x^2-x}
  =\lim_{x\rightarrow 1}\frac{(x-1)(x+2)}{x(x-1)}
  =\lim_{x\rightarrow 1}\frac{x+2}{x}=3,
\end{equation*}
or
\begin{align*}
  \lim_{x\rightarrow 0}\frac{\sqrt{x^2+100}-10}{x^2} &= \lim_{x\rightarrow 0}\frac{\sqrt{x^2+100}-10}{x^2} \frac{\sqrt{x^2+100}+10}{\sqrt{x^2+100+10}}\\
  &= \lim_{x\rightarrow 0}\frac{1}{\sqrt{x^2+100}+10}=\frac{1}{20}.
\end{align*}

\subsection{Squeezing theorem}

\begin{theorem}
  If $g(x)\leq f(x)\leq k(x)$ for $x\in(a,c)$, where $b\in(a,c)$, then if 
  $g(x)\to l$ and $k(x)\to l$ for $x\to b$, then $\lim_{x\to b}f(x)=l$. \qedwhite
\end{theorem}
\begin{example}
  Find the limit of:
  \begin{enumerate}
    \item $(\sin x)/x$ for $x\to 0$.
  
    Consider the circle of radius $1$ centred at the origin. For some angle 
    $x\in[0,\pi/2]$, suppose we trace out the following.
    
    \begin{marginfigure}
      \includegraphics{figs/circle1}
    \end{marginfigure}
    
    The area of the triangle OAP is $(1/2)(BP)(OA)=(1/2)\sin x$, while the area 
    of the triangle OQA is $(1/2)(OA)(AQ)=(1/2)\tan x$. The sector $OAP$ has 
    area $x/2$, and since the sector area is bounded between the are of the two 
    triangles, we have
    \begin{equation*}
      \frac{1}{2}\sin x<\frac{x}{2},\qquad \frac{x}{2}<\frac{1}{2}\tan x\qquad
      \Rightarrow \qquad
      \frac{\sin x}{x}<1,\qquad \cos x<\frac{\sin x}{x},
    \end{equation*}
    so $\cos x<(\sin x)/x<1$, and by squeezing theorem, $\lim_{x\to 0}(\sin x)/x 
    =1$.
    
    \item $(1-\cos x)/x$ for $x\to 0$
    \begin{equation*}
      \lim_{x\to0}\frac{1-\cos x}{x}=
      \lim_{x\to0}\frac{1-\cos x}{x}\frac{1+\cos x}{1+\cos x}
      =\lim_{x\to0}\left(\frac{\sin x}{x}\frac{\sin x}{1+\cos x}\right)
      =1
    \end{equation*}
    by the product rule.
  \end{enumerate}
\end{example}

\subsection{Rational functions when $x\to\pm\infty$}

\begin{example}
  Find the limits of the following:
  \begin{enumerate}
    \item a rational function where the degree of the numerator is smaller than 
    the denominator,
    \begin{equation*}
      \lim_{x\to\infty}\frac{5x^2+8x-3}{3x^3+1}=
      \lim_{x\to\infty}\frac{5+8/x-3/x^2}{3x+1/x^2}=0.
    \end{equation*}
    
    \item a rational function where the degree of the numerator is equal to the 
    denominator,
    \begin{equation*}
      \lim_{x\to\infty}\frac{5x^2+8x-3}{3x^2+1}=
      \lim_{x\to\infty}\frac{5+8/x-3/x^2}{3+1/x^2}=\frac{5}{3}.
    \end{equation*}
    
    \item a rational function where the degree of the numerator is less than the 
    denominator,
    \begin{equation*}
      \lim_{x\to\infty}\frac{5x^2+8x-3}{3x+1}=
      \lim_{x\to\infty}\frac{5+8/x-3/x^2}{3/x+1/x^2}=\infty.
    \end{equation*}
  \end{enumerate}
\end{example}

%-------------------------------------------------------------------------------

\section{Continuity}

A function $f$ is \Def{continuous} at a point $c$ if (i), $f(c)$
is defined, (ii) the limit of $f(x)$ as $x\to c$ exists, and (iii), if the
limit is there is $f(c)$. Trigonometric functions and polynomials are considered
to be continuous. On an interval, if $f$ is continuous at on $(a,b)$, then $f$
is continuous at each point in $(a,b)$. If $f$ is defined on $[a,b]$, $f$ is
continuous on $[a.b]$ if (i) $f$ is continuous at $(a,b)$, and (ii) if the right
and left limits exists.

Discontinuities may be (i) \Def{removable} (discontinuity at a point),
(ii) a \Def{jump} (discontinuity at an interval), (iii)
\Def{infinite} (vertical asymptotes).

\begin{theorem}[Intermediate value theorem]
  If $f$ is continuous at $(a,b)$ and $k$ is any number between $f(a)$ and
  $f(b)$, then there exists at least one value $c\in(a,b)$ such that $f(c)=k$. \qedwhite
\end{theorem}

\begin{example}
  Show that $f(x)=(\cos x\pi/2) - x^2$ for $x\in[0,1]$ has at least one root in
  $(0,1)$.
  
  Trigonometric and polynomials are continuous, so $f(x)$ is continuous here.
  Since $f(0)=1$ and $f(1)=-1$, there must be at least one $c\in(0,1)$ such that
  $f(c)=0$ by the intermediate value theorem.
\end{example}

%-------------------------------------------------------------------------------

\section{Differentiation}

We define the \Def{derivative} of $f(x)$ to be
\begin{equation*}
  f'(x)=\lim_{h\to 0}\frac{f(x+h)-f(x)}{h}.
\end{equation*}
The derivative exists provided the limit exists. Note that $h$ can be of either
sign, and that $\mbox{dom}\ f'\subset \mbox{dom} f$ since
\begin{equation*}
  \mbox{dom}\ f'=\left\{x\in\mbox{dom}\ f\ |\ \mbox{there exists}\ 
  \lim_{h\to 0}\frac{f(x+h)-f(x)}{h}\right\}.
\end{equation*}
\begin{example}
  By definition, with $f(x)=\sqrt{x}$,
  \begin{equation*}
    f'(x)=\lim_{h\to 0}\frac{\sqrt{x+h}-\sqrt{x}}{h}=
    \lim_{h\to 0}\frac{x+h-x}{h(\sqrt{x+h}+\sqrt{x})}=
    \lim_{h\to 0}\frac{1}{\sqrt{x+h}+\sqrt{x}}=\frac{1}{2\sqrt{x}}.
  \end{equation*}
\end{example}

\begin{proposition}
  If $f$ is differentiable at $x$, then $f$ is continuous at $x$.
\end{proposition}
\begin{proof}
  We have
  \begin{equation*}
    \lim_{h\to 0}\left[f(x+h)-f(x)\right]=
    \lim_{h\to 0}\left[\frac{f(x+h)-f(x)}{h}\right]\cdot\lim_{h\to 0}h.
  \end{equation*}
  Since $f$ is differentiable, the first limit on the right hand side is finite,
  and thus $\lim_{h\to 0}[f(x+h)-f(x)]=0$. So $\lim_{h\to 0}f(x+h)=f(x)$, and
  $f$ is continuous. \qed
\end{proof}
The converse is not true. for example, taking
\begin{equation*}
  f(x)=|x|=\begin{cases}x, & x\geq 0,\\-x, & x<0, \end{cases}
\end{equation*}
then at $x=0$,
\begin{equation*}
  \lim_{h\to 0}\frac{f(x+h)-f(x)}{h}=\lim_{h\to 0}\frac{|h|}{h},
\end{equation*}
which depends on the sign of $h$, so the derivative is not well-defined at
$x=0$ (although it is continuous there).

\begin{example}
  Is the function
  \begin{equation*}
    f(x)=\begin{cases}x, & x\leq 1,\\ (x+1)/2, & x\geq 1,\end{cases}
  \end{equation*}
  differentiable on (i) $[0,2]$, (ii) $[1,2]$?
  
  It may be checked that the two one-sided limits do not agree on $x=1$, so it
  is not differentiable on $[0,2]$. However, only the one sided limits are
  required for $[1,2]$, so the function is differentiable on that domain.
\end{example}

\begin{proposition}
  Let $f$ and $g$ be differentiable at $x$, then the following are
  differentiable there also:
  \begin{itemize}
    \item $f+g$;
    \item $\alpha f$ where $\alpha\in\mathbb{C}$;
    \item $fg$, with $(fg)'=f'g + fg'$ (\Def{product rule});
    \item $f/g$, with $(f/g)'=(fg'-fg')/g^2$ (\Def{quotient rule}, or
    write $1/g=g^-1$ and use chain rule);
    \item $(g\circ f)$, with $(g\circ f)'=g'(f(x))\cdot f'$ (\Def{chain
    rule}). \qedwhite
  \end{itemize}
\end{proposition}

%-------------------------------------------------------------------------------

\section{Higher order derivatives and Taylor approximations}

We recall that
\begin{equation*}
  f'(x_0)=\lim_{h\to 0}\frac{f(x_0+h)-f(x_0)}{h}.
\end{equation*}
Geometrically, this says that as $h\to 0$, the chord linking $f(x_0+h)$ and
$f(x_0)$ approaches the tangent to the curve at $x_0$. The line containing
$(x_0,f(x_0))$ and whose slope is $f'(x_0)$ is the tangent line to the graph
$f(x)$ at $(x_0,f(x_0))$. The tangent has the equation
\begin{equation*}
  y-f(x_0)=f'(x_0)(x-x_0),
\end{equation*}
and one can approximate $f(x)$ near $x_0$ by writing
\begin{equation*}
  f(x)\approx f(x_0)+f'(x_0)(x-x_0).
\end{equation*}
In this case, $f(x)$ is approximated by a polynomial of degree $1$ in $(x-x_0)$,
called the \Def{Taylor polynomial of degree $1$}.

One can obtain better approximations to $f(x)$ near $x_0$. Denoting $f^{(n)}(x)$
to be the $n^{\textnormal{th}}$ derivative of $f$, the \Def{Taylor
polynomial of degree $n$} is given by
\begin{align*}
  P_n(x-x_0) &= f(x_0)+f'(x_0)(x-x_0)+\frac{f''(x_0)}{2!}(x-x_0)^2+\cdots \\
    &= \sum_{j=0}^n\frac{f^{(j)}(x_0)}{j!}(x-x_0)^j.
\end{align*}

\begin{example}
  Find the Taylor polynomials of degree $n$ for $f(x)$ near $x_0$:
  \begin{enumerate}
    \item $f(x)=\cos x$, $n=4$, $x_0=0$.
    
    Note that the odd derivatives give a $\sin x$ which is zero at $x=0$, and
    since $\cos 0=1$,
    \begin{equation*}
      P_4(x)=1-\frac{x^2}{2!}+\frac{x^4}{4!},
    \end{equation*}
    which is only valid near $x=0$.
    
    \item $f(x)=\ex^x$, $n=3$, $x_0=1$.
    \begin{equation*}
      P_3(x)=\ex+\ex(x-1)+\frac{\ex}{2!}(x-1)^2+\frac{\ex}{3!}(x-1)^3.
    \end{equation*}
  \end{enumerate}
\end{example}

\begin{theorem}[Taylor's theorem]
  If $f$ has $n+1$ derivatives that are continuous on an open interval $I$,
  $x_0\in I$ for all $x\in I$, then
  \begin{align*}
    f(x) &= \sum_{j=0}^n\frac{f^{(j)}(x_0)}{j!}(x-x_0)^j + R_n(x), \\
    R_n(x) &= \frac{f^{(n+1)}(c)}{(n+1)!}(x-x_0)^{n+1},\qquad c\in(x_0, x),
  \end{align*}
  where this is an equality rather than an approximation. \qedwhite
\end{theorem}
$R_n(x)$ is the remainder and it is important to bound $R_n(x)$. One useful one
is
\begin{equation*}
  |R_n(x)|\leq \frac{|x-x_0|^{n+1}|}{(n+1)!}
  \max_{t\in(x_0,x)}\left|f^{(n+1)}(t)\right|.
\end{equation*}
If $R_n(x)\to 0$ as $n\to\infty$, then
\begin{equation*}
  f(x)=\sum_{j=0}^\infty\frac{f^{(j)}(x_0)}{j!}(x-x_0)^j,
\end{equation*}
and the right hand side is a \Def{Taylor series expansion} near $x_0$ the
converges to $f(x)$.

We have the following useful expansions
\begin{enumerate}
  \item if $f(x)$ is a polynomial then trivially the Taylor series is $f(x)$
  for all $x\in\mathbb{R}$;
  
  \item
  \begin{equation*}
    \ex^x = \sum_{j=0}^\infty \frac{x^j}{j!} = 1 + x + \frac{x^2}{2!}+\cdots,
    \qquad x\in\mathbb{R};
  \end{equation*}
  
  \item Remembering that $\cosh x + \sinh x = \ex^x$, we have that
  \begin{align*}
    \cosh x &= \sum_{j=0}^\infty \frac{x^{2j}}{(2j)!} = 
    1 + \frac{x^2}{2!} + \frac{x^4}{4!}+\cdots, \\
    \sinh x &= \sum_{j=0}^\infty \frac{x^{2j+1}}{(2j+1)!} = 
    x + \frac{x^3}{3!} + \frac{x^5}{5!}+\cdots
  \end{align*}
  
  \item using $\ex^{\zi\theta}=\cos\theta + \zi\sin\theta$, we have that
  \begin{align*}
    \cos x &= \sum_{j=0}^\infty (-1)^{j}\frac{x^{2j}}{(2j)!}
    = 1 - \frac{x^2}{2!} + \frac{x^4}{4!} + \cdots, \\
    \sin x &= \sum_{j=0}^\infty (-1)^{j}\frac{x^{2j+1}}{(2j+1)!}
    = x - \frac{x^3}{3!} + \frac{x^5}{5!} + \cdots.
  \end{align*}
  
  \item
  \begin{equation*}
    \log(1+x) = \sum_{j=1}^\infty (-1)^{j-1} \frac{x}{j}
    = x - \frac{x^2}{2} + \frac{x^3}{3} - \frac{x^4}{4}+\cdots,\qquad
    |x|<1
  \end{equation*}
  (note that $x=1$ is actually defined, with $\log 2=1-1/2+1/3+\cdots)$;
  
  \item
  \begin{equation*}
    (1+c)^c = 1 + \sum_{n=1}^\infty\begin{pmatrix}c\\ n\end{pmatrix}x^n,\qquad
    \begin{pmatrix}c\\ n\end{pmatrix} = \frac{c(c-1)\cdots(c-n+1)}{n!}.
  \end{equation*}
  
  \item
  \begin{equation*}
    \frac{1}{1-x} = 1 + \sum_{j=1}^\infty x^n,\qquad |x|<1.
  \end{equation*}
\end{enumerate}

\begin{example}
  Calculate a remainder estimate for $f(x)=\cos x$ near $x=0$.
  
  Since the derivative of $\cos x$ is bounded above by $1$, we have
  \begin{equation*}
    |R_n(x)|\leq 1\cdot\frac{|x|^{n+1}}{(n+1)!}.
  \end{equation*}
  Fixing $x$ and choosing $k\in\mathbb{Z}$ with $k>|x|$ and $n>k+1$, we have
  \begin{equation*}
    \frac{k^n}{k!}=\left(\frac{k^k}{k!}\right)
    \left(\frac{k}{k+1}+\frac{k}{k+2}+\cdots\frac{k}{n-1}\right)
    \left(\frac{k}{n}\right)<\frac{k^{k+1}}{k!}\frac{1}{n}.
  \end{equation*}
  Since $k>|x|\geq0$, we have
  \begin{equation*}
    0\leq\frac{|x|^{n+1}}{(n+1)!}<\frac{k^n}{k!}<\frac{k^{k+1}}{k!}\frac{1}{n},
  \end{equation*}
  so $R_n(x)\to 0$ as $n\to\infty$ by squeezing.
\end{example}

%-------------------------------------------------------------------------------

\section{Taylor approximations and limits}

Let $g$ by a single variable function defined on an interval containing $h$.
Then $g(h)$ is said to be $o(h)$ if $\lim_{h\to 0} g(h)/h=0$.

\begin{example}
  Find the limit of $\lim_{x\to 0} (\sin x)/x$.
  
  $\sin x$ admits the Taylor series $\sin x=x-x^3/3! + o(x^3)$, so
  \begin{equation*}
    \lim_{x\to 0}\frac{\sin x}{x}=\lim_{x\to 0}1-\frac{x^2}{3!}+o(x^2)=1,
  \end{equation*}
  which we know already.
\end{example}
\begin{example}
  Let $f(x)=\sin(\ex^{-x}/(x+1)-1)$. Find the Taylor polynomial of order $3$,
  and hence calculate the limit of $\lim_{x\to 0} (f(x)+2x)/\sinh x^2$.
  
  We note that
  \begin{equation*}
    \ex^{-x}=1-x+\frac{x^2}{2!}-\frac{x^3}{3!}+o(x^3),\qquad
    (1+x)^{-1}=1-x+x^2-x^3+o(x^3).
  \end{equation*}
  Then
  \begin{equation*}
    \frac{\ex^{-x}}{x+1}-1=-2x+\frac{5x^2}{2}-\frac{8x^3}{3}+o(x^3)=y.
  \end{equation*}
  Then since $\sin y=y-y^3/3!+o(y^3)$, we have
  \begin{equation*}
    f(x)=-2x+\frac{5}{2}x^2-\frac{4}{3}x^3+o(x^3).
  \end{equation*}
  
  Now, $\sinh x^2=x^2+o(x^3)$, so
  \begin{align*}
    \lim_{x\to 0}\frac{f(x)-2x}{\sinh x^2} &= \lim_{x\to 0}\frac{5x^2/2-4x^3/3-2x+2x+o(x^3)}{x^2+o(x^3)} \\
    &= \lim_{x\to 0}\frac{5/2-4x^2/3+o(x)}{1+o(x)}=\frac{5}{2}.
  \end{align*}
\end{example}

\begin{example}
  What is the maximum error when using $P_6(x)$ to approximate $\ex^x$,
  $x\in[0,1]$?
  
  With $\ex^x=P_6(x)+R_6(x)$, we note that $\max_{t\in(0,1)}|f^{(7)}(t)|$ occurs
  when $t=1$, i.e., $\ex$, so
  \begin{equation*}
    |R_6(x)|\leq\ex\frac{|x|^7}{7!}<\frac{\ex}{7!},
  \end{equation*}
  and the error is no more than $\ex/7!$.
\end{example}

\begin{example}
  Give an estimate of $\ex^{0.2}$ correct to 3d.p., i.e., find $n$ where
  $R_n(x)<0.0005$.
  
  We use $P_n(x)$ near $0$. We have
  \begin{equation*}
    P_n(x)+R_n(x)=1+x+\cdots\frac{x^n}{n!},
  \end{equation*}
  and we want $|R_n(0.2)|\leq 0.0005$. So
  \begin{equation*}
    |R_n(0.2)|\leq\ex^{0.2}\frac{|0.2|^{n+1}}{(n+1)!}<
    \ex\frac{|0.2|^{n+1}}{(n+1)!}=\frac{\ex}{5^{n+1}(n+1)!}<0.0005,
  \end{equation*}
  and we see $n=3$ satisfies this.
\end{example}

%-------------------------------------------------------------------------------

\section{Differentials}

By linearising to get
\begin{equation*}
  f(x)\approx f(x_0)+f'(x_0)(x-x_0),
\end{equation*}
we see that if we take $x_0=x$, $x=x+h$, we have
\begin{equation*}
  f(x+h)\approx f(x)+f'(x)h,\qquad
  \Delta f = f(x+h)-f(x)\approx f'(x)h.
\end{equation*}
Here, $\Delta f$ is the \Def{increment} of $f$ from $x$ to $x+h$, and
$f'(x)h$ is the \Def{differential} with increment $h$.

\begin{marginfigure}
  \includegraphics{differential}
\end{marginfigure}

$P_1$ is valid near $x$, and here $\mathrm{d}f$ is an approximation to $\Delta
f$. If
\begin{equation*}
  \frac{\Delta f-\mathrm{d}f}{h}\to 0
\end{equation*}
as $h\to 0$, then we can replace $\Delta f$ by $\mathrm{d}f$. For example, with
$f(x)=x^2$, $\Delta f=h^2+2x h$, while $\mathrm{d}f=2x h$, so $(\Delta
f-\mathrm{d}f)/\delta x=\delta x\to 0$.

Let $y=f(x)$ be differentiable. The differentiable $\mathrm{d}x$ is an
independent variable (replacing $h$ for $h\to0$). The differential is then
$\mathrm{d}y = f'(x)\mathrm{d}x$, so
\begin{equation*}
  f'(x)=\frac{\mathrm{d}y}{\mathrm{d}x}.
\end{equation*}
This is known as the \Def{differential form} of the derivative.

%-------------------------------------------------------------------------------

\section{Implicit differentiation}

When an equation is not expressed as $y=f(x)$ with $x$ as the sole variable, it
is an \Def{implicit function}. $y$ depends on $x$, and we differentiate
the whole equation with respect to $x$.
\begin{example}
  $x^2+y^2=1$ is an implicit function. We have
  \begin{equation*}
    \frac{\mathrm{d}(x^2)}{\mathrm{d}x}+\frac{\mathrm{d}(y^2)}{\mathrm{d}x}
    =\frac{\mathrm{d}1}{\mathrm{d}x}\qquad\Rightarrow\qquad
    2x+2y\frac{\mathrm{d}y}{\mathrm{d}x}=0,
  \end{equation*}
  and so $\mathrm{d}y/\mathrm{d}x=-x/y$.
\end{example}

Another method to calculate a differential implicitly is to manufacture $F(x,y)$
such that $F(x,y)=0$ encodes the content of the equation.
\begin{example}
  With the above, we have $F(x,y)=x^2+y^2-1=0$. Since $F(x,y)=0$, by chain rule,
  \begin{equation*}
    0=\frac{\mathrm{d}}{\mathrm{d}x}F(x,y)=
    \ddy{F}{x}+\ddy{F}{y}\frac{\mathrm{d}y}{\mathrm{d}x}=
    2x+2y\frac{\mathrm{d}y}{\mathrm{d}x},
  \end{equation*}
  and we cover the previous result. Note also that
  \begin{equation*}
    \frac{\mathrm{d}y}{\mathrm{d}x}=-\frac{\dy F/\dy x}{\dy F/\dy y}.
  \end{equation*}
\end{example}

%-------------------------------------------------------------------------------

\section{Partial derivatives}

Suppose we have functions of $n$ independent variables. Let $D$ be a set of
$n$-tuple $(x_1,\cdots x_n)$ real valued function on $D$. The function is a rule
which assigns an unique valued $(\omega_1,\cdots\omega_n)=f(x_1,\cdots x_n)$ to
each $n$-tuple on $D$. $\omega$'s here are the dependent variable, and $x$'s are
the independent variables. For $n=2$, we usually use $(x,y)$ or $(x,t)$ if we
are interested in two-dimensional space or space plus time dimension.

The \Def{partial derivative} with are defined analogously as
\begin{equation*}
  \ddy{f}{x}=\lim_{h\to0}\frac{f(x+h,y)-f(x,y)}{h},\qquad
  \ddy{f}{y}=\lim_{h\to0}\frac{f(x,y+h)-f(x,y)}{h},
\end{equation*}
provided the limits exist.
\begin{example}
  Find the partial derivatives of $f(x,y)=1-x+y+3x^2 y$.
  
  From first principles,
  \begin{align*}
    \ddy{f}{x} &= \lim_{h\to0}\frac{1-(x+h)+y+3(x+h)^2 y-(1-x+y+3x^2y)}{h} \\
      &= \lim_{h\to0}\frac{-h + 6xy h + 3y h^2}{h} = 6xy-1.
  \end{align*}
  Or, we treat $y$ as a constant and differentiate with respect to $x$.
  Similarly, we have
  \begin{equation*}
    \ddy{f}{y}=1-3x^2.
  \end{equation*}
\end{example}

Geometrically, we can see that, like the case of one variables, the derivative
generates a tangent line to the surface described by $z=f(x,y)$. Since there are
two tangent lines for each partial derivative, they span a tangent plan to the
surface at $(x_0,y_0)$.

Implicit partial differential is as before.
\begin{example}
  Find $\dy z/\dy x$ of $yz-\log z=x+y$, where $z$ is the independent variable.
  
  We have
  \begin{equation*}
    y\ddy{z}{x}-\frac{1}{z}\ddy{z}{x}=1\qquad\Rightarrow\qquad
    \ddy{z}{x}=\frac{z}{zy-1}.
  \end{equation*}
\end{example}

\begin{theorem}[Chain rule]
  If $\omega=f(x,y)$ has continuous partial derivatives $\dy f/\dy x$ and $\dy
  f/\dy y$, and if $x$ and $y$ are differentiable functions of $t$, then the
  composite function $\omega=f(x(t),y(t))$ is a differentiable function of $t$,
  with
  \begin{equation*}
    \frac{\mathrm{d}\omega}{\mathrm{d}t}=
    \ddy{\omega}{x}\frac{\mathrm{d}x}{\mathrm{d}t}+
    \ddy{\omega}{y}\frac{\mathrm{d}y}{\mathrm{d}t}.
  \end{equation*}
  \qedwhite
\end{theorem}
\begin{example}
  $\omega=x\sin $, $x=\cos t$, $y=t^2$, then
  \begin{equation*}
    \frac{\mathrm{d}\omega}{\mathrm{d}t}=-\sin y\sin t+2xt\cos y
    =-\sin t^2\sin t+2t\cos t\cos t^2.
  \end{equation*}
  We get the same answer if we first substitute $x$ and $y$ to give $\omega=\cos
  t\sin t^2$.
\end{example}

The chain rule generalises to functions of more than two variables. In
particular, if $\omega=f(x,y,z)$, and $x,y,z$ are differentiable functions of
$t$, we have
\begin{equation*}
  \frac{\mathrm{d}\omega}{\mathrm{d}t}=
  \ddy{\omega}{x}\frac{\mathrm{d}x}{\mathrm{d}t}+
  \ddy{\omega}{y}\frac{\mathrm{d}y}{\mathrm{d}t}+
  \ddy{\omega}{z}\frac{\mathrm{d}z}{\mathrm{d}t}
\end{equation*}

\begin{theorem}[Extended chain rule]
  Suppose $\omega=f(x,y,z)$, and $x,y,z$ are differentiable functions of $u$ and
  $t$, then $\omega$ has partial derivatives given by
  \begin{equation*}
    \ddy{\omega}{t}=\ddy{\omega}{x}\ddy{x}{t}+
    \ddy{\omega}{y}\ddy{y}{t}+\ddy{\omega}{z}\ddy{z}{t},
  \end{equation*}
  and similarly for $\dy\omega/\dy u$. \qedwhite
\end{theorem}
\begin{example}
  For $\omega=x+2y+z^2$ with $x=u/v$, $y=u^2+\log v$, $z=2u$,
  \begin{align*}
    \ddy{\omega}{u} &= 1\frac{1}{v}+2(2u)+2z(2)=12u+\frac{1}{v}, \\
    \ddy{\omega}{v} &= 1\left(-\frac{u}{v}\right)+2\frac{1}{v}+2z(0)=
    \frac{2v-u}{v^2}.
  \end{align*}
\end{example}

%-------------------------------------------------------------------------------

\section{Mixed partial derivatives}

A function $f(x,y)$ can have partial derivative with respect to $x$ and $y$ at a
point without being continuous at that point. The situation is different from
the case of single variable functions where existence of derivative implies the
continuity of the function. The limiting procedure in taking partial derivatives
is such that all independent variables but one are kept constant. The issue of
continuity requires that one takes the limit of the type
\begin{equation*}
  (x_1,\cdots x_n)\to(x_{0,1},\cdots x_{0,n}).
\end{equation*}
\begin{example}
  Investigate the continuity of the function
  \begin{equation*}
    f(x,y)=\frac{xy+y^3}{x^2+y^2}
  \end{equation*}
  at $(x,y)=(0,0)$.
  
  We must explore the various ways of limiting in which $(x,y)\to(0,0)$.
  Existence of one cases where the limit is inconsistent proves discontinuity.
  \begin{enumerate}
    \item $x=0$, $y\to0$ gives $\lim_{y\to 0} f(0,y)=\lim_{y\to 0} y^3/y^2=0$;
    \item $x\to0$, $y=0$ gives $\lim_{x\to 0} f(x,0)=\lim_{x\to 0} 0/x^2=0$;
    \item $y=2x$ gives
    \begin{equation*}
      \lim_{x\to 0}f(x,2x)=\lim_{x\to 0}\frac{2x^2+8x^3}{5x^2}=\frac{2}{5}.
    \end{equation*}
  \end{enumerate}
  We have an inconsistency, so $f(x,y)$ is not continuous at $(0,0)$.
\end{example}

For $f(x,y)$, we have the following order two derivatives:
\begin{equation*}
  \ddy{^2 f}{x^2},\qquad \ddy{^2 f}{y^2},\qquad \ddy{^2 f}{x\dy y},\qquad
  \ddy{^2 f}{y\dy x},
\end{equation*}
and so on up to order $n$.
\begin{theorem}
  If $f(x,y)$ and all its derivatives are defined throughout an open region
  containing $(a,b)$ and care continuous at $a$ and $b$, then
  \begin{equation*}
    \ddy{^2 f}{x\dy y}(a,b)=\ddy{^2 f}{y\dy x}(a,b).
  \end{equation*}
  \qedwhite
\end{theorem}

%-------------------------------------------------------------------------------

\section{Differentiability and the gradient}

Recall that differentiability for functions of one independent variable requires
\begin{equation*}
  \lim_{h\to0}\frac{f(x+h)-f(x)}{h}
\end{equation*}
to exist. To generalise it to $n$ variables, let $f(x_1,\cdots x_n)=f(\xb)$ and
$(h_1,\cdots h_n)=\boldsymbol{h}$. Since division of vectors does not exist, we
use the alternative form 
\begin{equation*}
  \Delta f=(\mathrm{d}f)\cdot\boldsymbol{h}+o(\boldsymbol{h})
\end{equation*}
to say that $f$ is differentiable if there exists $\yb$ such that
\begin{equation*}
  f(\xb+\boldsymbol{h})-f(\xb)=\yb\cdot\boldsymbol{h}+o(\boldsymbol{h}).
\end{equation*}
$\yb$ is unique is it exists, and is called the \Def{gradient} of $f$
evaluated at $\xb$, denoted
\begin{equation*}
  \yb=\nabla f(\xb).
\end{equation*}
Note that $\yb$ is a vector.
\begin{example}
  Find the gradient of $f(x,y,z)=2xy-3z^2$.
  \begin{equation*}\begin{aligned}
    f(\xb+\boldsymbol{h}) &=2(x+h_1)(y+h_2)+3(z h_3)^2\\
    &= 2yh_1 + 2x h_2 - 6z h_3 + 2h_1 h_2 - 3h_3^2\\
    &= \boldsymbol{h}\cdots(2y,2x,-6z)+o(\boldsymbol{h}).
  \end{aligned}\end{equation*}
  So
  \begin{equation*}
    \yb=\begin{pmatrix}2y\\ 2x\\ -6z\end{pmatrix}=\nabla f(\xb)
    =\begin{pmatrix}\dy f/\dy x\\ \dy f/\dy y\\ \dy f/\dy z\end{pmatrix}.
  \end{equation*}
\end{example}

\begin{theorem}
  If $f$ has continuous first partial derivatives in region of $\xb$, then $f$
  is differentiable, and $\nabla f(\xb)=(\dy f/\dy x_1,\cdots \dy f/\dy x_n)$ for
  $n$ variables. \qedwhite
\end{theorem}
With this,
\begin{equation*}
  \Delta f=\mathrm{d}f+o(\boldsymbol{h})\equiv f(\xb+\boldsymbol{h})-f(\xb)
  =\nabla f\cdot \boldsymbol{h}+o(\boldsymbol{h}).
\end{equation*}
In two variables it is common to write $\boldsymbol{h} =(\mathrm{d}x,
\mathrm{d}y)$, then
\begin{equation*}
  \mathrm{d}f=\left(\ddy{f}{x},\ddy{f}{y}\right)\cdot(\mathrm{d}x, \mathrm{d}y)
  =\ddy{f}{x}\mathrm{d} x+\ddy{f}{y}\mathrm{d}y,
\end{equation*}
the differential of $f$ at $\xb$. This may be extended to $n$ variables.

%-------------------------------------------------------------------------------

\section{Mean value theorem}

\begin{theorem}[Mean value theorem]
  If $f$ is a differentiable function on $(a,b)$, and if it is continuous on
  $[a,b]$, then there exists $c\in[a,b]$ such that
  \begin{equation*}
    f'(c)=\frac{f(b)-f(a)}{b-a},
  \end{equation*}
  i.e., $f'(c)$ is parallel to the chord between $f(b)$ and $f(a)$ for at least
  one value of $c$. \qedwhite
\end{theorem}

\begin{corollary}[Rolle's theorem]
  Suppose $f(a)=f(b)=0$, and $f$ satisfies the m ean value theorem, then there
  exists $c\in[a,b]$ such that $f'(c)=0$. \qedwhite
\end{corollary}
This is useful in proving how many roots a polynomial has.
\begin{example}
  Find the roots for $f(x)=x^3-4x$ on $[0,3]$.
  
  A polynomial is continuous. We observe that
  \begin{equation*}
    f'(c)=\frac{(3^3-12)-0}{3-0}=5,\qquad 5=3c^3-4\qquad\Rightarrow\qquad
    c^2=\pm\sqrt{3},
  \end{equation*}
  and since $c\in[0,3]$, $c=\sqrt{3}$
\end{example}
\begin{example}
  Show $p(x)=2x^3+5x-1$ has exactly one root.
  
  Since $p(0)=-1$ and $p(1)=6$, by the intermediate value theorem, there exists
  at least one root. Suppose there are two roots with $p(a)=p(b)=0$, then by
  Rolle's theorem, there exist $c\in(a,b)$ such that $p'(c)=0$, but
  $p'(c)=6c^2+5$ with no real $c$, so $c$ is unique and $p(x)$ only has one
  root.
\end{example}

\begin{corollary}[fundamental limits]
  We have, for $\alpha\in\mathbb{R}^+$,
  \begin{equation*}
    \lim_{x\to\infty}\frac{1}{\log x}=0, \qquad 
    \lim_{x\to\infty}\frac{x^\alpha}{\ex^x}=0,\qquad
    \lim_{x\to\infty}\frac{\log x}{x^\alpha}=0.
  \end{equation*}
\end{corollary}
\begin{proof}
  Observe that
  \begin{equation*}
    \frac{\mathrm{d}}{\mathrm{d}x}\log x=\frac{1}{x},
  \end{equation*}
  and since $\log x$ is defined on $(0,\infty)$. By the mean value theorem,
  $\log x$ is an increasing function. For $x>1$, let $n$ be the largest integer
  such that $x>2^n$. Then
  \begin{equation*}
    0>\log x>\log 2^n\qquad\Leftrightarrow\qquad
    0<\frac{1}{\log x}<\frac{1}{n\log 2},
  \end{equation*}
  so $1/\log x\to0$ as $n\to\infty$ by squeezing theorem.
  
  Now, for $f(x)=x/\ex^x$, we have $f'(x)=\ex^{-x}(1-x)$; by the mean value
  theorem, $f(x)$ is decreasing on $[1,\infty)$. Let $n$ be the largest integer
  such that $n<x$, and let $b=(\ex-1)>0$. We obtain
  \begin{equation*}
    0<f(x)<\frac{n}{\ex^n}=\frac{n}{(b+1)^n}<\frac{n}{1+nb+n(n-1)b^2/2}<
    \frac{2}{(n-1)b^2},
  \end{equation*}
  so $f(x)\to0$ as $n\to\infty$ since $b$ is a constant. So then we have
  \begin{enumerate}
    \item $y=\alpha\log x$, giving $x^\alpha=\ex^y$, and so
    \begin{equation*}
      \lim_{x\to\infty}\frac{y/\alpha}{\ex^y}=\frac{1}{\alpha}\lim_{y\to\infty}
      \frac{y}{\ex^y}=0,
    \end{equation*}
    and so we have $\log x/x^\alpha\to0$.
    
    \item let $z=\ex^x$, then $x=\log z$, and
    \begin{equation*}
      \lim_{z\to\infty}\frac{(\log z)^\alpha}{z}=\lim_{z\to\infty}
      \left(\frac{\log z}{z^{1/\alpha}}\right)=0,
    \end{equation*}
    so $x^\alpha/\ex^x\to0$. \qed
  \end{enumerate}
\end{proof}

\begin{corollary}[L'Hopital's rule]
  If $f(x)$ and $g(x)$ are differentiable at $a$, with $g'(a)\neq0$ and
  $f(a)=g(a)=0$, then
  \begin{equation*}
    \lim_{x\to a}=\frac{f(x)}{g(x)}=\frac{f'(a)}{g'(a)}.
  \end{equation*}
\end{corollary}
\begin{proof}
  We have
  \begin{align*}
    \lim_{x\to a}\frac{f(x)}{g(x)} &= \lim_{x\to a}\frac{f(x)-f(a)}{g(x)-g(a)} \\
      &= \lim_{x\to a}{(f(x)-f(a))/(x-a)}{(g(x)-g(a))/(x-a)} \\
      &= \frac{f'(a)}{g'(a)}.
  \end{align*}
  \qed
\end{proof}
\begin{example}
  \begin{equation*}
    \lim_{x\to 1}\frac{2\log(2x-1)}{x^2-1}=\frac{2(2/(2-1))}{2}=2
  \end{equation*}
  by L'Hopital's rule.
\end{example}
This may be re-iterated as appropriate to find the limit.

\begin{corollary}[Increasing/decreasing functions]
  Let $f$ be differentiable on an open interval $I$, and $x_1 < x_2$, $x_1,\
  x_2\in I$. If
  \begin{itemize}
    \item $f'(x)>0$ for all $x\in I$, then $f$ is increasing on $I$;
    \item $f'(x)<0$ for all $x\in I$, then $f$ is decreasing on $I$;
    \item $f'(x)=0$ for all $x\in I$, then $f$ is constant on $I$.
  \end{itemize}
\end{corollary}
\begin{proof}
  Let $x_1, x_2\in I$, $x_1 < x_2$, and $f$ is differentiable. Applying the mean
  valute theorem on $[x_1, x_2]\subseteq I$, there exists $c\in(x_1, x_2)$ such
  that
  \begin{equation*}
    f'(c)=\frac{f(x_2)-f(x_1)}{x_2 - x_1}.
  \end{equation*}
  Then it is easy to see that the above statements hold. \qed
\end{proof}
\begin{example}
  Show $x^3 - 2x^2 - 2x - 3 = 0$ has exactly one root on $[2,5]$.
  
  We have $f(2)=-7$ and $f(5)=2$, so by the intermediate value theorem, there is
  at least one root. Now, $f'(x)=3x^2 - 4x - 2$, and $f'(x)=0$ when
  $x=(2\pm\sqrt{10})/3$, with the positive root less than $2$. Thus for all
  $x\geq 2$, $f'(x)>0$, so it is a strictly increasing function and thus only
  has one root in the interval.
\end{example}

%-------------------------------------------------------------------------------

\section{Extreme values of continuous functions}

The (local) extrema of functions occurs where $f'(x)=0$; when $f''(x)>0$, we
have a minima, whilst for $f''(x)>0$, we have a local maxima.

\begin{example}
  Find and characterise the extrema of $f(x)=(x^3 - 3x^2/3 - 6x +2)/4$ on
  $[-2,\infty)$.
  
  We have
  \begin{equation*}
    f'(x)=\frac{3x^2 -3x-6}{4},\qquad f''(x)=\frac{6x-3}{4},
  \end{equation*}
  so we have a maxima at $x=1$ and minima at $x=2$.
\end{example}

%===============================================================================

\chapter{Integration}

%-------------------------------------------------------------------------------

\section{Definite integrals as limit to Riemann sums}

Let $f$ be a continuous function on $[a,b]$, and $\mathcal{P}_{ab}$ be a
\Def{partition} of $[a,b]$. $\mathcal{P}_{ab}=\{x_0, x_1,\cdots x_n\}$ is
a finite subset of $[a,b]$ which contains $a$ and $b$, and it breaks $[a,b]$
into $n$ subintervals $\Delta x_1,\cdots\Delta x_n$. The definite integral of
$f$ from $a$ to $b$ can be viewed as the limit of \Def{Riemann sums}
\begin{equation*}
  S^*(\mathcal{P}_{ab})=\sum_{i=1}^n f(x_1')\Delta x_1.
\end{equation*}
$S^*(\mathcal{P}_{ab})$ may be negative as $f(x_j)$ is unsigned. Defining the
\Def{norm} of a partition as
\begin{equation*}
  \|\mathcal{P}_{ab}\|=\max\{\Delta x_i,\ i=1,\cdots n\}\|,
\end{equation*}
then
\begin{equation*}
  \int_a^b f(x)\, \mathrm{d}x=\lim_{\|\mathcal{P}_{ab}\|\to 0}
  \sum_{i=1}^n f(x_i')\Delta x_i
\end{equation*}
provided the limit exists.

A Riemann sum represents the signed area of the rectangles and differences
between the signed area bounded by $f(x)$ and the $x$-axis. The integral is
defined is the actual error we make may be made arbitrarily small with shrinking
interval lengths.

%-------------------------------------------------------------------------------

\section{Functions defined by an integral}

\begin{theorem}
  Let
  \begin{equation*}
    F(x)=\int_a^x f(t)\, \mathrm{d}t,
  \end{equation*}
  $f$ continuous on $[a,b]$. If $F(x)$ is defined and continuous on $[a,b]$, and
  differentiable on $(a,b)$, then, for all $x\in(a,b)$,
  \begin{equation*}
    F'(x)=\frac{\mathrm{d}F}{\mathrm{d}x}=f(x).
  \end{equation*}
  \qedwhite
\end{theorem}
\begin{example}
  Find the derivatives of the following $F(x)$.
  \begin{enumerate}
    \item \begin{equation*}
      F(x)=\int_0^x \sin\pi t\, \mathrm{d}t\qquad\Rightarrow\qquad
      F'(x)=\sin x\pi
    \end{equation*}
    
    \item Using $u=x^3$,
    \begin{align*}
      F(x)  &= \int_0^x t\cos t\, \\
      F'(x) &= \frac{\mathrm{d}F}{\mathrm{d}u}\frac{\mathrm{d}u}{\mathrm{d}x} \\
        &=(u\cos u)3x^2 = 3x^5\cos x^3
    \end{align*}
    
    \item For $x>0$,
    \begin{equation*}
      F(x)=\int_x^{x^2}t\, \mathrm{d}t = 
      \int_0^{x^2}t\, \mathrm{d}t-\int_0^x t\, \mathrm{d}t
      \qquad\Rightarrow\qquad
      F'(x)=2x^3-x
    \end{equation*}
    
    \item Let
    \begin{equation*}
      F(x)=\int_0^x f(t)\, \mathrm{d}t,\qquad f(x)=
      \begin{cases} 2-x, & 0\leq x\leq 1,\\ 2+x, & 1<x\leq 3.\end{cases}
    \end{equation*}
    For $x\in(0,1)$,
    \begin{equation*}
      F(x)=\int_0^x (2-t)\, \mathrm{d}t\qquad\Rightarrow\qquad F'(x)=2-x,
    \end{equation*}
    while for $x\in(1,3)$,
    \begin{equation*}
      F(x)=\int_0^1 (2-t)\, \mathrm{d}t + \int_1^x (2+t)\, \mathrm{d}t
      \qquad\Rightarrow\qquad F'(x)=2+x,
    \end{equation*}
    so
    \begin{equation*}
      \begin{cases} 2-x, & 0\leq x\leq 1,\\ 2+x, & 1<x\leq 3.\end{cases}
    \end{equation*}
  \end{enumerate}
\end{example}

%-------------------------------------------------------------------------------

\section{Fundamental theorem of calculus}

Let $f$ be a continuous function on $[a,b]$, then $g$ is called the
\Def{anti-derivative} of $f$ on $[a,b]$ if $g$ is continuous and
$g'(x)=f(x)$ for all $x\in(a,b)$.
\begin{theorem}[Fundamental theorem of calculus]
  Let $f$ be continuous on $[a,b]$. If $g$ is the anti-derivative of $f$, then
  \begin{equation*}
    \int_a^b f(t)\, \mathrm{d}t=g(b)-g(a)=\left[g(t)\right]_a^b.
  \end{equation*}
  \qedwhite
\end{theorem}
Some observations:
\begin{itemize}
  \item If $f(t)>0$ for all $t\in[a,b]$, then the area bounded by $f(t)$ and the
  $x$-axis is $\in_a^b f(t)\, \mathrm{d}t$.
  
  \item If $f(t),g(t)>0$ for all $t\in[a,b]$, $f(t)\geq g(t)$, then the area
  bounded between $f(t)$ and $g(t)$ is
  \begin{equation*}
    \int_a^b f(t)\, \mathrm{d}t-\int_a^b g(t)\, \mathrm{d}t=
    \int_a^b[f(t)-g(t)]\, \mathrm{d}t.
  \end{equation*}
  
  \item Suppose $f$ is continuous on $[-a,a]$, $a>0$, then if $f$ is an odd
  or even function, we have respectively
  \begin{equation*}
    \int_{-a}^a f(t)\, \mathrm{d}t=0,\qquad \int_{-a}^a f(t)\, \mathrm{d}t
    =2\int_0^a f(t)\, \mathrm{d}t.
  \end{equation*}
\end{itemize}

The \Def{Gaussian integral} is deinfed as
\begin{equation*}
  \int_{-\infty}^\infty \ex^{-ax^2}\, \mathrm{d}x=\sqrt{\frac{\pi}{a}}
\end{equation*}
for $a>0$. This intergral shows up quite often in probability theory.

%===============================================================================

\chapter{Ordinary Differential Equations (ODEs)}

Let $x$ be the independent variable, $y(x)$ be dependent on $x$. An
\Def{ordinary differential equation} (ODE) is an equation which relates an
unknown function like $y(x)$ to one or more of its derivatives. The general form
of a first \Def{order} (order here refers to the highest derivative) ODE
is
\begin{equation*}
  Q(x,y)y'+P(x,y)=0\qquad\textnormal{or}\qquad
  Q(x,y)\,\mathrm{d}y + P(x,y)\,\mathrm{d}x=0.
\end{equation*}

%-------------------------------------------------------------------------------

\section{Forms of first order ODEs}

\subsection{Separable ODEs}

These have the form
\begin{equation*}
  Q(y)y'+P(x)=0\qquad\textnormal{or}\qquad
  Q(y)\,\mathrm{d}y + P(x)\,\mathrm{d}x=0.
\end{equation*}
These may be solved by integration.
\begin{example}
  Solve $(y+1)y'=x^2 y - y$.
  
  \begin{equation*}
    \int \left(1+\frac{1}{y}\right)\, \mathrm{d}y=\int (x^2-1)\, \mathrm{d}x
    \qquad\Rightarrow\qquad
    y+\log|y|=\frac{x^3}{3}-x+c.
  \end{equation*}
  We have one free constant in this case because the ODE is of first order.
\end{example}

\subsection{Exact ODEs}

If $\dy P/\dy y=\dy Q/\dy x$, then the ODE is exact, and there exists $F(x,y)$
such that
\begin{equation*}
  Q(x,y)=\ddy{F}{y},\qquad P(x,y)=\ddy{F}{x}.
\end{equation*}
Then, by the chain rule,
\begin{equation*}
  \frac{\mathrm{d}}{\mathrm{d}x}F(x,y)=
  \ddy{F}{x}+\ddy{F}{y}\frac{\mathrm{d}y}{\mathrm{d}x}
  =P(x,y)+Q(x,y)y',
\end{equation*}
so $F(x,y)=\textnormal{constant}$ is a solution of the ODE.
\begin{example}
  Solve $(xy-x^3)+(x^3-y)y'=0$.
  
  Note that $\dy P/\dy x=\dy Q/\dy x=2xy$, so ODE is exact. To find $F(x,y)$, we
  have
  \begin{equation*}
    xy^2-x^3=\ddy{F}{x}\qquad\Rightarrow\qquad
    F=\frac{x^2 y^2}{2}+\frac{x^4}{4}+\phi(y),
  \end{equation*}
  while
  \begin{equation*}
    xy^2-y=\ddy{F}{y}\qquad\Rightarrow\qquad
    F=\frac{x^2 y^2}{2}-\frac{y^2}{2}+\psi(x),
  \end{equation*}
  so
  \begin{equation*}
    F=\frac{x^2 y^2}{2}+\frac{x^4}{4}-\frac{y^2}{2}+k.
  \end{equation*}
  The solution is then
  \begin{equation*}
    \frac{x^2 y^2}{2}+\frac{x^4}{4}-\frac{y^2}{2}=c.
  \end{equation*}
\end{example}

\subsection{Linear ODEs}

If $Q(x,y)=Q(x)$ and $P(x,y)=R(x)y+S(x)$, we introduce the
\Def{integrating factor}
\begin{equation*}
  \mu(x)=\ex^{\int R(x)/Q(x)\, \mathrm{d}x}\qquad\qquad
  \left(\mu'(x)=\frac{R(x)}{Q(x)}\mu(x)\right).
\end{equation*}
The ODE could be written as
\begin{equation*}
  y'+\frac{R(x)}{Q(x)}y=-\frac{S(x)}{Q(x)}\qquad\Rightarrow\qquad
  \mu y'+\mu\frac{R(x)}{Q(x)}y=-\mu\frac{S(x)}{Q(x)}.
\end{equation*}
We notice that
\begin{equation*}
  \mu y'+\frac{R(x)}{Q(x)}\mu y=\left(\mu(x)y\right)',
\end{equation*}
so using this, integrating and rearranging gives
\begin{equation*}
  y(x)=-\frac{1}{\mu}\int\mu\frac{S(x)}{Q(x)}\, \mathrm{d}x.
\end{equation*}
\begin{example}
  Solve $xy'+2y=(\cos x)/x$.
  
  Rearranging to $y'+2y/x=(\cos x)/x^2$, we notice the integrating factor is
  $\exp(\int^x 2/s\, \mathrm{d}s)=x^2$, so
  \begin{align*}
    x^2 y'+2yx = \cos x\qquad &\Rightarrow \qquad x^2y=\int\cos x\, \mathrm{d}x= \sin x + c\\
    &\Rightarrow \qquad y=\frac{\sin x}{x^2}+\frac{c}{x^2}.
  \end{align*}
\end{example}

\subsection{Homogeneous ODEs}

If $Q(\lambda x,\lambda y)=\lambda^n Q(x,y)$ for $n\in\mathbb{R}$ and similarly
for $P$, then the ODE is homogeneous, with
\begin{equation*}
  \lambda^n Q(x,y) y'+\lambda^n P(x,y)=0.
\end{equation*}
To solve this, we let $y=v(x)x$, so that $y'=v'x+v$, and the resulting PDE
becomes separable in $v$ and $x$.
\begin{example}
  Solve $y'=y/x + \sin(y/x)$.
  
  The ODE is homogeneous with $n=0$, so the above substitution yields
  \begin{align*}
    xv'+v=v+\sin v \qquad &\Rightarrow \qquad
    \int\frac{\mathrm{d}v}{\sin v}=\frac{\mathrm{d}x}{x}\\
      &\Rightarrow \qquad \frac{1-\cos(y/x)}{\sin(y/x)}=Ax    
  \end{align*}
  after reverting for $y$ and $x$.
\end{example}

\subsection{ODEs that may be made exact}

For $Q(x,y)y'+P(x,y)=0$, if $\dy Q/\dy x\neq \dy P/\dy y$, we might be able to
find a factor $\mu$ so that $\mu Q(x,y)y'+P(x,y)\mu=0$ with $\dy(\mu P)/\dy
y=\dy(\mu Q)/\dy x$. This is potentially complicated because $\mu=\mu(x,y)$. We
may impose either of the following requirements on $\mu$, that $\mu$ is a
function of $x$ or $y$ alone. Without loss of generality, let $\mu=\mu(x)$, then
the equality of the partial derivatives yields
\begin{equation*}
  \mu\ddy{P}{y}=\frac{\mathrm{d}\mu}{\mathrm{d}x}Q+\ddy{Q}{x}\mu,
\end{equation*}
and separating this gives
\begin{equation*}
  \int\frac{\mathrm{d}\mu}{\mu}=
  \int\frac{1}{Q}\left(\ddy{P}{y}-\ddy{Q}{x}\right)\, \mathrm{d}x,
\end{equation*}
and so the integrating factor in this case is $\mu=\exp(\int^x r(s)\,
\mathrm{d}x)$. (If $r(x)$ is not a function of $x$ alone, then try $r(y)$.)
\begin{example}
  Solve $(2y^2+3x+2/x^2)+(2xy-y/x)y'=0$.
  
  We have that
  \begin{equation*}
    \ddy{Q}{x}=2y+\frac{y}{x^2},\qquad \ddy{P}{y}=4y,
  \end{equation*}
  so ODE is non-exact. However, we notice that
  \begin{equation*}
    r(x)=\frac{1}{Q}\left(\ddy{P}{y}-\ddy{Q}{x}\right)=\frac{1}{x},
  \end{equation*}
  so with the integrating factor $\mu=x$, we have the ODE
  $(2xy^2+3x^2+2/x)+(2x^2 y-y)y'=0$, and that $\dy\tilde{P}/\dy
  y=\dy\tilde{P}/\dy x=4xy$. Thus
  \begin{align*}
    F &= \int^x \tilde{P}\, \mathrm{d}s = x^2 y^2 + x^3 + 2\log|x| + \phi(y), \\
    F &= \int^y \tilde{Q}\, \mathrm{d}s = x^2 y^2 - \frac{y^2}{2} + \psi(x),
  \end{align*}
  and the solution is $x^2 y^2 + x^3 + \log x^2 - y^2 / 2 = c$.
\end{example}

\subsection{Bernouilli ODEs: ODEs that may be made linear}

When $Q(x,y)=1$ and $P(x,y)=R(x)y - S(x) y^n$, we have $y' + R(x)y = S(x)y^n$,
which is nonlinear in $y$. However, setting $z=y^{1-n}$, and so
$z'=(1-n)y^{-n}y'$, the ODE is transformed into
\begin{equation*}
  z'+(1-n)R(x)z = (1-n)S(x).
\end{equation*}
\begin{example}
  Solve $y' + 4y = 3\ex^{2x}y^2$.
  
  Dividing by $y^2$ gives $y'y^{-2} + 4y^{-1} = 3x^{2x}$. Letting $v^{-1}$, we
  notice that $v'=-y^{-2}y'$, so the resulting ODE becomes $v' - 4v = -3x^{2}$,
  which is linear. Using an integrating factor of $\mu=\ex^{-4x}$, we have
  \begin{equation*}
    \ex^{-4x}v = -\int 3\ex^{-2x}\, \mathrm{d}x \qquad\Rightarrow\qquad
    \frac{1}{y} = \frac{3}{2}\ex^{2x}+c\ex^{4x}.
  \end{equation*}
\end{example}

%-------------------------------------------------------------------------------

\section{Second order linear ODEs with constant coefficients}

These are of the form
\begin{equation*}
  f_2 y'' + f_1 y' + f_0 y = f(x).
\end{equation*}
A general solution of the ODE consists of the \Def{complementary function}
and the \Def{particular integral}. The first of these is obtained by
solving the homogeneous equation
\begin{equation*}
  f_2 y'' + f_1 y' + f_0 y = 0.
\end{equation*}
Since this is linear, several complementary functions may be added together to
form another solution (the \Def{principle of superposition}). We solve for
the complementary function by forming the \Def{auxiliary equation} by trying
solutions of the form $y(x)\sim\ex^{\lambda x}$, giving
\begin{equation*}
  f_2 \lambda^2 + f_1 \lambda + f_0 = 0.
\end{equation*}
If the determinant of the resulting quadratic equation is non-zero, we have
exactly two solutions for $\lambda\in\mathbb{C}$, and the complementary solution
takes the form
\begin{equation*}
  y_c(x) = A\ex^{\lambda_1 x} + B\ex^{\lambda_2 x}.
\end{equation*}
Otherwise, there is a degeneracy, but we may restore the degree of freedom by
letting
\begin{equation*}
  y_c(x) = A\ex^{\lambda x} + Bx\ex^{\lambda x}.
\end{equation*}
For the particular integral, we have to make an intelligent guess:
\begin{itemize}
  \item $f(x) = k = \textnormal{constant}$, then try $y_p(x) =
  \textnormal{constant}$;
  \item $f(x) = kx$, try $y_p(x) = bx + c$;
  \item $f(x) = A\ex^{bx}$, try $y_p(x) = \alpha\ex^{bx}$, with $\alpha$ not
  necessarily equal to $A$;
  \item $f(x) = m\cos bx$ or $m\sin bx$, try $y_p(x) = m\cos bx + n\sin bx$.
\end{itemize}
\begin{example}
  Solve for the following ODEs:
  \begin{enumerate}
    \item $y'' + 2y' + 5y = 10\ex^{-2x}$.
    
    The auxiliary equation has roots $\lambda = -1 \pm 2\zi$, and so
    \begin{equation*}
      y_c(x) = A\ex^{(-1 + 2\zi)x} + B\ex^{(-1 - 2\zi)x} =
      \ex^{-x}(C\cos 2x + D\zi\sin 2x),
    \end{equation*}
    where $C = A + B$, $D = A - B$. For the particular integral, we try $y_p =
    m\ex^{-2x}$, which gives $4m - 4m + 5m = 10$, so $m=2$, and the general
    solution is
    \begin{equation*}
      y = \ex^{-x}(A\cos2x + B\zi\sin 2x) + 2\ex^{-2x},
    \end{equation*}
    where the constants may be determined given sufficient number of conditions.
    
    \item $y'' - 4y = 12x$, given that $y(0) = 2$ and $y'(0) = 0$.
    
    The auxiliary equation yields $\lambda = \pm2$, so $y_c(x) = A\ex^{2x} +
    B\ex^{-2x}$. Trying $y_p(x) = ax+b$, this gives $a = -3$ and $b = 0$, so the
    general solution is $y = A\ex^{2x} + B\ex^{-2x} - 3x$. With the initial
    conditions, this implies $A = B = 1$, so the solution is
    \begin{equation*}
      y = \ex^{2x} + \ex^{-2x} - 3x.
    \end{equation*}
  \end{enumerate}
\end{example}
%===============================================================================

\chapter{Fourier series}

Suppose $f(x)$ is given. We consider expanding the function as
\begin{equation*}
  f(x) = \frac{1}{2}a_0 + \sum_{n=1}^\infty(a_n\cos nx) + b_n\sin nx),
\end{equation*}
and the coefficients $a_n$ and $b_n$ depends on the function $f(x)$.
\begin{example}
  Some easy examples:
  \begin{itemize}
    \item $f(x)=1$, $a_0=2$, otherwise zero.
    \item $f(x) = \cos2x$, $a_2=1$, otherwise zero.
    \item $f(x) = \sin^2 x = (1-\cos2x)/2$, $a_0=1$, $a_2=-1/2$, otherwise zero.
  \end{itemize}
\end{example}

To work out the coefficients in general, we first state this orthogonality
result.
\begin{proposition}
  Over the symmetric domain $(-\pi,\pi)$, we have the following orthogonality
  relations:
  \begin{align*}
    \int_{-\pi}^\pi \cos nx\sin mx &= 0,\\
    \int_{-\pi}^\pi \cos nx\cos mx &= \begin{cases} 
      0, & n\neq m,\\ \pi, & n=m=0\neq0,\\ 2\pi, & n=m=0, \end{cases}, \\
    \int_{-\pi}^\pi \sin nx\sin mx &= \begin{cases}
      0, & n\neq m,\\ \pi, & n=m. \end{cases}.
  \end{align*}
\end{proposition}
\begin{proof}
  Expand using compound angle formulae, occasionally use the fact that $\cos(x)$
  and $\sin(x)$ are even and odd on this interval, and that $\sin p\pi=0$ for
  $p\in\mathbb{Z}$. \qed
\end{proof}

The general procedure is then as follows:
\begin{enumerate}
  \item Multiply through by $\cos mx$, $m\in\mathbb{Z}^+$.
  \item Integrate over $(-\pi,\pi)$, which results in
  \begin{equation*}
    \int_{-\pi}^\pi f(x)\cos mx\, \mathrm{d}x = 
    a_m\int_{-\pi}^\pi \cos^2 mx\, \mathrm{d}x = a_m \pi    
  \end{equation*}
  by orthogonality, so
  \begin{equation*}
    a_m = \frac{1}{\pi}\int_{-\pi}^\pi f(x)\cos mx\, \mathrm{d}x,\qquad
    m=0,1,2,\cdots
  \end{equation*}
  \item A similar procedure for the $b_m$ terms gives
  \begin{equation*}
    b_m = \frac{1}{\pi}\int_{-\pi}^\pi f(x)\sin mx\, \mathrm{d}x,\qquad
    m=1,2,3,\cdots
  \end{equation*}
\end{enumerate}
The Fourier series may not converge on a general domain, so instead we define
the Fourier series on $(\pi,\pi)$ and see if series converges, and if so, how
the sum resembles $f(x)$ itself on this interval.

\begin{example}
  Find the Fourier transform $\mathcal{F}(f(x)=x)$ for $x\in(-\pi,\pi)$.
  
  We notice that $f(x)$ is an odd function, so the integrals associated with the
  $a_n$ coefficients are identically zero. On the other hand,
  \begin{align*}
    b_n &= \frac{1}{\pi}\int_{-\pi}^\pi x\sin nx\, \mathrm{d}x = \frac{2}{\pi}\int_{0}^\pi x\sin nx\, \mathrm{d}x = \frac{2}{n}(-1)^{n+1},
  \end{align*}
  after noticing the integrand is an even function, doing an integration by
  parts and that $\cos n\pi=(-1)^n$ for $n\in\mathbb{Z}$. So
  \begin{equation*}
    \mathcal{F}(x)=2\sum_{n=1}^\infty \frac{(-1)^{n+1}}{n}\sin nx,\qquad
    x\in(-\pi,\pi).
  \end{equation*}
  (Using MAPLE or otherwise, keeping up to about the twelfth term of the partial
  sum gives a close approximation to the original function, and the sum appears
  to converge; this may be expected from the \Def{Riemann--Lebesgue
  lemma}.)
\end{example}
\begin{remark}
  We have equality at $x=\pi/2$, and so since
  $f(\pi/2)=\pi/2=\mathcal{F}(f(x=\pi/2))$, we have
  \begin{equation*}
    \frac{\pi}{4} = \sum_{n=1}^\infty \frac{(-1)^{n+1}}{n}\cdot 1 = 
    1-\frac{1}{3}+\frac{1}{5}-\frac{1}{7}\cdots.
  \end{equation*}
  
  Outside of the interval $(-\pi,\pi)$, the Fourier transform gives a
  \Def{periodic extension} of $f(x)$ from $(-\pi,\pi)$. At the edge of the
  intervals however the Fourier transform may have jumps (as in this case, since
  $f(x)$ is not periodic).
\end{remark}

%-------------------------------------------------------------------------------

\section{Convergence of Fourier series}

\begin{theorem}[Dirichlet's theorem]
  For $f(x)$ defined on $(-\pi,pi)$, let $f_E(x)=f(x)$ on $(-\pi,pi)$ and
  $f_E(x)=f_E(x+2n\pi)$ be the periodic extension over the real line. Then
  $\mathcal{F}(f(x))$ converges to $f_E(x)$ whenever $f_E(x)$ is continuous, and
  to the average of its left and right limits at any finite jumps, provided that
  \begin{enumerate}
    \item all jumps are finite, and elsewhere $f'_E(x)$ is finite,
    \item in each $2\pi$ interval, $f_E(x)$ has finite number of discontinuities
    and extrema. \qedwhite
  \end{enumerate}
\end{theorem}

\begin{example}
  For $\mathcal{F}(f(x)=|x|)$ in $(-\pi,pi)$, $f(x)$ is an even function, so all
  the $b_n$ coefficients associated with $\sin(nx)$ are zero. We have that
  \begin{align*}
    a_0 &= \frac{2}{\pi}\int_0^\pi x\, \mathrm{d}x = \pi, \\
    a_n &= \frac{2}{\pi}\int_0^\pi x\cos nx\, \mathrm{d}x \\
      &= \frac{2}{\pi}\int_0^\pi x\left(\frac{\sin nx}{n}\right)\, \mathrm{d}x \\
      &= \frac{2}{\pi}\frac{(-1)^n - 1}{n^2},
  \end{align*}
  so
  \begin{equation*}
    \mathcal{F}(|x|) = \frac{\pi}{2} - 
    \frac{4}{\pi}\sum_{n=0}^\infty \frac{\cos(2n+1)x}{(2n+1)^2}.
  \end{equation*}
  We note that we have pointwise convergence here even though $f_E(x)$ is not
  differentiable at $x=2n\pi$, $n\in\mathbb{Z}$.
\end{example}

%-------------------------------------------------------------------------------

\section{Even and odd functions}

If $f(x)$ is even, then we have
\begin{equation*}
  b_n = 0, a_n = \frac{2}{\pi}\int_0^\pi f(x)\cos nx\, \mathrm{d}x,
\end{equation*}
whilst for $f(x)$ odd, we have
\begin{equation*}
  a_n = 0, b_n = \frac{2}{\pi}\int_0^\pi f(x)\sin nx\, \mathrm{d}x.
\end{equation*}
Some functions are neither even or odd, but we can always write a function as
\begin{equation*}
  f(x)= f_+(x) + f_-(x),\qquad f_\pm = \frac{f(x)\pm f(-x)}{2},
\end{equation*}
which are even and odd functions respectively. The Fourier series of $f_\pm$ are
the cosine and sine parts of $\mathcal{F}(f(x))$ respectively.

%-------------------------------------------------------------------------------

\section{Fourier series on $(-L,L)$}

For $f(x)$ defined on $(-L,L)$, we replace $x$ by $\pi x/L$, and we have
\begin{equation*}
  \mathcal{F}(f(x)) = \frac{a_0}{2} 
  + \sum_{n=1}^\infty \left(a_n\cos\frac{n\pi }{L} 
  + b_n\sin\frac{n\pi x}{L}\right),
\end{equation*}
where
\begin{align*}
  a_0 &= \frac{1}{L}\int_{-L}^L \cos\frac{n\pi }{L}\, \mathrm{d}x, \\
  a_n &= \frac{1}{L}\int_{-L}^L f(x)\cos\frac{n\pi }{L}\, \mathrm{d}x, \\
  b_n &= \frac{1}{L}\int_{-L}^L f(x)\sin\frac{n\pi }{L}\, \mathrm{d}x.
\end{align*}
\begin{example}
  Find $\mathcal{F}(x/2)$ for $x\in(0,2)$.
  
  Since $0<x<2$, we rescale this to $-\pi<\pi(x-1)<\pi$, so letting
  $y=\pi(x-1)$, this gives $x=1+y/\pi$, and so we consider instead the Fourier
  transform of $g(y) = (1+y/\pi)/2$, $y\in(-\pi,\pi)$. We have
  \begin{align*}
    \mathcal{F}\left(\frac{1}{2}+\frac{y}{2\pi}\right) &= \frac{1}{2\pi}\mathcal{F}(y) + \frac{1}{2} \\
      &= \frac{2}{2\pi}\sum_{n=1}^\infty (-1)^{n+1}\frac{\sin ny}{n} + \frac{1}{2} \\
      &= \frac{1}{\pi}\sum_{n=1}^\infty (-1)^{n+1} \frac{\sin n\pi(x-1)}{n} + 
    \frac{1}{2}.
  \end{align*}
  Using double angle formulae, we have
  \begin{equation*}
    \mathcal{F}\left(\frac{x}{2}\right) = \frac{1}{2} 
    - \frac{1}{\pi}\sum_{n=1}^\infty \frac{\sin n\pi x}{n},\qquad x\in(0,2).
  \end{equation*}
\end{example}

%-------------------------------------------------------------------------------

\section{Half range series}

Suppose we want to extend the Fourier series from $x\in(0,L)$ to $x\in(-L,L)$ or
something similar.
\begin{example}
  Using the above example, we might consider the \Def{even extension},
  with $f(x)=|x|/2$, $x\in(-2,2)$. Then
  \begin{equation*}
    \mathcal{F}_c(f) = \frac{a_0}{2}+ \sum_{n=1}^\infty a_n\cos\frac{n\pi x}{2}.
  \end{equation*}
  We have
  \begin{align*}
    a_0 &= 1, \\
    a_n &= \frac{1}{2} \int_{-2}^2 \frac{x}{2}\cos\frac{n\pi x}{2}\, \mathrm{d}x \\
      &= \left[\frac{x}{n\pi}\sin\frac{n\pi x}{2}\right]_0^2 - \frac{1}{n\pi}\int_0^2 \sin\frac{n\pi x}{2}\, \mathrm{d}x \\
      &= \frac{2}{(n\pi)^2}((-1)^n-1),
  \end{align*}
  so $a_0=1$ and $a_n=-4/(n\pi)^2$ only when $n$ is odd.
\end{example}
\begin{example}
  If instead we consider the \Def{odd extension} with $f(x)=x/2$,
  $x\in(-2,2)$, we should have
  \begin{equation*}
    \mathcal{F}_s(f) = \sum_{n=1}^\infty b_n\sin\frac{n\pi x}{2}.
  \end{equation*}
  So then
  \begin{align*}
    b_n &= \frac{1}{2} \int_{-2}^2 \frac{x}{2}\sin\frac{n\pi x}{2}\, \mathrm{d}x \\
      &= \left[-\frac{x}{n\pi}\cos\frac{n\pi x}{2}\right]_0^2 + \frac{1}{n\pi}\int_0^2 \cos\frac{n\pi x}{2}\, \mathrm{d}x\\
      &= \frac{2}{n\pi}(-1)^{n+1} + 0.
  \end{align*}
\end{example}

%-------------------------------------------------------------------------------

\section{An application to PDEs}

Suppose we have the diffusion equation on the interval $(0,L)$ subject to
initial and boundary conditions
\begin{equation*}
  \ddy{u}{t} = \kappa\ddy{^2 u}{x^2},\qquad u(),t) = u(L,t) = 0,\qquad
  u(x,0) = f(x).
\end{equation*}
We assume we have solutions of the form $u(x,t) = X(x) T(t)$. With this, we have
\begin{equation*}
  \kappa X T' = \kappa T X'' \qquad\Leftrightarrow\qquad
  \kappa\frac{X''}{X} = \frac{T'}{T} = -\alpha^2,
\end{equation*}
where $-\alpha^2$ is the \Def{constant of separation} due to $x$ and $t$
being independent of each other; $\alpha$ is chosen to be positive here so that
the boundary conditions are satisfied. We now have two ODEs to solve, i.e.,
\begin{equation*}
  X'' + \alpha^2 X = 0,\qquad T' + \kappa\alpha^2 T = 0,
\end{equation*}
related by $-\alpha^2$ that is to be determined.

For the spatial equation, we have $X(x) = A\cos\alpha x + B\sin\alpha x$. By the
boundary conditions, $A=0$, and we require $\sin\alpha L = 0$, i.e., $\alpha =
n\pi/ L$, so
\begin{equation*}
  X = X_n(x) = B_n \sin\frac{n\pi x}{L}.
\end{equation*}
The idea here is that, since the equations are linear, solutions may be added
accordingly to give the full solution (\Def{principle of superposition}).
For the time equation, we have $T(t) = C\ex^{-\kappa\alpha^2 t}$, so
\begin{equation*}
  T = T_n(t) = C_n \ex^{-\kappa(n\pi/L)^2 t},
\end{equation*}
and so the general form of the solution is
\begin{equation*}
  u(x,t) = D_n \sin\frac{n\pi x}{L} \ex^{-\kappa(n\pi/L)^2 t},\qquad
  n\in\mathbb{N}.
\end{equation*}
$D_n$ is determined by the initial condition. since
\begin{equation*}
  u(x,0) = f(x) = \sum_{n=0}^\infty D_n\sin\frac{n\pi x}{L},
\end{equation*}
we have
\begin{equation*}
  D_n = \frac{2}{\pi} \int_0^L f(x)\sin\frac{n\pi x}{L}\, \mathrm{d}x.
\end{equation*}

\begin{example}
  Suppose a bar of length $L$ has the middle half portion heated to 100$^\circ$,
  i.e.,
  \begin{equation*}
    f(x) = \begin{cases}100, & L/4 < x < 3L/4,\\ 0, \textnormal{otherwise}.
    \end{cases}
  \end{equation*}
  Thus
  \begin{align*}
    D_n &= \frac{2}{\pi} \int_{L/4}^{3L/4} 100\sin\frac{n\pi x}{L}\, \mathrm{d}x \\
      &= -\frac{200L}{n\pi^2}\left(\cos\frac{3\pi}{4} - \cos\frac{n\pi}{4}\right)\\
      ^= \frac{100 L}{n\pi^2}\sin\frac{n\pi}{2}\sin{n\pi}{4},
  \end{align*}
  using
  \begin{equation*}
    \cos A - \cos B = \frac{1}{2}\sin\frac{A+B}{2}\sin\frac{A-B}{2}.
  \end{equation*}
\end{example}

%===============================================================================

\chapter{Integration in higer dimensions}

We extend definitions of the one dimensional integral in the obvious way, by
dividing our domain of integration $D$ into partition $R_{ij}$ and take the sum
of it, i.e., $\sum f_{ij} \times \textnormal{area}(R_{ij})$.

Given a set of indices $A = \{1\leq i\leq m,\ 1\leq j\leq n\}$ and for each pair
$(i,j)$ there is a number $a_{ij}$, their sum is
\begin{equation*}
  \sum_{i=1}^m \sum_{j=1}^n a_{ij}.
\end{equation*}
Since addition is associative and commutative, we know that
\begin{equation*}
  \sum_{i=1}^m \sum_{j=1}^n a_{ij} 
  = \sum_{i=1}^m \left(\sum_{j=1}^n a_{ij}\right)
  = \sum_{j=1}^n \left( \sum_{i=1}^m a_{ij}\right).
\end{equation*}
Addition and multiplication are distributive, i.e., $a(b+c) = ab + ac$, so that
\begin{equation*}
  \sum_{i=1}^m \sum_{j=1}^n (\alpha a_{ij} + \beta b_{ij}) = 
  \alpha \sum_{i=1}^m \sum_{j=1}^n a_{ij} 
  + \beta \sum_{i=1}^m \sum_{j=1}^n b_{ij}.
\end{equation*}
If $a_{ij} = b_i c_j$ for all $(i,j)$, then
\begin{equation*}
  \sum_{j=1}^n a_{ij} = \sum_{j=1}^n b_i c_j = b_i \sum_{j=1}^n c_j,
\end{equation*}
so
\begin{equation*}
  \sum_{i=1}^m \sum_{j=1}^n a_{ij} 
  = \sum_{i=1}^m \left(b_i \sum_{j=1}^n c_j\right)
  = \left(\sum_{i=1}^m b_i\right) \left(\sum_{j=1}^n c_j\right).
\end{equation*}
Sometimes the summation depends on how we add the terms together (e.g., summing
the sequence $(-1)^n n^{-1}$). We may re-arrange sums if, however, $a_i\geq0$
for all $i$, or that the sequence $\{a_i\}$ is absolutely convergent.

%-------------------------------------------------------------------------------

\section{Integration in $\mathbb{R}^2$}

Consider a rectangle $R = [a_0,a_1]\times[b_0,b_1]$. If we partition in the $x$
and $y$ direction as
\begin{equation*}
  \mathcal{P}_x = [x_0,\cdots x_m],\qquad \mathcal{P}_y = [y_0,\cdots y_n],
\end{equation*}
then
\begin{equation*}
  R_{ij} = \{(x,y)\in\mathbb{R}^2\ |\ x_{i-1} \leq x \leq x_i,\ 
  y_{j-1} \leq y \leq y_j\}.
\end{equation*}
Suppose $f$ is continuous on $\mathbb{R}^2$, we let
\begin{equation*}
  m_{ij} = \min_{R_{ij}} f(x,y),\qquad M_{ij} = \max_{R_{ij}} f(x,y).
\end{equation*}
Forming the sums
\begin{equation*}
  L_{f}(\mathcal{P}) = \sum_{i=1}^m \sum_{j=1}^n m_{ij}\Delta x_i \Delta y_j,
  \qquad
  U_{f}(\mathcal{P}) = \sum_{i=1}^m \sum_{j=1}^n M_{ij}\Delta x_i \Delta y_j,
\end{equation*}
with $\Delta x_i = x_i - x_{i-1}$ and $\Delta y_j = y_j - y_{j-1}$, we can prove
that there is a unique $I$ such that
\begin{equation*}
  L_{f}(\mathcal{P}) \leq I \leq U_{f}(\mathcal{P})
\end{equation*}
for every partition $\mathcal{P}$.

The integral of $f$ covering a domain $D\in\mathbb{R}^2$ is defnoted by
\begin{equation*}
  \iint_D f\, \mathrm{d}A \qquad\textnormal{or}\qquad
  \iint_D f(x,y)\, \mathrm{d}x\,\mathrm{d}y.
\end{equation*}
The second form is useful as it makes the independent variables explicit.

\subsection{Repeated integration}

Looking at $L_f(\mathcal{P})$, fixing $i$ and summing over $j$ gives
\begin{equation*}
  \sum_{j=1}^n (m_{ij} \Delta y_j) x_i.
\end{equation*}
At a specific $x$, these upper and lower $j$ sums are close to
\begin{equation*}
  \int_{b_0}^{b_1} f(x,y)\, \mathrm{d}y.
\end{equation*}
Denoting this by $F(x)$, we have the double sum being equal to
\begin{equation*}
  \int_{a_0}^{a_1} F(x)\, \mathrm{d}x,
\end{equation*}
so that
\begin{equation*}
  \iint_R f\, \mathrm{d}A = \int_{a_0}^{a_1} \left( \int_{b_0}^{b_1}
  f(x,y)\, \mathrm{d}y \right) \mathrm{d}x.
\end{equation*}
It turns out that, usually, the ordering of integrals do not matter, so
\begin{equation*}
  \iint_R f\, \mathrm{d}A = \int_{b_0}^{b_1} \left( \int_{a_0}^{a_1}
  f(x,y)\, \mathrm{d}x \right) \mathrm{d}y.
\end{equation*}
\begin{example}
  Compute the integrals of
  \begin{enumerate}
    \item $f(x,y) = (1+x+y)^{-1}$ over $R = [0,1]^2$.
    \begin{align*}
      I &= \int_0^1 \left( \int_0^1 (1+x+y)^{-1}\, \mathrm{d}y \right)\mathrm{d}x \\
        &= \int_0^1 (\log 2 + x - \log 1 + x)\, \mathrm{d}x \\
        &= 3\log 3 - 4\log 2.
    \end{align*}
    
    \item $f(x,y) = xy^2$ over $R = [a,b]\times[0,c]$.
    \begin{align*}
      \int_a^b \left(\int_0^c xy^2\, \mathrm{d}y\right)\mathrm{d}x & = \int_a^b \left(x\int_0^c y^2\, \mathrm{d}y\right)\mathrm{d}x \\
        &= \left(\int_0^c y^2\, \mathrm{d}y\right) \left(\int_a^b x\, \mathrm{d}x\right) \\
        &= \frac{c^3}{3}\left(\frac{b^2 - a^2}{2}\right).
    \end{align*}
  \end{enumerate}
\end{example}

\subsection{More complex domains}

We can evaluate integrals over $D$ of types where
\begin{itemize}
  \item $D=\{a\leq x\leq b, \phi_1(x) \leq y \leq \phi_2(x)\}$, i.e., domain
  bounded by two curves varying in $x$,
  \item $D=\{\psi_1(y) \leq x \leq \psi_2(y), a\leq y\leq b\}$, i.e., domain
  bounded by two curves varying in $y$.
\end{itemize}
We have
\begin{equation*}
  I = \int_a^b \left(\int_{\phi_2(x)}^{\phi_1(x)} f(x,y)\, \mathrm{d}y\right)
  \mathrm{d}x \qquad\textnormal{or}\qquad
  I = \int_a^b \left(\int_{\psi_2(y)}^{\psi_1(y)} f(x,y)\, \mathrm{d}x\right)
  \mathrm{d}y.
\end{equation*}
We also have the following properties of integrals:
\begin{itemize}
  \item \Def{Linearity}: $\iint_D (\alpha f + \beta g)\, \mathrm{d}A =
  \alpha \iint_D f\, \mathrm{d}A + \beta \iint_D g\, \mathrm{d}A$.
  
  \item \Def{Order}: For $f \geq 0$ on $D$, $f(x,y) \geq 0$ FOR $(x,y)\in
  D$ implies that $\iint_D f\, \mathrm{d}A \geq 0$, and if $f(x,y) \geq g(x,y)$
  for $(x,y) \in D$, $\iint_D f\, \mathrm{d}A \geq \iint_D g\, \mathrm{d}A$.
  
  \item \Def{Additivity}: If $D$ is partition into disjoint regions $D_i$,
  then $\iint_D f\, \mathrm{d}A = \sum_{i=1}^n \iint_{D_i} f\, \mathrm{d}A$.
  
  \item \Def{Mean value}: there exists some $(x_0, y_0) \in D$ such that
  $\iint_D f\, \mathrm{d}A = f(x_0,y_0)\times\textnormal{area of }D$.
\end{itemize}

\begin{example}
  Integrate the following over the appropriate domains:  
  
  \begin{enumerate}
    \item $f(x,y) = (x\log(y+a))/(y-a)^2$ over $D$ where $D$ is the region
    between the circle of radius $a$ centred on $(0,a)$ and the $x$-axis.
  
    We integrate $x$ first, and to do that, we work out $\psi_{1,2}(y)$. The
    circle has formula $x^2 + (y-a)^2 = a^2$, so
    \begin{equation*}
      \psi_2(y) = a,\qquad \psi_1(y) = \sqrt{a^2 - (y-a)^2} = \sqrt{2ay-y^2}.
    \end{equation*}
    Thus
    \begin{align*}
      I = \iint_D f\, \mathrm{d}A &= \int_0^a \int_{\sqrt{2ay-y^2}}^a \frac{x\log(y+a)}{(y-a)^2}\, \mathrm{d}x\,\mathrm{d}y \\
        &= \int_0^a \left[\frac{x^2}{2}\right]_{\sqrt{2ay-y^2}}^a \frac{\log(y+a)}{(y-a)^2}\, \mathrm{d}y\\
        &= \frac{1}{2}\int_0^a \log(y+a)\, \mathrm{d}y.
    \end{align*}
    So this results in $I = (a/2)(\log4a -1)$.
    
    \item $f(x,y) = xy$ over the triangle with vertices at $\{(0,0), (1,1),
    (2,0)\}$.
    
    Doing this over $y$ first, we have
    \begin{equation*}
      \phi_1(x) = 0,\qquad \phi_2(x) = \begin{cases}x,& x\leq 1,\\ 2-x,& x>1,
      \end{cases}.
    \end{equation*}
    By additivity, with have
    \begin{equation*}
      \iint_D f\, \mathrm{d}A = \int_0^1 \int_0^x xy\, \mathrm{d}y\, \mathrm{d}x
      + \int_1^2 \int_0^{2-x} xy\, \mathrm{d}y\,\mathrm{d}x
      = \cdots = \frac{1}{3}.
    \end{equation*}
    \end{enumerate}
\end{example}

%-------------------------------------------------------------------------------

\section{Change of variables}

One notable co-ordinates system is the polar co-ordinate system where
\begin{equation*}
  x = r\cos\theta,\qquad y = r\sin\theta.
\end{equation*}
A change of variables from Cartesian to polar co-ordinates has a correction
\begin{equation*}
  \iint_D f\, \mathrm{d}A = \iint_D f\, r\, \mathrm{d}r\mathrm{d}\theta.
\end{equation*}
\begin{example}
  Compute the integral of $f(x,y)$ over the sector with vertices at $\{(0,0),
  (\sqrt{2},0), (1,1)\}$.
  
  We see that our domain satisfies $\theta \in [0, \pi/4]$ and $r \in [0,
  \sec\theta]$. So the integral is
  \begin{align*}
    \iint xy\, \mathrm{d}A &= \int_0^{\pi/4} \int_0^{\sec\theta} r^2\cos\theta\sin\theta\, r\, \mathrm{d}r\, \mathrm{d}\theta \\
      &= \int_0^{\pi/4} \left[\frac{r^4}{4} \sin\theta \cos\theta \right]_0^{\sec\theta}\, \mathrm{d}\theta \\
      &= \frac{1}{4} \int_0^{\pi/4} \frac{\sin\theta}{\cos^3\theta}\, \mathrm{d}\theta.
  \end{align*}
  A substitution with $y=\cos\theta$ results in $I = 1/8$.
\end{example}

\subsection{The Jacobian}

When we substitute for $x,y$ using $u,x$ we have $x=x(u,v)$ and $y=y(u,v)$.
Consider a box $[u, u + \Delta u]\times[v, v + \Delta v]$ and a point $(u +
\delta u, v + \delta v)$ in this box. Using Taylor's theorem,
\begin{align*}
  \delta x &= x(u+\delta u, v+\delta v) - x(u,v) = \ddy{x}{u}\delta u + \ddy{x}{v}\delta v + \cdots, \\
  \delta y &= y(u+\delta u, v+\delta v) - y(u,v) = \ddy{y}{u}\delta u + \ddy{y}{v}\delta v + \cdots.
\end{align*}
At leading order, we have
\begin{equation*}
  \begin{pmatrix}\delta x\\ \delta y\end{pmatrix} = 
  \begin{pmatrix}
    \dy x/\dy u & \dy x/\dy v\\ \dy y/\dy u & \dy y/\dy v
  \end{pmatrix}
  \begin{pmatrix}\delta u\\ \delta v\end{pmatrix}
  = \boldsymbol{\mathsf{J}}\begin{pmatrix}\delta u\\ \delta v\end{pmatrix}.
\end{equation*}
In $uv$-space, our box has area $\Delta u\Delta v$, and its image in $xy$ plane
has area
\begin{equation*}
  \Delta u\Delta v
  \left|\begin{matrix}
    \dy x/\dy u & \dy x/\dy v\\ \dy y/\dy u & \dy y/\dy v
  \end{matrix}\right| = |\boldsymbol{\mathsf{J}}|\Delta u\Delta v.
\end{equation*}
Here we replace the area element $\mathrm{d}x\, \mathrm{d}y$ with
$|\boldsymbol{\mathsf{J}}|\, \mathrm{d}u\, \mathrm{d}v$ when we substitute $u,v$
for $x,y$. For example, for polar co-ordinates, we have
\begin{equation*}
  \left|\begin{matrix}
    \cos\theta & -r\sin\theta\\ \sin\theta & r\cos\theta
  \end{matrix}\right| = r\cos^2\theta + r\sin^2\theta = r.
\end{equation*}

\begin{example}
  Find the area between the curve $x^2 - 4xy + 4y^2 - (2x+y) = 1$ and the line
  $y=2/5$.
  
  Note that we have $x^2 - 4xy + 4y^2 - (2x+y) = (x-2y)^2 - (2x+y) = 1$, and so
  letting $u=x-2y$ and $v=2x+y$, we the curve becomes $u^2-v=1$. On the other
  hand, $y=(x-2u)/5 = 2/5$, so the line becomes $v=2+2u$. We also have
  $|\boldsymbol{\mathsf{J}}|=1/5$. The two curves meet where $u^2-1 = 2u+2$, so
  $u=-1,3$, thus
  \begin{equation*}
    A = \int_{-1}^3 \int_{u^2-1}^{2u+2} \frac{1}{5}\, \mathrm{d}v\, \mathrm{d}u
    = \frac{1}{5} =\int_{-1}^3 (3+2u-u^2)\, \mathrm{d}u = \frac{32}{15}.
  \end{equation*}
\end{example}

%-------------------------------------------------------------------------------

\section{Triple integrals}

Considering a region $I$ in $\mathrm{R}^3$ which contains material of density
$f(x,y,z)$ at $(x,y,z) \in D$, then $\iiint_D \mathrm{d}V$ is the volume of $D$,
$M= \iiint_{D} f\, \mathrm{d}V$ is the mass of material in $D$, and $(\iint_x
f(x,y,z)\, \mathrm{d}V)/M$ is the centre of mass $x_0$.

\begin{example}
  Consider a square base pyramid with base $[-1,1]^2$ and apex at $(0,0,1)$.
  Find the volume and centre of mass assuming uniform density.
  
  We integrate over $z$ last. At height $z$, the $(x,y)$ cross-section is the
  square $[z-1, 1-z]^2$, so, using the substitution $u=1-z$,
  \begin{equation*}
    V = \int_01^1 \int_{z-1}^{1-z} \int_{z-1}^{1-z}
    \mathrm{d}x\,\mathrm{d}y\,\mathrm{d}z
    =\int_0^1 4(1-z)^2\, \mathrm{d}z = -4\int_1^0 u^2\,\mathrm{d}u=\frac{4}{3}.
  \end{equation*}
  
  Centre of mass is found by integrating in $x$, $y$ and $z$ respectively:
  \begin{equation*}
    \int_0^1 \int_{-(1-z)}^{1-z} \int_{-(1-z)}^{1-z} x
    \mathrm{d}x\,\mathrm{d}y\,\mathrm{d}z = 0
  \end{equation*}
  as $x$ is an odd function, and similar for $y$. We know that since we have
  uniform density and the volume is $4/3$, the mass $M=4/3$, so
  \begin{align*}
    z_0 &= \frac{3}{4}\int_0^1 z\left(\iint_{[z-1,1-z]^2} 1\,\mathrm{d}A\right)\; \mathrm{d}z \\
      &= \frac{3}{4} \int_0^1 z4(1-x)^2\, \mathrm{d}z \\
      &= -3 \int_0^1 (1-u)u^2\, \mathrm{d}u \\
      &= \frac{1}{4},
  \end{align*}
  so the centre of mass is located at $(0,0,1/4)$.
\end{example}

Changing variables is done as before, and this time we have
\begin{equation*}
  \boldsymbol{\mathsf{J}} = \begin{pmatrix}
    \dy x/\dy u & \dy x/\dy v & \dy x/\dy w \\
    \dy y/\dy u & \dy y/\dy v & \dy y/\dy w \\
    \dy z/\dy u & \dy z/\dy v & \dy z/\dy w
  \end{pmatrix}.
\end{equation*}
\begin{example}
  In spherical polar co-ordinates, we have:
  \begin{itemize}
    \item $r$ the distance from $(0,0,0)$ to $(x,y,z)$;
    \item $\theta$ the angle in $(x,y)$ plane from $x$-axis, $\theta \i n[0,
    2\pi)$;
    \item $\phi$ the polar angle fro $z$, $\phi\in [0,\pi]$.
  \end{itemize}
  The change of co-ordinates is
  \begin{equation*}
    x = r\sin\phi\cos\theta,\qquad y = r\sin\phi\sin\theta,\qquad z = r\cos\phi.
  \end{equation*}
  For the integral of $\iiint_D (x^2 + y^2 + z^2)^{-1}\, \mathrm{d}V$ over the
  region bounded by the sphere $x^2 + y^2 + z^2 = 1$ and $x^2 + y^2 + z^2 = 9$,
  we have $|\boldsymbol{\mathsf{J}}|=r^2\sin\phi$, so
  \begin{align*}
    I &= \int_1^3 \int_0^{2\pi} \int_0^\pi \frac{1}{r^2} r^2\sin\phi\, \mathrm{d}\phi\, \mathrm{d}\theta\, \mathrm{d}r\\
      &= -\int_1^3 \int_0^{2\pi} [\cos\phi]_0^\pi\,\mathrm{d}\theta\,\mathrm{d}r \\
      &= \int_1^3 4\pi\,\mathrm{d}r = 8\pi.
  \end{align*}
\end{example}

%===============================================================================

%%%%%%%%%%%%%%%%%%%%%%%%%%%%%%%%%%%%%%%%%

% r.5 contents
%\tableofcontents

%\listoffigures

%\listoftables

% r.7 dedication
%\cleardoublepage
%~\vfill
%\begin{doublespace}
%\noindent\fontsize{18}{22}\selectfont\itshape
%\nohyphenation
%Dedicated to those who appreciate \LaTeX{} 
%and the work of \mbox{Edward R.~Tufte} 
%and \mbox{Donald E.~Knuth}.
%\end{doublespace}
%\vfill

% r.9 introduction
% \cleardoublepage

%%%%%%%%%%%%%%%%%%%%%%%%%%%%%%%%%%%%%%%%%
% actual useful crap (normal chapters)
\mainmatter

%\part{Basics (?)}


%\backmatter

%\bibliography{refs}
\bibliographystyle{plainnat}

%\printindex

\end{document}

