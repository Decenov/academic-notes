%% ****** Start of file aiptemplate.tex ****** %
%%
%%   This file is part of the files in the distribution of AIP substyles for REVTeX4.
%%   Version 4.1 of 9 October 2009.
%%

%
% This is a template for producing documents for use with 
% % the REVTEX 4.1 document class and the AIP substyles.
% 
% Copy this file to another name and then work on that file.
% That way, you always have this original template file to use.

%\documentclass[aip,pof,graphicx,11pt]{revtex4-1}
\documentclass[10pt,notitlepage]{revtex4-1}
\usepackage{amssymb,amsbsy,times,fancyhdr,color}
\usepackage{amsmath}
\usepackage{mathrsfs,amsfonts}
%\usepackage[dvips]{epsfig,graphics,lscape}
%\usepackage{natbib}
\usepackage{latexsym,afterpage}
\usepackage{graphicx,subfigure,multirow}
\usepackage{color}
\newcommand{\dfd}{\mathrm{d}}
\newcommand{\imi}{\mathrm{i}}

%\draft % marks overfull lines with a black rule on the right

% extra macros; disable accordingly
\linespread{1}
%\setlength{\parindent}{0pt}

% ===========================================================================

\usepackage{psfrag}
%\usepackage{showlabels}
\usepackage{color}
\usepackage{xcolor}

% more pastel ones
\xdefinecolor{pastel-red}{rgb}{0.77,0.31,0.32}
\xdefinecolor{pastel-green}{rgb}{0.33,0.66,0.41}
\definecolor{pastel-blue}{rgb}{0.30,0.45,0.69} % crayola blue
\definecolor{gray}{rgb}{0.2,0.2,0.2} % dark gray

\xdefinecolor{orange}{rgb}{1,0.45,0}
\xdefinecolor{green}{rgb}{0,0.35,0}
\definecolor{blue}{rgb}{0.12,0.46,0.99} % crayola blue
\definecolor{gray}{rgb}{0.2,0.2,0.2} % dark gray

\xdefinecolor{cerulean}{rgb}{0.01,0.48,0.65}
\xdefinecolor{ust-blue}{rgb}{0,0.20,0.47}
\xdefinecolor{ust-mustard}{rgb}{0.67,0.52,0.13}

% ===========================================================================

\newtheorem{theorem}{Theorem}[section]
\newtheorem{lemma}[theorem]{Lemma}
\newtheorem{proposition}[theorem]{Proposition}
\newtheorem{corollary}[theorem]{Corollary}

\newenvironment{proof}[1][Proof]{\begin{trivlist}
\item[\hskip \labelsep {\bfseries #1}]}{\end{trivlist}}
\newenvironment{definition}[1][Definition]{\begin{trivlist}
\item[\hskip \labelsep {\bfseries #1}]}{\end{trivlist}}
\newenvironment{example}[1][Example]{\begin{trivlist}
\item[\hskip \labelsep {\bfseries #1}]}{\end{trivlist}}
\newenvironment{remark}[1][Remark]{\begin{trivlist}
\item[\hskip \labelsep {\bfseries #1}]}{\end{trivlist}}

\newcommand{\qed}{\nobreak \ifvmode \relax \else
      \ifdim\lastskip<1.5em \hskip-\lastskip
      \hskip1.5em plus0em minus0.5em \fi \nobreak
      \vrule height0.75em width0.5em depth0.25em\fi}

% ===========================================================================

% For multiletter symbols

\newcommand\Def[1]{{\color{green}#1}} % cf plain TeX's \Re and Reynolds number

\newcommand\Real{\mbox{Re}} % cf plain TeX's \Re and Reynolds number
\newcommand\Imag{\mbox{Im}} % cf plain TeX's \Im
\newcommand\Rey{\mbox{\textit{Re}}}  % Reynolds number
\newcommand\Pran{\mbox{\textit{Pr}}} % Prandtl number, cf TeX's \Pr product
\newcommand\Pen{\mbox{\textit{Pe}}}  % Peclet number
\newcommand\Ai{\mbox{Ai}}            % Airy function
\newcommand\Bi{\mbox{Bi}}            % Airy function

% Our general macros:

\newcommand{\dy}{\partial}
\newcommand{\ddy}[2]{\frac{\dy#1}{\dy#2}}

\newcommand{\ex}{\mathrm{e}}

\newcommand{\zi}{{\rm i}}

\newcommand{\grad}{\nabla}

\newcommand{\sech}{\mathrm{sech}}

% Notational macros for this ms only: ---------------------------------------

\newcommand{\eb}{\boldsymbol{e}}
\newcommand{\xb}{\boldsymbol{x}}
\newcommand{\ub}{\boldsymbol{u}}
\newcommand{\vb}{\boldsymbol{v}}
\newcommand{\wb}{\boldsymbol{w}}

\newcommand{\Ab}{{\boldsymbol{A}}}
\newcommand{\ab}{{\boldsymbol{a}}}
\newcommand{\Bb}{{\boldsymbol{B}}}
\newcommand{\bb}{{\boldsymbol{b}}}
\newcommand{\cb}{{\boldsymbol{c}}}
\newcommand{\Db}{{\boldsymbol{D}}}
\newcommand{\db}{{\boldsymbol{d}}}
\newcommand{\Fb}{{\boldsymbol{F}}}
\newcommand{\Ib}{{\boldsymbol{I}}}
\newcommand{\jb}{{\boldsymbol{j}}}
\newcommand{\Jb}{{\boldsymbol{J}}}
\newcommand{\kb}{{\boldsymbol{k}}}
\newcommand{\Lb}{{\boldsymbol{L}}}
\newcommand{\lb}{{\boldsymbol{l}}}
\newcommand{\mb}{{\boldsymbol{m}}}
\newcommand{\nb}{{\boldsymbol{n}}}
\newcommand{\pb}{{\boldsymbol{p}}}
\newcommand{\Sb}{{\boldsymbol{S}}}
\newcommand{\Vb}{{\boldsymbol{V}}}
\newcommand{\Ub}{{\boldsymbol{U}}}
\newcommand{\tb}{{\boldsymbol{t}}}
\newcommand{\Xb}{{\boldsymbol{X}}}
\newcommand{\yb}{\boldsymbol{y}}

\newcommand{\Ubar}{{\overline{U}}}
\newcommand{\ubar}{{\overline{u}}}
\newcommand{\Bbar}{{\overline{B}}}
\newcommand{\psibar}{{\overline{\psi}}}
\newcommand{\qbar}{{\overline{q}}}
\newcommand{\zbar}{{\overline{z}}}

\newcommand{\Ri}{R}
\newcommand{\Fr}{F}

\newcommand{\qhat}{{\hat{q}}}
\newcommand{\jhat}{{\hat{j}}}

\newcommand\comment[1]{{\color{red}#1}}

% ===========================================================================

\begin{document}

% Use the \preprint command to place your local institutional report number 
% on the title page in preprint mode.
% Multiple \preprint commands are allowed.
%\preprint{}

\title{Academic notes: 2H Complex Analysis} %Title of paper

% repeat the \author .. \affiliation  etc. as needed
% \email, \thanks, \homepage, \altaffiliation all apply to the current author.
% Explanatory text should go in the []'s, 
% actual e-mail address or url should go in the {}'s for \email and \homepage.
% Please use the appropriate macro for the type of information
% \affiliation command applies to all authors since the last \affiliation command. 
% The \affiliation command should follow the other information.

\author{J. Mak (\today) [Blended from notes of R. Gregory and J. Bolton, Durham]}
\email[]{julian.c.l.mak@googlemail.com}

%\email[]{Your e-mail address}
%\homepage[]{Your web page}
%\thanks{}
%\altaffiliation{}

%\affiliation{}

% Collaboration name, if desired (requires use of superscriptaddress option in \documentclass). 
% \noaffiliation is required (may also be used with the \author command).
%\collaboration{}
%\noaffiliation
%\date{\today}

%\begin{abstract}
%Notes on the vorticity wave mechanism in 2d non-Boussinesq setting.
%\end{abstract}

%\pacs{}% insert suggested PACS numbers in braces on next line

\maketitle %\maketitle must follow title, authors, abstract and \pacs

% Body of paper goes here. Use proper sectioning commands. 
% References should be done using the \cite, \ref, and \label commands

%===============================================================================

\section*{Misc. notes}

\begin{itemize}
  \item This was part of the Durham core second year modules. Involves more
  things to do with analysis in the complex plane, involving holomorphic
  functions, contour integrals, residue theorems, conform mappings, etc.
  \item The original course does not have geometry of complex numbers since that
  was covered in Core A (Geometry 1A), but for consistency reasons this has been
  moved here.
  \item[]
  \item \comment{(to be fixed?)} Diagrams to do
\end{itemize}

%===============================================================================

\section{Geometry of complex numbers}

%-------------------------------------------------------------------------------

\subsection{Complex numbers and the Argand diagram}

We define $\sqrt{-1}=\zi$, which is the basic unit imaginary number. A
\Def{complex number} is then a combination of real and imaginary parts
$z=a+b\zi$, with $a,b\in\mathbb{R}$. The complex numbers $\mathbb{C}$ then obeys
the same axioms for addition and multiplication as $\mathbb{R}$ (both are
\Def{fields}).

Consider instead $\mathbb{C}$ as a vector space $z=(x,y)$, where multiplication
is defined on $\mathbb{R}^2$ as
\begin{equation*}
	z_1\times z_2 = (x_1 x_2 - y_1 y_2, x_1 y_2 - x_2 y_1),
\end{equation*}
and this is commutative. $1=(1,0)$ is the identity. So we see that
$\mathbb{R}^2$ with this multiplication is a concrete visualisation of
$\mathbb{C}$, and is called the \Def{Argand diagram}.

Given $z=x+\zi y$, the \Def{conjugate} of $z$ is defined to be
$\zbar=x-\zi y$. Geometrically, this represents a reflection of $z$ in the
`real' axis. The \Def{real} and \Def{imaginary} part of $z$ is
given respectively by
\begin{equation*}
	\Real(z)=\frac{z+\zbar}{2},\qquad \Imag(z)=\frac{z-\zbar}{2}.
\end{equation*}

In polar form, $z=r(\cos\theta+\zi\sin\theta)$. $r$ is called the
\Def{modulus} of $z$ and is denoted $|z|$, whilst $\theta$ is called the
\Def{argument} of $z$, denoted $\mbox{arg}(z)$.

%-------------------------------------------------------------------------------

\subsection{Geometry of addition and multiplication in $\mathbb{C}$}

Addition is as in $\mathbb{R}^2$. From this, we can deduce the \Def{triangle
inequality}.
\begin{lemma}
	For $z_1,z_2\in\mathbb{C}$, $|z_1 +z_2|\leq|z_1|+|z_2|$, and we have an
	equality iff $\mbox{arg}(z_1)=\mbox{arg}(z_2)$. By corollary, we have $|z_2
	+z_2|\geq||z_1|-|z_2||$.
\end{lemma}
\begin{proof}
	Wlog, let $|z_1|>|z_2|$, then $|z_1|=|z_1 + z_2 +
	(-z_2)|\leq|z_1+z_2|+|z_2|$ by the triangle inequality for real numbers. So
	$|z_1|-|z_2|\leq|z_1+z_2|$, and since $|z_1|>|z_2|$, we have the corollary
	of the result as required.
\end{proof}

For multiplication, we observe that $|z_1 z_2|=|z_1||z_2|$ and $\mbox{arg}(z_1
z_2)=\mbox{arg}(z_1)+\mbox{arg}(z_2)$. Geometrically, this is a spiral scaling.

We can use the $\mathbb{C}$-plane to describe various geometrical objects.
\begin{example}
	A circle may be described by $|z-z_0|=a$, where $z_0$ is the centre of the
	circle and $a$ is the radius; expanding this accordingly, we see that $a^2 =
	(x-x_0)^2 + (y-y_0)^2$.
\end{example}
\begin{example}
	The equation $|z-x_0|+|z+x_0|=2r$ describes an ellipse, where $r>|x_0|$.
	This may be done via expansion in $(x,y)$. Alternatively, in polar form, we
	observe that, for $z=a+\zi b$, $|z\pm x_0|^2 = (a^2-b^2)\cos^2\theta\pm
	2ax_0\cos\theta + (x_0^2+b^2)$. If $x_0^2=(a^2-b^2)$, then this may be
	simplified to $|z\pm x_0|=a\pm x_0\cos\theta$ since $a>x_0$. With this, we
	obtain $|z-x_0|+|z+x_0|=2a$, thus, with $x=a\cos\theta$ and $y=b\sin\theta$,
	this describes an ellipse.
\end{example}
\begin{example}
	The locus of $|z-z_1|=|z-z_2|$ describes the line that is equidistant to the
	points $z_1$ and $z_2$. To see this, expanding everything in $x$ and $y$ and
	we obtain the equality
	\begin{equation*}
		x(x_2-x_1) + y(y_2-y_1) 
		= \frac{y_2^2-y_1^2}{2} + \frac{x_2^2 - x_1^2}{2},
	\end{equation*}
	and the normal to the line is $z_2-z_1$.
\end{example}

%-------------------------------------------------------------------------------

\subsection{de Moivre's theorem}

\begin{theorem}[de Moivre's theorem]
	For all $n\in\mathbb{Z}^+$ and angle $\theta$, $\cos n\theta+\zi \sin
	n\theta = (\cos\theta+\zi\sin\theta)^n$.
\end{theorem}
\begin{proof}
	We do this by induction. The $n=1$ case is trivial, so, assuming it is true
	for $n$, then we observe that
	\begin{align*}
		\cos(n+1)\theta+\zi\sin(n+1)\theta &=
		\cos n\theta\cos\theta + \zi^2 \sin\theta\sin n\theta
		+\zi\sin n\theta\cos\theta + \zi\sin\theta\cos n\theta\\
		&= (\cos n\theta+\zi\sin n\theta)(\cos\theta+\zi\sin\theta)\\
		&=(\cos\theta+\zi\sin\theta)^{n+1}.
	\end{align*}
\end{proof}

\begin{example}
	\begin{equation*}
		\cos2\theta+\zi\sin2\theta=(\cos\theta+\zi\sin\theta)^2
		=(\cos^2\theta-\sin^2\theta)+\zi(2\sin\theta\cos\theta),
	\end{equation*}
	and remembering the double angle formulae, the equality agrees. From de
	Moivre's theorem, we see that
	\begin{equation*}
		\cos n\theta=\Real(\cos\theta+\zi\sin\theta)^n,\qquad
		\sin n\theta=\Imag(\cos\theta+\zi\sin\theta)^n.
	\end{equation*}
\end{example}

We can also use the theorem to find $\sin$ or $\cos$ of rational multiples of
$\pi$.
\begin{example}
	Express $\sin4\theta/\cos\theta$ as a polynomial in $\sin\theta$, and hence
	find $\sin(\pi/4)$.
	
	\begin{equation*}
		\sin4\theta = \Imag(\cos\theta+\zi\sin\theta)^4
		= 4\cos^3\theta\sin\theta-4\cos\theta\sin^3\theta
		= 4\cos\theta(\sin\theta-2\sin^3\theta),
	\end{equation*}
	so $\sin4\theta/\cos\theta=4\sin\theta(1-2\sin^\theta)$. Evaluating this
	$\pi/4$, we see that the LHS is zero. Now, $4\sin(\pi/4)>0$, so we conclude
	that $\sin(\pi/4)=1/\sqrt{2}$, as expected.
\end{example}
\begin{example}
	Find $\cos(k\pi/6)$ for $k=1,2,3,4,5$.
	
	Letting $c=\cos\theta$ and $s=\sin\theta$, observe that
	\begin{equation*}
		\sin6\theta=sc(6c^4+6s^4-20s^2c^2)=sc(32c^4-32c^2+6)
		=2sc(4c^3-3)(4c^2-1).
	\end{equation*}
	Now, $\sin(k\pi)=0$, so LHS is zero, and since $\sin(k\pi/6)\neq0$, we have
	\begin{equation*}
		\cos^2(k\pi/6)=3/4,\quad \cos^2(k\pi/6)=1/4,\quad \cos\theta=0\quad
		\Rightarrow\qquad \cos(k\pi/6)=\pm\sqrt{3}/2,\ \pm1/2,\ 0.
	\end{equation*}
	Since $\cos\theta$ is a decreasing function in $[0,\pi]$, we have
	\begin{equation*}
		\cos(\pi/6)=\sqrt{3}/2,\quad \cos(2\pi/6)=1/2,\quad \cos(\pi/2)=0,\quad
		\cos(2\pi/3)=-1/2,\quad \cos(5\pi/6)=-\sqrt{3}/2.
	\end{equation*}
\end{example}

%-------------------------------------------------------------------------------

\subsection{Imaginary exponentials}

de Moivre's theorem hints at a deeper geometric significance of cosine and sine
functions and a way of encoding multiplication by imaginary numbers. Suppose
$f(\theta)=\cos\theta+\zi\sin\theta$, then we notice that $f'(\theta)=\zi
f(\theta)$, and, more generally, $f^{(n)}(\theta)=\zi^n f(\theta)$. We know that
also that the $n$-th derivative of $\ex^{\lambda x}$ is $\lambda^n\ex^{\lambda
x}$, so this suggests a link with exponential functions; indeed, we have
\Def{Euler's formula}
\begin{equation}
	\cos\theta+\zi\sin\theta=\ex^{\zi\theta}.
\end{equation}
By de Moivre's theroem then,
\begin{equation*}
	r(\cos n\theta+\zi\sin n\theta)=r(\cos\theta+\zi\sin\theta)^n
	=r\ex^{\zi n\theta}.
\end{equation*}
\begin{lemma}[Euler identity]
	$\ex^{\zi\pi}+1=0$.
\end{lemma}
\begin{example}
	Find all the roots of $z^6+4z^3+8=0$.
	
	Factorising the above gives $z^3=-2\pm2\zi$. So since $|z^3|=2\sqrt{2}$, we
	have $|z|=\sqrt{2}$. Now,
	\begin{equation*}
		\mbox{arg}(-2+2\zi)=\frac{3\pi}{4},\qquad
		\mbox{arg}(-2-2\zi)=\frac{5\pi/4},
	\end{equation*}
	and the argument of the roots $z$ satisfies
	\begin{equation*}
		\mbox{arg}(z)=\frac{3\pi/4 + 2n\pi}{3},\qquad
		\mbox{arg}(z)=\frac{5\pi/4 + 2n\pi}{3},
	\end{equation*}
	where the division by $3$ is to take into account the cube root, and the
	$2n\pi$ factors is to account for all the roots. This eventually yields
	\begin{equation*}
		z=\sqrt{2}(\ex^{\zi\pi/4}, \ex^{5\zi\pi/4}, \ex^{11\zi\pi/12},
		\ex^{13\zi\pi/12}, \ex^{19\zi\pi/12}, \ex^{21\zi\pi/21}).
	\end{equation*}
\end{example}

%===============================================================================

\section{Basics of complex functions}

A real function can for example be once differentiable, but not
twice. One example is $f(x) = x|x|$, where $f'(x)$ is not differentiable at
$x=0$.

\begin{theorem}
  If a complex function is once differentiable, it is differentiable as many
  times as you like.
\end{theorem}

It is possible for two real functions to agree on an interval but not
everywhere, assuming they are differentiable. One example is $f(x) = x|x|$ and
$g(x) = x^2$ for $x>0$.

\begin{theorem}
  If two complex differentiable functions agree on any disc in the complex
  plane, then they agree everywhere (subject to certain conditions...)
\end{theorem}

Recall that a real function assigns any real number $x$ to at most one real
number (i.e. it is injective). A \Def{complex function} therefore assigns any
complex number $z$ to at most one complex number. These include standard
polynomials, rational functions, transcendental functions, trigonometric
functions, hyperbolic functions, where the argument is in $z$. Some examples
have already been given above.

\begin{example}
  Solve $\ex^z = 1$.
  
  Writing $z = x + \zi y$ and using Euler's formula,
  \begin{equation*}
    \ex^x (\cos y + \zi \sin y) = 1,
  \end{equation*}
  and equating real and imaginary parts lead to
  \begin{equation*}
    \ex^x \cos y = 1, \qquad \ex^x \sin y = 0.
  \end{equation*}
  Considering the imaginary part, since $\ex^x \neq 0$, $y = n\pi$ for $n \in
  \mathbb{Z}$, but from the real part, since $\ex^x > 0$ and $\cos n\pi = \pm
  1$, we should only have $y = 2n\pi$ for $n \in \mathbb{Z}$. The real part then
  additionally implies that $x=0$ since $\cos 2n\pi = 1$, so $z = 2\zi n \pi$
  for $n \in \mathbb{Z}$.
\end{example}
Note that $|\ex^{\zi z}| \geq 0$ for all $z \in \mathbb{C}$.

\begin{example}
  Solve $\sin z = 0$.
  
  With the standard identity for sine with complex arguments, we have
  \begin{equation*}
    \frac{\ex^{\zi z} - \ex^{-\zi z}}{2\zi} = 0.
  \end{equation*}
  Equating real and imaginary parts lead to $z = m\pi$, $m \in \mathbb{Z}$.
\end{example}

The (natural) \Def{logarithm} we define by
\begin{equation}
  \log z = \log|z| + \zi \mbox{arg}z
\end{equation}
to give a complex version of the log function that satisfies the usual rules of
\begin{equation*}
  \log z = \log r\ex^{\zi\theta} = \log r + \zi \theta = \log|z| + \zi \mbox{arg}z.
\end{equation*}
Here we need to choose a \Def{branch}, and we take $\theta\in(-\pi, \pi)$ (the
\Def{principal branch}) to preserve the continuity property, so that $\log z$ is
undefined on the negative real axis, coinciding with the real case.

\begin{example}
  $\log(1-i) = \log\sqrt{2} - \zi(\pi / 4)$
\end{example}

We use $\log z$ to define powers of complex numbers. Recall that for real
numbers we have $x^a = \ex^{a \log a}$ for $a>0$, so for $z,w \in \mathbb{C}$,
we analogously define 
\begin{equation}
  z^w = \ex^{w \log z},
\end{equation}
choosing the principal branch unless otherwise stated.

\begin{example}
  \begin{align*}
    (1 + \zi \sqrt{3})^{1/2} = \exp\left[\frac{1}{2} \log(1 + \zi \sqrt{3})\right] = \exp\left[\frac{1}{2} \left(\log 2 + \zi\frac{\pi}{3}\right)\right] = \ex^{\log \sqrt{2}} \ex^{\zi(\pi/6)} = \sqrt{2}\ex^{\zi(\pi/6)},
  \end{align*}
  which in this case is could have been gotten from $(1 + \zi \sqrt{3}) = 2\ex^{\zi(\pi/3)}$.
\end{example}

\begin{example}
  \begin{align*}
    (1 - \zi)^{\zi} = \ex^{\zi \log(1-\zi)} = \ex^{\zi (\log \sqrt{2} - \zi \pi/4)} = \ex^{\pi/4} \ex^{\zi \log\sqrt{2}}.
  \end{align*}
\end{example}

We say a complex function $f(z)$ is \Def{complex differentiable at $z = z_0$} if
\begin{equation*}
  \lim_{z\to z_0} \frac{f(z) - f(z_0)}{z - z_0}
\end{equation*}
exists, or that
\begin{equation*}
  \lim_{h\to 0} \frac{f(z + h) - f(z)}{h}
\end{equation*}
exists at $z = z_0$. The derivative is denoted $f'(z)$ as usual.

\begin{example}
  For $f(z) = z^2$,
  \begin{equation*}
    \lim_{h \to 0} \frac{f(z + h) - f(z)}{h} = \lim_{h \to 0} \frac{z^2 + 2hz + h^2 - z^2}{h} = \lim_{h \to 0} 2z + h = 2z.
  \end{equation*}
  $f(z)$ is differentiable everywhere.
\end{example}
The usual rules for differentiation hold (linearity, product rule, chain rule
etc.)

Note that $f(x) = x|x|$ is real differentiable everywhere. $f(z) = z|z|$ on the
other hand is differentiable on the real axis, and complex differentiable at the
origin.

Complex differentiation is a much stronger condition. Recall that for the limit
to exist in the real case, the limit only needs to be equal when approached from
above or below on the real line. In the complex plane however there are an
infinite numbers of cases the limit can be approach, and thus a infinite number
of cases to check. We see that a necessary condition for complex
differentiability is that the limit needs to exist when $z_0$ is approached in
the lines parallel to the real and imaginary axis. If we set $f(z)$ to be
\begin{equation*}
  f(z) = u(x,y) + \zi v(x,y)
\end{equation*}
for some real functions $u$ and $v$, then it turns out that
\begin{align*}
  \lim_{z \to z_0} \frac{f(z) - f(z_0)}{z - z_0} = \ddy{u}{x} + \zi \ddy{v}{x} = \ddy{v}{y} - \zi \ddy{u}{y},
\end{align*}
when we take the limit in the direction parallel to the real and imaginary axis
respectively. It follows that a \emph{necessary} conditions for a function to be
complex differentiable is that
\begin{equation}
  \ddy{u}{x} = \ddy{v}{y} \qquad \textnormal{and} \qquad \ddy{u}{y} = -\ddy{v}{x}.
\end{equation}
These are known as the \Def{Cauchy--Riemann equations}, and we actually have the
following theorem.
\begin{theorem}
  If $f(z)$ is complex differentiable at $z = z_0$, then the Cauchy--Riemann
  equations hold at $(x_0, y_0)$ for $z_0 = x_0 + \zi y_0$, and that
  \begin{equation*}
    f'(z_0) = \left.\left(\ddy{u}{x} + \zi \ddy{v}{x}\right)\right|_{(x_0, y_0)} = \left.\left(\ddy{v}{y} - \zi \ddy{u}{y}\right)\right|_{(x_0, y_0)}.
  \end{equation*}
\end{theorem}

%===============================================================================

\section{Integration in the complex plane}

%===============================================================================

\section{Conform mapping}

%===============================================================================

\bibliography{/home/jclmak/Documents/tex-files/master_ref/refs}

%\begin{thebibliography}{48}
%\providecommand{\natexlab}[1]{#1}
%\providecommand{\url}[1]{\texttt{#1}}
%\expandafter\ifx\csname urlstyle\endcsname\relax
%  \providecommand{\doi}[1]{doi: #1}\else
%  \providecommand{\doi}{doi: \begingroup \urlstyle{rm}\Url}\fi

%\bibitem[Limaye(1986)]{lim86}
%S.S. Limaye.
%\newblock {J}upiter: new estimates of the mean zonal flow at the cloud level.
%\newblock \emph{Icarus}, 65:\penalty0 335--352, 1986.

%\end{thebibliography}

\end{document}




%%%%%%%%%%%%%%%%%%%%%%%%%%%%
% Appendices:
% \newpage
% 
% \appendix
% \ifnum\hylinks=1 \phantomsection \fi
% \addcontentsline{toc}{section}{Appendices}
% 
% \section{Sample Appendix}

%%%%%%%%%%%%%%%%%%%%%%%%%%%%%
