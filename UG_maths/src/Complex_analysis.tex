\documentclass[letter-paper]{tufte-book}

%%
% Book metadata
\title{Complex Analysis 2H}
\author[]{Inusuke Shibemoto}
%\publisher{Research Institute of Valinor}

%%
% If they're installed, use Bergamo and Chantilly from www.fontsite.com.
% They're clones of Bembo and Gill Sans, respectively.
\IfFileExists{bergamo.sty}{\usepackage[osf]{bergamo}}{}% Bembo
\IfFileExists{chantill.sty}{\usepackage{chantill}}{}% Gill Sans

%\usepackage{microtype}
\usepackage{amssymb}
\usepackage{amsmath}
%%
% For nicely typeset tabular material
\usepackage{booktabs}

%% overunder braces
\usepackage{oubraces}

%% 
\usepackage{xcolor}
\usepackage{tcolorbox}

\newtcolorbox[auto counter,number within=section]{derivbox}[2][]{colback=TealBlue!5!white,colframe=TealBlue,title=Box \thetcbcounter:\ #2,#1}                                                          

\makeatletter
\@openrightfalse
\makeatother

%%
% For graphics / images
\usepackage{graphicx}
\setkeys{Gin}{width=\linewidth,totalheight=\textheight,keepaspectratio}
\graphicspath{{figs/}}

% The fancyvrb package lets us customize the formatting of verbatim
% environments.  We use a slightly smaller font.
\usepackage{fancyvrb}
\fvset{fontsize=\normalsize}

\usepackage[plain]{fancyref}
\newcommand*{\fancyrefboxlabelprefix}{box}
\fancyrefaddcaptions{english}{%
  \providecommand*{\frefboxname}{Box}%
  \providecommand*{\Frefboxname}{Box}%
}
\frefformat{plain}{\fancyrefboxlabelprefix}{\frefboxname\fancyrefdefaultspacing#1}
\Frefformat{plain}{\fancyrefboxlabelprefix}{\Frefboxname\fancyrefdefaultspacing#1}

%%
% Prints argument within hanging parentheses (i.e., parentheses that take
% up no horizontal space).  Useful in tabular environments.
\newcommand{\hangp}[1]{\makebox[0pt][r]{(}#1\makebox[0pt][l]{)}}

%% 
% Prints an asterisk that takes up no horizontal space.
% Useful in tabular environments.
\newcommand{\hangstar}{\makebox[0pt][l]{*}}

%%
% Prints a trailing space in a smart way.
\usepackage{xspace}
\usepackage{xstring}

%%
% Some shortcuts for Tufte's book titles.  The lowercase commands will
% produce the initials of the book title in italics.  The all-caps commands
% will print out the full title of the book in italics.
\newcommand{\vdqi}{\textit{VDQI}\xspace}
\newcommand{\ei}{\textit{EI}\xspace}
\newcommand{\ve}{\textit{VE}\xspace}
\newcommand{\be}{\textit{BE}\xspace}
\newcommand{\VDQI}{\textit{The Visual Display of Quantitative Information}\xspace}
\newcommand{\EI}{\textit{Envisioning Information}\xspace}
\newcommand{\VE}{\textit{Visual Explanations}\xspace}
\newcommand{\BE}{\textit{Beautiful Evidence}\xspace}

\newcommand{\TL}{Tufte-\LaTeX\xspace}

% Prints the month name (e.g., January) and the year (e.g., 2008)
\newcommand{\monthyear}{%
  \ifcase\month\or January\or February\or March\or April\or May\or June\or
  July\or August\or September\or October\or November\or
  December\fi\space\number\year
}


\newcommand{\urlwhitespacereplace}[1]{\StrSubstitute{#1}{ }{_}[\wpLink]}

\newcommand{\wikipedialink}[1]{http://en.wikipedia.org/wiki/#1}% needs \wpLink now

\newcommand{\anonymouswikipedialink}[1]{\urlwhitespacereplace{#1}\href{\wikipedialink{\wpLink}}{Wikipedia}}

\newcommand{\Wikiref}[1]{\urlwhitespacereplace{#1}\href{\wikipedialink{\wpLink}}{#1}}

% Prints an epigraph and speaker in sans serif, all-caps type.
\newcommand{\openepigraph}[2]{%
  %\sffamily\fontsize{14}{16}\selectfont
  \begin{fullwidth}
  \sffamily\large
  \begin{doublespace}
  \noindent\allcaps{#1}\\% epigraph
  \noindent\allcaps{#2}% author
  \end{doublespace}
  \end{fullwidth}
}

% Inserts a blank page
\newcommand{\blankpage}{\newpage\hbox{}\thispagestyle{empty}\newpage}

\usepackage{units}

% Typesets the font size, leading, and measure in the form of 10/12x26 pc.
\newcommand{\measure}[3]{#1/#2$\times$\unit[#3]{pc}}

% Macros for typesetting the documentation
\newcommand{\hlred}[1]{\textcolor{Maroon}{#1}}% prints in red
\newcommand{\hangleft}[1]{\makebox[0pt][r]{#1}}
\newcommand{\hairsp}{\hspace{1pt}}% hair space
\newcommand{\hquad}{\hskip0.5em\relax}% half quad space
\newcommand{\TODO}{\textcolor{red}{\bf TODO!}\xspace}
\newcommand{\na}{\quad--}% used in tables for N/A cells
\providecommand{\XeLaTeX}{X\lower.5ex\hbox{\kern-0.15em\reflectbox{E}}\kern-0.1em\LaTeX}
\newcommand{\tXeLaTeX}{\XeLaTeX\index{XeLaTeX@\protect\XeLaTeX}}
% \index{\texttt{\textbackslash xyz}@\hangleft{\texttt{\textbackslash}}\texttt{xyz}}
\newcommand{\tuftebs}{\symbol{'134}}% a backslash in tt type in OT1/T1
\newcommand{\doccmdnoindex}[2][]{\texttt{\tuftebs#2}}% command name -- adds backslash automatically (and doesn't add cmd to the index)
\newcommand{\doccmddef}[2][]{%
  \hlred{\texttt{\tuftebs#2}}\label{cmd:#2}%
  \ifthenelse{\isempty{#1}}%
    {% add the command to the index
      \index{#2 command@\protect\hangleft{\texttt{\tuftebs}}\texttt{#2}}% command name
    }%
    {% add the command and package to the index
      \index{#2 command@\protect\hangleft{\texttt{\tuftebs}}\texttt{#2} (\texttt{#1} package)}% command name
      \index{#1 package@\texttt{#1} package}\index{packages!#1@\texttt{#1}}% package name
    }%
}% command name -- adds backslash automatically
\newcommand{\doccmd}[2][]{%
  \texttt{\tuftebs#2}%
  \ifthenelse{\isempty{#1}}%
    {% add the command to the index
      \index{#2 command@\protect\hangleft{\texttt{\tuftebs}}\texttt{#2}}% command name
    }%
    {% add the command and package to the index
      \index{#2 command@\protect\hangleft{\texttt{\tuftebs}}\texttt{#2} (\texttt{#1} package)}% command name
      \index{#1 package@\texttt{#1} package}\index{packages!#1@\texttt{#1}}% package name
    }%
}% command name -- adds backslash automatically
\newcommand{\docopt}[1]{\ensuremath{\langle}\textrm{\textit{#1}}\ensuremath{\rangle}}% optional command argument
\newcommand{\docarg}[1]{\textrm{\textit{#1}}}% (required) command argument
\newenvironment{docspec}{\begin{quotation}\ttfamily\parskip0pt\parindent0pt\ignorespaces}{\end{quotation}}% command specification environment
\newcommand{\docenv}[1]{\texttt{#1}\index{#1 environment@\texttt{#1} environment}\index{environments!#1@\texttt{#1}}}% environment name
\newcommand{\docenvdef}[1]{\hlred{\texttt{#1}}\label{env:#1}\index{#1 environment@\texttt{#1} environment}\index{environments!#1@\texttt{#1}}}% environment name
\newcommand{\docpkg}[1]{\texttt{#1}\index{#1 package@\texttt{#1} package}\index{packages!#1@\texttt{#1}}}% package name
\newcommand{\doccls}[1]{\texttt{#1}}% document class name
\newcommand{\docclsopt}[1]{\texttt{#1}\index{#1 class option@\texttt{#1} class option}\index{class options!#1@\texttt{#1}}}% document class option name
\newcommand{\docclsoptdef}[1]{\hlred{\texttt{#1}}\label{clsopt:#1}\index{#1 class option@\texttt{#1} class option}\index{class options!#1@\texttt{#1}}}% document class option name defined
\newcommand{\docmsg}[2]{\bigskip\begin{fullwidth}\noindent\ttfamily#1\end{fullwidth}\medskip\par\noindent#2}
\newcommand{\docfilehook}[2]{\texttt{#1}\index{file hooks!#2}\index{#1@\texttt{#1}}}
\newcommand{\doccounter}[1]{\texttt{#1}\index{#1 counter@\texttt{#1} counter}}

\newcommand{\studyq}[1]{\marginnote{Q: #1}}

\hypersetup{colorlinks}% uncomment this line if you prefer colored hyperlinks (e.g., for onscreen viewing)

% Generates the index
\usepackage{makeidx}
\makeindex

\setcounter{tocdepth}{3}
\setcounter{secnumdepth}{3}

%%%%%%%%%%%%%%%%%%%%%%%%%%%%%%%%%%%%%%%%%%%%%%%%%%%%%%%%%%%%%%
% custom commands

\newtheorem{theorem}{\color{pastel-blue}Theorem}[section]
\newtheorem{lemma}[theorem]{\color{pastel-blue}Lemma}
\newtheorem{proposition}[theorem]{\color{pastel-blue}Proposition}
\newtheorem{corollary}[theorem]{\color{pastel-blue}Corollary}

\newenvironment{proof}[1][Proof]{\begin{trivlist}
\item[\hskip \labelsep {\bfseries #1}]}{\end{trivlist}}
\newenvironment{definition}[1][Definition]{\begin{trivlist}
\item[\hskip \labelsep {\bfseries #1}]}{\end{trivlist}}
\newenvironment{example}[1][Example]{\begin{trivlist}
\item[\hskip \labelsep {\bfseries #1}]}{\end{trivlist}}
\newenvironment{remark}[1][Remark]{\begin{trivlist}
\item[\hskip \labelsep {\bfseries #1}]}{\end{trivlist}}

\hyphenpenalty=5000

% more pastel ones
\xdefinecolor{pastel-red}{rgb}{0.77,0.31,0.32}
\xdefinecolor{pastel-green}{rgb}{0.33,0.66,0.41}
\definecolor{pastel-blue}{rgb}{0.30,0.45,0.69} % crayola blue
\definecolor{gray}{rgb}{0.2,0.2,0.2} % dark gray

\xdefinecolor{orange}{rgb}{1,0.45,0}
\xdefinecolor{green}{rgb}{0,0.35,0}
\definecolor{blue}{rgb}{0.12,0.46,0.99} % crayola blue
\definecolor{gray}{rgb}{0.2,0.2,0.2} % dark gray

\xdefinecolor{cerulean}{rgb}{0.01,0.48,0.65}
\xdefinecolor{ust-blue}{rgb}{0,0.20,0.47}
\xdefinecolor{ust-mustard}{rgb}{0.67,0.52,0.13}

%\newcommand\comment[1]{{\color{red}#1}}

\newcommand{\dy}{\partial}
\newcommand{\ddy}[2]{\frac{\dy#1}{\dy#2}}

\newcommand{\ex}{\mathrm{e}}
\newcommand{\zi}{{\rm i}}

\newcommand\Real{\mbox{Re}} % cf plain TeX's \Re and Reynolds number
\newcommand\Imag{\mbox{Im}} % cf plain TeX's \Im

\newcommand{\zbar}{{\overline{z}}}

\newcommand\Def[1]{\textbf{#1}}

\newcommand{\qed}{\hfill$\blacksquare$}
\newcommand{\qedwhite}{\hfill \ensuremath{\Box}}

%%%%%%%%%%%%%%%%%%%%%%%%%%%%%%%%%%%%%%%%%%%%%%%%%%%%%%%%%%%%%%
% some extra formatting (hacked from Patrick Farrell's notes)
%  https://courses.maths.ox.ac.uk/node/view_material/4915
%

% chapter format
\titleformat{\chapter}%
  {\huge\rmfamily\itshape\color{pastel-red}}% format applied to label+text
  {\llap{\colorbox{pastel-red}{\parbox{1.5cm}{\hfill\itshape\huge\color{white}\thechapter}}}}% label
  {1em}% horizontal separation between label and title body
  {}% before the title body
  []% after the title body

% section format
\titleformat{\section}%
  {\normalfont\Large\itshape\color{pastel-green}}% format applied to label+text
  {\llap{\colorbox{pastel-green}{\parbox{1.5cm}{\hfill\color{white}\thesection}}}}% label
  {1em}% horizontal separation between label and title body
  {}% before the title body
  []% after the title body

% subsection format
\titleformat{\subsection}%
  {\normalfont\large\itshape\color{pastel-blue}}% format applied to label+text
  {\llap{\colorbox{pastel-blue}{\parbox{1.5cm}{\hfill\color{white}\thesubsection}}}}% label
  {1em}% horizontal separation between label and title body
  {}% before the title body
  []% after the title body

%%%%%%%%%%%%%%%%%%%%%%%%%%%%%%%%%%%%%%%%%%%%%%%%%%%%%%%%%%%%%%%%%%%%%%%%%%%%%%%%

\begin{document}

% Front matter
%\frontmatter

% r.3 full title page
%\maketitle

% v.4 copyright page

\chapter*{}

\begin{fullwidth}

\par \begin{center}{\Huge Complex Analysis 2H}\end{center}

\vspace*{5mm}

\par \begin{center}{\Large typed up by B. S. H. Mithrandir}\end{center}

\vspace*{5mm}

\begin{itemize}
  \item \textit{Last compiled: \monthyear}
  \item Blended from notes of R. Gregory and J. Bolton, Durham
  \item This was part of the Durham core second year modules. Involves more
  things to do with analysis in the complex plane, involving holomorphic
  functions, contour integrals, residue theorems, conform mappings, etc.
  \item The original course does not have geometry of complex numbers since that
  was covered in Core A (Geometry 1A), but for consistency reasons this has been
  moved here.
  \item[]
  \item \TODO Diagrams to do
\end{itemize}

\par

\par Licensed under the Apache License, Version 2.0 (the ``License''); you may not
use this file except in compliance with the License. You may obtain a copy
of the License at \url{http://www.apache.org/licenses/LICENSE-2.0}. Unless
required by applicable law or agreed to in writing, software distributed
under the License is distributed on an \smallcaps{``AS IS'' BASIS, WITHOUT
WARRANTIES OR CONDITIONS OF ANY KIND}, either express or implied. See the
License for the specific language governing permissions and limitations
under the License.
\end{fullwidth}


%===============================================================================

\chapter{Geometry of complex numbers}

%-------------------------------------------------------------------------------

\section{Complex numbers and the Argand diagram}

We define $\sqrt{-1}=\zi$, which is the basic unit imaginary number. A
\Def{complex number} is then a combination of real and imaginary parts
$z=a+b\zi$, with $a,b\in\mathbb{R}$. The complex numbers $\mathbb{C}$ then obeys
the same axioms for addition and multiplication as $\mathbb{R}$ (both are
\Def{fields}).

Consider instead $\mathbb{C}$ as a vector space $z=(x,y)$, where multiplication
is defined on $\mathbb{R}^2$ as
\begin{equation*}
	z_1\times z_2 = (x_1 x_2 - y_1 y_2, x_1 y_2 - x_2 y_1),
\end{equation*}
and this is commutative. $1=(1,0)$ is the identity. So we see that
$\mathbb{R}^2$ with this multiplication is a concrete visualisation of
$\mathbb{C}$, and is called the \Def{Argand diagram}.

Given $z=x+\zi y$, the \Def{conjugate} of $z$ is defined to be
$\zbar=x-\zi y$. Geometrically, this represents a reflection of $z$ in the
`real' axis. The \Def{real} and \Def{imaginary} part of $z$ is
given respectively by
\begin{equation*}
	\Real(z)=\frac{z+\zbar}{2},\qquad \Imag(z)=\frac{z-\zbar}{2}.
\end{equation*}

In polar form, $z=r(\cos\theta+\zi\sin\theta)$. $r$ is called the
\Def{modulus} of $z$ and is denoted $|z|$, whilst $\theta$ is called the
\Def{argument} of $z$, denoted $\mbox{arg}(z)$.

%-------------------------------------------------------------------------------

\section{Geometry of addition and multiplication in $\mathbb{C}$}

Addition is as in $\mathbb{R}^2$. From this, we can deduce the \Def{triangle
inequality}.
\begin{lemma}
	For $z_1,z_2\in\mathbb{C}$, $|z_1 +z_2|\leq|z_1|+|z_2|$, and we have an
	equality iff $\mbox{arg}(z_1)=\mbox{arg}(z_2)$. By corollary, we have $|z_2
	+z_2|\geq||z_1|-|z_2||$.
\end{lemma}
\begin{proof}
	Without loss of generalisation, let $|z_1|>|z_2|$, then $|z_1|=|z_1 + z_2 +
	(-z_2)|\leq|z_1+z_2|+|z_2|$ by the triangle inequality for real numbers. So
	$|z_1|-|z_2|\leq|z_1+z_2|$, and since $|z_1|>|z_2|$, we have the corollary
	of the result as required. \qed
\end{proof}

For multiplication, we observe that $|z_1 z_2|=|z_1||z_2|$ and $\mbox{arg}(z_1
z_2)=\mbox{arg}(z_1)+\mbox{arg}(z_2)$. Geometrically, this is a spiral scaling.

We can use the $\mathbb{C}$-plane to describe various geometrical objects.
\begin{example}
	A circle may be described by $|z-z_0|=a$, where $z_0$ is the centre of the
	circle and $a$ is the radius; expanding this accordingly, we see that $a^2 =
	(x-x_0)^2 + (y-y_0)^2$.
\end{example}
\begin{example}
	The equation $|z-x_0|+|z+x_0|=2r$ describes an ellipse, where $r>|x_0|$.
	This may be done via expansion in $(x,y)$. Alternatively, in polar form, we
	observe that, for $z=a+\zi b$, $|z\pm x_0|^2 = (a^2-b^2)\cos^2\theta\pm
	2ax_0\cos\theta + (x_0^2+b^2)$. If $x_0^2=(a^2-b^2)$, then this may be
	simplified to $|z\pm x_0|=a\pm x_0\cos\theta$ since $a>x_0$. With this, we
	obtain $|z-x_0|+|z+x_0|=2a$, thus, with $x=a\cos\theta$ and $y=b\sin\theta$,
	this describes an ellipse.
\end{example}
\begin{example}
	The locus of $|z-z_1|=|z-z_2|$ describes the line that is equidistant to the
	points $z_1$ and $z_2$. To see this, expanding everything in $x$ and $y$ and
	we obtain the equality
	\begin{equation*}
		x(x_2-x_1) + y(y_2-y_1) 
		= \frac{y_2^2-y_1^2}{2} + \frac{x_2^2 - x_1^2}{2},
	\end{equation*}
	and the normal to the line is $z_2-z_1$.
\end{example}

%-------------------------------------------------------------------------------

\section{de Moivre's theorem}

\begin{theorem}[de Moivre's theorem]
	For all $n\in\mathbb{Z}^+$ and angle $\theta$, $\cos n\theta+\zi \sin
	n\theta = (\cos\theta+\zi\sin\theta)^n$.
\end{theorem}
\begin{proof}
	We do this by induction. The $n=1$ case is trivial, so, assuming it is true
	for $n$, then we observe that
	\begin{align*}
		\cos(n+1)\theta&+\zi\sin(n+1)\theta \\
		&=\cos n\theta\cos\theta + \zi^2 \sin\theta\sin n\theta
		+\zi\sin n\theta\cos\theta + \zi\sin\theta\cos n\theta\\
		&= (\cos n\theta+\zi\sin n\theta)(\cos\theta+\zi\sin\theta)\\
		&=(\cos\theta+\zi\sin\theta)^{n+1}.
	\end{align*}
	\qed
\end{proof}

\begin{example}
  Since  
	\begin{equation*}
		\cos2\theta+\zi\sin2\theta=(\cos\theta+\zi\sin\theta)^2
		=(\cos^2\theta-\sin^2\theta)+\zi(2\sin\theta\cos\theta),
	\end{equation*}
	and remembering the double angle formulae, the equality agrees. From de
	Moivre's theorem, we see that
	\begin{equation*}
		\cos n\theta=\Real(\cos\theta+\zi\sin\theta)^n,\qquad
		\sin n\theta=\Imag(\cos\theta+\zi\sin\theta)^n.
	\end{equation*}
\end{example}

We can also use the theorem to find $\sin$ or $\cos$ of rational multiples of
$\pi$.
\begin{example}
	Express $\sin4\theta/\cos\theta$ as a polynomial in $\sin\theta$, and hence
	find $\sin(\pi/4)$.
	\begin{align*}
		\sin4\theta = \Imag(\cos\theta+\zi\sin\theta)^4
		&= 4\cos^3\theta\sin\theta-4\cos\theta\sin^3\theta\\
		&= 4\cos\theta(\sin\theta-2\sin^3\theta),
	\end{align*}
	so $\sin4\theta/\cos\theta=4\sin\theta(1-2\sin^\theta)$. Evaluating this
	$\pi/4$, we see that the LHS is zero. Now, $4\sin(\pi/4)>0$, so we conclude
	that $\sin(\pi/4)=1/\sqrt{2}$, as expected.
\end{example}
\begin{example}
	Find $\cos(k\pi/6)$ for $k=1,2,3,4,5$.\\
	
	Letting $c=\cos\theta$ and $s=\sin\theta$, observe that
	\begin{equation*}
		\sin6\theta=sc(6c^4+6s^4-20s^2c^2)=sc(32c^4-32c^2+6)
		=2sc(4c^3-3)(4c^2-1).
	\end{equation*}
	Now, $\sin(k\pi)=0$, so LHS is zero, and since $\sin(k\pi/6)\neq0$, we have
	\begin{equation*}
		\cos^2(k\pi/6)=3/4,\quad \cos^2(k\pi/6)=1/4,\quad \cos\theta=0,
	\end{equation*}
	which implies that
	\begin{equation*}
	  \cos(k\pi/6)=\pm\sqrt{3}/2,\ \pm1/2,\ 0.
	\end{equation*}
	Since $\cos\theta$ is a decreasing function in $[0,\pi]$, we have
	\begin{align*}
		\cos(\pi/6)=\sqrt{3}/2,\quad \cos(2\pi/6)=1/2,\quad \cos(\pi/2)=0,\\
		\cos(2\pi/3)=-1/2,\quad \cos(5\pi/6)=-\sqrt{3}/2.
	\end{align*}
\end{example}

%-------------------------------------------------------------------------------

\section{Imaginary exponentials}

de Moivre's theorem hints at a deeper geometric significance of cosine and sine
functions and a way of encoding multiplication by imaginary numbers. Suppose
$f(\theta)=\cos\theta+\zi\sin\theta$, then we notice that $f'(\theta)=\zi
f(\theta)$, and, more generally, $f^{(n)}(\theta)=\zi^n f(\theta)$. We know that
also that the $n$-th derivative of $\ex^{\lambda x}$ is $\lambda^n\ex^{\lambda
x}$, so this suggests a link with exponential functions; indeed, we have
\Def{Euler's formula}
\begin{equation}
	\cos\theta+\zi\sin\theta=\ex^{\zi\theta}.
\end{equation}
By de Moivre's theroem then,
\begin{equation*}
	r(\cos n\theta+\zi\sin n\theta)=r(\cos\theta+\zi\sin\theta)^n
	=r\ex^{\zi n\theta}.
\end{equation*}
\begin{lemma}[Euler identity]
	$\ex^{\zi\pi}+1=0$. \qedwhite
\end{lemma}
\begin{example}
	Find all the roots of $z^6+4z^3+8=0$.\\
	
	Factorising the above gives $z^3=-2\pm2\zi$. So since $|z^3|=2\sqrt{2}$, we
	have $|z|=\sqrt{2}$. Now,
	\begin{equation*}
		\mbox{arg}(-2+2\zi)=\frac{3\pi}{4},\qquad
		\mbox{arg}(-2-2\zi)=\frac{5\pi/4},
	\end{equation*}
	and the argument of the roots $z$ satisfies
	\begin{equation*}
		\mbox{arg}(z)=\frac{3\pi/4 + 2n\pi}{3},\qquad
		\mbox{arg}(z)=\frac{5\pi/4 + 2n\pi}{3},
	\end{equation*}
	where the division by $3$ is to take into account the cube root, and the
	$2n\pi$ factors is to account for all the roots. This eventually yields
	\begin{equation*}
		z=\sqrt{2}(\ex^{\zi\pi/4}, \ex^{5\zi\pi/4}, \ex^{11\zi\pi/12},
		\ex^{13\zi\pi/12}, \ex^{19\zi\pi/12}, \ex^{21\zi\pi/21}).
	\end{equation*}
\end{example}

%===============================================================================

\chapter{Basics of complex functions}

A real function can for example be once differentiable, but not
twice. One example is $f(x) = x|x|$, where $f'(x)$ is not differentiable at
$x=0$.

\begin{theorem}
  If a complex function is once differentiable, it is differentiable as many
  times as you like.
\end{theorem}

It is possible for two real functions to agree on an interval but not
everywhere, assuming they are differentiable. One example is $f(x) = x|x|$ and
$g(x) = x^2$ for $x>0$.

\begin{theorem}
  If two complex differentiable functions agree on any disc in the complex
  plane, then they agree everywhere (subject to certain conditions...)
\end{theorem}

Recall that a real function assigns any real number $x$ to at most one real
number (i.e. it is injective). A \Def{complex function} therefore assigns any
complex number $z$ to at most one complex number. These include standard
polynomials, rational functions, transcendental functions, trigonometric
functions, hyperbolic functions, where the argument is in $z$. Some examples
have already been given above.

\begin{example}
  Solve $\ex^z = 1$.\\
  
  Writing $z = x + \zi y$ and using Euler's formula,
  \begin{equation*}
    \ex^x (\cos y + \zi \sin y) = 1,
  \end{equation*}
  and equating real and imaginary parts lead to
  \begin{equation*}
    \ex^x \cos y = 1, \qquad \ex^x \sin y = 0.
  \end{equation*}
  Considering the imaginary part, since $\ex^x \neq 0$, $y = n\pi$ for $n \in
  \mathbb{Z}$, but from the real part, since $\ex^x > 0$ and $\cos n\pi = \pm
  1$, we should only have $y = 2n\pi$ for $n \in \mathbb{Z}$. The real part then
  additionally implies that $x=0$ since $\cos 2n\pi = 1$, so $z = 2\zi n \pi$
  for $n \in \mathbb{Z}$.
\end{example}
Note that $|\ex^{\zi z}| \geq 0$ for all $z \in \mathbb{C}$.

\begin{example}
  Solve $\sin z = 0$.\\
  
  With the standard identity for sine with complex arguments, we have
  \begin{equation*}
    \frac{\ex^{\zi z} - \ex^{-\zi z}}{2\zi} = 0.
  \end{equation*}
  Equating real and imaginary parts lead to $z = m\pi$, $m \in \mathbb{Z}$.
\end{example}

The (natural) \Def{logarithm} we define by
\begin{equation}
  \log z = \log|z| + \zi \mbox{arg}z
\end{equation}
to give a complex version of the log function that satisfies the usual rules of
\begin{equation*}
  \log z = \log r\ex^{\zi\theta} = \log r + \zi \theta = \log|z| + \zi \mbox{arg}z.
\end{equation*}
Here we need to choose a \Def{branch}, and we take $\theta\in(-\pi, \pi)$ (the
\Def{principal branch}) to preserve the continuity property, so that $\log z$ is
undefined on the negative real axis, coinciding with the real case.

\begin{example}
  $\log(1-i) = \log\sqrt{2} - \zi(\pi / 4)$
\end{example}

We use $\log z$ to define powers of complex numbers. Recall that for real
numbers we have $x^a = \ex^{a \log a}$ for $a>0$, so for $z,w \in \mathbb{C}$,
we analogously define 
\begin{equation}
  z^w = \ex^{w \log z},
\end{equation}
choosing the principal branch unless otherwise stated.

\begin{example}
  \begin{align*}
    (1 + \zi \sqrt{3})^{1/2} &= \exp\left[\frac{1}{2} \log(1 + \zi \sqrt{3})\right] \\
    &= \exp\left[\frac{1}{2} \left(\log 2 + \zi\frac{\pi}{3}\right)\right] \\
    &= \ex^{\log \sqrt{2}} \ex^{\zi(\pi/6)} \\
    &= \sqrt{2}\ex^{\zi(\pi/6)},
  \end{align*}
  which in this case is could have been gotten from $(1 + \zi \sqrt{3}) = 2\ex^{\zi(\pi/3)}$.
\end{example}

\begin{example}
  \begin{align*}
    (1 - \zi)^{\zi} = \ex^{\zi \log(1-\zi)} = \ex^{\zi (\log \sqrt{2} - \zi \pi/4)} = \ex^{\pi/4} \ex^{\zi \log\sqrt{2}}.
  \end{align*}
\end{example}

We say a complex function $f(z)$ is \Def{complex differentiable at $z = z_0$} if
\begin{equation*}
  \lim_{z\to z_0} \frac{f(z) - f(z_0)}{z - z_0}
\end{equation*}
exists, or that
\begin{equation*}
  \lim_{h\to 0} \frac{f(z + h) - f(z)}{h}
\end{equation*}
exists at $z = z_0$. The derivative is denoted $f'(z)$ as usual.

\begin{example}
  For $f(z) = z^2$,
  \begin{equation*}
    \lim_{h \to 0} \frac{f(z + h) - f(z)}{h} = \lim_{h \to 0} \frac{z^2 + 2hz + h^2 - z^2}{h} = \lim_{h \to 0} 2z + h = 2z.
  \end{equation*}
  $f(z)$ is differentiable everywhere.
\end{example}
The usual rules for differentiation hold (linearity, product rule, chain rule
etc.)

Note that $f(x) = x|x|$ is real differentiable everywhere. $f(z) = z|z|$ on the
other hand is differentiable on the real axis, and complex differentiable at the
origin.

Complex differentiation is a much stronger condition. Recall that for the limit
to exist in the real case, the limit only needs to be equal when approached from
above or below on the real line. In the complex plane however there are an
infinite numbers of cases the limit can be approach, and thus a infinite number
of cases to check. We see that a necessary condition for complex
differentiability is that the limit needs to exist when $z_0$ is approached in
the lines parallel to the real and imaginary axis. If we set $f(z)$ to be
\begin{equation*}
  f(z) = u(x,y) + \zi v(x,y)
\end{equation*}
for some real functions $u$ and $v$, then it turns out that
\begin{align*}
  \lim_{z \to z_0} \frac{f(z) - f(z_0)}{z - z_0} = \ddy{u}{x} + \zi \ddy{v}{x} = \ddy{v}{y} - \zi \ddy{u}{y},
\end{align*}
when we take the limit in the direction parallel to the real and imaginary axis
respectively. It follows that a \emph{necessary} conditions for a function to be
complex differentiable is that
\begin{equation}
  \ddy{u}{x} = \ddy{v}{y} \qquad \textnormal{and} \qquad \ddy{u}{y} = -\ddy{v}{x}.
\end{equation}
These are known as the \Def{Cauchy--Riemann equations}, and we actually have the
following theorem.
\begin{theorem}
  If $f(z)$ is complex differentiable at $z = z_0$, then the Cauchy--Riemann
  equations hold at $(x_0, y_0)$ for $z_0 = x_0 + \zi y_0$, and that
  \begin{equation*}
    f'(z_0) = \left.\left(\ddy{u}{x} + \zi \ddy{v}{x}\right)\right|_{(x_0, y_0)} = \left.\left(\ddy{v}{y} - \zi \ddy{u}{y}\right)\right|_{(x_0, y_0)}.
  \end{equation*}
\end{theorem}
\begin{proof}
  If we approach $z_0$ in a line parallel to the real axis, we have
  \begin{align*}
    f'(z_0) 
      &= \lim_{x \to x_0} \frac{u(x, y_0) + \zi v(x, y_0) - u(x_0, y_0) - \zi v(x_0, y_0)}{x - x_0}\\
      &= \lim_{x \to x_0} \left(\frac{u(x, y_0) - u(x_0, y_0)}{x - x_0} + \zi \frac{v(x, y_0) - v(x_0, y_0)}{x - x_0}\right)\\
      &= \left.\ddy{u}{x}\right|_{(x_0, y_0)} + \zi \left.\ddy{v}{x}\right|_{(x_0, y_0)}.
  \end{align*}
  We have the analogous result when approaching $z_0$ in a line parallel to the
  imaginary axis. \qed
\end{proof}

In actual fact, the Cauchy--Riemann equation holding is a necessary \emph{and}
sufficient condition for complex differentiability.

\begin{theorem}
  Let $f(z) = u(x,y) + \zi v(x,y)$. If the partial derivatives of $u$ and $v$
  exist in some disk centered on $(x_0, y_0)$ and are continuous at $z_0 = x_0 +
  \zi y_0$, and $u$ and $v$ satisfy the Cauchy--Riemann equation, then $f(z)$ is
  complex differentiable at $z_0$. \qedwhite
\end{theorem}

A function is said to be \Def{holomorphic} (or \Def{analytic}) at $z_0$ if it is
complex differentiable on some disk centred at $z_0$. A function is holomorphic
if it is analytic at all points where it is defined.

\begin{example}
  If $f(z) = y^3 - 3\zi x y^2$, fine where $f(z)$ is complex differentiable, and
  compute $f'(z)$.\\
  
  Note that for $u = y^3$ and $v = -3xy^2$,
  \begin{equation*}
    \ddy{u}{x} = 0, \quad \ddy{v}{x} = -3y^2, \quad \ddy{u}{y} = -3y^2, \quad \ddy{v}{y} = -6xy,
  \end{equation*}
  so it is differentiable if $-6xy = 0$ and $3y^2 = 3y^2$, which is only
  satisfied at $x=0$ or $y=0$, i.e. at the co-ordinate axes. In this case $f'(z)
  = -\zi 3y^2$, and that $f(z)$ is nowhere holomorphic.
\end{example}

\begin{theorem}
  Let $f(z)$ be holomorphic and $f(z) = u(x,y) + \zi v(x,y)$. Then $u$ and $v$
  are solutions to Laplace's equation in two dimensions.
\end{theorem}
\begin{proof}
  By Cauchy--Riemann equations and the holomorphic property,
  \begin{equation*}
    \ddy{^2 y}{x^2} = \ddy{}{x}\ddy{u}{x} = \ddy{v}{x\partial y} = \ddy{v}{y\partial x} = \ddy{}{y}\ddy{v}{x} = \ddy{}{y}\left(-\ddy{u}{y}\right) = -\ddy{^2 u}{y^2},
  \end{equation*}
  so that $\partial^2 u / \partial x^2 + \partial^2 u / \partial y^2 = 0$.
  Similarly for $v$. \qed
\end{proof}

Recall that if $f(x)$ is an infinitely differentiable real function, that its
Taylor series about $x = x_0$ is given by
\begin{equation*}
  \sum^\infty_{k=0} \frac{f^{(k)}(x_0)}{k!}(x - x_0)^k.
\end{equation*}
The complex counterpart is then if $f(z)$ is an infinitely complex
differentiable complex function, its Taylor series about $z = z_0$ is
\begin{equation*}
  \sum^\infty_{k=0} \frac{f^{(k)}(z_0)}{k!}(z - z_0)^k.
\end{equation*}
Since derivatives of standard functions are the same in the complex case, their
Taylor series are the same too.

\begin{theorem}
  Let $f(z)$ be complex differentiable. Then its Taylor series converges to
  $f(z)$ for all $z$ where it converges. \qedwhite
\end{theorem}
This implies any complex differentiable function is just a power series.

If we let $\sum^\infty_{n=0} b_n (z - z_0)^n$ be a power series centred on $z =
z_0$, then there exists some $R \in [0, \infty]$ where the power series
\begin{itemize}
  \item converges for $|z - z_0| < R$,
  \item diverges for $|z - z_0| > R$,
  \item inconclusive for $|z - z_0| = R$.
\end{itemize}
$R$ is called the \Def{radius of convergence}, and $\{z : |z - z_0| < R\}$ is
the \Def{disk of convergence}.

To find the disk of convergence we can often use the ratio test.

\begin{example}
  Find the radius of convergence for $f(z) = (1 - z)^{-1}$ around $z_0 = 0$.\\
  
  Recall that $f(z) = \sim^\infty_{n=0} z^n$, then we note that $\lim_{n \to
  \infty} |z^{n+1} / z^n| = |z|$, hence we have convergence if $|z| < 1$ by the
  ratio test, and the radius of convergence is $R = 1$.
\end{example}

\begin{example}
  For $f(z) = \sum^\infty_{n=0} n^2 (z - \zi)^{2n} / 2^n$ as a power series around
  $z_0 = \zi$, by the ratio test,
  \begin{equation*}
    \lim_{n \to \infty} \left|\frac{(n+1)^2 (z - \zi)^{2n+2}}{2^{n+1}}\frac{2^n}{n^2 (z-\zi)^{2n}}\right| = \frac{|z - \zi|^2}{2},
  \end{equation*}
  so we have convergence if $|z - \zi| < \sqrt{2}$, and the radius of
  convergence is $R = \sqrt{2}$.
\end{example}

\begin{theorem}
  If $\sum^\infty_{n=0} a_n (z - z_0)^n$ has radius of convergence $R$ and
  converges to $f(z)$ of its disk of convergence $D$, then $f(z)$ is complex
  differentiable, and $\sum^\infty_{n=0} a_n n (z - z_0)^{n-1}$ converges to
  $f'(z)$ in $D$. \qedwhite
\end{theorem}

\begin{theorem}
  If $\sum^\infty_{n=0} a_n (z - z_0)^n \to f(z)$ in its disk of convergence,
  then $f(z)$ is complex differentiable an infinite number of times, and
  $f^{(n)}(z_0) = n! a_n$.
\end{theorem}
\begin{proof}
  By previous theorem, we have
  \begin{align*}
    f(z)   &= a_0 + a_1 (z - z_0) +   a_2 (z - z_0)^2 + \ldots\\
    f'(z)  &=       a_1           + 2 a_2 (z - z_0)   + \ldots\\
    f''(z) &=                       2\cdot 1 a_2      + \ldots
  \end{align*}
  and so on. Hence the function is infinitely complex differentiable, and
  $f^{(n)}(z_0)$ is as required. \qed
\end{proof}

\begin{example}
  Find the Taylor series of $(1 - z)^{-2}$ about $z = 0$.\\
  
  We see that since $\mathrm{d}/\mathrm{d}z (1 - z)^{-1} = (1 - z)^{-2}$,
  \begin{equation*}
    \frac{1}{(1 - z)^2} = \sum^\infty_{n=1} n z^{n-1}, \qquad |z| < 1.
  \end{equation*}
\end{example}

\begin{example}
  Find the Taylor series for $\cosh(4z^3)$ about $z = 0$.\\
  
  Recall that
  \begin{equation*}
    \cosh y = 1 + \frac{y^2}{2!} + \ldots = \sum^\infty_{n=0} \frac{y^{2n}}{(2n)!},
  \end{equation*}
  so that
  \begin{equation*}
    \cosh(4z^3) = \sum^\infty_{n=0} \frac{(4z^3)^{2n}}{(2n)!} = \sum^\infty_{n=0} \frac{16^n z^{6n}}{(2n)!}, \qquad \forall z \in \mathbb{C}.
  \end{equation*}
\end{example}

\begin{example}
  Find the Taylor series of $z^3 / (1-5z)^2$ about $z = 0$.\\
  
  Using the identity from two examples ago,
  \begin{equation*}
    \frac{1}{(1-5z)^2} = \sum^\infty_{n=1} n (5z)^{n-1},
  \end{equation*}
  so that
  \begin{equation*}
    \frac{z^3}{(1-5z)^2} = \sum^\infty_{n=1} n 5^{n-1} z^{n+2}, \qquad |z| < \frac{1}{5}.
  \end{equation*}
\end{example}

\begin{example}
  Find the Taylor series for $3z(z+1)^{-1}(z-2)^{-1}$ about $z = 0$.\\
  
  First note that the radius of convergence cannot be greater than 1. By partial
  fractions,
  \begin{equation*}
    \frac{3z}{(z+1)(z-2)} = \frac{1}{z+1} + \frac{2}{z-2},
  \end{equation*}
  so that the Taylor series is
  \begin{equation*}
    \sum^\infty_{n=0} (-z)^n - \sum^\infty_{n=0} \left(\frac{z}{2}\right)^n = \sum^\infty_{n=0} \left[(-1)^n - \frac{1}{2^n}\right] z^n, \qquad |z| < 1.
  \end{equation*}
\end{example}

%===============================================================================

\chapter{Integration in the complex plane}

Recall that in the real case we have the \Def{indefinite integral} with
\begin{equation*}
  \int f(x)\; \mathrm{d}x = F(x),
\end{equation*}
where $F(x)$ is the primitive of $f$. We also have the \Def{definite integral}
where, by the fundamental theorem of calculus, gives
\begin{equation*}
  \int^a_b f(x)\; \mathrm{d}x = F(b) - F(a).
\end{equation*}
Although we can generalise the indefinite integral to the complex case, the
definite integral doesn't generalise directly, because we are essentially trying
to talk about a 2-dimensional surface in 4-space. So instead we integrate
complex functions along curves, or contours, in the complex plane.

%-------------------------------------------------------------------------------

\section{Curves in $\mathbb{C}$}

A differentiable curve in $\mathbb{C}$ is a function $\gamma : [a, b] \to
\mathbb{C}$ such that $\gamma(t) = \gamma_1(t) + \zi \gamma_2 (t)$, where
$\gamma_1$ and $\gamma_2$ are real differentiable functions in $t$.

\begin{example}
  One way to generate the unit circle is with
  \begin{equation*}
    \gamma : [0, 1] \to \mathbb{C}, \qquad \gamma(t) = \ex^{2\pi \zi t}.
  \end{equation*}
  Notice here that $\gamma$ is closed, and has a direction characterised by how
  $t$ is parameterised (in this case it is in the positive sense, or in the
  anti-clockwise). In general, a circle centred at $z_0$ with radius $r$ has the
  associated curve
  \begin{equation*}
    \gamma : [0, 1] \to \mathbb{C}, \qquad \gamma(t) = z_0 + r \ex^{2\pi \zi t}.
  \end{equation*}
\end{example}

\begin{example}
  Consider two curves
  \begin{equation*}
    \gamma(t) = t + \zi t, \quad 0 \leq t \leq 1, \qquad \beta(t) = \begin{cases}t, & 0 \leq t \leq 1, \\ 1 + (t-1)\zi & 1 \leq t \leq 2. \end{cases}
  \end{equation*}
  \TODO diagram
  
  Both curves connect the origin to $z = 1 + \zi$, but the path is difference.
  $\beta(t)$ here is piecewise differentiable. One question of course is whether
  the path matters (see later). In general, a vector from $z_0$ to $z_1$ may be
  parameterised as $\gamma(t) = z_0 + t (z_1 - z_0)$, for $t \in [0, 1]$.
\end{example}

%-------------------------------------------------------------------------------

\subsection{Contour integrals}

To integrate along the curve $z = \gamma(t)$ with $t\in[a, b]$, we have from
chain rule that $\mathrm{d}z = \gamma'(t)\; \mathrm{d}t$, so that
\begin{equation*}
  \int_{\gamma} f(z)\; \mathrm{d}z = \int^b_a f(\gamma(t)) \gamma'(t)\; \mathrm{d}t,
\end{equation*}
where the latter is as before since we are dealing with a function of a real
variable.

\begin{example}
  Compute the contour integrals of the following:
  \begin{enumerate}
    \item $f(z) = z^2$, $\gamma(t) = \ex^{\zi \pi t}$, $t \in [0, 1]$\\
    
    The path is the upper unit semi-circle, and we have
    \begin{equation*}
      \int_\gamma f(z)\; \mathrm{d}z = \zi \pi \int^1_0 \ex^{3\zi\pi t}\; \mathrm{d}t = -\frac{2}{3}.
    \end{equation*}
    
    \item $f(z) = z^2$, $\gamma(t) = \ex^{-\zi \pi t}$, $t \in [0, 1]$\\
    
    The path is the lower unit semi-circle, and we have
    \begin{equation*}
      \int_\gamma f(z)\; \mathrm{d}z = -\zi \pi \int^1_0 \ex^{-3\zi\pi t}\; \mathrm{d}t = -\frac{2}{3}.
    \end{equation*}
    
    Notice here the integral has the same value as the previous part, which in
    this case is not a coincidence.
    
    \item $f(z) = \overline{z}$, $\gamma(t) = 1 + \zi t$, $t \in [0, 2]$\\
    
    We have
    \begin{equation*}
      \int_\gamma f(z)\; \mathrm{d}z = \zi \int^2_0 (1 - \zi t)\; \mathrm{d}t = 2 + 2\zi.
    \end{equation*}
  \end{enumerate}
\end{example}

A \Def{contour} is a continuous curve made up a finite number of differentiable
curves. The contour itself does not need to be differentiable although the
individual pieces should. The integral of $f(z)$ along a contour is then the sum
of integrals along each individual differentiable curve.

\begin{proposition}
  We have the following properties for contour integrals:
  \begin{enumerate}
    \item Linearity, where
    \begin{equation*}
      \int_\gamma (\alpha f(z) + \beta g(z))\; \mathrm{d}z = \alpha \int_\gamma f(z)\; \mathrm{d}z + \beta \int_\gamma g(z))\; \mathrm{d}z.
    \end{equation*}
    
    \item If contours $\gamma_1$ and $\gamma_2$ have the same track in
    $\mathbb{C}$ and transverse it in the same direction, then
    \begin{equation*}
      \int_{\gamma_1} f(z)\; \mathrm{d}z = \int_{\gamma_2} f(z)\; \mathrm{d}z.
    \end{equation*}
    
    \item If $\gamma : [a,b] \to \mathbb{C}$ and $\mu : [-b, -a] \to \mathbb{C}$
    where $\mu(t) = \gamma(-t)$, i.e. $\mu$ is the `reverse' of $\gamma$, then
    \begin{equation*}
      \int_{\mu} f(z)\; \mathrm{d}z = -\int_{\gamma} f(z)\; \mathrm{d}z.
    \end{equation*}
    
    \item We have the inequality
    \begin{equation*}
      \left|\int_\gamma f(z)\; \mathrm{d}z\right| \leq \int_\gamma |f(\gamma(t))| \cdot |\gamma'(t)|\; \mathrm{d}t \leq \textnormal{length}(\gamma) \cdot \max_{\gamma} |f(\gamma(t))|.
    \end{equation*}
    
    \item (Fundamental Theorem of Calculus) Let $F(z)$ be holomorphic on an open
    set $D \subset \mathbb{C}$, and $F'(z) = f(z)$. Then for any contour $\gamma
    : [a,b] \to D$ with end points $z_0 = \gamma(a)$ and $z_1 = \gamma(b)$, we
    have
    \begin{equation*}
      \int_\gamma f(z)\; \mathrm{d}z = F(z_1) - F(z_0).
    \end{equation*}
  \end{enumerate}
\end{proposition}

\begin{proof}
  \begin{enumerate}
    \item Since we have linearity when the integrals are real, this one is just
    by definition:
    \begin{align*}
      \int_\gamma (\alpha f(z) + \beta g(z))\; \mathrm{d}z 
        &= \int^b_a [\alpha f(\gamma(t)) + \beta g(\gamma(t))]\gamma'(t)\; \mathrm{d}t\\
        &= \alpha \int^b_a f(\gamma(t)) \gamma'(t)\; \mathrm{d}t + \beta \int^b_a g(\gamma(t)) \gamma'(t)\; \mathrm{d}t\\
        & =\alpha \int_\gamma f(z)\; \mathrm{d}z + \beta \int_\gamma g(z))\; \mathrm{d}z.
    \end{align*}
    
    \item Let $\gamma_k : [a_k, b_k] \to \mathbb{C}$ with $k=1,2$, and assume
    that $\gamma_2(h(t)) = \gamma_1(t)$. Then taking a substitution $u = h(t)$
    and judicious use of chain rule gives
    \begin{align*}
      \int_{\gamma_2} f(z)\; \mathrm{d}z &= \int^{b_2}_{a_2} f(\gamma_2(u)) \gamma_2'(u)\; \mathrm{d}u \\
      &= \int^{b_1}_{a_1} f(\gamma_2(h(t))) \gamma_2'(h(t)) h'(t)\; \mathrm{d}t \\
      &= \int^{b_1}_{a_1} f(\gamma_1(t)) \gamma_1'(t)\; \mathrm{d}t \\
      &= \int_{\gamma_1} f(z)\; \mathrm{d}z.
    \end{align*}
    
    \item As in previous case but use different limits.
    
    \item Let $\theta = \mbox{arg} \int_\gamma f(z)\; \mathrm{d}z$, then
    \begin{align*}
      \left|\int_\gamma f(z)\; \mathrm{d}z\right| &= \ex^{-\zi \theta} \int_\gamma f(z)\; \mathrm{d}z \\
      &= \int_\gamma \ex^{-\zi \theta} f(z)\; \mathrm{d}z \\
      &= \mbox{Re} \left(\int_\gamma \ex^{-\zi \theta} f(z)\; \mathrm{d}z\right) \\
      &= \mbox{Re} \left(\int^b_a \ex^{-\zi \theta} f(\gamma(t)) \gamma'(t)\; \mathrm{d}t\right)\\
      &\leq \int^b_a \left|\ex^{-\zi \theta} f(\gamma(t)) \gamma'(t)\right|\; \mathrm{d}t \\
      &= \int^b_a |f(\gamma(t))| \cdot |\gamma'(t)|\; \mathrm{d}t\\
      &\leq \textnormal{length}(\gamma) \cdot \max_{\gamma} |f(\gamma(t))|.
    \end{align*}
    
    \item Let $F(\gamma(t)) = u(t) + \zi v(t)$, where $u$ and $v$ are real
    functions. By the chain rule, $u'(t) + \zi v'(t) = F'(\gamma(t))\gamma'(t)$,
    so
    \begin{align*}
      \int_\gamma f(z)\; \mathrm{d}z = \int^b_a F'(\gamma(t))\gamma'(t)\; \mathrm{d}t = \int^b_a [u'(t) + \zi v'(t)]\; \mathrm{d}t = F(b) - F(a).
    \end{align*}
  \end{enumerate}
\end{proof}

\begin{example}
  Let $\gamma(t) = R\ex^{\zi t}$, $t\in[0, 2\pi]$, then
  \begin{equation*}
    \int_\gamma \frac{1}{z}\; \mathrm{d}z = \int^{2\pi}_0 \frac{1}{R\ex^{\zi t}} R\zi \ex^{\zi t}\; \mathrm{d}t = 2\pi \zi,
  \end{equation*}
  and this is because the primitive is not well-defined at $z=0$.
\end{example}

\begin{theorem}[Path Independent Theorem]
  Let $f$ be continuous on an open connected set $D \subset \mathbb{C}$. Then
  the following statements are equivalent to each other:
  \begin{enumerate}
    \item integrals are path independent;
    \item if $\gamma$ is a closed curve in $D$, then $\oint_\gamma f(z)\; \mathrm{d}z = 0$;
    \item there exists a primitive $F(z)$ of $f(z)$ where $F'(z) = f(z)$,
    defined globally on $D$.
  \end{enumerate}
\end{theorem}

\begin{proof}
  We show that 1 is equivalent to 2, and 2 is equivalent to 3, so 1 is then
  equivalent to 3 by default.
  
  (1 $\Leftrightarrow$ 2) Suppose $\Gamma$ is a closed curve consisting of some
  arbitrary closed simple curves $\gamma_{0,1}$ as illustrated \TODO diagram
  
  Then
  \begin{equation*}
    \oint_\Gamma f(z)\; \mathrm{d}z = \left(\int_{\gamma_0} + \int_{-\gamma_1}\right) f(z)\; \mathrm{d}z = \left(\int_{\gamma_0} - \int_{\gamma_1}\right) f(z)\; \mathrm{d}z.
  \end{equation*}
  Since the integrals are path independent, we have $\int_{\Gamma} f(z)\;
  \mathrm{d}z = 0$. Conversely, if the integral is zero by assumption, since
  $\gamma_{0,1}$ are arbitrary, this implies path independence.
  
  (2 $\Leftrightarrow$ 3) Assuming there is a primitive, then the fundamental
  theorem of calculus implies that since we have the existence of the primitive,
  we have $\int_\gamma f(z)\; \mathrm{d}z = F(z_1) - F(z_0)$ regardless of path,
  so if $z_1 = z_0$ then $\oint_\gamma f(z)\; \mathrm{d}z = 0$.
  
  Conversely, let $z_0$ be any fixed point on $D$, and $z$ be any other point on
  $D$. Since $D$ is open and connected, the contour $\gamma$ joining $z_0$ to
  $z$ exists. Defining then $F(z) = \int_\gamma f(\zeta)\; \mathrm{d}\zeta$, by
  the assumption of of path independence, $F(z)$ is well-defined, and by the
  estimation property, $F'(z) = f(z)$, so there exists a primitive. \qed
\end{proof}

%-------------------------------------------------------------------------------

\subsection{Cauchy's theorem and residue theorem}

Cauchy's theorem is one of the centre pieces of complex analysis. Before the
statement, we need an extra tool from topology regarding simple closed curves.

\begin{theorem}[Jordan curve theorem]
  Let $\gamma$ be a simple closed contour, i.e. no self-intersections except at
  the end points. Then the compliment of $\gamma$ in $\mathbb{C}$ is the
  disjoint union of exactly two sets, where exactly one is bounded. \qedwhite
\end{theorem}

Intuitively this says that a simple closed curve splits the space into an
outside and inside (trivial as it may sound rigourously proofing this is not so
obvious...)

\begin{theorem}[Cauchy's theorem]
  Let $f(z)$ be holomorphic on and inside a simple closed curve $\gamma$. Then
  $\oint_\gamma f(z)\; \mathrm{d}z = 0$.
\end{theorem}

\begin{proof}
  Let $f = u + \zi v$ for real $u$ and $v$, then using Green's theorem (since
  the resulting integrands are real)
  \begin{align*}
    \oint_\gamma f(z)\; \mathrm{d}z &= \oint_\gamma (u + \zi v)(\mathrm{d}x + \zi \mathrm{d}y) \\
      &= \oint_\gamma \left[(u\; \mathrm{d}x - v\; \mathrm{d}y) + \zi(u\; \mathrm{d}y + v\; \mathrm{d}x) \right]\\
      &= \iint_A \left[\left(-\ddy{u}{y} -\ddy{v}{x}\right) + \zi \left(\ddy{u}{x} - \ddy{v}{y}\right)\right] = 0.
  \end{align*}
  The latter quality to zero is because $f(z)$ is holomorphic, so $u$ and $v$
  satisfies the Cauchy--Riemann equations, and thus the partials are continuous
  and equal. \qed
\end{proof}

\begin{example}

\end{example}
%-------------------------------------------------------------------------------

\section{Residue theorem}

%-------------------------------------------------------------------------------

\section{Applications for real integrals}

%===============================================================================

\chapter{More analysis topics}

%===============================================================================

\chapter{Conformal mapping}

%===============================================================================

%%%%%%%%%%%%%%%%%%%%%%%%%%%%%%%%%%%%%%%%%

% r.5 contents
%\tableofcontents

%\listoffigures

%\listoftables

% r.7 dedication
%\cleardoublepage
%~\vfill
%\begin{doublespace}
%\noindent\fontsize{18}{22}\selectfont\itshape
%\nohyphenation
%Dedicated to those who appreciate \LaTeX{} 
%and the work of \mbox{Edward R.~Tufte} 
%and \mbox{Donald E.~Knuth}.
%\end{doublespace}
%\vfill

% r.9 introduction
% \cleardoublepage

%%%%%%%%%%%%%%%%%%%%%%%%%%%%%%%%%%%%%%%%%
% actual useful crap (normal chapters)
\mainmatter

%\part{Basics (?)}


%\backmatter

%\bibliography{refs}
\bibliographystyle{plainnat}

%\printindex

\end{document}

