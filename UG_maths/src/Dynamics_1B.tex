\documentclass[letter-paper]{tufte-book}

%%
% Book metadata
\title{Dynamics 1H}
\author[]{Inusuke Shibemoto}
%\publisher{Research Institute of Valinor}

%%
% If they're installed, use Bergamo and Chantilly from www.fontsite.com.
% They're clones of Bembo and Gill Sans, respectively.
\IfFileExists{bergamo.sty}{\usepackage[osf]{bergamo}}{}% Bembo
\IfFileExists{chantill.sty}{\usepackage{chantill}}{}% Gill Sans

%\usepackage{microtype}
\usepackage{amssymb}
\usepackage{amsmath}
%%
% For nicely typeset tabular material
\usepackage{booktabs}

%% overunder braces
\usepackage{oubraces}

%% 
\usepackage{xcolor}
\usepackage{tcolorbox}

\newtcolorbox[auto counter,number within=section]{derivbox}[2][]{colback=TealBlue!5!white,colframe=TealBlue,title=Box \thetcbcounter:\ #2,#1}                                                          

\makeatletter
\@openrightfalse
\makeatother

%%
% For graphics / images
\usepackage{graphicx}
\setkeys{Gin}{width=\linewidth,totalheight=\textheight,keepaspectratio}
\graphicspath{{figs/}}

% The fancyvrb package lets us customize the formatting of verbatim
% environments.  We use a slightly smaller font.
\usepackage{fancyvrb}
\fvset{fontsize=\normalsize}

\usepackage[plain]{fancyref}
\newcommand*{\fancyrefboxlabelprefix}{box}
\fancyrefaddcaptions{english}{%
  \providecommand*{\frefboxname}{Box}%
  \providecommand*{\Frefboxname}{Box}%
}
\frefformat{plain}{\fancyrefboxlabelprefix}{\frefboxname\fancyrefdefaultspacing#1}
\Frefformat{plain}{\fancyrefboxlabelprefix}{\Frefboxname\fancyrefdefaultspacing#1}

%%
% Prints argument within hanging parentheses (i.e., parentheses that take
% up no horizontal space).  Useful in tabular environments.
\newcommand{\hangp}[1]{\makebox[0pt][r]{(}#1\makebox[0pt][l]{)}}

%% 
% Prints an asterisk that takes up no horizontal space.
% Useful in tabular environments.
\newcommand{\hangstar}{\makebox[0pt][l]{*}}

%%
% Prints a trailing space in a smart way.
\usepackage{xspace}
\usepackage{xstring}

%%
% Some shortcuts for Tufte's book titles.  The lowercase commands will
% produce the initials of the book title in italics.  The all-caps commands
% will print out the full title of the book in italics.
\newcommand{\vdqi}{\textit{VDQI}\xspace}
\newcommand{\ei}{\textit{EI}\xspace}
\newcommand{\ve}{\textit{VE}\xspace}
\newcommand{\be}{\textit{BE}\xspace}
\newcommand{\VDQI}{\textit{The Visual Display of Quantitative Information}\xspace}
\newcommand{\EI}{\textit{Envisioning Information}\xspace}
\newcommand{\VE}{\textit{Visual Explanations}\xspace}
\newcommand{\BE}{\textit{Beautiful Evidence}\xspace}

\newcommand{\TL}{Tufte-\LaTeX\xspace}

% Prints the month name (e.g., January) and the year (e.g., 2008)
\newcommand{\monthyear}{%
  \ifcase\month\or January\or February\or March\or April\or May\or June\or
  July\or August\or September\or October\or November\or
  December\fi\space\number\year
}


\newcommand{\urlwhitespacereplace}[1]{\StrSubstitute{#1}{ }{_}[\wpLink]}

\newcommand{\wikipedialink}[1]{http://en.wikipedia.org/wiki/#1}% needs \wpLink now

\newcommand{\anonymouswikipedialink}[1]{\urlwhitespacereplace{#1}\href{\wikipedialink{\wpLink}}{Wikipedia}}

\newcommand{\Wikiref}[1]{\urlwhitespacereplace{#1}\href{\wikipedialink{\wpLink}}{#1}}

% Prints an epigraph and speaker in sans serif, all-caps type.
\newcommand{\openepigraph}[2]{%
  %\sffamily\fontsize{14}{16}\selectfont
  \begin{fullwidth}
  \sffamily\large
  \begin{doublespace}
  \noindent\allcaps{#1}\\% epigraph
  \noindent\allcaps{#2}% author
  \end{doublespace}
  \end{fullwidth}
}

% Inserts a blank page
\newcommand{\blankpage}{\newpage\hbox{}\thispagestyle{empty}\newpage}

\usepackage{units}

% Typesets the font size, leading, and measure in the form of 10/12x26 pc.
\newcommand{\measure}[3]{#1/#2$\times$\unit[#3]{pc}}

% Macros for typesetting the documentation
\newcommand{\hlred}[1]{\textcolor{Maroon}{#1}}% prints in red
\newcommand{\hangleft}[1]{\makebox[0pt][r]{#1}}
\newcommand{\hairsp}{\hspace{1pt}}% hair space
\newcommand{\hquad}{\hskip0.5em\relax}% half quad space
\newcommand{\TODO}{\textcolor{red}{\bf TODO!}\xspace}
\newcommand{\na}{\quad--}% used in tables for N/A cells
\providecommand{\XeLaTeX}{X\lower.5ex\hbox{\kern-0.15em\reflectbox{E}}\kern-0.1em\LaTeX}
\newcommand{\tXeLaTeX}{\XeLaTeX\index{XeLaTeX@\protect\XeLaTeX}}
% \index{\texttt{\textbackslash xyz}@\hangleft{\texttt{\textbackslash}}\texttt{xyz}}
\newcommand{\tuftebs}{\symbol{'134}}% a backslash in tt type in OT1/T1
\newcommand{\doccmdnoindex}[2][]{\texttt{\tuftebs#2}}% command name -- adds backslash automatically (and doesn't add cmd to the index)
\newcommand{\doccmddef}[2][]{%
  \hlred{\texttt{\tuftebs#2}}\label{cmd:#2}%
  \ifthenelse{\isempty{#1}}%
    {% add the command to the index
      \index{#2 command@\protect\hangleft{\texttt{\tuftebs}}\texttt{#2}}% command name
    }%
    {% add the command and package to the index
      \index{#2 command@\protect\hangleft{\texttt{\tuftebs}}\texttt{#2} (\texttt{#1} package)}% command name
      \index{#1 package@\texttt{#1} package}\index{packages!#1@\texttt{#1}}% package name
    }%
}% command name -- adds backslash automatically
\newcommand{\doccmd}[2][]{%
  \texttt{\tuftebs#2}%
  \ifthenelse{\isempty{#1}}%
    {% add the command to the index
      \index{#2 command@\protect\hangleft{\texttt{\tuftebs}}\texttt{#2}}% command name
    }%
    {% add the command and package to the index
      \index{#2 command@\protect\hangleft{\texttt{\tuftebs}}\texttt{#2} (\texttt{#1} package)}% command name
      \index{#1 package@\texttt{#1} package}\index{packages!#1@\texttt{#1}}% package name
    }%
}% command name -- adds backslash automatically
\newcommand{\docopt}[1]{\ensuremath{\langle}\textrm{\textit{#1}}\ensuremath{\rangle}}% optional command argument
\newcommand{\docarg}[1]{\textrm{\textit{#1}}}% (required) command argument
\newenvironment{docspec}{\begin{quotation}\ttfamily\parskip0pt\parindent0pt\ignorespaces}{\end{quotation}}% command specification environment
\newcommand{\docenv}[1]{\texttt{#1}\index{#1 environment@\texttt{#1} environment}\index{environments!#1@\texttt{#1}}}% environment name
\newcommand{\docenvdef}[1]{\hlred{\texttt{#1}}\label{env:#1}\index{#1 environment@\texttt{#1} environment}\index{environments!#1@\texttt{#1}}}% environment name
\newcommand{\docpkg}[1]{\texttt{#1}\index{#1 package@\texttt{#1} package}\index{packages!#1@\texttt{#1}}}% package name
\newcommand{\doccls}[1]{\texttt{#1}}% document class name
\newcommand{\docclsopt}[1]{\texttt{#1}\index{#1 class option@\texttt{#1} class option}\index{class options!#1@\texttt{#1}}}% document class option name
\newcommand{\docclsoptdef}[1]{\hlred{\texttt{#1}}\label{clsopt:#1}\index{#1 class option@\texttt{#1} class option}\index{class options!#1@\texttt{#1}}}% document class option name defined
\newcommand{\docmsg}[2]{\bigskip\begin{fullwidth}\noindent\ttfamily#1\end{fullwidth}\medskip\par\noindent#2}
\newcommand{\docfilehook}[2]{\texttt{#1}\index{file hooks!#2}\index{#1@\texttt{#1}}}
\newcommand{\doccounter}[1]{\texttt{#1}\index{#1 counter@\texttt{#1} counter}}

\newcommand{\studyq}[1]{\marginnote{Q: #1}}

\hypersetup{colorlinks}% uncomment this line if you prefer colored hyperlinks (e.g., for onscreen viewing)

% Generates the index
\usepackage{makeidx}
\makeindex

\setcounter{tocdepth}{3}
\setcounter{secnumdepth}{3}

%%%%%%%%%%%%%%%%%%%%%%%%%%%%%%%%%%%%%%%%%%%%%%%%%%%%%%%%%%%%%%
% custom commands

\newtheorem{theorem}{\color{pastel-blue}Theorem}[section]
\newtheorem{lemma}[theorem]{\color{pastel-blue}Lemma}
\newtheorem{proposition}[theorem]{\color{pastel-blue}Proposition}
\newtheorem{corollary}[theorem]{\color{pastel-blue}Corollary}

\newenvironment{proof}[1][Proof]{\begin{trivlist}
\item[\hskip \labelsep {\bfseries #1}]}{\end{trivlist}}
\newenvironment{definition}[1][Definition]{\begin{trivlist}
\item[\hskip \labelsep {\bfseries #1}]}{\end{trivlist}}
\newenvironment{example}[1][Example]{\begin{trivlist}
\item[\hskip \labelsep {\bfseries #1}]}{\end{trivlist}}
\newenvironment{remark}[1][Remark]{\begin{trivlist}
\item[\hskip \labelsep {\bfseries #1}]}{\end{trivlist}}

\hyphenpenalty=5000

% more pastel ones
\xdefinecolor{pastel-red}{rgb}{0.77,0.31,0.32}
\xdefinecolor{pastel-green}{rgb}{0.33,0.66,0.41}
\definecolor{pastel-blue}{rgb}{0.30,0.45,0.69} % crayola blue
\definecolor{gray}{rgb}{0.2,0.2,0.2} % dark gray

\xdefinecolor{orange}{rgb}{1,0.45,0}
\xdefinecolor{green}{rgb}{0,0.35,0}
\definecolor{blue}{rgb}{0.12,0.46,0.99} % crayola blue
\definecolor{gray}{rgb}{0.2,0.2,0.2} % dark gray

\xdefinecolor{cerulean}{rgb}{0.01,0.48,0.65}
\xdefinecolor{ust-blue}{rgb}{0,0.20,0.47}
\xdefinecolor{ust-mustard}{rgb}{0.67,0.52,0.13}

%\newcommand\comment[1]{{\color{red}#1}}

\newcommand{\dy}{\partial}
\newcommand{\ddy}[2]{\frac{\dy#1}{\dy#2}}

\newcommand{\ab}{\boldsymbol{a}}
\newcommand{\bb}{\boldsymbol{b}}
\newcommand{\Bb}{\boldsymbol{B}}
\newcommand{\cb}{\boldsymbol{c}}
\newcommand{\db}{\boldsymbol{d}}
\newcommand{\eb}{\boldsymbol{e}}
\newcommand{\Eb}{\boldsymbol{E}}
\newcommand{\Fb}{\boldsymbol{F}}
\newcommand{\lb}{\boldsymbol{l}}
\newcommand{\Lb}{\boldsymbol{L}}
\newcommand{\nb}{\boldsymbol{n}}
\newcommand{\rb}{\boldsymbol{r}}
\newcommand{\tb}{\boldsymbol{t}}
\newcommand{\ub}{\boldsymbol{u}}
\newcommand{\vb}{\boldsymbol{v}}
\newcommand{\xb}{\boldsymbol{x}}
\newcommand{\wb}{\boldsymbol{w}}
\newcommand{\yb}{\boldsymbol{y}}

\newcommand{\Xb}{\boldsymbol{X}}

\newcommand{\ex}{\mathrm{e}}
\newcommand{\zi}{{\rm i}}

\newcommand\Real{\mbox{Re}} % cf plain TeX's \Re and Reynolds number
\newcommand\Imag{\mbox{Im}} % cf plain TeX's \Im

\newcommand{\zbar}{{\overline{z}}}

\newcommand\Def[1]{\textbf{#1}}

\newcommand{\qed}{\hfill$\blacksquare$}
\newcommand{\qedwhite}{\hfill \ensuremath{\Box}}

%%%%%%%%%%%%%%%%%%%%%%%%%%%%%%%%%%%%%%%%%%%%%%%%%%%%%%%%%%%%%%
% some extra formatting (hacked from Patrick Farrell's notes)
%  https://courses.maths.ox.ac.uk/node/view_material/4915
%

% chapter format
\titleformat{\chapter}%
  {\huge\rmfamily\itshape\color{pastel-red}}% format applied to label+text
  {\llap{\colorbox{pastel-red}{\parbox{1.5cm}{\hfill\itshape\huge\color{white}\thechapter}}}}% label
  {1em}% horizontal separation between label and title body
  {}% before the title body
  []% after the title body

% section format
\titleformat{\section}%
  {\normalfont\Large\itshape\color{pastel-green}}% format applied to label+text
  {\llap{\colorbox{pastel-green}{\parbox{1.5cm}{\hfill\color{white}\thesection}}}}% label
  {1em}% horizontal separation between label and title body
  {}% before the title body
  []% after the title body

% subsection format
\titleformat{\subsection}%
  {\normalfont\large\itshape\color{pastel-blue}}% format applied to label+text
  {\llap{\colorbox{pastel-blue}{\parbox{1.5cm}{\hfill\color{white}\thesubsection}}}}% label
  {1em}% horizontal separation between label and title body
  {}% before the title body
  []% after the title body

%%%%%%%%%%%%%%%%%%%%%%%%%%%%%%%%%%%%%%%%%%%%%%%%%%%%%%%%%%%%%%%%%%%%%%%%%%%%%%%%

\begin{document}

% Front matter
%\frontmatter

% r.3 full title page
%\maketitle

% v.4 copyright page

\chapter*{}

\begin{fullwidth}

\par \begin{center}{\Huge Dynamics 1H}\end{center}

\vspace*{5mm}

\par \begin{center}{\Large typed up by B. S. H. Mithrandir}\end{center}

\vspace*{5mm}

\begin{itemize}
  \item \textit{Last compiled: \monthyear}
  \item Adapted from notes of R. J. Johnson Durham
  \item Was part of the Durham Core B module given in the first year. Includes
  solving Newton's equations as a differential equation, and various other
  well-known differentiation equations.
  \item[]
  \item \TODO diagrams
\end{itemize}

\par

\par Licensed under the Apache License, Version 2.0 (the ``License''); you may not
use this file except in compliance with the License. You may obtain a copy
of the License at \url{http://www.apache.org/licenses/LICENSE-2.0}. Unless
required by applicable law or agreed to in writing, software distributed
under the License is distributed on an \smallcaps{``AS IS'' BASIS, WITHOUT
WARRANTIES OR CONDITIONS OF ANY KIND}, either express or implied. See the
License for the specific language governing permissions and limitations
under the License.
\end{fullwidth}

%===============================================================================

\chapter{Kinematics}

Mechanics is a model of the motion and interaction of everyday size objects
according to Newton's Laws.

\begin{example}[Newton's Laws]
  \begin{enumerate}
    \item A body stays at rest or in motion of constant velocity unless acted on
    by an external influence.
    
    \item A body's rate of change of momentum equals to the net force acting on
    it.
    
    \item Action and reaction are equal and opposite.
  \end{enumerate}
\end{example}

The aim is to identify forces acting on a body and solve the equations of
motion. This is a differential equation for the body's position as a function of
time.

We will normally assume a point mass object with no extent, which is a good
approximation when the object size is much less than the size of its trajectory.
\Def{Position} of the particle is described by its displacement vector from a given
origin $\rb$. Its \Def{velocity} is then given by its time-derivative
\begin{equation*}
  \dot{\rb} = \frac{\mathrm{d}\rb}{\mathrm{d}t}
  = \lim_{h\to0} \frac{\rb(t+h) - \rb(t)}{h}.
\end{equation*}
$\dot{\rb}$ is the tangent vector pointing in the direction of motion, with
speed $|\dot{\rb}(t)|$. If the reference axes are $\eb_{x,y,z}$ (assumed to be
unit length), then
\begin{equation*}
  \rb(t) = x(t)\eb_x + y(t)\eb_y + z(t)\eb_z, \qquad
  \vb = \dot{\rb} = \dot{x}\eb_x + \dot{y}\eb_y + \dot{z}\eb_z,
\end{equation*}
with speed $|\vb| = \sqrt{\vb\cdot\vb} = \sqrt{\dot{x}^2 + \dot{y}^2 +
\dot{z}^2}$.
\begin{example}
  \begin{equation*}
    \rb(t) = (t, 1, t^2) \qquad \Rightarrow \qquad \vb = (1, 0, 2t), \qquad
    |\vb| = \sqrt{1 + 4t^2}.
  \end{equation*}
\end{example}
\Def{Momentum} is defined to be $m\vb$, where $m$ is the mass of the
object. \Def{Acceleration} is defined to be $\ab = \dot{\vb}$.

Even if the speed is constant, acceleration can be non-zero if direction
changes.
\begin{example}
  For circular motion around the origin, we have, for $\rho$ the radius and
  $\theta$ the angle,
  \begin{equation*}
    \rb = (\rho\cos\theta, \rho\sin\theta),\qquad \theta = \theta(t).
  \end{equation*}
  So then
  \begin{equation*}
    \vb = \rho\dot{\theta}(-\sin\theta, \cos\theta),\qquad
    |\vb| = \rho|\dot{\theta}| = \textnormal{constant}
  \end{equation*}
  for uniform circular motion. However,
  \begin{equation*}
    |\ab| = \rho\dot{\theta} 
    = -\rho\dot{\theta}^2 (\cos\theta, \sin\theta) \neq 0,
  \end{equation*}
  so the acceleration is of magnitude $\rho\dot{\theta}^2$ in the direction
  $-\rb$, i.e., towards the origin.
\end{example}

Physical dimensions (e.g., length, mass and time) of a correct equation must
balance. Sometimes insight may be gained from just looking at the available
dimensions to the problem.
\begin{example}
  For a pendulum, the period of its oscillation has dimensions $T$, so the
  formula for a pendulum should only depend on $T$. The other available
  dimensions in the problem are the mass $M$, the length $L$ and the
  gravitational acceleration $g$; it may be shown that $T\sim\sqrt{L/g}$ (in
  fact the constant of proportionality is $2\pi$).
\end{example}

Suppose we have a particle with position $\rb$ with respect to a fixed origin,
but an observer is $\boldsymbol{R}$ from the origin. Then the motion with
respect to the observer is
\begin{equation*}
  \tilde{\rb} = \rb - \boldsymbol{R},\qquad
  \tilde{\vb} = \vb - \dot{\boldsymbol{R}},\qquad
  \tilde{\ab} = \ab - \ddot{\boldsymbol{R}}.
\end{equation*}
Newton's first law defines an \Def{inertial frame of reference} which is
un-accelerated, so that acceleration in the frame is zero if no forces act. An
inertial frame is relative (e.g., a car moving on Earth sees Earth as
stationary, whilst Earth moving in a galaxy sees the galaxy as stationary). We
can apply the remaining two laws in inertial frames.

\Def{Forces} is the influence that tends to make a body move, and this is
modelled by a vector with respect to the inertial frame. Newton's second law
then says that
\begin{equation*}
  \frac{\mathrm{d}}{\mathrm{d}t} m\vb = \Fb(\rb, \vb, \cdots, t),
\end{equation*}
where $\Fb$ is the sum of all forces. If $\Fb = 0$, (a \Def{free
particle}), then $m\vb$ is constant and so $\vb$ is constant, consistent with
the first law. It is our aim to solve this second order differential equation.
This is generally difficult to do, and there are a few things we can do to
simplify it: (i) choose axes to simplify the vectors involves; (ii) use
conservation laws to reduce the problem (e.g., energy, momentum, angular
momentum etc.)

%-------------------------------------------------------------------------------

\section{Sample one dimensional problems}

\begin{enumerate}
  \item Pushing a glass of Duff beer along the bar in Moe's Tavern at constant
  velocity. This implies that $\Fb = 0$, but, horizontally, we must have a
  balance between pushing and friction, whilst in the vertical we have weight
  and normal reaction balancing. The horizontal motion may be studied by solving
  $F = ma$.
  
  \item A cake of mass $m$ slides along a horizontal table, slowed by frictional
  force $F = -b\ex^{a v}$, where $v = v(t)$, and the other parameters are
  constant. If initial speed is $v_0$, how long does it take for the cake to
  come to rest?
  
  For $\eb_x$ to be the direction of travel, we have
  \begin{equation*}
    m\frac{\mathrm{d}v}{\mathrm{d}t}\eb_x = -b\ex^{av}\eb_x.
  \end{equation*}
  The ODE is separable with
  \begin{equation*}
    \int m\ex^{-av}\, \mathrm{d}v = -\int b\, \mathrm{d}t.
  \end{equation*}
  With $v(0) = v_0$, the general solution is
  \begin{equation*}
    \frac{m}{a}\ex^{-av} = bt + \frac{m}{a}\ex^{-av_0}.
  \end{equation*}
  At $t=T$, $v=0$, so equating gives
  \begin{equation*}
    T = \frac{m}{ab} \left(1 - \ex^{-av_0}\right).
  \end{equation*}
  
  \item For vertical motion under gravity $g$, a mass $m$ feels a force of $mg$
  downwards. Choosing the axis to be vertically upward from the ground, with
  surface where $z=0$, we have
  \begin{equation*}
    \frac{\mathrm{d}}{\mathrm{d}t} m\dot{z} = -mg.
  \end{equation*}
  At $t=0$, $z=z_0$ and $\vb = v_0$, and the general solution is
  \begin{equation*}
    z(t) = -\frac{gt^2}{2} + v_0 t + z_0.
  \end{equation*}
  How far up does it go if $v_0 = 15\ \mathrm{m}\ \mathrm{s}^{-1}$? The maximum
  point is where $\vb = 0$, with excursion $z - z_0$. From the velocity
  equation, we end up with $v_0 = gt$ so $t = v_0/g$, and substituting into the
  displacement equation, we have
  \begin{equation*}
    z - z_0 = -\frac{g}{2}\left(\frac{v_0}{g}\right)^2
      + v_0\left(\frac{v_0}{g}\right) = \frac{v_0^2}{2g},
  \end{equation*}
  so with $g = 10\ \mathrm{m}\ \mathrm{s}^{-2}$, we have $z - z_0 = 11.5\
  \mathrm{m}$.
  
  \item A parachutist drops from rest ($v(0) = 0$) at a great height, and
  gravity $\Fb_w = mg\eb_z$ acts on him downwards with air resistance $\Fb_r =
  -k\vb$ opposing motion. We thus have
  \begin{equation*}
    \frac{\mathrm{d}}{\mathrm{d}t} mv = mg - kv,
  \end{equation*}
  which results in the linear ODE
  \begin{equation*}
    \frac{\mathrm{d}v}{\mathrm{d}t} + \frac{k}{m}v = g.
  \end{equation*}
  With the integrating factor $\mu = \ex^{kt/m}$, we obtain
  \begin{equation*}
    v\ex^{kt/m} = \frac{gm}{k}\ex^{kt/m} + c.
  \end{equation*}
  $v(0) = 0$ sets $c = -gm/k$, so
  \begin{equation*}
    v(t) = \frac{gm}{k}\left(1 - \ex^{-kt/m}\right).
  \end{equation*}
  As $t\to\infty$, $v\to mg/k$, known as the \Def{terminal speed}.
  
  
  \item Restoring force which is proportional to displacement, acting such that
  the force is directed towards the centre $x=0$ (e.g., a spring say). We have
  \begin{equation*}
    \frac{\mathrm{d}}{\mathrm{d}t} mv = -kx \qquad\Rightarrow\qquad
    \ddot{x} + \frac{k}{m}x = 0.
  \end{equation*}
  Since the particular integral is zero, the complementary function is the full
  answer. The auxillary equation is $\lambda^2 + k/m = 0$, so $\lambda =
  \pm\sqrt{k/m}\zi$, and so
  \begin{equation*}
    x(t) = \alpha\cos\sqrt{\frac{k}{m}}t + \beta\sin\sqrt{\frac{k}{m}}t,
  \end{equation*}
  and the solution is periodic with period $T = 2\pi/\omega = 2\pi/\sqrt{k/m}$.
  This is \Def{simple harmonic motion}, and it can be shown that the
  amplitude of maximum excursion is $|x|_{\max} = \sqrt{\alpha^2 + \beta^2}$.

  \item For ballistics, a particle moves near the Earth's surface under its own
  weight. Here, $\Fb = -mg\eb_y$, so
  \begin{equation*}
    \frac{\mathrm{d}\vb}{\mathrm{d}t} = -g\eb_y.
  \end{equation*}
  Without loss of generality, choosing $\rb(0) = 0$ and $\vb(0) = v_0
  (\cos\alpha, \sin\alpha)$, the solution is
  \begin{equation*}
    \rb = -\frac{gt^2}{2}\eb_y + \vb_0 t 
    = (v_0 t \cos\alpha, v_0 t \sin\alpha - gt^2/2).
  \end{equation*}
  
  Suppose the time taken to return to $y=0$ is $t=T$. Then
  \begin{equation*}
    v_0 T \sin\alpha - \frac{gT^2}{2} = 
    T\left(v_0 \sin\alpha - \frac{gT}{2}\right) = 0,
  \end{equation*}
  so we have $T=0$ or $T=(2/g)v_0 \sin\alpha$.
  
  How far does it travel? This is the horizontal displacement until $y=0$ for
  the second time, so
  \begin{equation*}
    \rb(T) = \frac{2v_0^2}{g}\sin\alpha\cos\alpha = \frac{v_0^2}{g}\sin2\alpha.
  \end{equation*}
  
  The equation of trajectory is given by $y(x)$. With $t = x/(v_0\cos\alpha)$,
  we have
  \begin{equation*}
    y = v_0\left(\frac{x}{v_0\cos\alpha}\right)\sin\alpha 
      - \frac{g}{2}\left(\frac{x}{v_0\cos\alpha}\right)^2
      = x\tan\alpha - \frac{1}{2}\frac{g}{v_0^2\cos^2\alpha}x^2.
  \end{equation*}
  This is an equation for a parabola, as expected. The turning point is where it
  is highest, given by
  \begin{equation*}
    0 = \frac{\mathrm{d}y}{\mathrm{d}x} 
    = \tan\alpha - \frac{g}{v_0^2\cos\alpha}x,
  \end{equation*}
  which is
  \begin{equation*}
    x = \frac{\sin\alpha\cos\alpha v_0^2}{g} = \frac{v_0^2}{2g}\sin2\alpha
    =\frac{1}{2}\rb(t),
  \end{equation*}
  which is also expected.
  
  Suppose we want to choose than angle $\alpha$ to hit some target located at
  $(x_0, y_0)$, given that we launch at speed $v_0$. Thus
  \begin{equation*}
    y_0 = x_0 \tan\alpha - \frac{gx_0^2}{v_0^2}(1+\tan^2\alpha).
  \end{equation*}
  This is a quadratic in $\tan\alpha$, and we see that: (i) if there are two
  roots, then there are two possibilities; (ii) a repeated root implies target
  is at maximum range; (iii) complex roots implies target is out of range.
  
  \item Air resistance is modelled by $-\Lambda\vb$. Then
  \begin{equation*}
    \frac{\mathrm{d}\vb}{\mathrm{d}t} = \boldsymbol{g} - \frac{\Lambda}{m}\vb.
  \end{equation*}
  This is a linear equation with integrating factor $\ex^{(\Lambda/m)t}$. So
  that the solution is
  \begin{equation*}
    \vb = \frac{\boldsymbol{g}}{\alpha} 
      + \left(\vb_0-\frac{\boldsymbol{g}}{\alpha}\right)\ex^{-\alpha t},\qquad 
    \alpha = \frac{\Lambda}{m},\qquad \vb(0) = \vb_0.
  \end{equation*}
  As $t\to\infty$, $\vb\to m\boldsymbol{g}/\Lambda$, which is the terminal
  velocity along $\boldsymbol{g}$.
  
  \item A particle of mass $m$ with electric charge $q$ moving at velocity $\vb$
  through an electric field $\Eb$ and magnetic field $\Bb$ feels a force
  \begin{equation*}
    \Fb = m\frac{\mathrm{d}\vb}{\mathrm{d}t} = q\Eb + q(\vb\times\Bb),
  \end{equation*}
  known as the \Def{Lorentz force}. Suppose that $\Eb$ and $\vb$ are
  parallel and constant along $\eb_z$, then
  \begin{equation*}
    \frac{\mathrm{d}}{\mathrm{d}t} m\vb = qE\eb_z + qB\vb\times\eb_z,
  \end{equation*}
  which implies that
  \begin{equation*}
    m\ddot{x} = qB\dot{y},\qquad m\ddot{y} = -qB\dot{x},\qquad
    m\ddot{z} = qE.
  \end{equation*}
  The last equation implies that
  \begin{equation*}
    z(t) = \frac{1}{2}t^2 \frac{qE}{m} + c_1 t + c_2.
  \end{equation*}
  Taking $\rb(0) = 0$ and $\dot{\rb}(0) = v_0 \eb_x$, we have
  \begin{equation*}
    z(t) = \frac{1}{2}\frac{qE}{m} t^2.
  \end{equation*}
  The second equation may be integrated once and, using the initial conditions,
  \begin{equation*}
    m\dot{y} = -qBx.
  \end{equation*}
  From the first equation, this gives
  \begin{equation*}
    m\ddot{x} = qB\dot{y} = -q^2 B^2 \frac{x}{m} \qquad \Rightarrow \qquad
    \ddot{x} = \left(\frac{qB}{m}\right)^2 x = 0,
  \end{equation*}
  which, after using the initial conditions, has as solutions
  \begin{equation*}
    x(t) = \frac{m v_0}{qB} \sin\left(\frac{qB}{m} t\right).
  \end{equation*}
  Returning to the second equation, this yields
  \begin{equation*}
    y(t) = \frac{m v_0}{qB} \left(\cos\left(\frac{qB}{m} t\right) - 1\right).
  \end{equation*}
  We see that
  \begin{equation*}
    \left(y(t) + \frac{m v_0}{qB}\right)^2 + x^2(t) =
    \left(\frac{m v_0}{qB}\right),
  \end{equation*}
  which is an equation of a circle. Moreover, $x(t)$ and $y(t)$ is independent
  of $\Eb$, so $x$ and $y$ motion is influences by the magnetic field, whilst
  the $z$ motion is influences by the electric field. Since $z(t)\sim t^2$, the
  motion is a spiral of increasing pitch.
\end{enumerate}

%-------------------------------------------------------------------------------

\section{Energy}

Consider the special case where $\mathrm{d}(m\dot{x})/\mathrm{d} t = F(x)$. This
can be integrated, by multiplying both sides by $\dot{x}$ and integrating in
time as
\begin{equation*}
  \frac{1}{2}m \int m \frac{\mathrm{d}}{\mathrm{d}t}\dot{x}^2\, \mathrm{d}t
  = \int F(x) \frac{\mathrm{d}x}{\mathrm{d}t}\, \mathrm{d}t =
  \int F(x)\, \mathrm{d}x.
\end{equation*}
Integrating this then results in
\begin{equation*}
  E = \frac{1}{2}mv^2 - \int F(x)\, \mathrm{d}x,
\end{equation*}
where $E$ is the total energy (conserved when there is no forcing or
dissipation), the kinetic energy and the potential energy. The equation of
motion may be then be thought of as studying the evolution of the energy.

\begin{example}
  Motion under weight near the Earth's surface on a vertical line is given by
  \begin{equation*}
    m\ddot{x} = -mg,
  \end{equation*}
  so the potential energy $V(x) = mgx$. Energy is zero when $x = \dot{x} = 0$,
  so integrating the equation with this conditions fixing the constants leads to
  \begin{equation*}
    \frac{1}{2}m\dot{x}^2 + mgx = E,
  \end{equation*}
  where $E$ is a constant. Then, for two different states, they should satisfy
  \begin{equation*}
    \frac{1}{2}v_1^2 + gx_1 = \frac{1}{2}v_2^2 + gx_2.
  \end{equation*}
\end{example}

In general for a potential energy $V(x)$, this results in
\begin{equation*}
  \dot{x}^2 = \frac{2}{m} (E - V(x)) \qquad \Rightarrow \qquad
  \frac{\mathrm{d}x}{\mathrm{d}t} = \pm \sqrt{\frac{2}{m}(E - V(x))}.
\end{equation*}
The ODE is separable, which results in
\begin{equation*}
  \int\frac{\mathrm{d}x}{\sqrt{E - V(x)}} = \pm \sqrt{\frac{2}{m}}(t - t_0).
\end{equation*}

\begin{example}
  Suppose $F(x) = -kx$, so we have simple harmonic motion, with $k > 0$ and the
  potential being $V(x) = kx^2 / 2$, then the LHS integral is
  \begin{equation*}
    \int\frac{\mathrm{d}x}{\sqrt{E} - kx^2 /2} = 
    \frac{1}{\sqrt{E}}\int\frac{\mathrm{d}x}{\sqrt{E} - kx^2 / (2E)}.
  \end{equation*}
  Substituting with $u^2 = kx^2 / (2E)$ gives $u = \sqrt{k/(2E)}$ and
  $\mathrm{d}x = \sqrt{2E/k}\mathrm{d}u$, so
  \begin{equation*}
    \sqrt{\frac{2E}{k}}\frac{1}{\sqrt{E}} \int \frac{\mathrm{d}u}{1-u^2}
    = \sqrt{\frac{2}{k}}\arcsin\sqrt\frac{k}{2E}x,
  \end{equation*}
  and the general solution is
  \begin{equation*}
    x(t) = \pm \sqrt{\frac{2E}{k}}\sin\left(\sqrt{\frac{k}{m}}(t - t_0)\right).
  \end{equation*}
\end{example}

In three dimensions, a similar approach by taking the scalar product of
$\dot{\rb}$ gives
\begin{equation*}
  E = \frac{1}{2}m|\dot{\rb}|^2 - \int \Fb\cdot\mathrm{d}\rb.
\end{equation*}
\begin{example}
  For $\rb(t) = t\eb_x + \eb_y + t^2 \eb_z$, this gives the kinetic energy as
  $m(1+4t^2)/2$.
\end{example}

With potential energy $V(\rb) = -\int \Fb \cdot \mathrm{d}\rb$, the relation is
that $\Fb(\rb) = -\nabla V(\rb)$. $\Fb$ is \Def{conservative} is this
relation holds.

\begin{example}
  For gravity near Earth's surface,
  \begin{equation*}
    \Fb = m\boldsymbol{g} = -mg\eb_z,
  \end{equation*}
  so then $\dy V/\dy z = mg$ and $V(x,y,z) = mgz$.
\end{example}

\Def{Constraint forces} are forces that are always perpendicular to motion,
i.e., $\Fb\cdot\rb = 0$, and so does not contribute towards energy conservation.
For example, in a pendulum, string tension does not appear in the energy
equation.

\begin{example}
  A particle slides down the surface of a smooth sphere from rest at the top.
  Where does it leave the surface? (i.e., where is $F_n = 0$?)
  
  Let $\rho$ be the radius of the sphere and $\theta$ the angle from the north
  pole. Assuming that energy is conserved, then the energy balance is
  \begin{equation*}
    \frac{1}{2}m\rho^2\dot{\theta}^2 = mg\rho(1 - \cos\theta).
  \end{equation*}
  The radial force balance is [\Def{centrifugal force}]
  \begin{equation*}
    F_n = mg\cos\theta - m\rho\dot{\theta}^2.
  \end{equation*}
  Substituting one into the other results in
  \begin{equation*}
    F_n = mg(3\cos\theta - 2),
  \end{equation*}
  so the particle leaves when $\theta = \arccos(2/3)\approx 48^\circ$.
\end{example}

%-------------------------------------------------------------------------------

\section{Motion near equilibrium}

At equilibrium point $\rb = \rb_0$, $\Fb(\rb_0) = 0$. For $\rb(t) = \rb_0 =
\textnormal{constant}$, the derivatives are zero for all time, so $\Fb(\rb_0) =
0$ by Newton's second law. Conversely, if $\Fb(\rb_0) = 0$, then $\rb = \rb_0 =
\textnormal{constant}$, and obeys Newton's second law.

Consider the equilibrium in one dimension only, and assume forces are
conservative with $F(x) = -\mathrm{d}V/\mathrm{d}x$.. Therefore the equilibrium
point $x=x_0$ is a turning point of $V(x)$.

\begin{example}
  In SHM, $V(x) = kx^2 / 2$, and equilibrium point is at $x=0$ where it is a
  minimum of $V(x)$. Near the equilibrium point $x_0$ then, we can make a Taylor
  expansion about $x_0$, which results in
  \begin{equation*}
    m\ddot{x} = F(x) = F(x_0) + (x - x_0) F'(x_0) + \ldots.
  \end{equation*}
  Since $F(x_0) = 0$ and $F(x) = -V'(x)$ by definition, we have
  \begin{equation*}
    m\ddot{x} = -(x - x_0) V''(x_0) + \ldots.
  \end{equation*}
  Let $x = x_0 + \epsilon$, then $m\ddot{e} + V''(x_0) + \epsilon \approx 0$. If
  $V''(x_0) > 0$, $x_0$ is a stable equilibrium point, and particle executes SHM
  around $x_0$ of the form
  \begin{equation*}
    x(t) = x_0 + A\sin\omega t + B\cos\omega t, \qquad
    \omega^2 = \frac{V''(x_0)}{m}.
  \end{equation*}
  On the other hand, if $V''(x_0) < 0$, $x_0$ is unstable, and particle moves
  away from it as
  \begin{equation*}
    x(t) = x_0 + C\ex^{\lambda t} + D\ex^{-\lambda t}, \qquad
    \lambda^2 = -\frac{V''(x_0)}{m}.
  \end{equation*}
  If $V''(x_0)$ then we need to retain higher order terms.
\end{example}

\begin{example}
\begin{enumerate}
  \item Suppose $V(x) = x(x^2 - 1)$, then
  \begin{equation*}
    F(x) = -\frac{\mathrm{d}V}{\mathrm{d}x} = -3x^2 - 1 = 0 \qquad \Rightarrow
    \qquad x = \pm\frac{1}{\sqrt{3}}.
  \end{equation*}
  Then since $V''(x) = 6x$, $V''(1/\sqrt{3}) = 2\sqrt{3} > 0$ so $x =
  1/\sqrt{3}$ is a stable equilibrium. For an object of mass $m$, $\omega^2 =
  2\sqrt{3}/m$, so the period of oscillation is $T = 2\pi / \sqrt{2\sqrt{3}/m}$.
  
  \item A particle of mass $m$ moves along the $x$-axis in potential $V(x) =
  k(x^2 - 2) \ex^{-2x}$, $k > 0$. The equilibrium point is found as
  \begin{equation*}
    0 = V'(x) = 2k\ex^{-2x}(2-x)(1+x)
  \end{equation*}
  so it is $x = -1, 2$. Then either by drawing the potential or working out the
  second derivative, $x=-1$ is the stable equilibrium point. To find the period,
  let $x = -1 + \epsilon$, then $m\ddot{x} = -V'(x)$ leads to
  \begin{equation*}
    m\ddot{\epsilon} = -2k(3-\epsilon)\epsilon \ex^{-2\epsilon + 2}.
  \end{equation*}
  Noting that $\ex^{-2\epsilon} = 1 - 2\epsilon + 2\epsilon^2+\ldots$, then the
  linear term remaining is
  \begin{equation*}
    m\ddot{\epsilon} = -2k(3)\epsilon\ex^2 1 = -6k\ex^2 \epsilon.
  \end{equation*}
  Then $\omega^2 = 6k\ex^2 / m$ and $T = (2\pi/\ex)\sqrt{m/(6k)}$.
  
  What is the escape velocity away from the potential well? We require the
  initial kinetic energy to be greater than the potential difference involves
  (at $x=2$) to escape to $+\infty$, i.e.,
  \begin{equation*}
    E = mv_0^2 / 2 + V(-1) > V(2)
  \end{equation*}
  giving $v_0 > \sqrt{2(V(-2) - V(-1))/m}$, so
  \begin{equation*}
    v_0 > \sqrt{\frac{2k}{m}(2\ex^{-4} + \ex^2)}.
  \end{equation*}
  Observe that this initial speed of the particle can be towards either
  direction, and it will still speed off to $+\infty$ eventually.
\end{enumerate}
\end{example}

%-------------------------------------------------------------------------------

\section{Simple pendulum}

For a pendulum of length $\ell$ with mass $m$ on the end hang freely, ignoring
tension force. Let $\theta$ be the angle from the vertical, then the energy is
\begin{equation*}
  E = \mbox{PE} + \mbox{KE} = \frac{1}{2} m(\ell\dot{\theta})^2 - m g \ell (1 -
  \cos\theta),
\end{equation*}
since the vertical displacement from $\theta = 0$ is given by $z = \ell(1 -
\cos\theta)$, so that $V(z) = mgz = mg\ell(1-\cos\theta)$, and that for angular
velocity $\dot{\theta}$ on a circle of radius $\ell$, $\vb = \ell\dot{\theta}
\hat{\eb}_\theta$. We see that $V(\theta)$ has $\theta = 0, \pm\pi$ as
stationary points, only the former of which is stable.

From $\dot{E} = 0$, the equations of motion are
\begin{equation*}
  \ddot{\theta} + \frac{g}{\ell}\sin\theta = 0.
\end{equation*}
For $|\theta| \ll 1$, $\sin\theta \approx \theta + O(\theta^3)$, so equation
reduces to simple harmonic motion with solutions
\begin{equation*}
  \theta(t) = A\cos\omega t + B \sin\omega t, \qquad \omega^2 = \frac{g}{\ell}.
\end{equation*}

%-------------------------------------------------------------------------------

\section{Damped vibrations}

Assume an object of mass $m$ hangs from a spring where the spring has some
natural length, and denote $x(t)$ as the extension. For a spring with spring
constant $K$, $x$ evolves as
\begin{equation*}
  m\ddot{x} = mg - Kx.
\end{equation*}
Since there is no damping and $F(x) = mg - Kx$ is conservation, the equilibrium
point is $x(t) = mg/k$. Let $u(t) = x(t) - mg/k$, then the perturbation $u(t)$
evolves as $m\ddot{u} + Ku = 0$, so the solution with $x(0) = 0$ is 
\begin{equation*}
  x(t) = \frac{mg}{k} + \sqrt{\frac{2E}{k}} \sin\sqrt{\frac{K}{m}}t,
\end{equation*}
with $E$ the energy at equilibrium. The period of oscillation is thus $T =
2\pi/\omega = 2\pi\sqrt{m/K}$.

Suppose there is friction proportional to velocity, so that
\begin{equation*}
  m\ddot{x} = mg - Kx - \Lambda \dot{x},
\end{equation*}
where $\Lambda > 0$ is a damping constant. Then an equilibrium solution is still
$x(t) = mg/K$ but with $\dot{x} = 0$, so the perturbation $u(t)$ now evolves as
\begin{equation*}
  m\ddot{u} = -Ku - \Lambda \dot{u}.
\end{equation*}
Let $\mu = \Lambda / m$ and $\sigma = K / m$, then we obtain the second order
ODE
\begin{equation*}
  \ddot{u} + \mu\dot{u} + \sigma u = 0.
\end{equation*}
The auxiliary equation has solutions
\begin{equation*}
  \lambda_\pm = -\frac{\mu}{2} \pm \sqrt{\left(\frac{\mu}{2}\right)^2 - \sigma},
\end{equation*}
so that, depending on friction and spring constant, yields real or complex
solutions.

Suppose damping is small, with $(\mu/2)^2 - \sigma = -\omega^2 < 0$, then
$\lambda_\pm = -\mu/2 \pm \zi \omega$and solution is
\begin{equation*}
  u(t) = \ex^{-\mu t/2}(A \cos\omega t + B \sin \omega t),
\end{equation*}
which is a decaying oscillation with period
\begin{equation*}
  T = \frac{2\pi}{\omega} = \frac{2\pi}{\sqrt{\sigma - (\mu/2)^2}}.
\end{equation*}

If damping is large with $(\mu/2)^2 - \sigma = k^2 > 0$, then $\lambda_\pm =
-(\mu/2) \pm k < 0$, so solutions are
\begin{equation*}
  u(t) = A\ex^{\lambda_+ t} + B\ex^{\lambda_- t}, \qquad
  x(t\to\infty) \to \frac{mg}{K},
\end{equation*}
and there are no oscillations at all.

For critical damping, the characteristic equation becomes degenerate and so
solutions take the form
\begin{equation*}
  u(t) = A\ex^{\lambda t} + B t \ex^{\lambda t} \to 0
\end{equation*}
as $t\to\infty$, where $\lambda = -\mu/2$. There is a maximum of one
oscillation. From a practical point of view, shock absorbers on vehicles are set
slightly sub-critical to give a smooth return to equilibrium position soon, so
it is ready for the next obstacle. Having it too sub-critical can be a problem
(e.g. resonance).

%-------------------------------------------------------------------------------

\section{Forcing and resonance}

Consider the forced case of
\begin{equation*}
  m\ddot{x} = mg - Kx - \Lambda \dot{x} + D\sin\alpha t.
\end{equation*}
Normalising the constants by $m$, and defining $u = x - mg/K$, we obtain
\begin{equation*}
  \ddot{u} + \mu \dot{u} + \sigma u = C\sin\alpha t.
\end{equation*}
The complementary function always damps to zero as $t\to\infty$, with two
arbitrary constants fixed by initial conditions. This \Def{transient}
response vanishes rapidly. On the other hand, the particular integral has the
form
\begin{equation*}
  u_p = A\cos\alpha t + B\sin\alpha t,
\end{equation*}
so that
\begin{equation*}
  -\alpha^2 B - \mu\alpha A + \sigma B = C, \qquad
  -\alpha^2 + \mu\alpha B + \sigma A = 0,
\end{equation*}
which results in
\begin{equation*}
  A = -\frac{\mu\alpha C}{(\sigma - \alpha^2)^2 + (\mu\alpha)^2}, \qquad
  B = \frac{(\sigma - \alpha^2)C}{(\sigma - \alpha^2)^2 + (\mu\alpha)^2}.
\end{equation*}
Hence, independent of initial conditions, the \Def{steady state} response
is
\begin{equation*}
  u(t\to\infty) = \frac{C}{(\sigma - \alpha^2)^2 + (\mu\alpha)^2} \left[
  (\sigma - \alpha^2)\sin\alpha t - \mu\alpha\cos\alpha t\right],
\end{equation*}
which, upon using trigonometric identities, may be written as
\begin{equation*}
  u(t\to\infty) = \frac{C \sin(\alpha t - \phi)}{\sqrt{(\sigma - \alpha^2)^2 +
  (\mu\alpha)^2}}, \qquad \tan\phi = \frac{\mu\alpha}{\sigma - \alpha^2}.
\end{equation*}
The \Def{phase difference} $\phi$ is that between the input and output.

If the forcing frequency $\alpha$ changes then the response amplitude is maximum
if $\alpha^2 = \sigma$, independent of $\mu$. There is thus
\Def{resonance} if $\alpha = \sqrt{\sigma} = \sqrt{K/m}$, the medium's
\Def{natural frequency}. At resonance with forcing amplitude $C$, the
response amplitude is
\begin{equation*}
  \frac{C}{\mu\alpha} = \frac{mC}{\Lambda \alpha}, \qquad \phi = \frac{\pi}{2}.
\end{equation*}
Note that resonant response may be large if $\Lambda \alpha$ is small.

%-------------------------------------------------------------------------------

\section{Angular momentum}

The radial component of the equation of motion is given by
\begin{equation*}
  \rb\times\frac{\mathrm{d}}{\mathrm{d}t}(m\vb) = \rb \times \Fb.
\end{equation*}
So then defining the \Def{angular momentum} about $\rb = 0$ to be $\Lb =
\rb + m\vb$,
\begin{equation*}
  \frac{\mathrm{d}\Lb}{\mathrm{d}t}
    = m\dot{\rb}\times\vb + m\rb\times\dot{\vb}
    = m\vb\times\vb + \rb\times\Fb = \rb\times\Fb.
\end{equation*}
The force $\Fb$ exerts a \Def{torque} or \Def{moment} equal to
$\rb\times\Fb = 0$ about $\rb = 0$. (Angular momentum about $\rb_0$ is $(\rb -
\rb_0)\times m\vb$.)

\begin{example}
  For $\rb = (t, 1, t^2)$, $\vb = (1, 0, 2t)$, so $\Lb = m(2t, -t^2, -1)$.
\end{example}

Note that angular momentum is associated with \Def{rotation}, as non-zero
$\Lb$ needs $\vb$ and $\rb$ to be non-parallel, and that $|\Lb|\sim |\vb|$.
Addition, we have $|\Lb| = m|\rb|~|\vb|\sin\alpha$, and $\Lb\cdot\rb =
\Lb\cdot\vb = 0$ by definition.

%===============================================================================

\chapter{Central forces}

A \Def{central force} involves attraction or repulsion from a fixed point.
With that force as the origin, $\Fb$ is parallel to $\rb$ so that $\rb\times\Fb
= 0$, and thus $\Lb$ is conserved, and motion is in two-dimensions. For $\Fb =
f(r)\hat{\eb}_r$ where $r = |\rb|$, $f(r) < 0$ represents attraction while $f(r) >
0$ represents repulsion.

\begin{example}
  \begin{enumerate}
    \item Gravitational attraction between two masses $M$ and $m$, with force on
    $m$ due to $M$ is given by Newton's law
    \begin{equation*}
      \Fb = -m\frac{GM}{r^2}\hat{\eb}_r.
    \end{equation*}
    Near Earth's surface, $r = R = 6400\ \mathrm{km}$ and $g = 9.81\ \mathrm{m}\
    \mathrm{s}^{-2}$, so that $g = GM/R^2$. If $m\ll M$ then the bigger mass is
    essentially unaffected by the interaction, and its centre can be taken as $\rb
    = 0$.
    
    \item In electrostatic interactions, by Coulomb's law, two charges $q$ and
    $Q$ feel the Coulomb force
    \begin{equation*}
      \Fb = \frac{qQ}{r^2}\hat{\eb}_r.
    \end{equation*}
  \end{enumerate}
\end{example}
Not all forces results in angular momentum being conserved. Friction and the
electromagnetic force $\Fb = q\vb\times \Bb$ are two examples.

Suppose that $m\ddot{r} = \Fb$ and $\Fb$ is a central force, so that $\Lb =
\vb\times(m\rb)$ is constant. Since $\Fb$ is along $\rb$, $\ddot{\rb}$ is along
$\rb$ also. Choose $\Lb\sim\hat{\eb}_z$ and $\hat{\eb}_{x,y}$ fixed. Since $\Fb$
is a central force, this means it is more natural to take polar co=ordinates
with $\hat{\eb}_r$ along $\rb$ and $\hat{\eb}_\theta$ perpendicular. So
then
\begin{equation*}
  \hat{\eb}_r = \cos\theta\hat{\eb}_x + \sin\theta \hat{\eb}_y,\qquad
  \hat{\eb}_\theta = -\sin\theta\hat{\eb}_x + \cos\theta \hat{\eb}_y.
\end{equation*}
The disadvantage here is that $\hat{\eb}_{r,\theta}$ are time dependent.

Note then the position, velocity and acceleration are then respectively defined
by (after repeated use of chain rule)
\begin{equation*}
  \rb = r\hat{\eb}_r, \qquad
  \vb = \dot{r}\hat{\eb}_r + r\dot{\theta}\hat{\eb}_\theta,\qquad
  \ab = (\ddot{r} - r\dot{\theta}^2)\hat{\eb}_r + 
    (2\dot{r}\dot{\theta} + r\ddot{\theta})\hat{\eb}_\theta.
\end{equation*}
Note that (i) circulation motion is where $r$ is constant, and the expressions
for $\vb$ and $\ab$ agree since the derivatives of $r$ vanish, and (ii) the
kinetic energy is given by $E_k = m(\dot{r}^2 + r^2\dot{\theta}^2)/2$.

Furthermore, since $\Fb = f(r)\hat{\eb}_r = m\ab$, the equations separate into a
radial and transverse part given by
\begin{equation*}
  m(\ddot{r} - r\dot{\theta}^2) = f(r), \qquad
  m(2\dot{r}\dot{\theta} + r\ddot{\theta}) = 0.
\end{equation*}
Multiplying the transverse part by $r$, we see that $r^2\dot{\theta}$ is a
constant of motion. Noting that
\begin{equation*}
  \Lb = m\rb\times\dot{\rb} = mr\hat{\eb}_r \times(\dot{r}\hat{\eb}_r +
  r\dot{\theta}\hat{\eb}_\theta)
\end{equation*}
with $\hat{\eb}_r \times \hat{\eb}_\theta = 0$, then $\Lb =
mr^2\dot{\theta}\hat{\eb}_z$, and so $|\Lb|$ is a constant of motion, which we
know already. The problem may be be reduced to a one dimensional problem since
$\dot{\theta} = |\Lb|/(mr^2)$. Indeed, the radial part now becomes
\begin{equation*}
  m\left(\ddot{r} - \frac{L^2}{m^2 r^3}\right) = f(r), \qquad
  \Fb = \frac{L^2}{mr^3} + f(r),
\end{equation*}
where $L = |\Lb|$.

\begin{example}
  A point mass alien of mass $m$ moves under the influence of a force $mc^2a^4 /
  r^5$, attracting her to some fixed point $O$ at distance $r$. She was set in
  motion at a distance $r=a$ from $O$, with speed $u$ perpendicular to $\rb$.
  Show her orbit is circular if $u = c$.
  
  Note that for a circular orbit we required $f(r) < 0$. Then, notice that $\Lb
  = mr^2\dot{\theta} = mr(r\dot{\theta})$ so that $L = mau$. Orbit is circular
  if $\rb = \ab$ is constant and that $\ddot{r} = 0$. Substituting in gives
  \begin{equation*}
    \ddot{r} - \frac{(mau)^2}{m^2 a^3} = -\frac{c^2 a^4}{a^5},
  \end{equation*}
  and so $\ddot{r} = 0$ if and only if $c = u$.
\end{example}

For a central force, note that since $m\ddot{\rb} = f(r)\hat{\eb}_r$,
\begin{equation*}
  \frac{1}{2}m\dot{\rb}\cdot\dot{\rb} = \int f(r) \hat{\eb}_r \cdot \dot{\rb}\,
    \mathrm{d}t.
\end{equation*}
For $\dot{\rb} = \dot{r}\hat{\eb}_r + r\dot{\theta}\hat{\eb}_{\theta}$ and
defining $V(r) = -\int f(r)\, \mathrm{d}r$ so that $f(r) =
-\mathrm{d}V/\mathrm{d}r$, the energy may be defined as
\begin{equation*}
  E = \frac{1}{2} m\left(\dot{r}^2 + r^2\dot{\theta}^2\right) + V(r).
\end{equation*}

In summary:
\begin{enumerate}
  \item $F(\rb) = f(r)\hat{\eb}_r$ results in planar motion perpendicular to the
  angular momentum vector $\Lb = mr^2\dot{\theta}$;
  \item the energy is
  \begin{equation*}\begin{aligned}
    E &= \frac{1}{2} m\left(\dot{r}^2 + r^2\dot{\theta}^2\right) + V(r)\\
      &= \frac{1}{2} m\left(\dot{r}^2 + \frac{L^2}{m^2}{r^2}\right) + V(r)
  \end{aligned}\end{equation*}
  where $f(r) = -\mathrm{d}V/\mathrm{d}r$ and $\dot{\theta} = L / mr^2$, with
  $L$ and $E$ fixed by the initial vectors $\rb(0)$ and $\vb(0)$.
\end{enumerate}
These are separable first order ODEs for $r(t)$ and $\theta(t)$, describing
motion in a plane perpendicular to $\Lb$.

\begin{example}
  Suppose our alien was set in motion at distance $r=a$ with transverse speed
  $\sqrt{5/8} c$. Find the turning points of her subsequent orbit and show that
  $r(t)\leq q$.
  
  Since $f(r) = mc^2a^4/r^5$, the potential is
  \begin{equation*}
    V(r) = -\int f(r)\, \mathrm{d}r = -\frac{Mc^2 a^4}{4r^4}.
  \end{equation*}
  With $\vb(0) = \sqrt{5/8}c \hat{\eb}_{\theta}$,
  \begin{equation*}
    E = \frac{1}{2}m\frac{5}{8}c^2 - \frac{mc^2a^4}{4a^4} = \frac{1}{16}mc^2,
    \qquad L = ma\sqrt{\frac{5}{8}}c.
  \end{equation*}
  Since $\dot{\theta} = L/(mr^2)$, this gives
  \begin{equation*}
    \dot{\theta} = \frac{ac}{r^2}\sqrt{\frac{5}{8}}.
  \end{equation*}
  The energy equation $E = (m/2)(\dot{r}^2 + r^2\dot{\theta}^2) + V(r)$
  then reads
  \begin{equation*}
    \frac{1}{6}mc^2 = \frac{1}{2}m \left(\dot{r}^2 + 
      r^2\frac{a^2 c^2}{r^4}\sqrt{\frac{5}{8}}\right) - \frac{mc^2 a^4}{4r^4}.
  \end{equation*}
  This re-arranges to
  \begin{equation*}
    \dot{r}^2 = \frac{1}{8}c^2 \left(1 - \frac{a^2}{r^2}\right)
      \left(1 - \frac{4a^2}{r^2}\right).
  \end{equation*}
  At the turning points, $\dot{r}=0$, so the turning points are $r=a, 2a$.
  However, $\dot{r}^2 < 0$ for $a < r < 2a$ and so $r > a$ is not permitted.
\end{example}

%-------------------------------------------------------------------------------

\section{Motion under gravity}

Motion under this particular central force follows the inverse square law where
\begin{equation*}
  \Fb = -F\frac{Mm}{r^2} \hat{\eb}_r \qquad\Rightarrow\qquad
  V(r) = -G \frac{Mm}{r}.
\end{equation*}
So over distances with respect to Earth's radius,
\begin{equation*}
  E = \frac{1}{2}mv^2 - G\frac{Mm}{r} = \textnormal{constant}.
\end{equation*}
If $E<0$, we have \Def{bound orbits}, otherwise we have
\Def{orbits}.

\begin{example}
  To work out the escape speed from Earth, suppose a particle of mass $m$ start
  at $r = R$ with speed $v$ has energy
  \begin{equation*}
    E = \frac{1}{2}mv^2 - G\frac{Mm}{R} = \textnormal{constant},
  \end{equation*}
  where $M$ is the Earth's mass. To reach $r = \infty$, we need the kinetic term
  to be larger than the potential term. For $GM = gR^2$, $v\geq \sqrt{2gR}$ is
  required, which equates to about $v = 11.2\ \mathrm{km}\ \mathrm{s}^{-1}$. It
  is of interest to note that for any massive body, $v_{E} = \sqrt{2gR}$ is
  dependent on the planet's size and gravitational field strength, but not on
  the escaper's mass.
\end{example}

\begin{example}
  Show that if $f(r) = -G(Mm/r^2)$, there is a solution to the equation of
  motion with $r(t) = r_0$, $v(t) = v_0$ describing steady motion around a
  circle. Express $L$ in terms of $(r_0, v_0$ and show that $E < 0$.
  
  Radial equation of motion is $m(\ddot{r} - r^2 \dot{\theta}^2) = -G(Mm/r^2)$,
  and for steady motion $\ddot{r} = 0$, thus
  \begin{equation*}
    m\frac{v_0^2}{r_0} = G\frac{Mm}{r_0^2},
  \end{equation*}
  so $L = m r_0 v_0$. For the energy equation,
  \begin{equation*}
    E = \frac{1}{2} m v_0^2 - G\frac{Mm}{r_0} = -\frac{1}{2} m v_0^2 < 0.
  \end{equation*}
  The circle is covered once in time $T = 2\pi r_0 / v_0$, and so also from the
  first part,
  \begin{equation*}
    T^2 = \frac{2\pi^2 r_0^2}{v_0^2} = \left(\frac{4\pi^2}{GM}\right) r_0^3.
  \end{equation*}
  This is \Def{Kepler's third law}, where $T^2\sim r^3$, independent of
  the satellite.
\end{example}

\begin{example}[Example: Geostationary satellites]
  For $T = 24\ \mathrm{hrs}$, using Kepler's third law,
  \begin{equation*}
    r_o = \left(\frac{g R^2 T^2}{4\pi^2}\right) \approx 42000\ \mathrm{km}.
  \end{equation*}
  So a geostationary satellite occupies an orbit at around 35600 km in the
  equatorial plane.
\end{example}

\begin{example}[Example: Hubble space telescope]
  This is in a low-Earth orbit at an altitude of around 600 km, so the orbital
  radius is $r_o = 600 + 6400 = 7000\ \mathrm{km}$. Since $T^2$ is proportional
  to $r^3$, the orbital period of the Hubble telescope is given by
  \begin{equation*}
    \left(\frac{T_H}{T_G}\right)^2 = \left(\frac{r_{o,H}}{r_{o,G}}\right)^3
    \qquad \Rightarrow \qquad
    T_H = \left(\frac{7000}{420000}\right)^{3/2}\ \mathrm{s} \approx
    1\ \mathrm{hr}\ 38\ \mathrm{mins}.
  \end{equation*}
\end{example}

\begin{example}[Example: Saturn's ``year'' and radius of orbit]
  Saturn has an orbital period of around 30 years, so $(T_S / T_E)^2 = (r_{o,S}
  / r_{o, E})^3$ leads to Saturn's orbit being $30^{2/3} \approx 10$ times the
  Earth's orbital radius (so around 10 AU).
\end{example}

A mass $m$ may trace out a non-circular orbit around mass $M$ with shape
characterised by its turning points of $r(t) = |\rb(t)|$. At a turning point,
$\dot{r} = 0$, and since
\begin{equation*}
  E = \frac{m}{2}\left(\dot{r} + r^2 \dot{\theta}^2 \right) - \frac{GMm}{r},
  \qquad
  L = mr^2 \dot{\theta}
\end{equation*}
are constants with $\dot{\theta} = L / (mr^2)$, we have the following quadratic
for $r$
\begin{equation*}
  Er^2 + GMmr - \frac{L^2}{2m} = 0,
\end{equation*}
which has roots $r_{1,2}$ satisfying $E(r - r_1)(r - r_2) = 0$. We see that
\begin{equation*}
  r_1 + r_2 = -\frac{GMm}{E}, \qquad r_1 r_2 = -\frac{L^2}{2mE}.
\end{equation*}
Thus if $E < 0$, the particle traces an elliptical path with a bounded orbit,
and if $E > 0$ it traces a hyperbola with an unbounded orbit.
\begin{example}
  A spacecraft in a circular Earth orbit at radius $E$ fires its motor very
  briefly, and increases its speed from $v_0$ to $u$. Find the nature of the
  subsequent orbit.
  
  Before the start of the motor, the speed is $v_0$ so $v_0^2 = GM/D$. Just
  afterwards, $r = D$, and thereafter, $E = mu^2 / 2 - GMm / D$. At the start of
  the orbit, the velocity is transverse, and so $L = mr(r\dot{\theta}) = mD u$.
  Then it may be seen that $r_1 = D$ is a turning point. Then
  \begin{equation*}
    r_2 = -\frac{GMm}{E} - r_1 = \ldots = D\frac{u^2}{2v_0^2 - u^2}.
  \end{equation*}
  We then note that when $u > v_0$, then $r_2 > D$, and $r_2 \to \infty$
  as $u^2 \to 2v_0^2 = 2GM/D$. Indeed, when $u^2 > 2v_0^2$, $E > 0$. On the
  other hand, if $u < v_0$, then $r_2 < D$, and the orbit is bounded. If $v_0
  \to u$, then $r_2 < r_1$, and if $r_2 < R$, then the spacecraft crashes.
\end{example}

If an object comes towards $r = 0$ from $r = \infty$ where it has speed $v$ and
is on a line to miss by $b$ (the \Def{impact parameters}) in the absence
of gravity, then $L = mbv$ and $E = mv^2 / 2 > 0$. The orbit is open and its
distance of closest approach $r_{\rm min}$ is then the single positive root of
\begin{equation*}
  Er^2 + GMmr - \frac{L^2}{2m} = 0.
\end{equation*}

\begin{example}[Example: Asteroids]
  We hope $r_{\rm min} > R$, so that
  \begin{equation*}
    Er^2_{\rm min} + GMmr_{\rm min} > ER^2 + GMmR,
  \end{equation*}
  or equivalently,
  \begin{equation*}
    \frac{L^2}{2m} > ER^2 + GMmR.
  \end{equation*}
  Let $GM = gR^2$, then
  \begin{equation*}
    \frac{L^2}{m} >: 2R^2 (E + mgR), \qquad L = mbv,
  \end{equation*}
  so that for avoiding impact, the impact parameter needs to satisfy
  \begin{equation*}
    b > R \sqrt{1 + \frac{2gR}{v^2}}.
  \end{equation*}
  Thus the Earth's \Def{gravitational cross-section} at speed $v$ is $\pi
  R^2 (1+ 2gR / v)$.
\end{example}

%-------------------------------------------------------------------------------

\section{Kepler's orbits}

Generally speaking $r(t)$ is only solvable in terms of elliptic functions. If
instead we consider solutions of the form $r(\theta)$ of the particle using
\begin{equation*}
  \dot{r} = \dot{\theta} \frac{\mathrm{d}r}{\mathrm{d}\theta} = \frac{L}{mr^2} r'(\theta),
\end{equation*}
then
\begin{align*}
  E &= \frac{M}{2}\left[\left(\frac{L^2}{mr^2}\right)^2 \left(\frac{\mathrm{d}r}{\mathrm{d}\theta}\right)^2 + r^2 \left(\frac{L^2}{mr^2}\right)^2\right] - \frac{GMm}{r}\\
  &= \frac{L^2}{2m}\left[\left(\frac{1}{r^2}\frac{\mathrm{d}r}{\mathrm{d}\theta}\right)^2 + \frac{1}{r}^2\right] - \frac{GMm}{r}.
\end{align*}
Letting $u = 1/r$, then $\mathrm{d}u/\mathrm{d}\theta = -(1/r^2)
(\mathrm{d}r/\mathrm{d}\theta)$, so
\begin{equation*}
  E = \frac{L^2}{2m} \left[\left(\frac{\mathrm{d}u}{\mathrm{d}\theta}\right)^2 +
  u^2\right] - GMmu.
\end{equation*}
The equation is separable, leading to
\begin{equation*}
  \int\; \mathrm{d}\theta = \pm \int \frac{L}{\sqrt{2mE + 2GM^2u - L^2 u^2}}\; \mathrm{d}u.
\end{equation*}
Completing the square leads to
\begin{equation*}
  \theta - \theta_0 = \pm \int \frac{L}{\sqrt{2mE + (GMm^2 / L)^2 - (GMm^2 / L - L u)^2}}\; \mathrm{d}u.
\end{equation*}
Substituting $v = Lu - GMm^2 / L$ and noting that $\int\mathrm{d}x /
\sqrt{a^2 - x^2} = \mathrm{arcsin}(x/a)$ leads to
\begin{align*}
  \theta - \theta_0 &= \pm \int \frac{1}{\sqrt{2mE + (GMm^2 / L)^2 - v^2}}\; \mathrm{d}v\\
  &= \pm \mathrm{arcsin}\left(\frac{Lu - GMm^2 / L}{\sqrt{2mE + (GMm^2 / L)^2}}\right).
\end{align*}
With $\theta_0 = \pi / 2$ unwrapping $u = 1/r$, this results in
\begin{equation*}
  \cos\theta \sqrt{2mE + \left(\frac{GMm^2}{L}\right)^2} = \frac{L}{r} -
  \frac{GMm^2}{L}.
\end{equation*}
Rearranging to
\begin{equation*}
  \frac{GMm^2}{L^2}\left(\cos\theta\sqrt{1 + \frac{2EL}{G^2M^2m^3}} + 1\right) = \frac{1}{r}
\end{equation*}
we have
\begin{equation*}
  r = \frac{A}{1 + \epsilon \cos \theta}, \qquad \epsilon = \sqrt{1 + \frac{2EL}{G^2M^2m^3}}, \qquad A = \frac{L^2}{GMm^2},
\end{equation*}
where $\epsilon$ is the \Def{eccentricity} of the orbit and $A$ is an
amplitude.

If $\epsilon = 0$ then we have a circular orbit, and $r = r_0 = A$. The energy
is
\begin{equation*}
  E = -\frac{1}{2}m^3 \left(\frac{GM}{L}\right)^2 = -\frac{1}{2}\frac{GMm}{r_0} < 0.
\end{equation*}

When $\epsilon < 1$, we note that $E < 0$ and $r(\theta) < \infty$, so there
have a bounded orbit. We note that the orbit is closed ($r(\theta) = r(\theta +
2\pi)$), symmetric ($r(\theta) = r(-\theta)$) and that $r$ increases as $\theta
\to \pi$. Noting that $r_{\rm min/max} = A / (1 \pm \epsilon)$, the orbit is an
elliptical curve, and is near circular when $\epsilon \approx = 0$. Note that
$r_{\rm min/max}$ are turning points, and that
\begin{align*}
  r_{\rm min} r_{\rm max} &= \frac{A}{1 + \epsilon} \frac{A}{1 - \epsilon} = -\frac{L^2}{2ME},\\
  r_{\rm min} + r_{\rm max} &= A\left(\frac{2}{1 - \epsilon^2} \right) = -\frac{GMm}{E},
\end{align*}
which are consistent with previous calculations.

(For completeness, note that $\epsilon_{\rm Earth} = 0.017$, where as
$\epsilon_{\textnormal{Halley's comet}} = 0.967$.)

When $\epsilon > 0$, $e > 0$ and we have an open hyperbolic orbit, approaching
infinity when $\cos \theta \to -1/\epsilon$. When $\epsilon = 1$, this traces
out instead a parabolic open orbit, approaching infinity when $\theta \to \pi$.

Kepler formulated his three laws of planetary motion:
\begin{enumerate}
  \item Planetary orbits are ellipses, with the Sun at one of the focus;
  \item The radius from the Sun to planet sweeps equal areas in equal time;
  \item $T^2 = Ka^3$, where $T$ is the planet's period of orbit and $a = (r_{\rm
  min} + r_{\rm max}) / 2$ is the semi-major axis.
\end{enumerate}
\begin{proof}
\begin{enumerate}
  \item This is just another way of saying idealised point planets with no
  mutual interaction.
  
  \item The area is $(r/2)r\dot{\theta} = \textnormal{const}$ is equivalent to
  statement of angular momentum conservation, or saying that the force is
  central.
  
  \item For $0 \leq \epsilon < 1$,
  \begin{equation*}
    T = \int\mathrm{d}t = \int_0^{2\pi} \frac{\mathrm{d}\theta}{\dot{\theta}}
      = \frac{m}{L} \left(\frac{L^3}{2Mm^2}\right) \int_0^{2\pi} \frac{\mathrm{d}\theta}{(1 + \epsilon \cos \theta)^2} = \frac{L^3}{(GM)^2 m^3} \frac{2\pi}{(1 - \epsilon^2)^{3/2}}.
  \end{equation*}
  Note that
  \begin{equation*}
    a = \frac{r_{\rm min} + r_{\rm max}}{2} = \frac{L^2}{GMm^2} \left(\frac{1}{1 - \epsilon^2}\right), \qquad gR^2 = GM,
  \end{equation*}
  this implies that
  \begin{equation*}
    T = \frac{L^3}{(GM)^2 m^3} \left(\frac{aGMm^2}{L^2}\right)^{3/2} 2\pi,
  \end{equation*}
  so that
  \begin{equation*}
    T^2 = \frac{4\pi^2}{GM}a^3 = \frac{4\pi^2}{gR^2} a^3.
  \end{equation*}
\end{enumerate}
\end{proof}

%===============================================================================

\chapter{Vibrating Strings}

Here we mostly deal with the \Def{wave equation}
\begin{equation*}
  \ddy{^2 u}{x^2} = \frac{1}{c^2} \ddy{^2 u}{t^2}, 
    \qquad u = u(x,t), \quad c > 0, \quad c \in \mathbb{R},
\end{equation*}
where $u$ is the transverse displacement of an idealised string, $x$ is the
distance along the string, $t$ is time, and $c$ is the wave speed. This is a
linear homogeneous partial differential equation, and so can be solved in terms
of a Fourier series via separation of variables.

%-------------------------------------------------------------------------------

\section{Derivation}

Let $\rho$ be the density and $T$ the tension, considered to be constants. Let
$u$ be a small transverse displacement from equilibrium, with the line between
$u(x + \delta x)$ and $u(x)$ connected by a line at angle $\theta$. Then the
transverse component satisfies
\begin{equation*}
  (\rho \delta x) \ddy{^2 u}{t^2} = 
    (T \sin \theta)_{x + \delta x} - (T \sin \theta)_{x}.
\end{equation*}
For small displacement, $\theta$ is small so $\theta \approx \tan\theta = \dy u/
\dy x$, and
\begin{equation*}
  (\rho \delta x) \ddy{^2 u}{t^2} = (T \ddy{u}{x})_{x + \delta x} - (T \ddy{u}{x})_{x}.
\end{equation*}
By the Mean Value Theorem,
\begin{equation*}
  (\rho \delta x) \ddy{^2 u}{t^2} = T \delta x \left(\ddy{^2 u}{x^2} \right)_{\rm x_c},
\end{equation*}
and so as $\delta x \to 0$,
\begin{equation*}
  \rho \ddy{^2 u}{t^2} = T \ddy{^2 u}{x^2},
\end{equation*}
and with $c^2 = T / \rho$ we obtain the wave equation. Derivation using the
longitudinal component gives the same PDE.

Solutions of the wave equation takes the form
\begin{equation*}
  u(x,t) = f(x - ct) + g(x + ct),
\end{equation*}
where $f$ and $g$ are arbitrary functions that are twice differentiable. This may
be confirmed by substituting in and using the chain rule.

\begin{example}
  $u(x,t) = x$ and $u(x,t) = t$ are solutions, since
  \begin{equation*}
    x = \frac{1}{2}\left((x-ct) + (x+ct)\right), \qquad
    t = \frac{1}{2}\left((x-ct) - (x+ct)\right)
  \end{equation*}
\end{example}

%-------------------------------------------------------------------------------

\section{Travelling waves}

Consider the case where $u(x,t) = f(x - ct)$. Now, $p = x - ct =
\textnormal{const}$ on the curve $f(p)$, so $x = \textnormal{const} + ct$, so
$f(x - ct)$ describes waves travelling towards $x = +\infty$ with speed $c$.
Conversely, $g(x + ct)$ describe waves travelling towards $x=-\infty$. These
waves are \Def{non-dispersive}, i.e. the shape of the wave does not change
overall. Since the PDE is linear and homogeneous, by the
\Def{superposition principle}, solutions may be built up as linear
combinations of other solutions; physically, this is the \Def{interference
of waves}.

%-------------------------------------------------------------------------------

\section{d'Alembert's formula}

The wave $u(x,t)$ is completely determined by specifying the string's initial
shape and velocity $u(x, 0) = R(x)$ and $u_t (x,y) = S(x)$.

From $u(x,y) = f(x-ct) + g(x+ct)$, we have $u_t (x,t)= -cf'(x-ct) + cg'(x+ct)$,
so that
\begin{equation*}
  R(x) = f(x) + g(x), \qquad S(x) = -cf'(x) + cg'(x).
\end{equation*}
Then rearranging gives
\begin{equation*}
  -\frac{1}{c}\int_a^x S(z)\; \mathrm{d}z = f(x) - g(x),
\end{equation*}
so that
\begin{align*}
  f(x) &= \frac{1}{2}\left( R(x) - \frac{1}{c} \int_a^x S(z)\; \mathrm{d}z\right) 
       = \frac{1}{2}\left( R(x) + \frac{1}{c} \int_x^a S(z)\; \mathrm{d}z\right),\\
  g(x) &= \frac{1}{2}\left( R(x) + \frac{1}{c} \int_a^x S(z)\; \mathrm{d}z\right).
\end{align*}
So that
\begin{equation*}
  u(x,t) = \frac{1}{2}\left(R(x-ct) + R(x+ct) + \frac{1}{c} \int_{x-ct}^{x+ct} S(z)\; \mathrm{d}z\right).
\end{equation*}

\begin{example}
For a plucked string, $S(x) = 0$, so
\begin{equation*}
  u(x,t) = \frac{1}{2}\left(R(x-ct) + R(x+ct) \right).
\end{equation*}
Waves go either side from original position with half the magnitude. For a
struck string with $R(x) = 0$,
\begin{equation*}
  u(x,t) = \frac{1}{c} \int_{x-ct}^{x+ct} S(z)\; \mathrm{d}z.
\end{equation*}
\end{example}

%-------------------------------------------------------------------------------

\section{Finite string}

d'Alembert's formular does not include any boundary conditions; on a finite
string the spreading waves may bounce back and forth. Consider an ideal string,
with density $\rho$ under tension $T$ between fixed points a distance $L$ apart.
We aim to solve
\begin{equation*}
  \ddy{^2 u}{x^2} = \frac{1}{c^2}\ddy{^2 u}{t^2}, \qquad
  u(0,t) = u(L,t) = 0,
\end{equation*}
with the initial conditions $u(x, 0) = R(x)$ and $u_t (x, 0) = S(x)$.

For \Def{separation of variables}, consider solutions of the form
\begin{equation*}
  u(x, t) = \sum_{i=1}^\infty X_i(x) T_i(t).
\end{equation*}
The idea is that each product term itself solves the PDE, and we construct all
solutions by the principle of linear superposition. We observe that
\begin{equation*}
  \frac{1}{X_i}\ddy{^2 X_i}{x^2} = \frac{1}{c^2}\frac{1}{T_i}\ddy{^2 T_i}{t^2} = K,
\end{equation*}
where $K$ is a \Def{constant of separation} (since LHS is a function of
$x$ only and RHS is a function of $t$ only, they equal when they are a
constant). For boundary conditions to be obeys, $K < 0$, so taking $K =
-k^2$ results in
\begin{equation*}
  \ddy{^2 X_k}{x^2} - k X_k = 0, \qquad
  \ddy{^2 T_k}{t^2} - ck T_k = 0.
\end{equation*}
So the individual solutions are
\begin{equation*}
  u_k(x,t) = \left(A \sin kx + B\cos kx\right) \left(C \sin kct + D\cos kct\right).
\end{equation*}
To satisfy the spatial boundary conditions, $B = 0$ for all $k$, while $kL =
n\pi$ for $n \in \mathbb{N}$, so
\begin{equation*}
  u(x, t) = \sum_{n=1}^\infty \sin\frac{n\pi x}{L} \left(a_n \cos\frac{n\pi ct}{L} + b_n \sin \frac{n\pi ct}{L} \right),
\end{equation*}
and the constants $a_n$ and $b_n$ are determined by the initial conditions:
with
\begin{equation*}
  R(x) = \sum_{n=1}^\infty a_n \sin \frac{n\pi x}{L}, \qquad
  S(x) = \sum_{n=1}^\infty b_n \frac{n\pi c}{L}\sin \frac{n\pi x}{L},
\end{equation*}
the coefficients are obtained by noting the orthogonality relations of of
trigonometric functions, so
\begin{equation*}
  a_n = \frac{2}{L} \int_0^L R(x) \sin\frac{n\pi x}{L}\; \mathrm{d}x, \qquad
  b_n = \frac{2}{n\pi c} \int_0^L S(x) \sin\frac{n\pi x}{L}\; \mathrm{d}x.
\end{equation*}

\begin{example}
An ideal string, fixed at $x = 0, L$ is pulled aside a distance $h$ at its
midpoint ($x = L / 2$) into a triangular shape, and released from rest.

Since $S(x) = 0$, $b_n = 0$. On the other hand,
\begin{equation}
  R(x) = 
  \begin{cases}
    2h x / L, & \quad 0 < x < L/2\\
    2h(L - x) / L, & L / 2 < x < L.
  \end{cases}
\end{equation}
Now, $R(x)$ is even about $x = L / 2$, and thus $R(x) \sin n\pi x /L$ is odd
about $x = L / 2$ for even $n$ and even about $x = L / 2$ for odd $n$, so the
coefficients $a_n$ are zero when $n$ is odd. Thus
\begin{align*}
  a_n &= 2 \frac{2}{L} \int^{L/2}_0 2h \frac{x}{L} \sin\frac{n\pi x}{L}\; \mathrm{d}x \\
  &= \frac{8 h}{L^2}\left[\frac{L}{2}\left(-\frac{L}{n\pi}\right)\cos\frac{n\pi}{2} +
  \left(\frac{L}{n\pi}\right)^2 \sin\frac{n\pi}{2}\right]
\end{align*}
for $n = 2k+1$. Thus the full solution is
\begin{equation*}
  u(x, t) = \frac{8h}{\pi^2} \sum_{k=0}^\infty \frac{(-1)^k}{(2k + 1)^2} 
    \sin\frac{(2k+1) \pi x}{L}\cos\frac{(2k+1) \pi ct}{L}.
\end{equation*}
\end{example}

\begin{example}
If $R(x) = \sin(5\pi x / L)$ and $S(x) = 0$, then the only non-zero coefficient
is $a_5 = 1$, and the solution is $u(x, t) = \sin(5\pi x/L)\sin(5\pi ct / L)$.
\end{example}

%-------------------------------------------------------------------------------

\section{Standing waves}

Taking the case $S(x) = 0$, the solution is
\begin{align*}
  u(x,t) &= \sum_{n=1}^\infty a_n \sin\frac{n\pi x}{L} \cos\frac{n\pi ct}{L}\\
  &= \frac{1}{2} \sum_{n=1}^\infty a_n\left(\sin\frac{n\pi (x-ct)}{L} + \sin\frac{n\pi (x+ct)}{L}\right)
\end{align*}
by trigonometric identities. Each term in the sum (the \Def{harmonic}) is
a pair of left and right propagating waves, reflected to and fro and
\Def{superimposed} to give a \Def{standing wave}.

A standing wave has $x$ dependence but is otherwise independent of $t$. The
factor $\sin(n\pi x / L)$ gives $n+1$ \Def{nodes} (the zeros), and the
wave has \Def{wavelength} $\lambda$ where $L = n\lambda / 2$. Each
harmonic has \Def{angular frequency} $\omega = n\pi c/ L$, with $c =
\sqrt{T / \rho}$.

%===============================================================================

%%%%%%%%%%%%%%%%%%%%%%%%%%%%%%%%%%%%%%%%%

% r.5 contents
%\tableofcontents

%\listoffigures

%\listoftables

% r.7 dedication
%\cleardoublepage
%~\vfill
%\begin{doublespace}
%\noindent\fontsize{18}{22}\selectfont\itshape
%\nohyphenation
%Dedicated to those who appreciate \LaTeX{} 
%and the work of \mbox{Edward R.~Tufte} 
%and \mbox{Donald E.~Knuth}.
%\end{doublespace}
%\vfill

% r.9 introduction
% \cleardoublepage

%%%%%%%%%%%%%%%%%%%%%%%%%%%%%%%%%%%%%%%%%
% actual useful crap (normal chapters)
\mainmatter

%\part{Basics (?)}


%\backmatter

%\bibliography{refs}
\bibliographystyle{plainnat}

%\printindex

\end{document}

